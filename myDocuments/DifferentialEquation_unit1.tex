\section{Introduction to differential equations and modeling}
\subsection{A secret Function}
\begin{ex}
  Can you guess my secret function $y(t)$?
\end{ex}
\paragraph{Clue} It satisfies the differential equation \\
\begin{equation*}
  \dot{y} = 3y,
\end{equation*}
which could model population growth in some biological system, like the growth of yeast cells.
Note that \emph{\color{blue} $\dot{y}$} is common notation for the derivative with respect to time;
$\dot{y}$, $y'$, and $\frac{dy}{dt}$ all mean the same thing here.\\
$y = e^{3t}$ is a solution to the differential equation above because substituting it into the DE gives
$3e^{3t}=3e^{3t}$. But it's not the function I was thinking of! Some other solutions are
$y = 7e^{3t}$, $y = -5e^{3t}$, $y = 0$, etc. From calculus we know that the family of functions
\begin{equation*}
  y = ce^{3t}
\end{equation*}
is the \emph{\color{blue} general solution}.
\begin{itemize}
\item for each $c$, the function $y=ce^{3t}$ is a solution, and
\item there are no other solutions besides these.
\end{itemize}
So there is a 1-parameter family of solutions to this DE.
The constant $c$ is a \emph{\color{blue} parameter}.

\paragraph{Clue 2:} The function satisfies the \emph{\color{blue} initial condition} $y(0)=5$.
\begin{align*}
  y(0) &= ce^{0} \\
  5 &= c  
\end{align*}
\emph{The solution} satisfying the initial condition $y(0) = 5$ is
\begin{equation*}
  y(t) = 5e^{3t}
\end{equation*}
\begin{mydef}
  An \emph{\color{blue} initial value problem}
  is a differential equation together with initial conditions.
\end{mydef}

\emph{\color{orange} Important} Checking a solution to a DE is usually easier
than finding the solution in the first place, so it is often worth doing.
Just plug in the function to both sides, and also check that it satisfies the initial condition.\\

In the solution to the DE, only one initial condition was needed,
since only one parameter $c$ needed to be recovered.
\clearpage

\subsection{Cell division}
Here we will see how the differential equation for our secret function appears when modeling a natural phenomenon - the population growth of a colony of cells.\\

In this example we'll model the number of yeast cells in a batch of dough.
As we work through this example, pay careful attention to the assumptions we make, and how the \emph{\color{blue} initial condition} plays a role in the resulting differential equation.

\subsubsection{\color{blue} The system}
For our system, we assume we have a colony of yeast cells in a batch of bread dough.
The first step is to identify the variables, the units, and give them names.
\begin{align*}
  y\qquad &\text{number of cells}\\
  t\qquad &\text{time measured in seconds}
\end{align*}
We also need to set some initial condition, $y_0$, the number of cells that we begin with at $t=0$.
In this system, this might be the number of yeast cells in a yeast packet.
\subsubsection{\color{blue} A differential model}
If $y$ denotes the number of yeast cells, what can we say about the derivative $\dot{y}$?
The derivative represents the rate at which the number of cells is growing.
How should it depend on the number of cells?
In nature, cells given plenty of space and food tend to divide through mitosis regularly.
If we assume that each cell is dividing independently of all other cells, then doubling the number of cells should double the rate at which new cells are born.
In fact, multiplying the number of cells by any scalar factor should do the same to the derivative.
So this directly implies that the growth rate of cells is proportional to the number of cells:

\begin{equation*}
  \dot{y} \propto y.  
\end{equation*}
We can make this into a true equation by simply inserting a proportionality constant $a$, such that
\begin{equation*}
  \dot{y} = ay. 
\end{equation*}
We say that $1/a$ is a ``characteristic'' timescale for our problem, setting the rate at which the cells divide. A solution to the above differential equation is
\begin{equation*}
  y = y_0e^{at},
\end{equation*}
where $y_0$ is the number of yeast cells we started with at $t=0$.
In our case, we assume that $y_0$ is the number of yeast cells in a packet, which is about $180$ billion yeast cells.
\clearpage

\subsection{Classification of differential equations:}
There are two kinds:
\begin{itemize}
\item An \emph{\color{blue} ordinary differential equation (ODE)}  involves derivatives of a function of \emph{only one} variable.
\item A \emph{\color{blue} partial differential equation (PDE)} involves \emph{partial derivatives} of a \emph{multivariable} function.
\end{itemize}

Here we will study ordinary differential equations. In \textit{Fourier Series and PDEs}, we will study partial differential equations. When we consider ODEs, we will often regard the independent variable to be time.\\

Notation for higher derivatives of a function $y(t)$:
\begin{align*}
  \text{first derivative: } \qquad  &\dot{y}\qquad y'\qquad \frac{dy}{dt}\\
  \text{second derivative: } \qquad  &\ddot{y}\qquad y''\qquad \frac{d^2y}{dt^2}\\
  \text{third derivative: } \qquad  &y^{(3)}\qquad \qquad \frac{d^3y}{dt^3}\\
                                    &\centerdot \\
                                    &\centerdot \\
                                    &\centerdot \\
  n^{th} \text{derivative: } \qquad &y^{(n)}\qquad \qquad \frac{d^ny}{dt^n}                              
\end{align*}

\emph{\color{orange} Warning: } The dot notation $\dot{y}$ should only be used to refer to a time derivative. If for example $y$ is a function of a spacial variable $y=y(x)$, we will only use the notation $y'$ to denote the derivative with respect to $x$.\\

\begin{mydef}
  The \emph{\color{blue} order} of a DE is the highest $n$ such that the $n^{th}$ derivative of the function appears.
\end{mydef}
\begin{ex}
  is 
\end{ex}
\begin{align*}
  \displaystyle  \displaystyle 9375 \cos (t^5) \, \ddot{y}^4 + 982 (y + t^3)^7 \,
  \dot{y} - 387 y^{(3)} y^{(4)} + 2t y^{(5)} \\
  \displaystyle {}=- 30987\sqrt {y}+ 4087 \sin (t/y) + 7092 \ln (1- t^7) + 3786 e^{y^8 - t^3} + 18.031 \,
\end{align*}
an ODE or a PDE?
\begin{mybox}{gray}{Answer}
  There is only only one dependent variable $y$ and independent variable $y$, So ODE.  
\end{mybox}
\clearpage

\begin{problem}
  Order of differential equation 
\end{problem}
What is the order of the differential equation in the above example?
\begin{mybox}{gray}{Answer}
  The highest $n$ is such that $n^{th}$ derivative of $y$ is $y^{(5)}$. 
\end{mybox}
\clearpage

\subsection{More terminology}
\begin{enumerate}
\item A \emph{\color{blue} homogeneous linear ODE} is a differential equation such as
  \begin{equation*}
    e^ t \,  {\color{blue}{\ddot{y}}}  + 5 \,  {\color{blue}{\dot{y}}}  + t^9 \,  {\color{blue}{y}}  = 0
  \end{equation*}
  in which each summand is a function of $t$ times one of $\color{blue}{\ddot{y}}$,
  $\color{blue}{\dot{y}}$, $\color{blue}{\dot{y}}$, $\centerdot$ $\centerdot$  $\centerdot$
  \textbf{Most general} $n^{th}$ \textbf{order homogeneous linear ODE:}
  \begin{equation*}
    p_ n(t) \,  {\color{blue}{y^{(n)}}}  + p_{n-1}(t) \,  {\color{blue}{y^{(n-1)}}}  \dotsc + p_1(t) \,  {\color{blue}{\dot{y}}}
    + p_0(t) \,  {\color{blue}{y}}  = 0
  \end{equation*}
  for some function $p_n(t), \cdots, p_0(t)$ called the {\color{blue}{coefficient}}.
\item An \emph{\color{blue} inhomogeneous linear ODE} is the same except that it has also
  \emph{\color{orange} one term that is a function of $t$ only} . \\
  For example,
  \begin{equation*}
    e^ t \,  {\color{blue}{\ddot{y}}}  + 5 \,  {\color{blue}{\dot{y}}}  + t^9 \,
    {\color{blue}{y}}  = {\color{orange}{7 \sin t}}
  \end{equation*}
  is a second-order inhomogeneous linear ODE.\\
  \textbf{Most general $n^{th}$ order inhomogeneous linear ODE:}
  \begin{equation*}
    p_ n(t) \,  {\color{blue}{y^{(n)}}}  + p_{n-1}(t) \,  {\color{blue}{y^{(n-1)}}}  \dotsb + p_1(t) \,
    {\color{blue}{\dot{y}}}  + p_0(t) \,  {\color{blue}{y}}  = {\color{orange}{q(t)}}
  \end{equation*}
  for some functions $p_n(t), \cdots , p_0(t), {\color{orange}, q(t)}$. \\
\item A \textbf{\color{blue} linear ODE} is an ODE that can be rearranged into either of the two types above.
\item Either of the two forms above can be reduced further by dividing the entire DE by $p_n(t)$,
  the coefficient of the highest derivative $y^{(n)}$ becomes 1.
  A differential equation written in either of the two forms above but with leading coefficient $1$ is said to be in standard linear form .
  \begin{align*}
    &{\color{blue}{y^{(n)}}}  + p_{n-1}(t) \,  {\color{blue}{y^{(n-1)}}}  \dotsb + p_1(t) \,
    {\color{blue}{\dot{y}}}  + p_0(t) \,  {\color{blue}{y}}  = 0\\
    &{\color{blue}{y^{(n)}}}  + p_{n-1}(t) \,  {\color{blue}{y^{(n-1)}}}  \dotsb + p_1(t) \,
    {\color{blue}{\dot{y}}}  + p_0(t) \,  {\color{blue}{y}}  = {\color{orange}{q(t)}}
  \end{align*}
\end{enumerate}

\begin{remark}
  We insist that solutions $y(t)$ are defined on an entire open interval $I$, as opposed to disjoint intervals.
  We will always assume the functions $p_n(t), p_{n−1}(t), \cdots, p_0(t), q(t)$ are well behaved,
  for example continuous, on $I$.
  (We'll look at more complicated functions in our class Transfer Functions and the Laplace Transform.)
\end{remark}

\begin{mybox}{gray}{What is open interval?}
  An \textbf{\color{blue} open interval} is a connected set of real numbers without endpoints.
  The notation $(a, b)$ denotes the open interval of all points $x$ such that $a < x < b$.
  It is conventional to allow one or both end points to be infinite:
  $(a, b)$, $(- \infty, b)$, $(a, \infty)$.
  The open interval $(-\infty, \infty)$ denotes the entire real line,
  which is often denoted by $\mathbb{R}$.
\end{mybox}

\begin{remark}
  If you already know that an ODE is linear, there is an easy test to decide if it is homogeneous or not:
  Plug in the constant function $y=0$.
\end{remark}
\begin{itemize}
\item if $y = 0$ is a solution, the ODE is homogeneous.
\item if $y = 0$ is not a solution, the ODE is inhomogeneous. 
\end{itemize}

\begin{remark}
  The cell division example gives us a homogeneous equation.
  If you start with no yeast cells, the population stays at $0$.
\end{remark}
For an ODE to be {\color{blue}nonlinear}, the functions $y, \dot{y},\cdots $ must enter the equation in a more complicated way: raised to powers, multiplied by each other, or with nonlinear functions applied to them.

\begin{problem}
  Liner vs. nonlinear concept check
\end{problem}
Which of the following ODEs are linear?
\begin{enumerate}
\item $\ddot{y} -7ty \dot{y} = 0$
\item $\ddot{y} = e^t (y+t^2)$
\item $\dot{y} - y^2 = 0$
\item $\dot{y}^2 - t y = \sin t$
\item $\dot{y} = \cos (y+t)$
\end{enumerate}

\begin{mybox}{gray}{Answer}
  $\ddot{y} -7ty \dot{y} = 0$, $7ty \dot{y}$ multiplied by each other.\\
  $\ddot{y} = e^t (y+t^2)$ is standard linear form. \\
  $\dot{y} - y^2 = 0$, $y^2$ is raised powers.\\
  $\dot{y}^2 - t y = \sin t$, $\dot{y}^2$ is raised powers. \\
  $\dot{y} = \cos (y+t)$, $\cos (y+t)$ with nonlinear function applied.   
\end{mybox}
\clearpage

\subsection{Natural growth and decay equations}
We've been introduced to a few basic forms of differential equations so far.
The first equation we saw was a basic growth equation,
\begin{equation*}
  \dot{y} = ay, 
\end{equation*}
which, when $a$ is a positive constant, governs systems like bank accounts and cell populations.
If we put a negative sign in front of $a$ we get the decay equation
\begin{equation*}
  \dot{y} = -ay, 
\end{equation*}
which can be used to describe things like radioactive decay of materials.
\begin{problem}
  Classification of ODEs concept check
\end{problem}
How would you classify the differential equations $\dot{y} = ay$ and
$\dot{y} = −ay$ just discussed?
\begin{mybox}{gray}{Answer}
  $\dot{y} = −ay$ rearranged
  \begin{align*}
    \dot{y} - ay = 0, \\
    \dot{y} + ay = 0\\
  \end{align*}
  $\dot{y}$ first order, only one term of $y$ linear and homogeneous ODE.   
\end{mybox}
\clearpage

\subsection{Introduction to modeling}
There are two kinds of modeling.
We're not going to talk about the kind that involves fancy clothes and photographs.
The other kind, mathematical modeling , is converting a real-world problem into mathematical equations.\\
Guideline: 
\begin{enumerate}
\item Identify relevant quantities, both known and unknown, and give them symbols. Find the units for each.
\item Identify the independent variable(s).
  The other quantities will be functions of them, or constants. Often \textbf{time} is the only independent variable.
\item Write down equations expressing how the functions change in response to small changes in the independent variable(s).
  Also write down any “laws of nature" relating the variables.
  As a check, make sure that all summands in an equation have the same units.
\end{enumerate}
Often simplifying assumptions need to be made;
the challenge is to simplify the equations so that they can be solved but so that they still describe the real-world system well.\\
(Once we have the language of input signals and system response, we'll refine these guidelines.)
\subsubsection{Model of a savings account}
\begin{ex}
  I have a savings account earning interest compounded daily,
  and I make frequent deposits or withdrawals into the account.
  Find an ODE with initial condition to model the balance.
\end{ex}
\textbf{\color{blue} Simplifying assumptions:}
\begin{itemize}
\item Daily compounding is almost the same as continuous compounding, so let's assume that interest is paid continuously instead of at the end of each day.
\item Similarly, let's assume that my deposits/withdrawals are frequent enough that they can be approximated by a continuous money flow at a certain rate, the {\color{blue}net deposit rate}
  (which is negative when I am withdrawing).
\end{itemize}
\textbf{\color{blue} Variables and functions(with unis): } Define the following:
\begin{align*}
  &P \qquad \text{the initial amount that the account starts with (\textit{dollars})} \\ 
  &t \qquad \text{time from the start \textit{years}} \\
  &x \qquad \text{balance \textit{dollars}} \\
  &I \qquad \text{Interest rate}  (year^{-1}; \text{for example} 4\%/year = 0.04yesr^{-1}) \\
  &q \qquad \text{the net deposit rate} (dollars/yes).
\end{align*}
Here $t$ is the independent variable, $P$ is a constant, and $x$, $I$, $q$ are functions of $t$. \\
\textbf{\color{blue} Equations:} Now we want to decide how the balance changes as time changes.
We'll estimate the change in the balance $\Delta x$ as time increases from some time $t$
to a time $t+ \Delta t$. We can approximate the interest earned per dollar to be:
\begin{equation*}
  \displaystyle \text{interest earned {\color{orange}per dollar }}  \approx I(t) \Delta t 
\end{equation*}
The balance in the account at time $t$ is $x(t)$.
Therefore, the total interest earned from time $t$ to $t + \Delta t$  is approximately
\begin{equation*}
  \displaystyle \text{net amount deposited into the account} \approx q(t) \Delta t 
\end{equation*}
Putting it all together, the change in balance is
\begin{align*}
  \displaystyle &\Delta x = \qquad \approx I(t) x(t) \Delta t + q(t) \Delta t \\
  \displaystyle &\Longrightarrow \frac{\Delta x}{\Delta t} \approx I(t) x(t) + q(t).
\end{align*}
The smaller $\Delta t$ is, the better the approximation becomes, and in the limit as $\Delta t \to 0$
\begin{equation*}
  \displaystyle  \lim \limits _{\Delta t\rightarrow 0}\frac{\Delta x}{\Delta t}=\frac{dx}{dt}.
\end{equation*}
This yields the {\color{orange} differential equation}
\begin{equation*}
  \displaystyle  \frac{dx}{dt}=I(t)x(t)+q(t).
\end{equation*}
Note that the units in each of the three terms are $dollars/year$.
Also, there is the initial condition $x(0)=P$.
Thus we have an ODE with initial condition:
\begin{equation*}
  \displaystyle \dot{x} = I(t)x + q(t), \qquad x(o) = P
\end{equation*}

\begin{remark}
  The notation we chose suggested that the interest rate $I$ depended only on time.
  However, I could have depended on $x$ as well.
  This would not change the modeling process.
  If I does not depend on $x$, we obtain a linear differential equation.
  If it does, the equation is nonlinear.
\end{remark}
\clearpage

\subsection{Application: mixing salt water solution} 
Now suppose we want to set up an experiment on ocean fish.
\begin{figure}[ht!]
  \centering
  \includegraphics[width=0.5\textwidth]{image-WaterTank}
  \caption{Image of Water Tank}
  \label{fig:Figure}
\end{figure}
These fish typically live in saltwater that contains a salt concentration of 34 grams per liter.
To make their environment, we're going to start with a tank that contains fresh water, with the volume going by $v$ of $t$.
We're going to hook up a pipe that allows a continuous flow of a concentrated salt solution
to enter the tank.
And then we'll also have a pipe that allows for drainage from the tank. \\
So let's give some numbers to these quantities so we can solve this problem.
\begin{align*}
  \text{Given : } &\qquad 800L (\text{ flash water}) \\
  \text{flow in : } &\qquad 5 Liter/min , \text{concentration } 75 g/L \\
  \text{flow out : } &\qquad 3 Liter/min 
\end{align*}
we're going to assume that the fluid inside the tank is instantaneously well mixed.
The Goal is model the change in salt concentration over time.
\emph{Guidelines :}
\begin{enumerate}
\item Identify relevant quantities.
  \begin{align*}
    t &\qquad \text{Time in minutes} (min) \\    
    v &\qquad \text{total liters fluid in tank} (L/min) \\
    x &\qquad \text{Total number of salt in the tank} (g/L) \\
  \end{align*}
  The concentration is
  \begin{equation*}
    \displaystyle concentration = \frac{x(t)}{v(t)}. 
  \end{equation*}
\item Identify the independent variable.\\
  The independent variable is time in minutes$t$.
\item Write down equations. \\
  The concentration $\displaystyle \frac{x(t)}{v(t)}$, both numerator and denominator are change
  over time, which makes this quantity really complicated to model.
  Instead, it turns out it's much easier to just think about modeling $x$ of $t$ alone
  and then dividing by an expression for $v$ of $t$.
  \begin{equation*}
    \displaystyle \frac{dx}{dt} = \text{Rate in} - \text{Rate out}
  \end{equation*}
  The rate of salt enter in is
  \begin{equation*}
    \displaystyle \text{Rate in } = (75 \frac{g}{L}) \cdot (5 \frac{L}{min}) = 375 \frac{g}{min},
  \end{equation*}
  and the rate of salt out is
  \begin{equation}
    \label{eq:1}
    \displaystyle \text{Rate out} = \text{concentration of existing fluid in} \cdot 3 \frac{L}{min} 
  \end{equation}
  What is the concentration of existing fluid in is?
  This is where we use the assumption that the fluid inside the tank is instantaneously well mixed.
  We said that concentration is $\displaystyle \frac{x(t)}{v(t)}$.
  And since we're ignoring complicated mixing dynamics, we're going to assume that this is true everywhere
  instantaneously in the tank.
  Specifically, it's true where the fluid is exiting the tank.
  And so we can say that the concentration of {\color{orange}the exiting fluid} is
  $\displaystyle \frac{x(t)}{v(t)}$ grams per liter.
  But we can also write down an expression for $v$ of $t$.
  \begin{equation*}
    \displaystyle v(t) = 800 L + (5 \frac{L}{min} - 3 \frac{L}{min}) \cdot t\\
     = 800 + 2t [L] \\
  \end{equation*}
  We substitute this into \ref{eq:1} is 
  \begin{equation*}
    \displaystyle \text{Rate out} = \frac{x(t)}{800 + 2t} \cdot 3 \frac{L}{min}.  
  \end{equation*}
  So our DEs is
  \begin{align*}
    \displaystyle \frac{dx}{dt} &= \text{Rate in} - \text{Rate out} \\
    \displaystyle \frac{dx}{dt} &= 375 - \frac{3x}{800 + 2t}, 
  \end{align*}
  is fist linear inhomogeneous DEs. 
\end{enumerate}


%%% Local Variables:
%%% mode: latex
%%% TeX-master: "NoteForDifferentialEquation"
%%% End:

\section{Sinusoidal Functions}
\subsection{Solutions to the spring-mass-dashpot system}
\begin{exercise}
  Review: spring-mass-dashpot system with distinct complex roots
\end{exercise}
\begin{figure}[ht!]
  \centering
  \includegraphics[width=0.6\textwidth]{image-spring_mass_dashpot}
  \caption{Spring Mass Dashpot System}
\end{figure}

Recall that the position of the mass in a spring-mass-dashpot system without external force can be modelled
by the second order linear homogeneous ODE.
\begin{equation*}
  m {\color{blue}{\ddot{x}}}  + b {\color{blue}{\dot{x}}}  + k {\color{blue}{x}}  =0.
\end{equation*}

Let the damping constant be small enough so that $b^2 < 4mk$ and the characteristic polynomial has
$2$ distinct complex roots.\\
Find the two characteristic roots $r_1$ and $r_2$ of this system in terms of the mass $m$,
the spring constant $k$, and the damping constant $b$.\\
(Enter your answer so that $r_1$ is the root with positive imaginary part, and $r_2$ is the root
with negative imaginary part.)

\begin{mybox}{gray}{Answer}
  The root of characteristic equation is
  \begin{equation*}
    mr^2 + br + k = 0 
  \end{equation*}
  in condition of $b^2 < 4mk$, the two distinct imaginary root of is
  \begin{equation*}
    \displaystyle \frac{-b \pm \sqrt{b^2 - 4mk}}{2m} \displaystyle =
    \displaystyle \frac{-b \pm \sqrt{(-1)(4mk -b^2)}}{2m} \displaystyle =
    \displaystyle \frac{-b \pm i\sqrt{4mk -b^2}}{2m}
  \end{equation*}
\end{mybox}

\begin{exercise}
  Review: basis of solutions
\end{exercise}
Consider the same spring-mass-dashpot system as above.
Let the two distinct characteristic roots be $ -s\pm i\omega _ d$
($s$ and $\omega d$ can be expressed in terms of $m$, $k$, and $b$).
\begin{mybox}{gray}{Answer}
  The exponential solutions is
  \begin{equation*}
    \displaystyle e^{\frac{-b + i\sqrt{4mk -b^2}}{2m}} ; \quad
    \displaystyle e^{\frac{-b - i\sqrt{4mk -b^2}}{2m}}
  \end{equation*}
  as a basis of the collection of all solutions.
  $b^2 - 4mk < 0$,  then the real and imaginary parts of either of the exponential solutions
  \begin{eqnarray*}
    \displaystyle \mathrm{Re\, }\left(e^{\frac{-b+\sqrt {b^2-4mk}}{2m}t}\right)
    & \displaystyle =& \displaystyle e^{-\frac{b}{2m}t}\cos \left(\frac{\sqrt {4mk-b^2}}{2m}t\right)\\
    \displaystyle \mathrm{Re\, }\left(e^{\frac{-b+\sqrt {b^2-4mk}}{2m}t}\right)
    & \displaystyle =& \displaystyle e^{-\frac{b}{2m}t}\sin \left(\frac{\sqrt {4mk-b^2}}{2m}t\right)
  \end{eqnarray*}
  form another basis of the solutions.
\end{mybox}
\clearpage

\subsection{Real solutions to spring-mass-dashpot system}
\begin{figure}[ht!]
  \centering
  \includegraphics[width=0.5\textwidth]{image-spring_mass_dashpot}
  \caption{Spring Mass Dashpot System}
\end{figure}

In the last lecture, we modeled the spring-mass-dashpot system with
no external force by the second order linear homogenous DE
\begin{equation*}
  m {\color{blue}{\ddot{x}}}  + b {\color{blue}{\dot{x}}}  + k{\color{blue}{x}}  =0 \qquad m>0, \, \, b,\, k\, \geq 0.
\end{equation*}
When there are two distinct characteristic complex roots, that is,
when the discriminant $\, b^2-4km<0,\,$
the real basis that you found in the previous problem gives the general real solution:
\begin{equation*}
  \displaystyle  e^{(-b/2m)t} \left( c_1\cos \left(\omega _ d t\right)+c_2 \sin \left(\omega _ d t\right)\right)
  \qquad c_1,\, c_2 \in \mathbb {R},
\end{equation*}
where $\,  \displaystyle \omega _ d\, =\, \sqrt {\frac{k}{m}-\frac{b^2}{4m^2}}\,$
is called the \textbf{{\color{blue} damped frequency}} of the system.
\begin{figure}[ht!]
  \centering
  \includegraphics[width=0.5\textwidth]{image-damped_frequency}
  \caption{Damped Frequency}
\end{figure}
It is useful to contrast the solutions of the general spring-mass-dashpot system with
the solutions to the system with no dashpot (damping), i.e. $\, b=0 (<4mk).\,$.
If the system has no damping, the DE reduces to
\begin{equation*}
  m {\color{blue}{\ddot{x}}}  + k {\color{blue}{x}}  =0\qquad m>0, \, \, k\, \geq 0,
\end{equation*}
and the general real solution reduces to
\begin{equation*}
  \displaystyle  c_1 \cos \left(\omega _ n\right)+c_2\sin \left(\omega _ n\right)
  \qquad c_1,\, c_2 \in \mathbb {R},
\end{equation*}
where $\,  \displaystyle \omega _ n\, =\, \sqrt {\frac{k}{m}}\,$
is called the \textbf{{\color{blue} natural frequency}} of system.
We will continue to call the value of $\omega _ n$ the natural frequency
of system even if damping is present ($b > 0$).
\begin{question}
  What do these general solutions look like? 
\end{question}
Both of these general solutions involve linear combinations of $\cos (\omega t)$ and
$\sin (\omega t)$ for some angular frequency $\omega$ that is the same for both
the sine and cosine functions. What so the graph of such linear combinations looks like?\\
To make graph these solutions easy, we will first rewrite the linear combination
$\, a\cos (\theta )+b\sin (\theta )\,$ in \textit{polar form}.
\clearpage

\subsection{From rectangular to polar form}
\begin{equation*}
  \displaystyle \frac{1}{1 + (\frac{w}{k})^2}
  \left( \cos \omega t + \frac{w}{k} \sin \omega t \right)
\end{equation*}
Convert $\cos \omega t + \frac{w}{k} \sin \omega t$ into other form.
Using\\
\begin{SCfigure}[][h] % position of SCfigure here 
  \begin{minipage}{.5\textwidth}
    \begin{equation*}
        a \cos \theta + b \sin \theta = C \cos \left( \theta - \phi \right)
    \end{equation*}
  \end{minipage}%
  \begin{minipage}{0.5\textwidth}
    \centering
    \includegraphics[width=0.4\textwidth]{image-cosine_triangle}
  \end{minipage}
\end{SCfigure}

Apply it to $\cos \omega t + \frac{w}{k} \sin \omega t$\\
\begin{SCfigure}[][h] 
  \begin{minipage}{.6\textwidth}
    \begin{equation*}
      \displaystyle \frac{1}{1 + (\frac{w}{k})^2} \sqrt{1 + (\frac{w}{k})}
       \cos (\omega t - \phi) \quad \text{ where, } \phi = \arctan \frac{w}{k}. 
    \end{equation*}
  \end{minipage}%
  \begin{minipage}{0.4\textwidth}
    \centering
    \includegraphics[width=0.4\textwidth]{image-cosine_triangle_2}
  \end{minipage}
\end{SCfigure}

There are two ways of expressing any \textbf{\color{blue} sinusoidal function},
in rectangular form and in \textbf{\color{orange}polar form}. They are related as follows:
\begin{equation*}
  \displaystyle  \underbrace{{\color{orange}{A}}
    \cos (\theta -{\color{orange}{\phi }} )}_{{\color{orange}{\text {polar form}}} }
  \qquad {\color{blue}{a}} ,\,
  {\color{blue}{b}} ,\,  {\color{orange}{\phi }}  \in \mathbb {R}, \  \  \
  {\color{orange}{A}} \geq 0 \in \mathbb {R},
\end{equation*}
where $A$ and $\phi$ in terms of $a$ and $b$ are given implicitly by the following diagram:
\begin{figure}[ht!]
  \centering
  \includegraphics[width=0.5\textwidth]{image-rectangular_form}
  \caption{Rectangular Form}
\end{figure}
\clearpage

That is,
\begin{align*}
  &A \quad = \quad \sqrt{a^2 + b^2} \\
  &\phi \quad : \quad
    \text {Angle between the positive horizontal axis and the ray to the point }\, (a, b).
\end{align*} 
Note that $\, (A,\phi )\,$ are polar coordinates of the point with rectangular coordinates
$\, (a,b).\,$ We will use the convention $\, A\geq 0.\, \,$
As usual for polar coordinate, the angle $\, \phi \,$ is well-defined
only up to addition of integer multiples of $2\pi$. \\

It is amazing that the sum of two sinusoids is another sinusoid!
And while graphing the sinusoid in rectangular form is a mystery,
graphing the polar form is easy. We will do this shortly. 
\clearpage

\subsection{Worked Example}
In practice, the argument of the cosine and sine terms is often a function rather than a constant angle.
In this course, we usually have
\begin{equation*}
  \theta = \omega t \qquad \text{for some } \omega > 0
\end{equation*}

\begin{example}
  Convert $\,  -\cos (5t) - \sqrt {3} \sin (5t)$ to polar form.
\end{example}
\Solution \\
Given: $\, a=-1,\, b=-\sqrt {3},\, \theta (t)=\omega t\,$ where $\omega = 5$.\\
Want: $\, A,\, \phi .\,$ These are polar coordinates of the point with rectangular coordinates(a,b). 
\begin{figure}[ht!]
  \centering
  \includegraphics[width=0.5\textwidth]{image-example_polar_rectangular}
  \caption{Diagram of  rectangular form}
\end{figure}

Using the diagram above,
\begin{eqnarray*}
  \displaystyle A \quad & \displaystyle = & \displaystyle \sqrt{(-1)^2 + (\sqrt{3})^2} = 2\\
  \displaystyle \phi \quad & \displaystyle = & \displaystyle - \frac{2 \pi}{3}. 
\end{eqnarray*}

Therefore, the answer is
\begin{equation*}
  \displaystyle A \cos (\omega t - \phi) \displaystyle = \displaystyle 2 \cos (5t + \frac{2 \pi}{3} )
\end{equation*}

Note that $4 \pi / 3$(or $-\frac{2\pi }{3}+2\pi n$) also works since $\phi$ is
well-defined up to addition of $2 \pi$. 
\clearpage

\subsection{Proof}
\subsubsection{Proof of the trigonometric identity}
\begin{equation*}
  a \cos \theta + b \sin \theta = C \cos \left( \theta - \phi \right)
\end{equation*}

\paragraph{The high school proof}\footnote{
  \begin{align*}
    C \cos (\theta - \phi) &= C \left(\cos\theta \cos(-\phi) - \sin \theta \sin(-\phi) \right) \\
    &= C \left(\cos\theta \cos \phi + \sin \theta \sin \phi \right) \\
    &= (C cos \phi) \cos \theta + (C \sin \phi) \sin \theta
  \end{align*}
  Compare two form
  \begin{equation*}
    a \cos \theta + b \sin \theta \quad \text{with} \quad
    (C \cos \phi) \cos \theta + (C \sin \phi) \sin \theta
  \end{equation*}
  \begin{eqnarray*}
    a &=& C \cos \phi\\
    b &=& C \sin \phi \\
  \end{eqnarray*}
  How can we use these to find values for $A$ and $\phi$?
  \begin{align*}
    a^2 + b^2 &= C^2 (\cos \phi)^2 + C^2(\sin \phi)^2 \\
              &= C^2(\cos^2 \phi + \sin^2 \phi) \\
              &= C^2
  \end{align*}
  So, $\displaystyle C \displaystyle = \displaystyle \sqrt{a^2 + b^2}$ and
  $\phi$ is
  \begin{eqnarray*}
    \displaystyle \frac{C \sin \phi}{C \cos \phi}
    &\displaystyle = & \displaystyle \frac{b}{a}\\
    \tan \phi &=& \frac{b}{a}                                                                     
  \end{eqnarray*}
}

The thing that's no good about it is that the direction in which it goes is from right to left side.
Well, everybody knew that. If I gave you this and told you write it out
in terms of cosine and sine, I would assume and dearly hope that practically all of you could do that.
Unfortunately, when you want to use the formula, it's this way you want to use it in the opposite direction.
You're starting with this and want to convert it to that.
Now the proof, therefore, will not be of much help.
It requires you to go in the backwards direction and match up coefficients.
It's much better to go forwards.
Now there are two proofs that go forwards.
\clearpage

\paragraph{The 18.02 proof} is
\begin{figure}[ht!]
  \centering
  \includegraphics[width=0.5\textwidth]{image-18_02_proof}
  \caption{18.02 Proof}
  \label{fig:18.02 Proof}
\end{figure}

First of all, the $a$ and the $b$ are the given.
So I'm going to put in that vector $< a, b>$.
Now, there's another vector lurking around.
It's the unit vector whose-- I'll write it this way, $\hat u$ because it's
a unit vector and theta to indicate that its angle is $\theta$.
Now the reason for doing that is because you
see that the left hand side is a dot product of two vectors.
\begin{equation*}
  <a , b> \cdot <\cos \theta, \sin \theta>
\end{equation*}
The left hand side of the identity is the dot product of the vector $<a, b>$ with the vector
whose components are cosine $\theta$ and sine $\theta$.
That's what I'm calling this unit vector.
It's a unit vector because cosine squared plus size squared is one.
Now all this formula is is saying that scale of product, the dot product of those two vectors,
can be evaluated if you know their components by the left hand side of the formula.
And if you don't know their components, it can be evaluated in another way.
The geometric evaluation, which goes--What is it?
It's the magnitude of one times the magnitude of the other times the cosine of the included angle.
Now what's the included angle?
Well, $\theta$ is this angle from the horizontal to that unitvector.
The angle $phi$ is this angle from figure \ref{fig:18.02 Proof} here.
And therefore the included angle between $\theta$ and my pink vector is theta minus $\phi$.
That's the formula.
\begin{equation*}
  <a , b> \cdot < \cos \theta, \sin \theta > = |<a, b>| \cdot 1 \cdot \cos (\theta - \phi)
\end{equation*}
It comes from two ways of calculating the scalar product of the vector whose coefficients are
$<a, b>$ and the unit vector whose components are cosine theta and sine theta.\\

\paragraph{The 18.03 proof} is uses complex numbers.
Take the left side. Instead of viewing it as the dot product of two vectors,
there's another way.
You can think of it as part of the product of two complex numbers.
So the 18.03 argument, really the complex number argument, says look.
Multiply together $a -bi$ and the complex number $\cos \theta + i \sin \theta$.
\begin{equation*}
  (a - bi)(\cos \theta + i \sin \theta)
\end{equation*}
There are different ways of explaining why I want to put the $-i$ there instead of $i$.
But the simplest is because, when I take the real part to get the left hand side, I will.
If I take the real part of this, I'm going to get $a \cos \theta + b \sin \theta$
because $-i$ and $i$ make one multiplied together.
That's the left hand side. And now the right hand side, I'm going
to use polar representation instead. What's the polar representation of this guy?
Well, if $<a, b>$ has the angle $\theta$, then $a -bi$, goes down below on figure \ref{fig:18.02 Proof}.
It has the angle $- \ phi$. 
In its polar representation, $\sqrt{a^2 + b^2} e^{-\phi}$ not positive $\phi$
because of $a -bi$ goes below the axis if $a$ and $b$ are positive.
How about the second guy?
The second guy is $e^{i \theta}$. 
So what's the product?
\begin{align*}
  (a - bi)(\cos \theta + i \sin \theta) &= \sqrt{a^2 + b^2} e^{-\phi} \cdot e^{i \theta}\\
  &= \sqrt{a^2 + b^2} e^{i(\theta - \phi)}
\end{align*}
And now what do I want? The real part of this.
And I want the real part of this.
So let's just say take the real parts of both sides.
\begin{equation*}
  a \cos \theta + b \sin \theta = \sqrt{a^2 + b^2} \cos(\theta - \phi)
\end{equation*}

There are three ways to write a sinusoid function:
\begin{itemize}
\item \textbf{\color{blue}amplitude-phase form } : $\displaystyle A \cos \left(\theta -\phi \right)$
\item \textbf{\color{blue}complex form } : $\mathrm{Re\, }\left(c e^{i\omega t} \right)$, where $c$ is a complex number
\item \textbf{\color{blue}linear combination } : $a \cos \omega t + b \sin \omega t$, where $a$ and $b$ are real numbers
\end{itemize}

Different forms are useful in different contexts,
so we'll need to know how to convert between them. The following proposition explains how.
\begin{proposition}
  If
  \begin{equation*}
    \qquad \overline{c}=\, A e^{i\phi }\, =\, a+ bi,\,
    \qquad \qquad c\in \mathbb {C}, \quad A,a,b\in \mathbb {R},
  \end{equation*}
  then
  \begin{equation*}
    \mathrm{Re\, }\left(c e^{i\omega t}\right)\, =\,  A\cos (\omega t -\phi )\, =\,
    a\cos \omega t+ b\sin \omega t,\qquad \omega >0\in \mathbb {R}.
  \end{equation*}
\end{proposition}

\textbf{\color{orange} Use these key equations to convert between the three forms!} \\

\Warning Don't forget that it is $\overline{c}$ and not $c$ itself that appears
in the key equations.
An equivalent form of the key equations (obtained by taking complex conjugates) is
\begin{equation*}
  c = Ae^{-i \phi} = a - bi. 
\end{equation*}

If you ever forget the key equations above, you can do the conversion manually by going through the steps in the proofs below.

\Proofs \\
\begin{enumerate}
\item
  \begin{align*}
    \displaystyle  \mathrm{Re\, }\left(c e^{i\omega t} \right)
    & \displaystyle = \mathrm{Re\, }\left(A e^{-i \phi } \,  e^{i\omega t} \right) \\
    & \displaystyle = \mathrm{Re\, }\left( A e^{i(\omega t - \phi )} \right) \\
    & \displaystyle = A \cos (\omega t - \phi ).
  \end{align*}
\item
  \begin{align*}
    \displaystyle  \mathrm{Re\, }\left(c e^{i\omega t} \right)
    & \displaystyle = \mathrm{Re\, }\left( (a-bi) (\cos \omega t + i \sin \omega t) \right) \\
    & \displaystyle = \mathrm{Re\, }\left(a \cos \omega t + b \sin \omega t + i(\cdots ) \right) \\
    & \displaystyle = a \cos \omega t + b \sin \omega t.
  \end{align*}
\item Using $\cos (x-y)=\cos x \cos y + \sin x \sin y,\,$, we have
  \begin{align*}
    \displaystyle  A \cos (\omega t - \phi )
    & \displaystyle = A \cos \omega t \cos \phi + A \sin \omega t \sin \phi \\
    & \displaystyle = a \cos \omega t + b \sin \omega t.
  \end{align*}
  The last step follows from $ a = A \cos \phi$ and $b = A \sin \phi$, since
  $A, \phi$ are polar coordinates of $(a,b)$.
\item
  \begin{align*}
    &\displaystyle a \cos \omega t + b \sin \omega t \\
    &\displaystyle = (a,b) \cdot (\cos \omega t,\sin \omega t) \\
    &\displaystyle = |(a,b)| \;  |(\cos \omega t,\sin \omega t)| \;
     \cos \left(\text {angle between the vectors } (a,b) \,
     \text {and}\,  (\cos \omega t,\sin \omega t)\right) \\
    &\displaystyle \qquad \qquad (\text {by the geometric interpretation of the dot product)} \\
    &\displaystyle = A \cos (\omega t - \phi ).
  \end{align*}
\end{enumerate}
\clearpage

\subsection{Graphing sinusoidal functions}
In polar form, it is easy to see that the graph of the sinusoidal function
$\, f(t)=A \cos (\omega t-\phi )\,$
is a rescaled and shifted version of the cosine graph, as shown below:

\begin{figure}[ht!]
  \centering
  \includegraphics[width=0.5\textwidth]{image-graph_sinusoidal_function}
  \caption{Graph of $f(t) = A \cos \left( \theta t - \phi \right)$}
\end{figure}

This graph of $f(t)$ can be described geometrically in terms of
\begin{itemize}
\item \textbf{\color{blue} A}, its \textbf{\color{blue} amplitude}, how high the graph raises above
  the $t$-axix at its maximum;
\item \textbf{\color{blue} P} (in \textbf{\color{orange} seconds} or \textbf{\color{orange}seconds per cycle)},
  its \textbf{\color{blue} period}, the time for one complete oscillation, or equivalently the width between successive maxima;
\item ${\color{blue} t_0}$ (in \textbf{\color{orange} seconds), its \textbf{\color{blue}} time lag}(relative to the cosine curve),
  a $t$-value at which a maximum is attained. The time lag is well defined up to the addition to integer multiples of a period.
\end{itemize}

\begin{question}
  How are the parameters describling the graph, $P$ and $t_0$, related to the parameters,
  $\omega$ and $\phi$, in function $f(t)$?
\end{question}
\Answer \\
\begin{itemize}
\item $\displaystyle P = \frac{2 \pi }{\omega },\,$ since adding $\, \displaystyle \frac{2\pi }{\omega }\,$
  to $t$ increases the angle $\omega t - \phi$ by $2 \pi$.
\item $\displaystyle t_0= \frac{\phi }{\omega },\,$, since $t_0$ is the $t$-value for which the angle
  $\, \omega t - \phi =0,\,$. 
\end{itemize}

There is also \textbf{\color{blue} frequency} $\displaystyle \nu \colon =\frac{1}{P}\,$
(in \textbf{\color{orange}Hertz}=\textbf{\color{orange}cycles per second}), the number of
complete oscillations per second. To convert from frequency $\nu$ to angular frequency
$\omega$, multiply by $\frac{2\pi \textrm{ radians}}{1 \textrm{ cycle}}$; thus
$\omega = 2\pi \nu = 2\pi /P$, which is consistent with formula $P = 2\pi /\omega$
above.\\

\paragraph{Steps to graph} $\, f(t)=A \cos (\omega t - \phi )$\\
Start with the curve $y(\theta )=\cos \theta .\, \,$ Then ``work from the outside in''
\begin{enumerate}
\item Amplify (stretch vertically) by a factor of ${\color{blue}A}$,
  the {\color{blue}amplitude}. You now have the graph of $y(\theta )=A\cos (\theta )$
\item hift the graph by $\, {\color{blue}{\phi }}$(in {\color{orange}radians})) to the
  right. You now have the graph of $\, y(\theta )=A\cos (\theta -\phi ).\, \,$\\
  
  The angle $\phi$ is called the {\color{blue}phase lag} or {\color{blue}phase shift} (relative th the cosine curve).
  The first maximum of graph of $\, A\cos (\theta -\phi )\,$ is at $\, \theta =\phi ,\,$ shifted from $\theta =0\,$
  in graph of $\, y(\theta )=A\cos (\theta -\phi ).\, \,$
\item Compress the result horizontally by \textbf{dividing} by the scale factor ${\color{blue}{\omega }} \,$
  (in {\color{orange}radains/second}), the {\color{blue} angular frequency}.
  The result is the graph of  $\, f(t)=A\cos (\omega t-\phi ).\, \,$\\
  
  With the horizontal rescaling, the horizontal axis change from the $\theta$-axis(in radains) to
  $t$-axis(in seconds). The first maximum of the rescaled graph is at $t_0=\phi /\omega$ instead of
  at $\theta =\phi ,\,$ and complete oscillation takes $2\pi /\omega \,$ instead of $\, 2\pi$.  
\end{enumerate}

\textbf{Remark:} The phase lag $\phi$ tells us by what \textbf{frequency of cycle} the graph
of $f(t)$ is shifted from a cosine graph of the same frequency.
For example, the graph of $\, f(t)=A\cos (\omega t-\pi /2)\,$ is shifted by $1/4$ of a cycle
to the right from the graph of $A\cos (\omega t).\, \,$
On the other hand, the time lag $t_0$ tell us by \textbf{how much time} the graph of 
$\, f(t)=A\cos (\omega t-\pi /2)\, =\, A\cos \left(\omega (t-t_0)\right)\,$ is shifted to
the right from the graph of $A\cos (\omega t).\, \,$ Also note that
$\displaystyle \, \frac{\phi }{2\pi }= \frac{t_0}{P}.$

\begin{exercise}
  Units of phi
\end{exercise}
What are the units of $\phi$?
\begin{mybox}{gray}{Answer}
  The unit of $\phi$ is radians, but this is the same as being unitless.
  The length $l$ (meter) of an arc spanning a center angle $\theta$
  radians in a circle of radius $r$ (meters) is given by $\, l=r\theta .$
\end{mybox}
\clearpage


\subsubsection{Worked example}

\begin{enumerate}
\item Write real part of
  \begin{equation*}
    \displaystyle \frac{e^{i \omega t}}{2 + 3i}
  \end{equation*}
  in polar and rectangular from.\\

  Change denominator into polar form, the modular is $\displaystyle \sqrt{2^2 + 3^2}$
  and the angle is $\displaystyle \tan \phi = \frac{3}{2}$, then
  \begin{align*}
    \displaystyle \frac{e^{i \omega t}}{2 + 3i}
    &\displaystyle = \displaystyle \frac{e^{i \omega t}}{\sqrt{13} e^{i \phi}} \\
    &\displaystyle = \displaystyle \frac{1}{\sqrt{13}} e^{i(\omega t - \phi)}
  \end{align*}
  The polar complex number is $\displaystyle \frac{1}{\sqrt{13}} e^{i(\omega t - \phi)}$.\\
  To find the real part, use Euler's formula
  \begin{equation*}
    e^{i \theta} = \cos \theta + i \sin \theta
  \end{equation*}
  The real part of polar form is 
  \begin{equation*}
    \displaystyle \mathrm{Re\, } \left( \frac{1}{\sqrt{13}} \cos (\omega t - \phi)  +
      i \frac{1}{\sqrt{13}} \sin (\omega t - \phi) \right)
    \displaystyle = \frac{1}{\sqrt{13}} \cos (\omega t - \phi). 
  \end{equation*}

  The Rectangular form is 
  \begin{align*}
    \displaystyle \mathrm{Re\, } \left( \frac{\cos \omega t + i \sin \omega t}{2 + 3i} \right)
    \displaystyle &=  \displaystyle \mathrm{Re\, } \left(
                    \frac{(\cos \omega t + i \sin \omega t) \cdot (2 - 3i)} 
                    {(2 + 3i) \cdot (2 - 3i)} \right)\\
    \displaystyle &= \displaystyle \mathrm{Re\, } \left( \frac{1}{13}
                    \left(2 \cos \omega t + i 2 \sin \omega t -i 3 \cos \omega t
                    + 3 \sin \omega t \right) \right) \\
    \displaystyle &= \displaystyle \mathrm{Re\, } \left( \frac{1}{13}
                    \left( \left( 2 \cos \omega t + 3 \sin \omega t \right)
                    + i \left(2 \sin \omega t - 3 \cos \omega t \right) \right) \right) \\
    \displaystyle &= \frac{1}{13} \left( 2 \cos \omega t + 3 \sin \omega t \right)
  \end{align*}

\item What is the circular frequency, amplitude and phase lag?\\

  The circular frequency is just $\omega$,
  from polar form the amplitude is $\displaystyle \frac{1}{\sqrt{13}}$
  , and the phase lag is $\phi = \arctan \frac{3}{2}$. 
\end{enumerate}
\clearpage

\subsection{Application to spring-mass system}
\begin{exercise}
  Solution in amplitude-phase form
\end{exercise}

\begin{figure}[ht!]
  \centering
  \includegraphics[width=0.5\textwidth]{image-damped_frequency}
  \caption{Damped Frequency}
\end{figure}

A mass of $1 kg$ is attached to a spring with spring constant of (approximately)
$\pi^2 N/m$. It is pulled to 1m to the left of the equilibrium and then pushed and released
at a velocity of $\pi m/s$ to the right.\\

(Note that the initial velocity has been corrected from a previous version of this problem.)\\

What is the position $x(t)$ of the mass at time $t$ seconds after being released?
Answer in amplitude-phase form $x(t) = A \cos⁡ (\omega t - \phi)$ by entering the values of
$A$, $\omega$, and $\phi$.\\

(Use the convention that $ A > 0$ and $\omega > 0$.)

\begin{mybox}{gray}{Answer}
  From Newton's second law and Hooke's law
  \begin{equation*}
    F = ma \qquad F = -kx
  \end{equation*}
  Let $x(t)$ position function of time, then the differential equation is 
  \begin{align*}
    &m \ddot{x} = -kx \\
    &m \ddot{x} + kx = 0
  \end{align*}
  with initial condition
  \begin{equation*}
    x(0) = 1 \,m \quad \text{ and, } \dot{x}(0) = - \pi\, m/s .
  \end{equation*}
  Put numbers in equation $m \ddot{x} + kx = 0$
  \begin{equation*}
    1 \ddot{x} + \pi^2 x = 0 
  \end{equation*}
  The characteristic polynomial is
  \begin{equation*}
    r^2 + \pi^2 = 0 
  \end{equation*}
  The root of characteristic polynomial is
  \begin{equation*}
    \frac{\pm \sqrt{-4 \pi^2}}{2} = \pm i \pi
  \end{equation*}
\end{mybox}

\begin{mybox}{gray}{Answer continuous}
  The root of characteristic polynomial is two distinct complex number,
  the general solution is
  \begin{equation*}
    x(t) = c_1 \cos (\pi t) + c_2 \sin (\pi t) 
  \end{equation*}
  Apply initial condition $x(0) = -1$
  \begin{eqnarray*}
    x(0) &=& c_1 \cos (0) + c_2 \sin (0) \\
    c_1  &=& -1
  \end{eqnarray*}
  Apply another initial condition $\dot{x}(0) = \pi $
  \begin{eqnarray*}
    \dot{x}(0) &=& \pi \cdot \sin(\pi \cdot 0) + c_2 \cdot \pi \cos(\pi \cdot 0) \\
    \dot{x}(0) &=&  c_2 \cdot \pi  \\
    \pi &=& c_2 \cdot \pi \\
    c_2  &=& 1
  \end{eqnarray*}

  The solution is
  \begin{equation*}
    x(t) = -\cos \pi t + \sin \pi t
  \end{equation*}
  Change it as amplitude-phase form
  \begin{equation*}
    \displaystyle \sqrt{2} \cos(\pi t - \phi) 
    \quad \text{, where } \phi = \arctan \frac{1}{-1}
  \end{equation*}
  The amplitude is $\sqrt{2}$, $\omega = \pi$, and $\phi = \frac{3 \pi}{4}$.
  Note that $\, \displaystyle \left(\frac{3\pi }{4}+2\pi n\right)\,$
  for any integer $n$ would work as the value of $\phi$ as well.
\end{mybox}

\begin{exercise}
 Units 
\end{exercise}
An amplitude-phase form, the solution to the problem above is
$x(t) = A \cos ⁡(\omega t - \phi)$, What are the units of
$A$, $\omega$ and $\phi$?
\begin{mybox}{gray}{Answer}
  Since the position $x$ of the mass is measured in meters,
  and the output of cosine function is unitless, $A$ must has the same units as $x$, and so is in meters.\\
  Since $(\omega t - \phi)$ is the argument of the cosine function,
  each term must be in radians. This means $\phi$ is in radians, and $\omega$ in radians per second.  
\end{mybox}
\clearpage

\begin{exercise}
  Period, frequency, and time lag
\end{exercise}
What are the units of the period $P$, frequency $\nu$ and time lag $t_0$?
What are the values of the period $P$, the frequency $\nu$, and the time lag $t_0$,
for the solution $x(t)$ to the spring-mass system above?

\begin{mybox}{gray}{Answer}
  \begin{itemize}
  \item The period $P$ is the time for one complete oscillation and so
    has units of $seconds$. Its value is given by
    \begin{equation*}
      \displaystyle P=\frac{2\pi }{\omega }
      \left(\, \frac{\text {rad}}{\text {rad}/\text {sec}}\right)=2(\text {sec})
    \end{equation*}
  \item The frequency $\nu$ is the number of cycles per second, and given by
    $\displaystyle \, \nu =\frac{1}{P} =\frac{1}{2}
    \left(\frac{\text {cycles}}{\text {seconds}}\right).$
  \item The time lag $t_0$ is the $t$-value that the graph of $\, \cos (\omega t)\,$
    is shifted to the right to get to the graph of
    $\, \cos \left(\omega (t-t_0)\right) \, =\cos (\omega t-\phi )$.
    Therefore, it is in $seconds$, and its value is
    $\, \displaystyle t_0\, =\, \frac{\phi }{\omega }\, =\, \frac{3\pi /4}{\pi }\, =\, \frac{3}{4}.$    
  \end{itemize}
\end{mybox}

\begin{exercise}
  Sketch the solution
\end{exercise}
Sketch the graph of $x(t)$, the position function of the mass in the spring-mass system above
(which you have found in the first problem above).
\begin{figure}[ht!]
  \centering
  \includegraphics[width=0.5\textwidth]{image-exercise_2_7}
  \caption{Sketch Soluation}
\end{figure}
\clearpage

\subsection{Amplitude phase forms of solutions to spring-mass-dashpot system}
\begin{figure}[ht!]
  \centering
  \includegraphics[width=0.5\textwidth]{image-spring_mass_dashpot}
  \caption{Spring mass dashpot system}
\end{figure}
The solution a spring-mass-dashpot system with $\, m=1,\, b=4,\, k=5\,$
and initial conditions $\, y(0)\, =\, 1,\, \, y'(0)\, =\, 0$ is
\begin{equation*}
  \displaystyle  e^{-2t}\left(\cos t+2 \sin t\right).
\end{equation*}

\subsubsection{Sketching the solution to an underdamped spring-mass}
\begin{equation*}
  \displaystyle  e^{-2t}\left(\cos t+2 \sin t\right) \displaystyle =
  \displaystyle \sqrt{5} e^{-2t} \cos \left(t - \phi \right) \quad
  \text{, where } \phi = \arctan 2
\end{equation*}
The graph is
\begin{figure}[ht!]
  \centering
  \includegraphics[width=0.5\textwidth]{image-underdamped}
  \caption{Underdamped Frequency}
\end{figure}

The main difference between solutions to an undamped system and a damped system
(where $b$ is small enough so there are two complex roots) is
that real solutions to the damped system have an additional overall exponential factor.\\

The general real solution to $m\ddot{y}+b\dot{y}+ky=\, 0\,$(where $\, b^2<4km)\,$) is:
\begin{align*}
  \displaystyle  \displaystyle y(t)
  & \displaystyle = \displaystyle  e^{(-b/2m)t}
    \left( c_1\cos \left(\omega _ d t \right)+c_2 \sin \left(\omega _ d t\right)\right) \\
  & \displaystyle = \displaystyle  e^{(-b/2m)t} \left( A \cos \left(\omega _ d t -\phi \right) \right) \\
  & \displaystyle = \displaystyle  e^{(-b/2m)t} \left( A \cos \left(\omega _ d t -\phi \right) \right) \\
  & \displaystyle = \displaystyle  \left(A e^{(-b/2m)t}\right) \cos \left(\omega _ d t -\phi \right).    
\end{align*}

Since cosine oscillates between $1$ and $-1$, the function $y(t)$ oscillates between $A e^{(-b/2m)t}$
and $-A e^{(-b/2m)t}$.
We call the graph of $\pm A e^{(-b/2m)t}$ the {\color{blue}envelope} of the oscillation.

\begin{exercise}
  Crossing equilibrium
\end{exercise}
Let
\begin{equation*}
  \displaystyle  \left(A e^{-pt}\right) \cos \left(\omega _ d t -\phi \right)
\end{equation*}
where $\,  A,\, p, \omega _ d ,\, \phi \,$ are non-zero real numbers.\\
How many times does $x(t)$ cross the $t$-axis as $t$ increases?\\
Are the time intervals between the zeros (the $t$-values when $x(t)$ crosses the $t$-axis)
always the same?
\begin{mybox}{gray}{Answer}
  Since $\, A e^{(-b/2m)t} \neq 0 \,$ for all $t\, x(t)=0\,$ whenever $\, \cos (\omega _ d t - \phi )=0.\, \,$
  Therefore, the zeros of $x(t)$ are the same as the zeros for the pure sinusoid
  $\, \cos (\omega _ d t - \phi ),\,$ which are at the $t$-values satisfying.
  \begin{eqnarray*}
    \displaystyle  \displaystyle \omega _ d t -\phi
    &\displaystyle =& \displaystyle  \pi /2 + n\pi \qquad n \text { any integer} \\
    \displaystyle \implies t &\displaystyle =& \displaystyle  \frac{\pi /2 + n\pi + \phi }{\omega _ d}.
  \end{eqnarray*}
  Hence, $x(t)$ has infinitely many equally spaced zeros.
  The exponential factor changes the amplitude of the oscillation but not the frequency.
\end{mybox}
\clearpage

\subsection{Damped sinusoid revisited}
The system response for an unforced spring-mass-dashpot system with small damping term
$\, (b<4km)\,$ is called a {\color{blue}damped sinusoid} and is of the form
\begin{equation*}
  \displaystyle  \displaystyle x(t) \displaystyle =
  \displaystyle  \left(A e^{(-b/2m)t}\right) \cos \left(\omega t -\phi \right).
\end{equation*}

The exponential factor does not affect the frequency of the oscillations:
$x(t)$ crosses the equilibrium whenever the pure sinusoid $\, \cos \left(\omega t -\phi \right)\,$
does.\\

Therefore, the period $P$ and the time lag $t_0$ still make sense and have same formulas as
for the pure sinusoid $\, \cos \left(\omega t -\phi \right)\, :$
\begin{eqnarray*}
  t_0 \qquad &=& \qquad \phi / \omega\\
  P \qquad &=& \qquad 2 \pi / \omega
\end{eqnarray*}
Because the graph of $x(t)$ is not truly periodic-its amplitude id decreasing, we call $P$
the {\color{blue}pseudo-period}.\\

For damped sinusoids, the time lag $t_0$ is easier to see from the shifts of zeros.
The value of $t_0$ is the time difference between a zero of $x(t)$ and a corresponding
zero of $\, \cos (\omega t)\,$.\\

(You can check using calculus that the local extrema of $x(t)$ are no longer
at the same $t$-values as those of the pure sinusoid $\, \cos \left(\omega t -\phi \right).\, \,$
However, they are still equally spaced.
Consecutive local maxima of a damped sinusoid are still a full period $P$ apart.)\\
\clearpage

\subsection{Beats}

\textbf{\color{blue} Bests} occur when two very nearby pitches are sounded simultaneously.\\

\begin{problem}
  Consider two sinusoid sound waves of angular frequencies $\omega +\epsilon$ and
  $\omega -\epsilon$, say $\cos ((\omega +\epsilon )t)$ and $\cos ((\omega -\epsilon )t)$,
  where $\epsilon$ is much smaller than $\omega$.
  What happens when they are superimposed? 
\end{problem}

\Solution The sum is 
\begin{align*}
  \displaystyle  \cos ((\omega +\epsilon )t) + \cos ((\omega -\epsilon )t)
  \displaystyle &= \mathrm{Re\, }(e^{i(\omega +\epsilon ) t}) + \mathrm{Re\, }(e^{i(\omega -\epsilon ) t}) \\
  \displaystyle &= \mathrm{Re\, }(e^{i\omega t} (e^{i \epsilon t} + e^{-i\epsilon t})) \,
  \text{ Inverse Euler's Formula\footnotemark}\\
  \displaystyle &= \mathrm{Re\, }(e^{i\omega t} (2 \cos \epsilon t))  \\
  \displaystyle &= (2 \cos \epsilon t) \mathrm{Re\, }(e^{i\omega t}) \\
  \displaystyle &= 2 (\cos \epsilon t) (\cos \omega t).
\end{align*}
\footnotetext{Inverse Euler's Formula
  \begin{align*}
    e^{i\theta} + e^{-i\theta} &= \cos \theta + i \sin \theta + \cos \theta -i \sin \theta\\
                               & = 2 \cos \theta
  \end{align*}
  therefore, $ 2 \cos \theta = \frac{e^{i\theta} + e^{-i\theta}}{2} $
}

The function $\cos \omega t$ oscillates rapidly between $\pm 1$.
Multiplying it by the slowly varying function $2 \cos \epsilon t$ produces a rapid oscillation between
$\pm 2 \cos \epsilon t$, so one hears a sound wave of angular frequency $\omega$
whose amplitude is the slowly varying function $|2 \cos \epsilon t|$.



\begin{figure}[ht!]
  \centering
  \includegraphics[width=0.5\textwidth]{image-beats}
  \caption{A plot of $2\left(\cos \epsilon t\right)\left(\cos \omega t\right)$
  for $\epsilon =1/8$ and $\omega =4$.}
\end{figure}

\textbf{Practical application}\\

You hear beats when tuning the strings of an instrument, or in tuning one instrument to another.
The waa waa waa sound you hear is exactly theses beats.
The higher the frequency of the beats, the more out of tune.
As the instruments or strings become closer and closer in tune,
the frequency of the beats diminish until you cannot hear them at all. \\
Observe that changing the phase of one signal with respect to the other doesn't change
the frequency of the beats!
This is important. If you could hear phase shifts,
it would be very difficult to tune your instrument by ear.\\

\begin{exercise}
  Envelope of the beats
\end{exercise}
In general, the sum of the two sinusoids with different frequencies and
\textbf{different amplitudes} can be written as the imaginary part of a complex valued function as follows:
\begin{align*}
  \displaystyle  \displaystyle f(t)+g(t)
  \displaystyle &= \displaystyle \sin (t)+A \sin (\omega t) \\
  \displaystyle &= \displaystyle  \mathrm{Im\, }( R(t) e^{i \theta (t)}) \qquad R(t) \, \text {real} \\
  \displaystyle &= \displaystyle  R(t) \sin (\theta (t))
\end{align*}

Notice that $|f(t)+g(t)|\leq R(t).\, \,$. This function $R(t)$ is called the
{\color{blue}envelope of the beats}.\\

Find $R(t)$.\\
(Your answer should be in terms of $\, A,\, \omega ,\, t.$)

\begin{mybox}{gray}{Answer}
  The envelope $R(t)$ is the modulus of the complex function whose imaginary part is
  $f(t)+g(t)$. Since
  \begin{align*}
    \displaystyle  \displaystyle f(t)+g(t)
    \displaystyle &= \displaystyle \sin (t)+A \sin (\omega t) \\
    \displaystyle &= \displaystyle  \mathrm{Im\, }\left(e^{it} +A e^{i\omega t}\right) \\
    \displaystyle &= \displaystyle \mathrm{Im\, }
                    \left(\cos (t)+A\cos (\omega t) +i \left(\sin (t)+A\sin (\omega t)\right)\right),
  \end{align*}
  we have
  \begin{align*}
    \displaystyle  \displaystyle R(t)
    \displaystyle &= \displaystyle
                    \left|\cos (t)+A\cos (\omega t) +i \left(\sin (t)+A\sin (\omega t)\right)\right| \\
    \displaystyle &= \displaystyle \sqrt
                    {\left(\cos (t)+A\cos (\omega t)\right)^2 +
                    \left(\sin (t)+A\sin (\omega t)\right)^2 } \\
    \displaystyle &= \displaystyle
                    \sqrt { 1 + A^2+2 A\left(\cos (t)\cos (\omega t)
                    +\sin (t)\sin (\omega t)\right) } \\
    \displaystyle &= \displaystyle \sqrt { 1 + A^2+2 A\left(\cos ((1-\omega ) t)\right)}.
  \end{align*}
  \textbf{Note:} When $A=1$, the amplitudes of the two sinusoids in $f(t)$ are the same,
  and $R(t)$ reduces to what we computed in the example above:
  \begin{align*}
    \displaystyle  \displaystyle R(t)
    \displaystyle &= \displaystyle  \sqrt {2(1+\cos (1-\omega t))} \\
    \displaystyle &= \displaystyle  \sqrt {4\cos ^2\left(\frac{1-\omega }{2}t\right)}\qquad
                    \left(\cos ^2\theta \, =\, \frac{\cos (2\theta )+1}{2}\right) \\
    \displaystyle &= \displaystyle  2 \left|\cos \left(\frac{1-\omega }{2}t\right)\right|.
  \end{align*}
\end{mybox}
\clearpage

\subsection{Recitation}
\subsubsection{The different forms of a sinusoid function}
\begin{problem}
  Linear combination to amplitude-phase form practice 
\end{problem}
Write $\, \cos (2t)+\sin (2t)\,$ in the form $A\cos (\omega t-\phi ),\,$ ?
\begin{mybox}{gray}{Answer}
  The amplitude is
  \begin{equation*}
    \sqrt{1^2 + 1^2} = \sqrt{2}. 
  \end{equation*}
  The $\omega$ is $2$ and $\phi$ is $\arctan \frac{1}{1}\, = \, \frac{\pi}{4}\, .$
\end{mybox}

\begin{problem}
  Linear combination to amplitude-phase form practice 
\end{problem}
Write $\, \cos (\pi t)-\sqrt {3}\sin (\pi t)\,$ in the form $A\cos (\omega t-\phi ),\,$ ?
\begin{mybox}{gray}{Answer}
  The amplitude is
  \begin{equation*}
    \sqrt{1^2 + {\sqrt{3}}^2} = 2
  \end{equation*}
  The $\omega$ is $\pi$ and $\phi$ is $\arctan \frac{-\sqrt{3}}{1}\, = \, - \frac{\pi}{3}\, .$
\end{mybox}


\begin{problem}
  Amplitude-phase form to linear combination practice
\end{problem}
Write $5\cos \left(3t+\frac{3\pi }{4}\right)$ in the form $a\cos (\omega t)+b\sin (\omega t)\,$ ?
\begin{mybox}{gray}{Answer}
  From $\phi = \arctan{ \frac{b}{a}} = \frac{-3 \pi}{4}$,
  \begin{align*}
    \tan (\arctan{ \frac{b}{a}}) &= \tan \frac{-3 \pi}{4} \\
    \frac{b}{a} &= 1 \\
    b &= a
  \end{align*}
  Substitutes $a = b$ into $\sqrt{a^2 + b^2} = 25$   
  \begin{align*}
    \sqrt{a^2 + a^2} &= 5\\
    2a^2 &= 25 \\
    a^2 &= \frac{25}{\sqrt{2}}\\
    a &= \pm \frac{5}{\sqrt{2}}. 
  \end{align*}
\end{mybox}

\begin{mybox}{gray}{Answer Continuous}
    The $\phi$ is the angel is in third quadrant, the sign of $a \, b\,$ should be negative.
  So, the linear combination form is
  \begin{equation*}
    -\frac{5}{\sqrt{2}} \cos 3t  -\frac{5}{\sqrt{2}} \sin 3t 
  \end{equation*}
\end{mybox}
\clearpage

\subsubsection{Graphing a damped sinusoid}

\begin{problem}
  Solution in amplitude-phase form
\end{problem}

\begin{figure}[ht!]
  \centering
  \includegraphics[width=0.5\textwidth]{image-spring_mass_dashpot}
  \caption{Spring Mass Dashpot}
\end{figure}

The spring-mass-dashpot system is set up with mass $m \, = \, 1(Kg)$,
spring constant $\, \displaystyle k= \frac{17}{16} \, (\text {N/m}),\,$
and damping constant $\, \displaystyle b=\frac{1}{2}\, (\text {N s /m})$. \\

The mass is pulled to $1\, m$ to the right of the equilibrium and then pushed
and released at a velocity of $\, 0.75\text {m/s}\,$ the the right. \\

What is the position $x(t)$ of the mass at time $t$ seconds after being release?
Answer in amplitude phase form $x(t) = Ae^{-pt}\cos (\omega t- \phi )$ ?

\begin{mybox}{gray}{Answer}
  The differential equation is
  \begin{equation*}
    \ddot{x} + \frac{1}{2} \dot{x} + \frac{17}{16} x = 0 
  \end{equation*}
  with initial condition
  \begin{equation*}
    x(0) = 1 \, m \qquad \text{and \,} \dot{x}(0) = 0.75 \, m/s
  \end{equation*}
  The characteristic polynomial have two distinct complex roots, when the
  discriminant $\displaystyle \sqrt{\frac{1}{4} - 4\cdot \frac{17}{16} } \, < 0 $. 
  So, the general real solution:
  \begin{equation*}
    \displaystyle e^{-\frac{1}{4} t} \left( c_1 \cos t + c_2 \sin t\right)
  \end{equation*}
  Apply initial condition $\, x(0) = 1 ,\,$ the $\, c_1 = 1$.
  \begin{align*}
    \displaystyle \dot{x}
    \displaystyle &= \displaystyle \frac{d}{dt}
                    \left( e^{-\frac{1}{4} t} \left( \cos t + c_2 \sin t\right) \right) \\
    \displaystyle &= -\frac{1}{4} e^{-\frac{1}{4} t} \left( \cos t + c_2 \sin t\right) 
                    + e^{-\frac{1}{4} t} \left( -\sin t + c_2 \cos t \right) \\
    \displaystyle \dot{x}(0)
    \displaystyle &= -\frac{1}{4} + c_2 = 0.75 \\
    \displaystyle c_2 &= 1.
  \end{align*}
  The solution of differential equation is
  \begin{equation*}
    \displaystyle e^{-\frac{1}{4} t} \left( \cos t + \sin t\right). 
  \end{equation*}
\end{mybox}
\clearpage

\begin{mybox}{gray}{Answer Continuous}
  Change linear combination
  \begin{equation*}
    \displaystyle e^{-\frac{1}{4} t} \left( \cos t + \sin t\right) 
  \end{equation*}
  to the amplitude phase form
  \begin{equation*}
    \displaystyle \sqrt{2} e^{-\frac{1}{4} t} \left( \cos (t - \phi) \right) \,
    \text{where, } \, \phi = \arctan \frac{1}{1}
  \end{equation*}
  
\end{mybox}

\begin{problem}
  Half-time
\end{problem}
Let us consider the envelope $\, \pm \epsilon (t)\,$ with
$\, \epsilon (t)=Ae^{-pt}\,$ of the solution $x(t)$ that you found above.
The decaying exponential envelope is a main feature that distinguishes
a damped sinusoid from a pure sinusoid.\\

How long does it take for the envelope to decay by half? In other words, find
$\, t_{1/2}\,$ (in seconds) such that
$\, \displaystyle \epsilon \left(t_{1/2}\right)\, =\, \frac{1}{2} \epsilon (0).$
This is called the {\color{blue}half-time} of the decaying exponential envelope.

\begin{mybox}{gray}{Answer}
  Let start time in 0 second as $t_0$ and half-time is $t_h$,
  \begin{align*}
    \displaystyle e^{-\frac{1}{4} t_h}
    \displaystyle &= \displaystyle \frac{1}{2} e^{-\frac{1}{4} t_0} \\
    \displaystyle &= \displaystyle \frac{1}{2}  \, \text{, where } t_0 = 0\\
    \displaystyle \ln {e^{-\frac{1}{4} t_h}}
    \displaystyle &= \displaystyle \ln{\frac{1}{2}} \\
    \displaystyle -\frac{1}{4} t_h \displaystyle &= \displaystyle \ln{\frac{1}{2}} \\
    \displaystyle t_h \displaystyle &= \displaystyle -4 \ln{\frac{1}{2}} 
  \end{align*}
\end{mybox}

\begin{problem}
  Pseudo-period, pseudo-frequency and time lag
\end{problem}
What are the values of the (pseudo)-period $P$,
(in seconds) the (pseudo)-frequency $\nu \,$
(in cycles per second) and the time lag $\, t_0 \,$ (in seconds) for
$\,x(t) ,\,$ of the solution that you found above?

\begin{mybox}{gray}{Answer}
  The period is $\, P = \frac{2 \pi}{\omega} = 2 \pi \,$.
  The frequency $\, \nu \,$ is $\, \frac{1}{P} = \frac{1}{2 \pi} \,$.
  The time lag $t_0$ is $\, \frac{\phi}{\omega} = \frac{\pi}{4} \,$. 
\end{mybox}
\clearpage

\begin{problem}
  Zeros of the graph
\end{problem}
When are the first $4$ times that the mass crosses the equilibrium position?
That is, Find the 4 smallest $t$-values for $\, t>0\,$ such that $\, x(t)=0$?
Let $t_1,\, t_2,\, t_3,\, t_4\,$ be the 4 $t$-values in increasing order.
\begin{mybox}{gray}{Answer}
  In
  \begin{equation*}
    \sqrt{2} e^{-\frac{1}{4} t} \left( \cos (t - \frac{\pi}{4}) \right) 
  \end{equation*}
  the exponential function $\displaystyle \, e^{-\frac{1}{4} t} \,$
  can not be zero. Find the $t$ values that $\, \cos (t - \frac{\pi}{4}) = 0$.
  \begin{align*}
    t - \frac{\pi}{4} &= \frac{\pi}{2} \\
    t &= \frac{3 \pi}{4} + n \cdot \pi \quad , 
    \text{where } n = 0, 1, 2, \cdots 
  \end{align*}
\end{mybox}

\begin{problem}
  Sketch the graph  
\end{problem}
Sketch the solution $\, x(t)\,$.

\begin{figure}[ht!]
  \centering
  \includegraphics[width=0.5\textwidth]{image-Problem_2_9}    
  \caption{ Graph of $\displaystyle \sqrt{2}
    e^{-\frac{1}{4} t} \left( \cos (t - \frac{\pi}{4}) \right)$ }
\end{figure}
\clearpage

\subsection{Simple harmonic oscillator}

Let us review what we have learned about the spring-mass-dashpot system with no external force.
We will first review the cases where oscillations are present.\\

If there is \textbf{no damping}, the DE that models the position of the mass is
\begin{eqnarray*}
  m \ddot{x} + kx \quad &=& \quad 0 \quad (m,\, k \, > \,0)\\
  \text{standard linear form:} \quad \ddot{x} + \frac{k}{m} x \quad &=& \quad 0 \\
  \text{ or } \ddot{x} + \omega ^2 _n x \quad &=& \quad 0 \quad
                                                  \text{where } \, \omega _n : \, = \, \sqrt{\frac{k}{m}}
\end{eqnarray*}

\textbf{Summary of results:}
\begin{eqnarray*}
  \text{Characteristic polynomial:} & \quad
  & P(r) =  mr^2 + k \, \text{(defined up to a constant multiple)}\\
  \text{Roots:}  & \quad
  & \pm i \omega _ n,\, \text{where } \, {\color{blue}{\omega _ n = \sqrt {\frac{k}{m}} }} \\
  \text{Basis of solution space:} & \quad
  & e^{i \omega _ n t}\, , \, e^{-i\omega _ n t}. \\
  \text{Real-valued basis:} & \quad
  & \cos \omega _ n t, \, \sin \omega _ n t \\
  \text{General real solution:}	& \quad
  & {\color{blue}{ a \cos \omega _ n t + b \sin \omega _ n t}} , \,
    \text{ where, $a, \, b$ are real constants} \\
  \text{amplitude-phase Form:}	& \quad
  & {\color{blue}{A \cos (\omega _ n t - \phi )}}
    \text{ where, $A \, >  \, 0$ and $\phi$ are real constants}     
\end{eqnarray*}

In other words, the real-valued solutions are all the sinusoidal functions of
angular frequency $\omega _n$. \\

This system, or any other system governed by the same DE, is also called a
{\color{blue} simple harmonic oscillator}.
The angular frequency $\omega _n$ is also called the {\color{blue} natural frequency}
( or {\color{blue} resonant frequency}) of oscillator.

\begin{exercise}
  Undamped Spring
\end{exercise}
A spring system that is undamped is called the harmonic oscillator, or an ideal spring.
What is the period of a nonzero solution of $\ddot x + 4x = 0$?
\begin{mybox}{gray}{Answer}
  Characteristic equation:  $r^2 + 4 = 0$. \\
  Roots: $r^2 = -4 \,  \Rightarrow \,  r = \pm 2i$. \\
  Complex exponential solutions: $z_1 = e^{i(2t)} , \, \, z_2 = e^{-i(2t)}$. \\
  Basic real solutions:  $\textrm{Re}(z_1) = \cos (2t)$, $\textrm{Im}(z_1) = \sin (2t)$. \\
  General real solution:
  \begin{equation*}
    x = c_1\cos (2t) + c_2\sin (2t) = A\cos (2t-\phi )
  \end{equation*}
  To find the period, we think of what $t$ has to do to take $2t$ from $0$ to $2 \pi$.
  Thus the period is $\, 2\pi /2=\pi$. 
\end{mybox}
\clearpage

\subsubsection{Review}
\begin{figure}[ht!]
  \centering
  \includegraphics[width=0.5\textwidth]{image-spring_mass_dashpot}
  \caption{Spring Mass Dashpot}
\end{figure}
Constants $k, \, m, \, b$, the DE equation is
\begin{eqnarray*}
  \text{Differential equation: } & \quad & m \ddot{x} + b \dot{x} + kx = 0\\
  \text{Standard Form: } & \quad & \ddot{x} + \frac{b}{m} \dot{x} + \frac{k}{m} x = 0\\
  \text{Change $x$ to $y$: } & \quad & \ddot{y} + 2p \dot{y} + {\omega _n}^2 y = 0
\end{eqnarray*}

\paragraph{Oscillation} which characteristic polynomial's root are two distinct complex number.
If root is real number that is overdamp case which does not get any oscillations.
The characteristic equation is
\begin{equation*}
  r^2 + 2pr + {\omega _n}^2 = 0. 
\end{equation*}
The root is
\begin{equation*}
  r = -p \pm \sqrt{p^2 - {\omega _n}^2}
\end{equation*}
The first case is $p \, = \, 0$,(means the mass $m$ can not be a zero, so there is no
dashpot) the oscillations is undamped.
\begin{equation*}
  \ddot{y} + \underbrace{{\omega _n}^2}_{angular frequency} y = 0 
\end{equation*}
The general solution for this particular case is
\begin{equation*}
  y = c_1 \cos \omega_n t + c_2 \sin \omega_n t = A \left( \cos(\omega_n t + \phi) \right)
\end{equation*}

\begin{exercise}
  Effect of the spring constant
\end{exercise}

\begin{mybox}{gray}{Answer}
  The frequency of the oscillations is directly proportion to the angular frequency
  $\omega _n$. Since $\displaystyle \, \omega _ n=\sqrt {\frac{k}{m}},\,$ if $k$ increases
  $\omega _n$ also increases.
  This agrees with intuition: the stiffer the spring, the faster the oscillations.  
\end{mybox}
\clearpage

\subsection{Damped Harmonic Oscillator}
When damping is present, the DE that models the position of the mass is
\begin{equation*}
  \displaystyle  \displaystyle m \ddot{x} + b \dot{x} + k x \quad \displaystyle = 
  \displaystyle  0 \qquad m,k>0,\,  b\geq 0
\end{equation*}

\textbf{Summary of results:}
\begin{eqnarray*}
  \text{Characteristic polynomial:} & \quad
  & P(r) =  mr^2 + + b r + k \\
  \displaystyle \text{Roots:}  & \quad
  & \displaystyle \displaystyle - \frac{b \pm \sqrt {b^2-4mk}}{2m}\, \text{ (by the quadratic formula)} \\
                                    & \quad
  & =\displaystyle \,
    -\frac{b}{2m} \pm \sqrt {\left(\frac{b}{2m}\right)^2 - \omega _ n^2}                                      
\end{eqnarray*}

There are three cases, depending on the sign of $b^2-4mk.\, \,$
The behavior of the solutions in these 3 cases are qualitatively different in these 3 cases.

\subsubsection{Case 1: underdamped}
\textbf{Case 1:} $b^2 < 4mk\,$ {\color{blue} underdamped} \\

There are two complex roots, and we will give names to the real and imaginary parts. Since the real part is always negative, we call it $-p, \, \,$ with $\displaystyle \, p=\frac{b}{2m}.\, \,$
The imaginary part is either positive or negative of the {\color{blue} underdamped frequency}
$\, \omega _ d,\,$ given by
\begin{eqnarray*}
  \displaystyle  \displaystyle {\color{blue}{\omega _ d}}
  & : \displaystyle = & \displaystyle {\color{blue}{\frac{\sqrt {4mk-b^2}}{2m} }} \\
  & : \displaystyle = & \displaystyle  {\color{blue}{ \sqrt {\omega _ n^2-p^2} }} \
                        ,\qquad \text {where } \, {\color{blue}{\omega _ n=\sqrt {\frac{k}{m}} }}
                        \, \, \text {is the natural frequency}.
\end{eqnarray*}
Note that both $p$ and $\omega _d$ are positive.\\

\textbf{Summary of results:}
\begin{eqnarray*}
  \text{Roots:}  & \quad
  & -p \pm i \omega _ d \\
  \text{Basis of solution space:} & \quad
  & \displaystyle e^{(-p + i \omega _ d) t}\, , \, e^{(-p -i\omega _ d) t}. \\
  \text{Real-valued basis:} & \quad
  & \displaystyle e^{-p t}  i\cos \omega _ d t, \, e^{-p t} \sin \omega _ d t \\
  \text{General real solution:}	& \quad
  & \displaystyle {\color{blue}{e^{-p t} (a \cos (\omega _ d t) + b \sin (\omega _ d t))}}
    , \, \text{ where $a, \, b \, $ are real constants.} \\
  \text{Amplitude-phase Form:}	& \quad
  & \displaystyle {\color{blue}{A e^{-pt}\cos (\omega _ d t - \phi ) }}
    , \, \text{for some $A$ and $\phi$. }
\end{eqnarray*}

This is a sinusoid multiplied by a decaying exponential. Each nonzero solution tends to $0 , \,$
but changes sign infinitely many times along the way.
The system is called {\color{blue} underdamped}, because there was not enough damping
to eliminate the oscillation completely.\\

The damping not only causes the solution to decay exponentially, but also
\textbf{ changes the frequency of the sinusoid}.
The new angular frequency, $\omega _ d$, is called {\color{blue}damped (angular) frequency}( or
sometimes \textit{pseudo (angular) frequency}). \\

The damped frequency $\omega _d$ is less than the natural(undamped) frequency $\omega _n$, as evident
from the formula $\, \omega _ d=\sqrt {\omega _ n^2-p^2}.\, \,$.
When $\, b=0,$ \, $\omega _ d\,$ is the same as $\, \omega _ n$. \\

The damped solutions are not actually periodic:
they don't repeat exactly, because of the decay.
Therefore $\, 2\pi /\omega _ d$  is called the {\color{blue}pseudo-period}.

\subsubsection{Review: the underdamped case}

\begin{equation*}
  \ddot{y} + 2p \dot{y} + {\omega _n}^2 y = 0,
  \quad \text{ where, } \, p = 2 * \frac{m}{k} , \, \omega _n = \sqrt{\frac{k}{m}} 
\end{equation*}
The roots of characteristic polynomial of above DEs is
\begin{equation*}
  r = -p \pm \sqrt{p^2 - {\omega _n}^2}
\end{equation*}
To get oscillation is $ \sqrt{p^2 - {\omega _n}^2} < 0 ,\,$
so, the condition for oscillation is $ p < \omega _n ^2 , \, $
where $\, p \, \text{and} \, \omega _n \, $ are positive.
In other words,  the damping should be less than the frequency,
the angular frequency except $p$ is not the damping.\\

Note that $p = 1/2 \frac{k}{m}$. \\
What is solutions looks like?
\begin{figure}[ht!]
  \centering
  \includegraphics[width=0.5\textwidth]{image-review_underdamped_case}
  \caption{Graph Underdamped Case }
\end{figure}

%%% Local Variables:
%%% mode: latex
%%% TeX-master: "NoteForDifferentialEquation"
%%% End:

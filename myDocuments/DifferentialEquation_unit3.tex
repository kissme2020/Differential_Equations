\section{Sinusoidal Functions}
\subsection{Solutions to the spring-mass-dashpot system}
\begin{exercise}
  Review: spring-mass-dashpot system with distinct complex roots
\end{exercise}
\begin{figure}[ht!]
  \centering
  \includegraphics[width=0.6\textwidth]{image-spring_mass_dashpot}
  \caption{Spring Mass Dashpot System}
\end{figure}

Recall that the position of the mass in a spring-mass-dashpot system without external force can be modelled
by the second order linear homogeneous ODE.
\begin{equation*}
  m {\color{blue}{\ddot{x}}}  + b {\color{blue}{\dot{x}}}  + k {\color{blue}{x}}  =0.
\end{equation*}

Let the damping constant be small enough so that $b^2 < 4mk$ and the characteristic polynomial has
$2$ distinct complex roots.\\
Find the two characteristic roots $r_1$ and $r_2$ of this system in terms of the mass $m$,
the spring constant $k$, and the damping constant $b$.\\
(Enter your answer so that $r_1$ is the root with positive imaginary part, and $r_2$ is the root
with negative imaginary part.)

\begin{mybox}{gray}{Answer}
  The root of characteristic equation is
  \begin{equation*}
    mr^2 + br + k = 0 
  \end{equation*}
  in condition of $b^2 < 4mk$, the two distinct imaginary root of is
  \begin{equation*}
    \displaystyle \frac{-b \pm \sqrt{b^2 - 4mk}}{2m} \displaystyle =
    \displaystyle \frac{-b \pm \sqrt{(-1)(4mk -b^2)}}{2m} \displaystyle =
    \displaystyle \frac{-b \pm i\sqrt{4mk -b^2}}{2m}
  \end{equation*}
\end{mybox}

\begin{exercise}
  Review: basis of solutions
\end{exercise}
Consider the same spring-mass-dashpot system as above.
Let the two distinct characteristic roots be $ -s\pm i\omega _ d$
($s$ and $\omega d$ can be expressed in terms of $m$, $k$, and $b$).
\begin{mybox}{gray}{Answer}
  The exponential solutions is
  \begin{equation*}
    \displaystyle e^{\frac{-b + i\sqrt{4mk -b^2}}{2m}} ; \quad
    \displaystyle e^{\frac{-b - i\sqrt{4mk -b^2}}{2m}}
  \end{equation*}
  as a basis of the collection of all solutions.
  $b^2 - 4mk < 0$,  then the real and imaginary parts of either of the exponential solutions
  \begin{eqnarray*}
    \displaystyle \mathrm{Re\, }\left(e^{\frac{-b+\sqrt {b^2-4mk}}{2m}t}\right)
    & \displaystyle =& \displaystyle e^{-\frac{b}{2m}t}\cos \left(\frac{\sqrt {4mk-b^2}}{2m}t\right)\\
    \displaystyle \mathrm{Re\, }\left(e^{\frac{-b+\sqrt {b^2-4mk}}{2m}t}\right)
    & \displaystyle =& \displaystyle e^{-\frac{b}{2m}t}\sin \left(\frac{\sqrt {4mk-b^2}}{2m}t\right)
  \end{eqnarray*}
  form another basis of the solutions.
\end{mybox}
\clearpage

\subsection{Real solutions to spring-mass-dashpot system}
\begin{figure}[ht!]
  \centering
  \includegraphics[width=0.5\textwidth]{image-spring_mass_dashpot}
  \caption{Spring Mass Dashpot System}
\end{figure}

In the last lecture, we modeled the spring-mass-dashpot system with
no external force by the second order linear homogenous DE
\begin{equation*}
  m {\color{blue}{\ddot{x}}}  + b {\color{blue}{\dot{x}}}  + k{\color{blue}{x}}  =0 \qquad m>0, \, \, b,\, k\, \geq 0.
\end{equation*}
When there are two distinct characteristic complex roots, that is,
when the discriminant $\, b^2-4km<0,\,$
the real basis that you found in the previous problem gives the general real solution:
\begin{equation*}
  \displaystyle  e^{(-b/2m)t} \left( c_1\cos \left(\omega _ d t\right)+c_2 \sin \left(\omega _ d t\right)\right)
  \qquad c_1,\, c_2 \in \mathbb {R},
\end{equation*}
where $\,  \displaystyle \omega _ d\, =\, \sqrt {\frac{k}{m}-\frac{b^2}{4m^2}}\,$
is called the \textbf{{\color{blue} damped frequency}} of the system.
\begin{figure}[ht!]
  \centering
  \includegraphics[width=0.5\textwidth]{image-damped_frequency}
  \caption{Damped Frequency}
\end{figure}
It is useful to contrast the solutions of the general spring-mass-dashpot system with
the solutions to the system with no dashpot (damping), i.e. $\, b=0 (<4mk).\,$.
If the system has no damping, the DE reduces to
\begin{equation*}
  m {\color{blue}{\ddot{x}}}  + k {\color{blue}{x}}  =0\qquad m>0, \, \, k\, \geq 0,
\end{equation*}
and the general real solution reduces to
\begin{equation*}
  \displaystyle  c_1 \cos \left(\omega _ n\right)+c_2\sin \left(\omega _ n\right)
  \qquad c_1,\, c_2 \in \mathbb {R},
\end{equation*}
where $\,  \displaystyle \omega _ n\, =\, \sqrt {\frac{k}{m}}\,$
is called the \textbf{{\color{blue} natural frequency}} of system.
We will continue to call the value of $\omega _ n$ the natural frequency
of system even if damping is present ($b > 0$).
\begin{question}
  What do these general solutions look like? 
\end{question}
Both of these general solutions involve linear combinations of $\cos (\omega t)$ and
$\sin (\omega t)$ for some angular frequency $\omega$ that is the same for both
the sine and cosine functions. What so the graph of such linear combinations looks like?\\
To make graph these solutions easy, we will first rewrite the linear combination
$\, a\cos (\theta )+b\sin (\theta )\,$ in \textit{polar form}.
\clearpage

\subsection{From rectangular to polar form}
\begin{equation*}
  \displaystyle \frac{1}{1 + (\frac{w}{k})^2}
  \left( \cos \omega t + \frac{w}{k} \sin \omega t \right)
\end{equation*}
Convert $\cos \omega t + \frac{w}{k} \sin \omega t$ into other form.
Using\\
\begin{SCfigure}[][h] % position of SCfigure here 
  \begin{minipage}{.5\textwidth}
    \begin{equation*}
        a \cos \theta + b \sin \theta = C \cos \left( \theta - \phi \right)
    \end{equation*}
  \end{minipage}%
  \begin{minipage}{0.5\textwidth}
    \centering
    \includegraphics[width=0.4\textwidth]{image-cosine_triangle}
  \end{minipage}
\end{SCfigure}

Apply it to $\cos \omega t + \frac{w}{k} \sin \omega t$\\
\begin{SCfigure}[][h] 
  \begin{minipage}{.6\textwidth}
    \begin{equation*}
      \displaystyle \frac{1}{1 + (\frac{w}{k})^2} \sqrt{1 + (\frac{w}{k})}
       \cos (\omega t - \phi) \quad \text{ where, } \phi = \arctan \frac{w}{k}. 
    \end{equation*}
  \end{minipage}%
  \begin{minipage}{0.4\textwidth}
    \centering
    \includegraphics[width=0.4\textwidth]{image-cosine_triangle_2}
  \end{minipage}
\end{SCfigure}

There are two ways of expressing any \textbf{\color{blue} sinusoidal function},
in rectangular form and in \textbf{\color{orange}polar form}. They are related as follows:
\begin{equation*}
  \displaystyle  \underbrace{{\color{orange}{A}}
    \cos (\theta -{\color{orange}{\phi }} )}_{{\color{orange}{\text {polar form}}} }
  \qquad {\color{blue}{a}} ,\,
  {\color{blue}{b}} ,\,  {\color{orange}{\phi }}  \in \mathbb {R}, \  \  \
  {\color{orange}{A}} \geq 0 \in \mathbb {R},
\end{equation*}
where $A$ and $\phi$ in terms of $a$ and $b$ are given implicitly by the following diagram:
\begin{figure}[ht!]
  \centering
  \includegraphics[width=0.5\textwidth]{image-rectangular_form}
  \caption{Rectangular Form}
\end{figure}
\clearpage

That is,
\begin{align*}
  &A \quad = \quad \sqrt{a^2 + b^2} \\
  &\phi \quad : \quad
    \text {Angle between the positive horizontal axis and the ray to the point }\, (a, b).
\end{align*} 
Note that $\, (A,\phi )\,$ are polar coordinates of the point with rectangular coordinates
$\, (a,b).\,$ We will use the convention $\, A\geq 0.\, \,$
As usual for polar coordinate, the angle $\, \phi \,$ is well-defined
only up to addition of integer multiples of $2\pi$. \\

It is amazing that the sum of two sinusoids is another sinusoid!
And while graphing the sinusoid in rectangular form is a mystery,
graphing the polar form is easy. We will do this shortly. 
\clearpage

\subsection{Worked Example}
In practice, the argument of the cosine and sine terms is often a function rather than a constant angle.
In this course, we usually have
\begin{equation*}
  \theta = \omega t \qquad \text{for some } \omega > 0
\end{equation*}

\begin{example}
  Convert $\,  -\cos (5t) - \sqrt {3} \sin (5t)$ to polar form.
\end{example}
\Solution \\
Given: $\, a=-1,\, b=-\sqrt {3},\, \theta (t)=\omega t\,$ where $\omega = 5$.\\
Want: $\, A,\, \phi .\,$ These are polar coordinates of the point with rectangular coordinates(a,b). 
\begin{figure}[ht!]
  \centering
  \includegraphics[width=0.5\textwidth]{image-example_polar_rectangular}
  \caption{Diagram of  rectangular form}
\end{figure}

Using the diagram above,
\begin{eqnarray*}
  \displaystyle A \quad & \displaystyle = & \displaystyle \sqrt{(-1)^2 + (\sqrt{3})^2} = 2\\
  \displaystyle \phi \quad & \displaystyle = & \displaystyle - \frac{2 \pi}{3}. 
\end{eqnarray*}

Therefore, the answer is
\begin{equation*}
  \displaystyle A \cos (\omega t - \phi) \displaystyle = \displaystyle 2 \cos (5t + \frac{2 \pi}{3} )
\end{equation*}

Note that $4 \pi / 3$(or $-\frac{2\pi }{3}+2\pi n$) also works since $\phi$ is
well-defined up to addition of $2 \pi$. 
\clearpage

\subsection{Proof}
\subsubsection{Proof of the trigonometric identity}
\begin{equation*}
  a \cos \theta + b \sin \theta = C \cos \left( \theta - \phi \right)
\end{equation*}

\paragraph{The high school proof}\footnote{
  \begin{align*}
    C \cos (\theta - \phi) &= C \left(\cos\theta \cos(-\phi) - \sin \theta \sin(-\phi) \right) \\
    &= C \left(\cos\theta \cos \phi + \sin \theta \sin \phi \right) \\
    &= (C cos \phi) \cos \theta + (C \sin \phi) \sin \theta
  \end{align*}
  Compare two form
  \begin{equation*}
    a \cos \theta + b \sin \theta \quad \text{with} \quad
    (C \cos \phi) \cos \theta + (C \sin \phi) \sin \theta
  \end{equation*}
  \begin{eqnarray*}
    a &=& C \cos \phi\\
    b &=& C \sin \phi \\
  \end{eqnarray*}
  How can we use these to find values for $A$ and $\phi$?
  \begin{align*}
    a^2 + b^2 &= C^2 (\cos \phi)^2 + C^2(\sin \phi)^2 \\
              &= C^2(\cos^2 \phi + \sin^2 \phi) \\
              &= C^2
  \end{align*}
  So, $\displaystyle C \displaystyle = \displaystyle \sqrt{a^2 + b^2}$ and
  $\phi$ is
  \begin{eqnarray*}
    \displaystyle \frac{C \sin \phi}{C \cos \phi}
    &\displaystyle = & \displaystyle \frac{b}{a}\\
    \tan \phi &=& \frac{b}{a}                                                                     
  \end{eqnarray*}
}

The thing that's no good about it is that the direction in which it goes is from right to left side.
Well, everybody knew that. If I gave you this and told you write it out
in terms of cosine and sine, I would assume and dearly hope that practically all of you could do that.
Unfortunately, when you want to use the formula, it's this way you want to use it in the opposite direction.
You're starting with this and want to convert it to that.
Now the proof, therefore, will not be of much help.
It requires you to go in the backwards direction and match up coefficients.
It's much better to go forwards.
Now there are two proofs that go forwards.
\clearpage

\paragraph{The 18.02 proof} is
\begin{figure}[ht!]
  \centering
  \includegraphics[width=0.5\textwidth]{image-18_02_proof}
  \caption{18.02 Proof}
  \label{fig:18.02 Proof}
\end{figure}

First of all, the $a$ and the $b$ are the given.
So I'm going to put in that vector $< a, b>$.
Now, there's another vector lurking around.
It's the unit vector whose-- I'll write it this way, $\hat u$ because it's
a unit vector and theta to indicate that its angle is $\theta$.
Now the reason for doing that is because you
see that the left hand side is a dot product of two vectors.
\begin{equation*}
  <a , b> \cdot <\cos \theta, \sin \theta>
\end{equation*}
The left hand side of the identity is the dot product of the vector $<a, b>$ with the vector
whose components are cosine $\theta$ and sine $\theta$.
That's what I'm calling this unit vector.
It's a unit vector because cosine squared plus size squared is one.
Now all this formula is is saying that scale of product, the dot product of those two vectors,
can be evaluated if you know their components by the left hand side of the formula.
And if you don't know their components, it can be evaluated in another way.
The geometric evaluation, which goes--What is it?
It's the magnitude of one times the magnitude of the other times the cosine of the included angle.
Now what's the included angle?
Well, $\theta$ is this angle from the horizontal to that unitvector.
The angle $phi$ is this angle from figure \ref{fig:18.02 Proof} here.
And therefore the included angle between $\theta$ and my pink vector is theta minus $\phi$.
That's the formula.
\begin{equation*}
  <a , b> \cdot < \cos \theta, \sin \theta > = |<a, b>| \cdot 1 \cdot \cos (\theta - \phi)
\end{equation*}
It comes from two ways of calculating the scalar product of the vector whose coefficients are
$<a, b>$ and the unit vector whose components are cosine theta and sine theta.\\

\paragraph{The 18.03 proof} is uses complex numbers.
Take the left side. Instead of viewing it as the dot product of two vectors,
there's another way.
You can think of it as part of the product of two complex numbers.
So the 18.03 argument, really the complex number argument, says look.
Multiply together $a -bi$ and the complex number $\cos \theta + i \sin \theta$.
\begin{equation*}
  (a - bi)(\cos \theta + i \sin \theta)
\end{equation*}
There are different ways of explaining why I want to put the $-i$ there instead of $i$.
But the simplest is because, when I take the real part to get the left hand side, I will.
If I take the real part of this, I'm going to get $a \cos \theta + b \sin \theta$
because $-i$ and $i$ make one multiplied together.
That's the left hand side. And now the right hand side, I'm going
to use polar representation instead. What's the polar representation of this guy?
Well, if $<a, b>$ has the angle $\theta$, then $a -bi$, goes down below on figure \ref{fig:18.02 Proof}.
It has the angle $- \ phi$. 
In its polar representation, $\sqrt{a^2 + b^2} e^{-\phi}$ not positive $\phi$
because of $a -bi$ goes below the axis if $a$ and $b$ are positive.
How about the second guy?
The second guy is $e^{i \theta}$. 
So what's the product?
\begin{align*}
  (a - bi)(\cos \theta + i \sin \theta) &= \sqrt{a^2 + b^2} e^{-\phi} \cdot e^{i \theta}\\
  &= \sqrt{a^2 + b^2} e^{i(\theta - \phi)}
\end{align*}
And now what do I want? The real part of this.
And I want the real part of this.
So let's just say take the real parts of both sides.
\begin{equation*}
  a \cos \theta + b \sin \theta = \sqrt{a^2 + b^2} \cos(\theta - \phi)
\end{equation*}



%%% Local Variables:
%%% mode: latex
%%% TeX-master: "NoteForDifferentialEquation"
%%% End:

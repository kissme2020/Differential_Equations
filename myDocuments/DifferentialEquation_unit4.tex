\section{UNIT4}
\clearpage

\subsection{Operators and Exponential Response}

\subsubsection{Objective}

\textbf{\color{blue}Objectives}
\begin{itemize}
\item Write {\color{blue} linear constant coefficient} differential equations
  using {\color{blue} operator} notation, that is using
  {\color{blue} linear time invariant (LTI) operators}.
\item Use {\color{blue} time-shifts } to solve LIT differential equations.
\item Solve inhomogeneous constant coefficient ODEs with exponential inputs
  using the {\color{blue} Exponential Response Formula (ERF)}.
\item Recognize when ERF {\color{blue} fails} and how to apply the {\color{blue} ERF'} and
  the {\color{blue} Generalized ERF} in these situations.
\item Apply superposition to higher order, linear {\color{blue} inhomogeneous }
  differential equations. 
\end{itemize}
\clearpage

\subsubsection{Operator Notation}

If $y_1$ and $y_2$ are two solutions to a homogeneous, linear differential equation
\begin{equation*}
  \ddot y + p\dot y + qy = 0,
\end{equation*}
Why is $c_1y_1 + c_2y_2$ a solution?\\

You know superposition for linear homogeneous differential equations.
We want an elegant proof because it will help us with more complicated problems,
including solving higher order inhomogeneous equations.\\

\paragraph{An elegant proof of superposition}
The superposition principle says exactly that if $y_1$ and $y_2$ are solution to a  linear homogeneous ODE. \\
\footnote{in fact, it can be of higher order too, In other words, you don't have to stop
with a second derivative. You could add a third derivative and fourth derivative.
As long as the form remained the same, then that implies, automatically, that $c_1 y_1 + c_2 y_2$
is a solution.}
Now the way to do that nicely is to take a little detour and talk a little bit about
{\color{blue}linear operators}.
And since we're going to be using these for the rest of the term, this is the natural place for you
to learn a little bit about what they are.
So I'm going to do it.
Ultimately, I am aimed at a proof of this statement, but there are going to be certain side
excursions I have to make.\\

The first side excursion is to write the differential equation in a different way.
So I'm going to just write its transformations.
So first, I'll simply recopy it, 
\begin{equation*}
  \ddot y + p\dot y + qy = 0. 
\end{equation*}
That's the first form. \\
The second form, I'm going to replace this by the differentiation operator $D$
\begin{equation*}
  D^2 y + pDy qy = 0, 
\end{equation*}

$D^2$ means differentiate it twice. $D$ it, and then $D$ it again.
$D$ means only have to differentiate once, so I'll write that as $p Dy$, $p$ times the derivative of $y$.\\
Now I'm going to formally factor out the $y$.
\begin{equation*}
  \left( D^2 + pD + q \right)y = 0 
\end{equation*}
So this I'm going to turn into $D^2 +  pD + q$.
Now everybody reads $D^2 +  pD + q$ times $y$ equals $0$.
But what it means is $(D^2 + pD + q)y$ means this is shorthand for
$D^2 y + pDy qy$.
I'm multiplying $q$ times $y$, but I'm not multiplying $D$ times $y$.
I'm applying $D$ to $y$. \\
And now I'll take the final step.
I'm going to view $(D^2 + pD + q)$ as a guy all by itself, a {\color{blue} linear operator}.
This is called a linear operator.
And I'm going to simply abbreviate it by the letter $L$.
And so the final version of this equation
\begin{equation*}
  Ly = 0
\end{equation*}

\clearpage
Now what's L? You could think of L as
\begin{equation*}
  L = D^2 + pD + q 
\end{equation*}

But you can think of L the way to think of it is as a black box.
\begin{figure}[ht!]
  \centering
  \includegraphics[width=0.5\textwidth]{image-linear_operator}
  \caption{Linear Operators}
\end{figure}

What goes into this black box? Well if this were a function box, what
would go in would be a number and what would come out would be a number.
But it's not that kind of a black box. It's an {\color{blue} operator box}.
And therefore what goes in is a function of $x$, and what comes out is another function of $x$,
the result of applying this operator to that.\\
So from this point of view, trying to solve the differential equation means what should come
out, you want to come out 0 and the question is what should you put in?
That's what it means; solving differential equations in an {\color{blue} inverse problem}.
The easy thing is to put in a function and see what comes out.
You just calculate. The hard thing is to ask.
You say, I want such and such a thing to come out, for example, $0$.
What should I put in? That's the difficult question.
And that's what we're spending the term answering.\\

Now the key thing about this is that this is a linear operator
and what that means is that it behaves in a certain way with respect to functions.
The easiest way to say it is, I like to make two laws of it,
\begin{align*}
  L(u _1 + u _2 ) &= L(u_1) + L(u_2)
  \text{ Where } u _1 \text{ and } u _2 \text{ a function of } x \\
  L(c u) &= c L(u) \text{ Where } c \text{ is a any constant. } 
\end{align*}
These are the two laws of linearity.
An operator is linear if it satisfies these two laws.
Now for example, the differentiation operator is such an operator; D is linear.
Why?
\begin{align*}
  (u _1 + u _2)' &= u' _1 + u' _2 \\
  (cu)' &= c u' 
\end{align*}

All that was a prelude to proving this simple theorem, the superposition principle.
So finally, what's the proof?
Well, proof of the superposition principle, if you believe that the operator is linear,
in other words, the ODE is $L$.
$L$ is $D^2 +  pD +  q$. So the ODE is $Ly = 0$.
And what am I being asked to prove?
I'm being asked to prove that if $y _1$ and $y _2$ are solutions,
\begin{align*}
  L(c_1 y _1 + c_2 y_2 \footnotemark) &= L(c_1 y _1) + L(c _2 y _2) \\
                                      &= c _1 L(y _1) + c _2 L(y _2) \\
                                      &= c _1 0 + c _2 0 = 0                                        
\end{align*}
\footnotetext{
  $c_1 y _1 + c_2 y_2$ called a linear combination.
  This expression is called a linear combination of $y_1$ and $y_2$.
  It means that particular sum with constant coefficients.
}
I'm trying to prove that this is $0$.
Well, what is $L(y _1)$?
At this point, I use the fact that $y_1$ is a solution.
Because it's a solution, this is $0$. That's what it means to solve that differential equation.
It means when you apply the linear operator $L$ to the function, you get $0$.
In the same, way $y _2$ is a solution. So that's $0$. 
That's the argument.
Why it's so is because the operator, this differential equation, is expressed as a linear operator applied
to $y$ is $0$.
And the only properties that are really being used is the fact that this operator is linear.
That's the key point. $L$ is linear.
It's a linear operator.\\

\textbf{\color{blue} The operator $D$}
\begin{itemize}
\item A function takes an input number and returns another number.
\item An {\color{blue} operator} takes an input \textbf{function} and return another function. 
\end{itemize}

For example, the {\color{blue} differential operator} $\frac{d}{dt}$ takes an input function
$y(t)$ and returns $\frac{dy}{dt}$.
This operator is also called {\color{blue} $D$}.
For instance $De^{4t} = 4e^{4t}$. The operator $D$ is {\color{blue} linear} ,
which means that
\begin{equation*}
  D(f+g) = Df + Dg, \qquad D(af) = a Df
\end{equation*}

for any functions $f$ and $g$, and any number $a$. Because of this, $D$
behaves well with respect to linear combinations, namely
\begin{equation*}
  D(c_1 f_1 + \cdots + c_ n f_ n) = c_1 \,  D f_1 + \cdots + c_ n \,  Df_ n
\end{equation*}
for any numbers $c_1,\ldots ,c_ n$ and functions $f_1,\ldots ,f_ n$.

\begin{example}
  $Dt^3 = 3t^2$. 
\end{example}

\Warning You can't take this equation and substitute $t = 2$ to get $D8 = 12$.
The only way to interpret ``8'' in ``$D8$'' is as a constant function,
which of course has derivative zero: $D8 = 0$.
The point is that in order to know the function $Df(t)$ at a particular value of $t$,
say $t = a$, you need to know more than just $f(a)$;
you need to know how $f(t)$ is changing near $a$ as well.
This is characteristic of operators;
in general you have to expect to need to know the whole function $f(t)$ in order to evaluate an operator on it.

\begin{mydef}
  In general, a linear operator $L$ is any operator that satisfies
  \begin{equation*}
    L(f+g) = Lf + Lg, \qquad L(af) = a\, Lf
  \end{equation*}
  for any function $f$ and $g$, and any number $a$. 
\end{mydef}

\begin{example}
  The operator $L = D^2 + p(t)D + q(t)$ where $p(t)$ and $q(t)$ are
  any functions of $t$ is a linear operator. 
\end{example}
Why is $L$ linear? You know that $D$ is linear, and similarly, $D^2$ is linear.
To see that $L$ is linear, verify that a linear combination of linear operators is again a linear operator.

\clearpage
\subsubsection{Multiplying and adding operators}
To apply a product of two operators, apply each operator in succession.
For instance, $DDy$ means take the derivative of $y$, and then take the derivative of the result;
therefore we write $D ^2 y = \ddot{y}$.\\

To apply a sum of two operators, apply each operator to the function and add the results.
For instance,
\begin{equation*}
  (D^2 + D) y = D^2 y + D y = \ddot{y} + \dot{y}
\end{equation*}

Any number can be viewed as the ``multiply-by-the-number'' operator:
for instance, the operator $5$ transforms the function $\sin ⁡x$ into the function $5 \sin ⁡x$.\\

Similarly, we can multiply by any function of $t$, and this is also a linear operator.
For instance, the operator $t^2$ transforms the function $\sin ⁡x$ into the function $t ^22 \sin⁡ x$.

\begin{example}
  The ODE
\end{example}
\begin{equation*}
  2 \ddot{y} + 3 \dot{y} + 5 y = 0,
\end{equation*}
whose characteristic polynomial is $P(r) = 2r^2+3r+5$, can be rewritten as
\begin{align*}
  (2D^2 + 3D + 5) y &= 0\\
  P(D) y &= 0 
\end{align*}

The same argument shows that every constant-coefficient homogeneous linear ODE
\begin{equation*}
  a_ n y^{(n)} + \cdots + a_0 y = 0
\end{equation*}
can be written simply as
\begin{equation*}
  P(D) y = 0,
\end{equation*}
where $P$ is the characteristic polynomial.

\begin{exercise}
  Operator notation concept check
\end{exercise}
Consider the differential equation
\begin{equation*}
  \displaystyle  \frac{d^4x}{dt^4}+2\frac{d^2x}{dt^2}+x=0.
\end{equation*}
If we want to write the differential equation in the form
$P\left(D\right)x=0$, what is the differential operator $P\left(D\right)$?\\

The characteristic polynomial of the differential equation
\begin{equation*}
  \displaystyle  \frac{d^4x}{dt^4}+2\frac{d^2x}{dt^2}+x=0
\end{equation*}
is $P\left(r\right)=r^4+2r^2+1$, so the differential equation can be written in the form
$P\left(D\right) x = 0$, where
\begin{equation*}
  \displaystyle  P\left(D\right)=D^4+2D^2+1
\end{equation*}

\begin{exercise}
  Linear operators concept check
\end{exercise}
Which of the following differential operators are linear? Check all that apply.
\begin{enumerate}
\item $4D ^n + 3$
\item $D + t ^2$
\item $D ^2 + tD + t ^2$
\item $mD ^2 + bD + k$
\item $a_0D^ n + a_1D^{n-1} + \cdots + a_ n$
\end{enumerate}

All of these operators are linear.
In particular, every linear differential equation can be written in terms of a linear differential operator.\\

The second and third choices are linear with variable coefficients.
The first, fourth, and fifth are linear with constant coefficients.\\

We'll only have formulas for solutions when the operators involved have constant coefficients.
\clearpage

\subsubsection{Time Invariance}
In this course we'll focus on polynomial differential operators with constant coefficients,
that is operators of the form
\begin{equation*}
  \displaystyle  P\left(D\right)=a_ nD^ n+a_{n-1}D^{n-1}+\dots a_1D+a_0,
\end{equation*}

where all of the coefficients $a_k$ are numbers (as opposed to functions of $t$).
All operators of this form are linear.
In addition to being linear operators, they are also {\color{blue}time-invariant} operators, which means:
\begin{equation*}
  \text{If } x(t) \text{solves } P\left(D\right)x=f\left(t\right), \,
  \text{then } y\left(t\right)=x\left(t-t_0\right) \text{ solves }
  P\left(D\right)y=f\left(t-t_0\right)
\end{equation*}

In words, this says that ``delaying the input signal $f(t)$ by $t_0$
seconds delays the output signal $x(t)$ by $t_0$ seconds.''
If we know that $x(t)$ is a solution to $P\left(D\right)x=f\left(t\right)$,
we can solve $P\left(D\right)y=f\left(t-t_{0}\right)$ by replacing $t$ by $t - t_ 0$.
This is a useful property because gives us the solutions to many differential equations for free.

\begin{example}
  The function $x(t) = \sin (t)$ solves the differential equation
  $\dot(x) = \cos t$.  
\end{example}
What is a solution to the differential equation
\begin{equation*}
  \dot y = \cos (t +\pi /2) ?
\end{equation*}
By time-invariance, one solution is $y=\sin (t+\pi /2)$. \\
Note that
\begin{eqnarray*}
  \dot{y} &=& \cos (t + \pi / 2) \\
  \dot{y} &=& \cos t \cos \pi / 2  - \sin t \sin \pi / 2 \\
  \dot{y} &=& - \sin t 
\end{eqnarray*}

And
\begin{eqnarray*}
  y &=& \sin (t + \pi / 2) \\
  y &=& \sin t \cos  \pi / 2 + \cos t \sin  \pi / 2 \\
  y &=& \cos t 
\end{eqnarray*}

\begin{example}
  Consider the differential equation
\end{example}
\begin{equation*}
  \displaystyle  \dot{x}+x=\cos t.
\end{equation*}
Using integrating factors $e^t$
\begin{eqnarray*}
  e^t \dot{x} + e^t x &=& e^t \cos t \\
  \int (e^t x)' dt &=& \int e^t \cos t dt \footnotemark \\
  e^t x &=& \frac{e^{t}}{2} \cos t + \frac{1}{2} \sin t + c_1  \\
  x &=& \frac{1}{2} \cos t + \frac{1}{2} \sin t + c_1 e^{-t}.
\end{eqnarray*}

\footnotetext{
  $\cos t$ is a real part of $e^{it}$,
  \begin{align*}
    \int e^t \cos t dt  &= \int e^t e^{it} dt \\
                        &= \int  e^{(1 + i)t} dt \\
                        &= \frac{1}{1+i} e^{t} e^{it} + c \\
                        &= \frac{e^{t}}{2} \left( (1 - i) (\cos t + i \sin t)\right) + c \\
                        &= \frac{e^{t}}{2} \left( (1 - i) (\cos t + i \sin t)\right) + c 
  \end{align*}
  The $\mathbb{Re} \left( (1 - i) (\cos t + i \sin t)\right)$ is
  $\cos t + \sin t$.
  So,
  \begin{equation*}
    \int e^t \cos t dt = \frac{1}{2} \cos t + \frac{1}{2} \sin t + c
  \end{equation*}
}

Now suppose that we want to solve the equation $\dot{y}+y=\sin t$.
Since $\sin t=\cos \left(t-\pi /2\right)$, time invariance tells us that
\begin{align*}
  \displaystyle  y\left(t\right)
  &\displaystyle =x\left(t-\frac{\pi }{2}\right)
    =\frac12\cos \left(t-\pi /2\right) + \frac12 \sin \left(t-\pi /2\right)+c_1e^{-\left(t-\pi /2\right)} \\
  &\displaystyle =
    \frac12\sin \left(t\right) - \frac12 \cos \left(t\right)+
    \underbrace{c_1e^{\pi /2}}_{\textrm{call this }c_{2}}e^{-t} \\
  &\displaystyle =\frac12\sin \left(t\right) - \frac12 \cos \left(t\right)+c_2e^{-t}
\end{align*}

should solve $\dot{y}+y=\sin t$.
This agrees with the solution we'd get if we had used variation of parameters or integrating factors,
but we had to do almost no work to get this solution from the first.
\begin{figure}[ht!]
  \centering
  \includegraphics[width=0.5\textwidth]{image-LTI_function}
  \caption{LTI Function}
\end{figure}

\begin{exercise}
  LTI concept check
\end{exercise}

Which of the following differential operators are LTI operators? Check all that apply.
\begin{itemize}
\item $4D^ n + 3$
\item $D + t^2$
\item $D^2 +tD + t^2$
\item $mD^2 +bD +k$
\item $a_0D^ n + a_1D^{n-1} + \cdots + a_ n$
\end{itemize}

All of these operators are linear, and all of them that are polynomials with constant coefficients are time-invariant.
The exceptions are $tD + 2$ and $D^2 +tD + t^2$. \\
To see that $P\left(D\right)=tD + 2$ for instance is not time-invariant,
notice that the general solution to
\begin{equation*}
  \displaystyle  P\left(D\right)x=t\dot{x}+2x=t^5
\end{equation*}
on $\left(0,+\infty \right)$ is
\begin{equation*}
  \displaystyle  x\left(t\right)=\frac{1}{7}t^{5}+ct^{-2}
\end{equation*}

Now if the operator were time invariant, then
\begin{equation*}
  \displaystyle  y\left(t\right)
  =x\left(t-t_{0}\right)=\frac{1}{7}\left(t-t_0\right)^{5}+c\left(t-t_0\right)^{-2}
\end{equation*}
would solve
\begin{equation*}
  \displaystyle  t\dot{y}+2y=\left(t-t_{0}\right)^5,
\end{equation*}
for any $t_{0}>0$. However,
\begin{equation*}
  \displaystyle t\dot{y}+2y
  =\frac{2}{7}\left(t-t_{0}\right)^5+\frac{5}{7}t\left(t-t_{0}\right)^4
  +2c\left(t-t_0\right)^{-2}-2ct\left(t-t_{0}\right)^{-3}.
\end{equation*}
It should be clear from this that $t\dot{y}+2y\neq \left(t-t_{0}\right)^5$,
so the linear operator $tD + 2$ tD + 2. 
\clearpage

\subsubsection{Superposition for an inhomogeneous linear ODE}
To understand the general solution $y$ to an inhomogeneous linear ODE

\begin{equation*}
  \text {inhomogeneous equation:}
  \quad p_ n(t) \,  {\color{blue}{y^{(n)}}}  + \cdots + p_0(t) \,
  {\color{blue}{y}}  = {\color{orange}{q(t)}} ,
\end{equation*}
do the following:
\begin{enumerate}
\item List all solutions to the associated homogeneous equation
  \begin{equation*}
    \text {homogeneous equation:}
    \quad p_ n(t) \,  {\color{blue}{y^{(n)}}}  +
    \cdots + p_0(t) \,  {\color{blue}{y}}  = 0;
  \end{equation*}
\item Find (in some way) any \textbf{one} particular solution $y_p$ to the
  \textbf{inhomogeneous} ODE.
\item Add $y_p$ to all the solutions of the homogeneous ODE to get all the solutions to the inhomogeneous ODE.
\end{enumerate}

\Summary \\
\begin{equation*}
  \underset {{\color{blue}{\text {general solution}}} }{y}
  = \underset {{\color{blue}{\text {particular solution}}} }{y_ p}
  + \underset {\text {general homogeneous solution}}{y_ h}.
\end{equation*}

Why does this work? \textbf{Proof.} Let $L$ be the linear operator
$L = p_ n(t)D^ n + \dotsb + p_1(t)D+p_0(t)$.
Then the differential equations become
\begin{eqnarray*}
  \displaystyle  \text {inhomogeneous equation} \qquad Ly
  \displaystyle &=& \displaystyle q(t) \\
  \displaystyle  \text {homogeneous equation} \qquad Ly
  \displaystyle &=& \displaystyle 0   
\end{eqnarray*}
Let $y_p$ be a particular solution to the inhomogeneous equation $L(y_p)=q$.
Let $y_h$ be a homogeneous solution $L(y_h)=0$. Then
\begin{equation*}
  L(y_ p+y_ h) =q+0=q.
\end{equation*}
Suppose that $y$ is also a solution to $Ly=q$. That is
\begin{eqnarray*}
  \displaystyle  Ly \qquad \displaystyle &=& \qquad q \\
  \displaystyle  Ly_p \qquad \displaystyle &=& \qquad q.
\end{eqnarray*}
Then it follows by linearity that
\begin{equation*}
  L(y-y_ p) = L(y)-L(y_ p) = q - q = 0
\end{equation*}
Therefore $y−y_p=y_h$ is a solution to the associated homogeneous equation. In other words,
\begin{equation*}
  y=yp+yh.
\end{equation*}

{\color{blue} The result of this section is the key point of linearity in the inhomogeneous case.}
It lets us build \textbf{all} the solutions to an inhomogeneous DE out of
\textbf{one particular solution} provided that you have already solved the associated homogeneous ODE.

\paragraph{Consequence of superposition for inhomogeneous equations}
So the {\color{blue}main theorem} about solving the inhomogeneous equation.
\begin{equation*}
  \ddot{y} + p\dot{y} + q = f(x) 
\end{equation*}
I'm going to make the left hand side a linear operator $L$.
And I'm going to write the equation
\begin{equation*}
  Ly = f(x). 
\end{equation*}
And what the theorem says is that the solution has the following form.
\begin{equation*}
  \text{Solution : } y_p + y_c\footnotemark 
\end{equation*}
\footnotetext{
  $y_c$ stands for the complementary solution.\\
  The general solution of differential equation is
  \begin{equation*}
    y = y_p + c_1 y_1 + c_2 y_2 \text{, where } y_h = c_1 y_1 + c_2 y_2
  \end{equation*}
}
What is $y_p$? $p$ stands for \textbf{particular}, the most confusing word in this subject.\\

The procedure for solving this equation is composed of two steps.
First, to find complementary solution $y_c$.
solve not the equation you are given, but the reduced equation.
The second step is to find $y_p$.
$y_p$ is a particular solution to the whole equation. Yeah, but which one?
Any one. Well, if it's any one, then it's not a particular solution.
So particular means any one solution.\\


First prove the theorem
It's extremely simple if you just use the fact that $L$ is a linear operator.
We've got two things to prove.
\begin{enumerate}
\item All the $y_p + c_1 y_1 + c_2 y_2$ are solutions.\\
  You plug it into the equation and you see if it satisfies the equation.
  \begin{eqnarray*}
    L(y_p + c_1 y_1 + c_2 y_2)
    &=& \underbrace{L(y_p)}_{f(x) \text{ since particular solution} }
        + \underbrace{L(c_1 y_1 + c_2 y_2)}_{0 \text{ since homogeneous solution}} \\
    L(y_p + c_1 y_1 + c_2 y_2) &=& f(x)    
  \end{eqnarray*}
  They satisfy the whole inhomogeneous differential equation
  \begin{equation*}
    L(y) = f(x).  
  \end{equation*}

\item There are no other solutions. \\
  So, Let $u(x)$ as a solution.
  \begin{equation*}
    L(u) = f(x).
  \end{equation*}
  The $L(y_p)$ is
  \begin{equation*}
    L(y_p) = f(x). 
  \end{equation*}
  Subtract two equations
  \begin{equation*}
    L(u - y_p) = 0 
  \end{equation*}
  The homogeneous solution $L(c_1 y_1 + c_2 y_2) = 0$, 
  \begin{equation*}
    u - y_p = \tilde{c_1}y_1 + \tilde{c_1} y_1 
  \end{equation*}
  I'll put a $\tilde{ }$ on to indicate it's a particular one. 
  Therefor
  \begin{equation*}
    u  = y_p + \tilde{c_1}y_1 + \tilde{c_1} y_1 
  \end{equation*}  
\end{enumerate}
All we have to do is find to solve equations which are inhomogeneous,
all we have to do is find a particular solution.
Find one solution. It doesn't matter which one.
\clearpage

\subsubsection{Why exponential inputs?}
Our task for today is to find particular solutions.
We're talking about the second order equation with constant coefficients,
which you can think of as modeling springs or simple electrical circuits.
\begin{equation*}
  y'' + Ay' + By = f(x)
\end{equation*}
The problem is remember that to find a particular solution $y_p$. 
And the reason why we want to do that is then the general solution will be
of the form is $y$ equals that particular solution, plus the complementary solution
the general solution to the reduced equation,
\begin{equation*}
  y = y_p + c_1 y_1 + c_2 y_2. 
\end{equation*}

So all the work depends upon finding out what that $y_p$ is.
And that's what we're going to talk about today--
or rather, talk about for two weeks.
But the point is not all functions
that you could write on the right-hand side
are equally interesting.
There's one kind, which is far more interesting or more
important in the applications than all the others.
And that's the one out of which, in fact,
as you'll see later on this week and next week,
an arbitrary function can be built out
of these simple functions.
So the important function is on the right-hand side
to be able to solve it when it's a simple exponential.
But if you allow me to make it a complex exponential,
so here are the important right-hand sides we want.

We want to be able to do it when it's of the form e to the ax.
In general, that will be--
in most applications, a is not a growing exponential,
but a decaying exponential.
So typically, a is negative, but it doesn't have to be.
I'll put it in parentheses though often.
That's not any assumption that I'm going to make today.
It's just culture.
Well, we want to be able to do it for sine omega
x and cosine omega x, in other words, when the right-hand side
is a pure oscillation.
That's another important type of input both for electrical
circuits-- think alternating current--
or the spring systems that's a pure vibration is being--
you're imposing a pure vibration on the spring mass dashpot
system, and you want to see how it responds to that.
Or you can put them together and make these decaying
oscillations.
So we could also have something like e to the ax times
sine omega x or times cosine omega x.

Now, the point is all of these together
are really just special cases of one general thing--
exponential if you allow the exponent
not to be a real number, but to be a complex number.
So they are all special cases of e
to the, I'll write it alpha x.
Well, why don't we write it of a plus i omega x, right?
If omega is 0, then I've got this case.
If a is 0, then I've got this case
separating it into its real and imaginary parts.
And if neither is 0, I have this case.

But I don't want to keep writing a plus i omega all the time.
So I'm going to write that simply as e to the alpha x.
And you understand that alpha is a complex number now.
Looks different.
It doesn't look like a real number.
OK, so it's a complex number.
\clearpage
%%% Local Variables:
%%% mode: latex
%%% TeX-master: "NoteForDifferentialEquation"
%%% End:

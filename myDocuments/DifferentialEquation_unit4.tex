\section{UNIT4}
\clearpage

\subsection{Operators and Exponential Response}

\subsubsection{Objective}

\textbf{\color{blue}Objectives}
\begin{itemize}
\item Write {\color{blue} linear constant coefficient} differential equations
  using {\color{blue} operator} notation, that is using
  {\color{blue} linear time invariant (LTI) operators}.
\item Use {\color{blue} time-shifts } to solve LIT differential equations.
\item Solve inhomogeneous constant coefficient ODEs with exponential inputs
  using the {\color{blue} Exponential Response Formula (ERF)}.
\item Recognize when ERF {\color{blue} fails} and how to apply the {\color{blue} ERF'} and
  the {\color{blue} Generalized ERF} in these situations.
\item Apply superposition to higher order, linear {\color{blue} inhomogeneous }
  differential equations. 
\end{itemize}
\clearpage

\subsubsection{Operator Notation}

If $y_1$ and $y_2$ are two solutions to a homogeneous, linear differential equation
\begin{equation*}
  \ddot y + p\dot y + qy = 0,
\end{equation*}
Why is $c_1y_1 + c_2y_2$ a solution?\\

You know superposition for linear homogeneous differential equations.
We want an elegant proof because it will help us with more complicated problems,
including solving higher order inhomogeneous equations.\\

\paragraph{An elegant proof of superposition}
The superposition principle says exactly that if $y_1$ and $y_2$ are solution to a  linear homogeneous ODE. \\
\footnote{in fact, it can be of higher order too, In other words, you don't have to stop
with a second derivative. You could add a third derivative and fourth derivative.
As long as the form remained the same, then that implies, automatically, that $c_1 y_1 + c_2 y_2$
is a solution.}
Now the way to do that nicely is to take a little detour and talk a little bit about
{\color{blue}linear operators}.
And since we're going to be using these for the rest of the term, this is the natural place for you
to learn a little bit about what they are.
So I'm going to do it.
Ultimately, I am aimed at a proof of this statement, but there are going to be certain side
excursions I have to make.\\

The first side excursion is to write the differential equation in a different way.
So I'm going to just write its transformations.
So first, I'll simply recopy it, 
\begin{equation*}
  \ddot y + p\dot y + qy = 0. 
\end{equation*}
That's the first form. \\
The second form, I'm going to replace this by the differentiation operator $D$
\begin{equation*}
  D^2 y + pDy qy = 0, 
\end{equation*}

$D^2$ means differentiate it twice. $D$ it, and then $D$ it again.
$D$ means only have to differentiate once, so I'll write that as $p Dy$, $p$ times the derivative of $y$.\\
Now I'm going to formally factor out the $y$.
\begin{equation*}
  \left( D^2 + pD + q \right)y = 0 
\end{equation*}
So this I'm going to turn into $D^2 +  pD + q$.
Now everybody reads $D^2 +  pD + q$ times $y$ equals $0$.
But what it means is $(D^2 + pD + q)y$ means this is shorthand for
$D^2 y + pDy qy$.
I'm multiplying $q$ times $y$, but I'm not multiplying $D$ times $y$.
I'm applying $D$ to $y$. \\
And now I'll take the final step.
I'm going to view $(D^2 + pD + q)$ as a guy all by itself, a {\color{blue} linear operator}.
This is called a linear operator.
And I'm going to simply abbreviate it by the letter $L$.
And so the final version of this equation
\begin{equation*}
  Ly = 0
\end{equation*}

\clearpage
Now what's L? You could think of L as
\begin{equation*}
  L = D^2 + pD + q 
\end{equation*}

But you can think of L the way to think of it is as a black box.
\begin{figure}[ht!]
  \centering
  \includegraphics[width=0.5\textwidth]{image-linear_operator}
  \caption{Linear Operators}
\end{figure}

What goes into this black box? Well if this were a function box, what
would go in would be a number and what would come out would be a number.
But it's not that kind of a black box. It's an {\color{blue} operator box}.
And therefore what goes in is a function of $x$, and what comes out is another function of $x$,
the result of applying this operator to that.\\
So from this point of view, trying to solve the differential equation means what should come
out, you want to come out 0 and the question is what should you put in?
That's what it means; solving differential equations in an {\color{blue} inverse problem}.
The easy thing is to put in a function and see what comes out.
You just calculate. The hard thing is to ask.
You say, I want such and such a thing to come out, for example, $0$.
What should I put in? That's the difficult question.
And that's what we're spending the term answering.\\

Now the key thing about this is that this is a linear operator
and what that means is that it behaves in a certain way with respect to functions.
The easiest way to say it is, I like to make two laws of it,
\begin{align*}
  L(u _1 + u _2 ) &= L(u_1) + L(u_2)
  \text{ Where } u _1 \text{ and } u _2 \text{ a function of } x \\
  L(c u) &= c L(u) \text{ Where } c \text{ is a any constant. } 
\end{align*}
These are the two laws of linearity.
An operator is linear if it satisfies these two laws.
Now for example, the differentiation operator is such an operator; D is linear.
Why?
\begin{align*}
  (u _1 + u _2)' &= u' _1 + u' _2 \\
  (cu)' &= c u' 
\end{align*}

All that was a prelude to proving this simple theorem, the superposition principle.
So finally, what's the proof?
Well, proof of the superposition principle, if you believe that the operator is linear,
in other words, the ODE is $L$.
$L$ is $D^2 +  pD +  q$. So the ODE is $Ly = 0$.
And what am I being asked to prove?
I'm being asked to prove that if $y _1$ and $y _2$ are solutions,
\begin{align*}
  L(c_1 y _1 + c_2 y_2 \footnotemark) &= L(c_1 y _1) + L(c _2 y _2) \\
                                      &= c _1 L(y _1) + c _2 L(y _2) \\
                                      &= c _1 0 + c _2 0 = 0                                        
\end{align*}
\footnotetext{
  $c_1 y _1 + c_2 y_2$ called a linear combination.
  This expression is called a linear combination of $y_1$ and $y_2$.
  It means that particular sum with constant coefficients.
}
I'm trying to prove that this is $0$.
Well, what is $L(y _1)$?
At this point, I use the fact that $y_1$ is a solution.
Because it's a solution, this is $0$. That's what it means to solve that differential equation.
It means when you apply the linear operator $L$ to the function, you get $0$.
In the same, way $y _2$ is a solution. So that's $0$. 
That's the argument.
Why it's so is because the operator, this differential equation, is expressed as a linear operator applied
to $y$ is $0$.
And the only properties that are really being used is the fact that this operator is linear.
That's the key point. $L$ is linear.
It's a linear operator.\\

\textbf{\color{blue} The operator $D$}
\begin{itemize}
\item A function takes an input number and returns another number.
\item An {\color{blue} operator} takes an input \textbf{function} and return another function. 
\end{itemize}

For example, the {\color{blue} differential operator} $\frac{d}{dt}$ takes an input function
$y(t)$ and returns $\frac{dy}{dt}$.
This operator is also called {\color{blue} $D$}.
For instance $De^{4t} = 4e^{4t}$. The operator $D$ is {\color{blue} linear} ,
which means that
\begin{equation*}
  D(f+g) = Df + Dg, \qquad D(af) = a Df
\end{equation*}

for any functions $f$ and $g$, and any number $a$. Because of this, $D$
behaves well with respect to linear combinations, namely
\begin{equation*}
  D(c_1 f_1 + \cdots + c_ n f_ n) = c_1 \,  D f_1 + \cdots + c_ n \,  Df_ n
\end{equation*}
for any numbers $c_1,\ldots ,c_ n$ and functions $f_1,\ldots ,f_ n$.

\begin{example}
  $Dt^3 = 3t^2$. 
\end{example}

\Warning You can't take this equation and substitute $t = 2$ to get $D8 = 12$.
The only way to interpret ``8'' in ``$D8$'' is as a constant function,
which of course has derivative zero: $D8 = 0$.
The point is that in order to know the function $Df(t)$ at a particular value of $t$,
say $t = a$, you need to know more than just $f(a)$;
you need to know how $f(t)$ is changing near $a$ as well.
This is characteristic of operators;
in general you have to expect to need to know the whole function $f(t)$ in order to evaluate an operator on it.

\begin{mydef}
  In general, a linear operator $L$ is any operator that satisfies
  \begin{equation*}
    L(f+g) = Lf + Lg, \qquad L(af) = a\, Lf
  \end{equation*}
  for any function $f$ and $g$, and any number $a$. 
\end{mydef}

\begin{example}
  The operator $L = D^2 + p(t)D + q(t)$ where $p(t)$ and $q(t)$ are
  any functions of $t$ is a linear operator. 
\end{example}
Why is $L$ linear? You know that $D$ is linear, and similarly, $D^2$ is linear.
To see that $L$ is linear, verify that a linear combination of linear operators is again a linear operator.

\clearpage
\subsubsection{Multiplying and adding operators}
To apply a product of two operators, apply each operator in succession.
For instance, $DDy$ means take the derivative of $y$, and then take the derivative of the result;
therefore we write $D ^2 y = \ddot{y}$.\\

To apply a sum of two operators, apply each operator to the function and add the results.
For instance,
\begin{equation*}
  (D^2 + D) y = D^2 y + D y = \ddot{y} + \dot{y}
\end{equation*}

Any number can be viewed as the ``multiply-by-the-number'' operator:
for instance, the operator $5$ transforms the function $\sin ⁡x$ into the function $5 \sin ⁡x$.\\

Similarly, we can multiply by any function of $t$, and this is also a linear operator.
For instance, the operator $t^2$ transforms the function $\sin ⁡x$ into the function $t ^22 \sin⁡ x$.

\begin{example}
  The ODE
\end{example}
\begin{equation*}
  2 \ddot{y} + 3 \dot{y} + 5 y = 0,
\end{equation*}
whose characteristic polynomial is $P(r) = 2r^2+3r+5$, can be rewritten as
\begin{align*}
  (2D^2 + 3D + 5) y &= 0\\
  P(D) y &= 0 
\end{align*}

The same argument shows that every constant-coefficient homogeneous linear ODE
\begin{equation*}
  a_ n y^{(n)} + \cdots + a_0 y = 0
\end{equation*}
can be written simply as
\begin{equation*}
  P(D) y = 0,
\end{equation*}
where $P$ is the characteristic polynomial.

\begin{exercise}
  Operator notation concept check
\end{exercise}
Consider the differential equation
\begin{equation*}
  \displaystyle  \frac{d^4x}{dt^4}+2\frac{d^2x}{dt^2}+x=0.
\end{equation*}
If we want to write the differential equation in the form
$P\left(D\right)x=0$, what is the differential operator $P\left(D\right)$?\\

The characteristic polynomial of the differential equation
\begin{equation*}
  \displaystyle  \frac{d^4x}{dt^4}+2\frac{d^2x}{dt^2}+x=0
\end{equation*}
is $P\left(r\right)=r^4+2r^2+1$, so the differential equation can be written in the form
$P\left(D\right) x = 0$, where
\begin{equation*}
  \displaystyle  P\left(D\right)=D^4+2D^2+1
\end{equation*}

\begin{exercise}
  Linear operators concept check
\end{exercise}
Which of the following differential operators are linear? Check all that apply.
\begin{enumerate}
\item $4D ^n + 3$
\item $D + t ^2$
\item $D ^2 + tD + t ^2$
\item $mD ^2 + bD + k$
\item $a_0D^ n + a_1D^{n-1} + \cdots + a_ n$
\end{enumerate}

All of these operators are linear.
In particular, every linear differential equation can be written in terms of a linear differential operator.\\

The second and third choices are linear with variable coefficients.
The first, fourth, and fifth are linear with constant coefficients.\\

We'll only have formulas for solutions when the operators involved have constant coefficients.
\clearpage

\subsubsection{Time Invariance}
In this course we'll focus on polynomial differential operators with constant coefficients,
that is operators of the form
\begin{equation*}
  \displaystyle  P\left(D\right)=a_ nD^ n+a_{n-1}D^{n-1}+\dots a_1D+a_0,
\end{equation*}

where all of the coefficients $a_k$ are numbers (as opposed to functions of $t$).
All operators of this form are linear.
In addition to being linear operators, they are also {\color{blue}time-invariant} operators, which means:
\begin{equation*}
  \text{If } x(t) \text{solves } P\left(D\right)x=f\left(t\right), \,
  \text{then } y\left(t\right)=x\left(t-t_0\right) \text{ solves }
  P\left(D\right)y=f\left(t-t_0\right)
\end{equation*}

In words, this says that ``delaying the input signal $f(t)$ by $t_0$
seconds delays the output signal $x(t)$ by $t_0$ seconds.''
If we know that $x(t)$ is a solution to $P\left(D\right)x=f\left(t\right)$,
we can solve $P\left(D\right)y=f\left(t-t_{0}\right)$ by replacing $t$ by $t - t_ 0$.
This is a useful property because gives us the solutions to many differential equations for free.

\begin{example}
  The function $x(t) = \sin (t)$ solves the differential equation
  $\dot(x) = \cos t$.  
\end{example}
What is a solution to the differential equation
\begin{equation*}
  \dot y = \cos (t +\pi /2) ?
\end{equation*}
By time-invariance, one solution is $y=\sin (t+\pi /2)$. \\
Note that
\begin{eqnarray*}
  \dot{y} &=& \cos (t + \pi / 2) \\
  \dot{y} &=& \cos t \cos \pi / 2  - \sin t \sin \pi / 2 \\
  \dot{y} &=& - \sin t 
\end{eqnarray*}

And
\begin{eqnarray*}
  y &=& \sin (t + \pi / 2) \\
  y &=& \sin t \cos  \pi / 2 + \cos t \sin  \pi / 2 \\
  y &=& \cos t 
\end{eqnarray*}

\begin{example}
  Consider the differential equation
\end{example}
\begin{equation*}
  \displaystyle  \dot{x}+x=\cos t.
\end{equation*}
Using integrating factors $e^t$
\begin{eqnarray*}
  e^t \dot{x} + e^t x &=& e^t \cos t \\
  \int (e^t x)' dt &=& \int e^t \cos t dt \footnotemark \\
  e^t x &=& \frac{e^{t}}{2} \cos t + \frac{1}{2} \sin t + c_1  \\
  x &=& \frac{1}{2} \cos t + \frac{1}{2} \sin t + c_1 e^{-t}.
\end{eqnarray*}

\footnotetext{
  $\cos t$ is a real part of $e^{it}$,
  \begin{align*}
    \int e^t \cos t dt  &= \int e^t e^{it} dt \\
                        &= \int  e^{(1 + i)t} dt \\
                        &= \frac{1}{1+i} e^{t} e^{it} + c \\
                        &= \frac{e^{t}}{2} \left( (1 - i) (\cos t + i \sin t)\right) + c \\
                        &= \frac{e^{t}}{2} \left( (1 - i) (\cos t + i \sin t)\right) + c 
  \end{align*}
  The $\mathbb{Re} \left( (1 - i) (\cos t + i \sin t)\right)$ is
  $\cos t + \sin t$.
  So,
  \begin{equation*}
    \int e^t \cos t dt = \frac{1}{2} \cos t + \frac{1}{2} \sin t + c
  \end{equation*}
}

Now suppose that we want to solve the equation $\dot{y}+y=\sin t$.
Since $\sin t=\cos \left(t-\pi /2\right)$, time invariance tells us that
\begin{align*}
  \displaystyle  y\left(t\right)
  &\displaystyle =x\left(t-\frac{\pi }{2}\right)
    =\frac12\cos \left(t-\pi /2\right) + \frac12 \sin \left(t-\pi /2\right)+c_1e^{-\left(t-\pi /2\right)} \\
  &\displaystyle =
    \frac12\sin \left(t\right) - \frac12 \cos \left(t\right)+
    \underbrace{c_1e^{\pi /2}}_{\textrm{call this }c_{2}}e^{-t} \\
  &\displaystyle =\frac12\sin \left(t\right) - \frac12 \cos \left(t\right)+c_2e^{-t}
\end{align*}

should solve $\dot{y}+y=\sin t$.
This agrees with the solution we'd get if we had used variation of parameters or integrating factors,
but we had to do almost no work to get this solution from the first.
\begin{figure}[ht!]
  \centering
  \includegraphics[width=0.5\textwidth]{image-LTI_function}
  \caption{LTI Function}
\end{figure}

\begin{exercise}
  LTI concept check
\end{exercise}

Which of the following differential operators are LTI operators? Check all that apply.
\begin{itemize}
\item $4D^ n + 3$
\item $D + t^2$
\item $D^2 +tD + t^2$
\item $mD^2 +bD +k$
\item $a_0D^ n + a_1D^{n-1} + \cdots + a_ n$
\end{itemize}

All of these operators are linear, and all of them that are polynomials with constant coefficients are time-invariant.
The exceptions are $tD + 2$ and $D^2 +tD + t^2$. \\
To see that $P\left(D\right)=tD + 2$ for instance is not time-invariant,
notice that the general solution to
\begin{equation*}
  \displaystyle  P\left(D\right)x=t\dot{x}+2x=t^5
\end{equation*}
on $\left(0,+\infty \right)$ is
\begin{equation*}
  \displaystyle  x\left(t\right)=\frac{1}{7}t^{5}+ct^{-2}
\end{equation*}

Now if the operator were time invariant, then
\begin{equation*}
  \displaystyle  y\left(t\right)
  =x\left(t-t_{0}\right)=\frac{1}{7}\left(t-t_0\right)^{5}+c\left(t-t_0\right)^{-2}
\end{equation*}
would solve
\begin{equation*}
  \displaystyle  t\dot{y}+2y=\left(t-t_{0}\right)^5,
\end{equation*}
for any $t_{0}>0$. However,
\begin{equation*}
  \displaystyle t\dot{y}+2y
  =\frac{2}{7}\left(t-t_{0}\right)^5+\frac{5}{7}t\left(t-t_{0}\right)^4
  +2c\left(t-t_0\right)^{-2}-2ct\left(t-t_{0}\right)^{-3}.
\end{equation*}
It should be clear from this that $t\dot{y}+2y\neq \left(t-t_{0}\right)^5$,
so the linear operator $tD + 2$ tD + 2. 
\clearpage

\subsubsection{Superposition for an inhomogeneous linear ODE}
To understand the general solution $y$ to an inhomogeneous linear ODE

\begin{equation*}
  \text {inhomogeneous equation:}
  \quad p_ n(t) \,  {\color{blue}{y^{(n)}}}  + \cdots + p_0(t) \,
  {\color{blue}{y}}  = {\color{orange}{q(t)}} ,
\end{equation*}
do the following:
\begin{enumerate}
\item List all solutions to the associated homogeneous equation
  \begin{equation*}
    \text {homogeneous equation:}
    \quad p_ n(t) \,  {\color{blue}{y^{(n)}}}  +
    \cdots + p_0(t) \,  {\color{blue}{y}}  = 0;
  \end{equation*}
\item Find (in some way) any \textbf{one} particular solution $y_p$ to the
  \textbf{inhomogeneous} ODE.
\item Add $y_p$ to all the solutions of the homogeneous ODE to get all the solutions to the inhomogeneous ODE.
\end{enumerate}

\Summary \\
\begin{equation*}
  \underset {{\color{blue}{\text {general solution}}} }{y}
  = \underset {{\color{blue}{\text {particular solution}}} }{y_ p}
  + \underset {\text {general homogeneous solution}}{y_ h}.
\end{equation*}

Why does this work? \textbf{Proof.} Let $L$ be the linear operator
$L = p_ n(t)D^ n + \dotsb + p_1(t)D+p_0(t)$.
Then the differential equations become
\begin{eqnarray*}
  \displaystyle  \text {inhomogeneous equation} \qquad Ly
  \displaystyle &=& \displaystyle q(t) \\
  \displaystyle  \text {homogeneous equation} \qquad Ly
  \displaystyle &=& \displaystyle 0   
\end{eqnarray*}
Let $y_p$ be a particular solution to the inhomogeneous equation $L(y_p)=q$.
Let $y_h$ be a homogeneous solution $L(y_h)=0$. Then
\begin{equation*}
  L(y_ p+y_ h) =q+0=q.
\end{equation*}
Suppose that $y$ is also a solution to $Ly=q$. That is
\begin{eqnarray*}
  \displaystyle  Ly \qquad \displaystyle &=& \qquad q \\
  \displaystyle  Ly_p \qquad \displaystyle &=& \qquad q.
\end{eqnarray*}
Then it follows by linearity that
\begin{equation*}
  L(y-y_ p) = L(y)-L(y_ p) = q - q = 0
\end{equation*}
Therefore $y−y_p=y_h$ is a solution to the associated homogeneous equation. In other words,
\begin{equation*}
  y=yp+yh.
\end{equation*}

{\color{blue} The result of this section is the key point of linearity in the inhomogeneous case.}
It lets us build \textbf{all} the solutions to an inhomogeneous DE out of
\textbf{one particular solution} provided that you have already solved the associated homogeneous ODE.

\paragraph{Consequence of superposition for inhomogeneous equations}
So the {\color{blue}main theorem} about solving the inhomogeneous equation.
\begin{equation*}
  \ddot{y} + p\dot{y} + q = f(x) 
\end{equation*}
I'm going to make the left hand side a linear operator $L$.
And I'm going to write the equation
\begin{equation*}
  Ly = f(x). 
\end{equation*}
And what the theorem says is that the solution has the following form.
\begin{equation*}
  \text{Solution : } y_p + y_c\footnotemark 
\end{equation*}
\footnotetext{
  $y_c$ stands for the complementary solution.\\
  The general solution of differential equation is
  \begin{equation*}
    y = y_p + c_1 y_1 + c_2 y_2 \text{, where } y_h = c_1 y_1 + c_2 y_2
  \end{equation*}
}
What is $y_p$? $p$ stands for \textbf{particular}, the most confusing word in this subject.\\

The procedure for solving this equation is composed of two steps.
First, to find complementary solution $y_c$.
solve not the equation you are given, but the reduced equation.
The second step is to find $y_p$.
$y_p$ is a particular solution to the whole equation. Yeah, but which one?
Any one. Well, if it's any one, then it's not a particular solution.
So particular means any one solution.\\


First prove the theorem
It's extremely simple if you just use the fact that $L$ is a linear operator.
We've got two things to prove.
\begin{enumerate}
\item All the $y_p + c_1 y_1 + c_2 y_2$ are solutions.\\
  You plug it into the equation and you see if it satisfies the equation.
  \begin{eqnarray*}
    L(y_p + c_1 y_1 + c_2 y_2)
    &=& \underbrace{L(y_p)}_{f(x) \text{ since particular solution} }
        + \underbrace{L(c_1 y_1 + c_2 y_2)}_{0 \text{ since homogeneous solution}} \\
    L(y_p + c_1 y_1 + c_2 y_2) &=& f(x)    
  \end{eqnarray*}
  They satisfy the whole inhomogeneous differential equation
  \begin{equation*}
    L(y) = f(x).  
  \end{equation*}

\item There are no other solutions. \\
  So, Let $u(x)$ as a solution.
  \begin{equation*}
    L(u) = f(x).
  \end{equation*}
  The $L(y_p)$ is
  \begin{equation*}
    L(y_p) = f(x). 
  \end{equation*}
  Subtract two equations
  \begin{equation*}
    L(u - y_p) = 0 
  \end{equation*}
  The homogeneous solution $L(c_1 y_1 + c_2 y_2) = 0$, 
  \begin{equation*}
    u - y_p = \tilde{c_1}y_1 + \tilde{c_1} y_1 
  \end{equation*}
  I'll put a $\tilde{ }$ on to indicate it's a particular one. 
  Therefor
  \begin{equation*}
    u  = y_p + \tilde{c_1}y_1 + \tilde{c_1} y_1 
  \end{equation*}  
\end{enumerate}
All we have to do is find to solve equations which are inhomogeneous,
all we have to do is find a particular solution.
Find one solution. It doesn't matter which one.
\clearpage

\subsubsection{Why exponential inputs?}
Our task for today is to find particular solutions.
We're talking about the second order equation with constant coefficients,
which you can think of as modeling springs or simple electrical circuits.
\begin{equation*}
  y'' + Ay' + By = f(x)
\end{equation*}
The problem is remember that to find a particular solution $y_p$. 
And the reason why we want to do that is then the general solution will be
of the form is $y$ equals that particular solution, plus the complementary solution
the general solution to the reduced equation,
\begin{equation*}
  y = y_p + c_1 y_1 + c_2 y_2. 
\end{equation*}
So all the work depends upon finding out what that $y_p$ is.
And that's what we're going to talk about. \\

But the point is not all functions that you could write on the right-hand side are equally interesting.
There's one kind, which is far more interesting or more important in the applications than all the others.
And that's the one out of which, an arbitrary function can be built out of these simple functions.
So the important function is on the right-hand side to be able to solve it when it's a simple exponential.
But if you allow me to make it a complex exponential, so here are the important right-hand sides we want.
We want to be able to do $f(x)$ when it's of the form $e^{ax}$.
in most applications, a is not a growing exponential, but a decaying exponential.
So typically, $a < 0$, but it doesn't have to be.
That's not any assumption that I'm going to make today. It's just culture.
\begin{equation*}
   f(x) = e^{ax} \text{ or , } \left\{
    \begin{array}{rl}
      \sin \omega x \\
      \cos \omega x
    \end{array} \right. 
\end{equation*}
In other words, when the right-hand side is a pure oscillation.
That's another important type of input both for electrical circuits(alternating current) or the spring systems that's a pure vibration is being. 
you're imposing a pure vibration on the spring mass dashpot system, and you want to see how it responds to that. \\

You can put them together and make these decaying oscillations.
\begin{equation*}
  f(x) =  \left\{
    \begin{array}{rl}
      e^{ax } \sin \omega x \\
      e^{ax} \cos \omega x
    \end{array} \right. 
\end{equation*}

Now, the point is all of these together are really just special cases of one general thing--
exponential if you allow the exponent not to be a real number, but to be a complex number.
So they are all special cases of
\begin{equation*}
  f(x) = e^{(a + i \omega) x}. 
\end{equation*}

If $\omega$ is $0$, then I've got $f(x) = e^{ax}$.
If $a = 0$, then I've got $\sin \omega x$ or $\cos \omega x$ separating it into its real and imaginary parts.
And if neither is $0$, I have a $e^{(a + i \omega) x}$.

But I don't want to keep writing $a + i \omega$ all the time.
So I'm going to write that simply as $e^{\alpha x}$.
And you understand that alpha is a complex number now.

\clearpage
\subsubsection{Shortcut}

\textbf{Warm-up problem: } if $r$ is a number, what is $(2D^2+3D+5) e^{rt}$?\\
\Solution First, $D e^{rt} = r e^{rt}$ and $D^2 e^{rt} = r^2 e^{rt}$ (keep applying the chain rule). Thus
\begin{align*}
  \displaystyle  (2D^2+3D+5) e^{rt}  \displaystyle &= 2 r^2 e^{rt} + 3 r e^{rt} + 5 e^{rt} \\
  \displaystyle &= (2r^2+3r+5) e^{rt}.
\end{align*}

The operator is $P(D) = 2D^2+3D+5$. Notice that $P(r) = 2r^2+3r+5$
is the characteristic polynomial. The same calculation, but with an arbitrary polynomial, proves the following general rule.

\begin{theorem}
  For any polynomial $P$ and any number $r$,
  \begin{equation*}
    {\color{blue}{P(D) e^{rt} = P(r) e^{rt}.}}
  \end{equation*}
\end{theorem}

\paragraph{Setup}
There are just a couple of basic formulas that we're going to use all the time.\\
The first is that if you apply $P(D)$ to a complex exponential or a real one-- it doesn't matter--
the answer is 
\begin{equation*}
  P(D) e^{\alpha x} = P(\alpha) e^{\alpha x}. 
\end{equation*}
The $P(\alpha)$ is now just an ordinary complex number.
So that's a basic formula.
It's called, the substitution rule, because the heart of it is you substitute for the $D$,
you substitute $\alpha$.\\

\begin{proof}
  How would I prove that?
  Well, just calculate it out.
  \begin{align*}
    \left( D^2 + AD + B\right) e^{\alpha x}
    &= D^2 e^{\alpha x} + AD e^{\alpha x} + B  e^{\alpha x} \\
    &= \alpha ^2 e^{\alpha x} + A \alpha e^{\alpha x} + B  e^{\alpha x} \\
  \end{align*}

It's obviously true for the operator $D$ and the operator $D^2$.
In other words,
\begin{equation*}
  D  e^{\alpha x} = \alpha e^{\alpha x}, \qquad D^2  e^{\alpha x} = \alpha^2 e^{\alpha x} 
\end{equation*}
And therefore, it's true for linear combinations of these as well by linearity.
So therefore, also true for $P(D)$.
\end{proof}
\clearpage
\subsubsection{Exponential response formula}
\textbf{Question:} For any polynomial $P$ and any number $r$, what is a particular solution to
\begin{equation*}
  P(D)y = e^{rt}
\end{equation*}

{\color{blue} Answer(superposition):} We have
\begin{equation*}
  P(D) {\color{blue}{e^{rt}}}  = {\color{orange}{P(r) e^{rt}}} .
\end{equation*}

This is off by a factor of $P(r)$. So multiply through by $1/P(r)$ and we can use linearity to get
\begin{equation*}
  P(D)\left(\frac{1}{P(r)}e^{rt} \right) = e^{rt}.
\end{equation*}

\textbf{\color{orange} Conclusion:} This is called the {\color{blue} Exponential Response Formula (ERF).}\\
In words, for any polynomial $P$ and any number $r$ such that $P(r) \neq 0$,
\begin{equation*}
  {\color{blue}{\frac{1}{P(r)} e^{rt}}}  \quad
  \text { is a particular solution to }
  \quad P(D) y = {\color{blue}{e^{rt}}} .
\end{equation*}

Remember, this is just one particular solution.
To get the general solution, we need to add the solution to the associated homogeneous equation.

\begin{example}
  Find the general solution to $\ddot{y} + 7\dot{y} + 12 y = {\color{blue}{-5e^{2t}}}$. 
\end{example}

\Solution\\
\begin{align*}
  \text{Characteristic polynomial:} \qquad
  &P(r) = r^2+7r+12 = (r+3)(r+4)\\
  \text{Roots:} \qquad
  & -3, -4\\
  \text{General solution to \textbf{homogeneous} equation:} \qquad
  & {\color{orange}{y_ h}}  = c_1 e^{-3t} + c_2 e^{-4t}    
\end{align*}

ERF says:
\begin{equation*}
  {\color{blue}{\frac{1}{P(2)} e^{2t}}}
  \quad \text {is a particular solution to }
  \quad P(D) y = {\color{blue}{e^{2t}}} ;
\end{equation*}
i.e.,
\begin{equation*}
  {\color{blue}{\frac{1}{30} e^{2t}}}
  \quad \text { is a particular solution to }
  \quad \ddot{y} + 7 \dot{y} + 12 y = {\color{blue}{e^{2t}}} ,
\end{equation*}
so
\begin{equation*}
  \underset {{\color{blue}{ \text {call this } y_ p}} }{-\frac{1}{6} e^{2t}}
  \quad \textrm{ is a particular solution to }
  \quad \ddot{y} + 7 \dot{y} + 12 y = {\color{blue}{-5e^{2t}}} .
\end{equation*}

General solution to inhomogeneous equation:
\begin{align*}
  \displaystyle y \displaystyle &= {\color{blue}{y_ p}}  + {\color{orange}{y_ h}} \\
  \displaystyle &= -\frac{1}{6} e^{2t} + c_1 e^{-3t} + c_2 e^{-4t}.
\end{align*}
\clearpage
\subsubsection{ERF and complex roots}

\begin{example}
  Use ERF to find a particular solution to $\ddot x + x = e^{2it}$.
\end{example}

\Solution In operator notation, this differential equation can be written as
\begin{equation*}
  (D^2+1)y = e^{2it}.
\end{equation*}
By the ERF, a particular solution is given by
\begin{equation*}
  y_ p = \frac{e^{rt}}{r^2+1} = \frac{e^{2it}}{(2i)^2+1} = \frac{-1}{3}e^{2it}.
\end{equation*}

\begin{exercise}
  Practice problem 1
\end{exercise}
Use ERF to find a particular solution to
\begin{equation*}
  (D^2+1)y = e^{-t}.
\end{equation*}
By the ERF, a particular solution is
\begin{equation*}
  y _p = \frac{e^{-t}}{r^2 + 1} = \frac{e^{-t}}{(-1)^2 + 1} = \frac{e^{-t}}{2} 
\end{equation*}

\begin{exercise}
  Practice problem 2
\end{exercise}
Use ERF and superposition to find a particular solution to
\begin{equation*}
  (D^2+1)y = e^{-t} - 3e^{2it} .
\end{equation*}
By superposition
\begin{eqnarray*}
  (D^2+1)(c_1 y_1 + c_2 y_2) &=& e^{-t} - 3e^{2it} \\
  (D^2+1)(c_1 y_1) + (D^2+1)(c_2 y_2) &=& e^{-t} - 3e^{2it} \\
  c_1 (D^2+1)(y_1) + c_2(D^2+1)(y_2) &=& e^{-t} - 3e^{2it}
\end{eqnarray*}
By ERF, the particular solution of $ c_1 (D^2+1)(y_1) = e^{-t}$ is
\begin{equation*}
  y_{1p} = \frac{e^{-t}}{r^2 + 1} = \frac{e^{-t}}{(-1)^2 + 1} = \frac{e^{-t}}{2}. 
\end{equation*}
By ERF, the particular solution of $ c_2 (D^2+1)(y_2) = -3 e^{-2it}$ is
\begin{equation*}
  -3 y_{2p} = -3 \left(\frac{e^{2it}}{r^2 + 1} \right)
  = -3 \left(\frac{e^{2it}}{(2i)^2 + 1} \right)
  = -3 \left(\frac{-1}{3} e^{2it} \right) = e^{2it}. 
\end{equation*}
The particular solution is
\begin{equation*}
  y _p = y_{1p} + y_{2p} =  \frac{e^{-t}}{2}  + e^{2it}. 
\end{equation*}

\clearpage
\subsubsection{The generalized exponential response formula}

\begin{exercise}
  When the ERF fails
\end{exercise}
For which of the following functions $q\left(t\right)=e^{at}$ does the differential equation
$\ddot{x}+x = q(t)$ \textbf{not} have a solution of the form $Ae^{at}$?

\begin{itemize}
\item $e^t$
\item $e^{-t}$
\item $e^{it}$
\item $e^{-it}$
\item $e^{2it}$
\item $e^{3t}$
\item $1$
\end{itemize}
The characteristic polynomial is $P(r) = r^2+1$.
This polynomial is zero when $r=\pm i$.
Thus we can not apply the Exponential Response Formula to find a solution of the form
$e^{at}$ is when $a=\pm i$.\\

The existence and uniqueness theorem says that
\begin{equation*}
  P(D) y = e^{rt}
\end{equation*}
should have a solution even if $P(r) = 0$(when ERF does not apply). Let's start with the one case:\\
{\color{blue} ERF'} Suppose that $P$ is a polynomial and $P(r _0) = 0$, but
$P'(r _0) \neq 0$ for some number $r_0$. Then
\begin{equation*}
  \boxed {{\color{blue}{x_ p=\frac{1}{P'(r_0)} t e^{r_0t}}}  \
    quad \text { is a particular solution to }\quad P(D) x = {\color{blue}{e^{r_0t}}} .}
\end{equation*}

\begin{proof}
  We want to solve $P(D)y = e^{r_0t}$. We know that
  \begin{equation*}
    P(D)e^{r_0t} = P(r_0)e^{r_0t},
  \end{equation*}
  for all $r$. However, since $P(r_0)$ is zero, we cannot divide by it.\\
  Instead, let us look at what happens near this point $r_0$ and differentiate with respect to $r$.
  \begin{equation*}
    \displaystyle  \displaystyle \frac{\partial }{\partial r}\left(P(D) e^{rt} \right)
    \displaystyle = \displaystyle  \frac{\partial }{\partial r}\left(P(r) e^{rt} \right)
    = P'(r)e^{rt} + P(r)te^{rt}.
  \end{equation*}
  We need to take the derivative of the left hand side. Recall that
  $\frac{\partial }{\partial r}\frac{\partial }{\partial t}
  = \frac{\partial }{\partial t}\frac{\partial }{\partial r}$.
  In this context, this means the same thing as
  \begin{equation*}
    \frac{\partial }{\partial r}D =
    D\frac{\partial }{\partial r} \qquad \left(D = \frac{d}{dt} \right).
  \end{equation*}
  Hence
  \begin{equation*}
    \frac{\partial }{\partial r}P(D) = P(D)\frac{\partial }{\partial r}
  \end{equation*}
  by linearity. \\
  The left hand side becomes.
  \begin{align*}
    \displaystyle  \displaystyle \frac{\partial }{\partial r}\left(P(D) e^{rt} \right)
    \displaystyle &= \displaystyle  P(D) \left( \frac{\partial }{\partial r} e^{rt}\right) \\
    \displaystyle &= \displaystyle  P(D) \left( te^{rt}\right)
  \end{align*}
  Let $r\rightarrow r_0$. Therefore since $P(r _0) = 0$
  \begin{align*}
    \displaystyle  P(D) \left( te^{r_0t}\right)
    \displaystyle &= \displaystyle  P'(r_0)e^{r_0t} + P(r_0)te^{r_0t} \\
    \displaystyle &= \displaystyle  P'(r_0)e^{r_0t}
  \end{align*}
  If $P'(r _0) \neq 0$. then we can divide through by $P'(r _0)$
  \begin{equation*}
    P(D) \left( \frac{te^{r_0t}}{P'(r_0)}\right) = e^{r_0t},
  \end{equation*}
  and $\displaystyle y_ p = \frac{te^{r_0t}}{P'(r_0)}$ is a particular solution to $P(D)y = e^{r_0t}$
\end{proof}

\begin{example}
  Find a particular solution to $\ddot{x} -4x = e^{-2t}$
\end{example}

\Solution The characteristic polynomial is $P(r) = r^2 - 4$, thus $P(-2) = 0$, But
$P'(2) = 2(-2) = -4 \ne 0$. Therefore this is a cae where we can apply ERF'.
which gives us a particular solution
\begin{equation*}
  x_ p = \frac{te^{rt}}{P'(r)} = \frac{te^{-2t}}{-4}.
\end{equation*}

\begin{example}
  Solve the system $\ddot{x} + x = e^{it}$ with initial conditions $x(0)=1$ and $\dot{x}(0) = 0$.
\end{example}
The characteristic polynomial of the differential equation $\ddot{x} + x = e^{it}$ is $P(r) = r^2 + 1$
, with roots $\pm i$.
Since $i$ is a root of the characteristic polynomial, the {\color{blue}Generalized ERF} tells
us to find the smallest integer $s$ for which $P(s)(i) \ne 0$.
In this case, $P'(r) = 2r$, so $P'(i) = 2i$, which is not zero, so $s = 1$.
The Generalized ERF tells us that
\begin{equation*}
  \displaystyle  x_ p\left(t\right)=\frac{1}{P^{\prime }\left(i\right)}t^{1}e^{it}
  =\frac{1}{2i}te^{it}
\end{equation*}
is particular solution to the inhomogeneous ODE.
Since $\frac{1}{i} = e^{-i \pi/2}$ \footnote{
  \begin{align*}
    \frac{1}{i} &= \frac{1}{i} \cdot \frac{i}{i} \\
    \frac{1}{i} &= -i \\
    \frac{1}{i} &= \cos (-\pi / 2) + i \sin (-\pi / 2) \\
    \frac{1}{i} &= e^{i \frac{-\pi}{2}}
                  \quad \text{where } e^{i \theta} = \cos \theta + i \sin theta     
  \end{align*}
}
, we can write $x _P$ as
\begin{equation*}
  \displaystyle  x_ p\left(t\right)=\frac{1}{2}te^{it-i\pi /2}.
\end{equation*}
Since the characteristic polynomial has roots $\pm i$, we know that the pair
\begin{equation*}
  \displaystyle  e^{it},e^{-it}
\end{equation*}
form a basis for the space of solutions to the homogeneous equation
$\ddot{x}+x=0$.
Hence, the general solution to the inhomogeneous equation
$\ddot{x}+x=e^{it}$ is
\begin{equation*}
  \displaystyle  x\left(t\right)
  =\underbrace{c_{1}e^{it}+c_{2}e^{-it}}_{\text {general solution to homogeneousODE}}
  +\underbrace{\frac{1}{2}te^{it-i\pi /2}.}_{{\color{orange}{\text {particular solution}}} }
\end{equation*}
Now we just need to solve for the constants $c_1$ and $c_2$
from the initial conditions initial conditions $x(0)=1$ and $\dot{x}(0) = 0$.
We see that
\begin{equation*}
  \displaystyle  x\left(0\right)=1=c_{1}+c_{2},
\end{equation*}
so $c _2 = 1 - c _1$, and we can write $x(t)$ as
\begin{equation*}
  \displaystyle  x\left(t\right)
  =c_{1}e^{it}+\left(1-c_{1}\right)e^{-it}+\frac{1}{2}te^{it-i\pi /2}.
\end{equation*}
We see that
\begin{equation*}
  \displaystyle  \dot{x}\left(t\right)
  =ic_{1}e^{it}-i\left(1-c_{1}\right)e^{-it}+\frac{1}{2}e^{it-i\pi /2}+\frac{1}{2}te^{it},
\end{equation*}
so
\begin{align*}
  \displaystyle  \dot{x}\left(0\right)
  \displaystyle &=ic_{1}-i\left(1-c_{1}\right)+\frac{1}{2}e^{-i\pi /2}\\
  \displaystyle &=ic_{1}-i\left(1-c_{1}\right)-\frac{i}{2}.
\end{align*}

Using the initial condition $\dot{x}(0) = 0$ and solving for $c_1$ we see that $c_1 = 3/4$.
Substituting this into the equation $c_2 = 1 - c_1$, we see that $c_2 = 1/4$.
Thus the solution to the system $\ddot{x} + x = e^{it}$ with initial conditions $x(0)=1$ and $\dot{x}(0) = 0$
is
\begin{equation*}
  \displaystyle  x\left(t\right)=
  \frac{3}{4}e^{it}+\frac{1}{4}e^{-it}+\frac{1}{2}te^{it-i\pi /2}.
\end{equation*}


\textbf{Generalized exponential response formula.} if $P$ is a polynomial and $r_0$ is a number
such that
\begin{equation*}
  P(r_0) = P'(r_0) = \dotsb = P^{(m-1)}(r_0) = 0 \qquad P^{(m)}(r_0) \neq 0,
\end{equation*}
then
\begin{equation*}
  P(D)\left( t^ me^{r_0t} \right) = P^{(m)}(r_0)e^{r_0t}
\end{equation*}
and
\begin{equation*}
  \boxed {{\color{blue}{y_ p=\frac{1}{P^{(m)}(r_0)} t^ m e^{r_0t}}}
    \quad \text { is a particular solution to }
    \quad P(D) y = {\color{blue}{e^{r_0t}}} .}
\end{equation*}

\begin{proof}
  We want to solve $P(D)y = e^{rt}$. We know that
  \begin{equation*}
    P(D)e^{r_0t} = P(r_0)e^{r_0t},
  \end{equation*}
  for all $r$. However, since $P(r_0)$ is zero, we cannot divide by it.
  Instead, let us look at what happens near this point $r_0$ and differentiate with
  respect $r$.
  \begin{eqnarray*}
    \displaystyle  \displaystyle P(D)e^{rt} \qquad
    \displaystyle &=&
                      \qquad \displaystyle  P(r)e^{rt} \\
    \displaystyle \frac{\partial }{\partial r}\left(P(D)e^{rt}\right) \qquad
    \displaystyle &=&
                      \qquad \displaystyle \frac{\partial }{\partial r}\left( P(r)e^{rt}\right) \\
    \displaystyle P(D)\left(\frac{\partial }{\partial r}e^{rt}\right) \qquad
    \displaystyle &=&
                      \qquad \displaystyle  P'(r)\  e^{rt} + P(r)\  t\  e^{rt}
  \end{eqnarray*}
  In the case that $r_0$ is a repeated root with multiplicity $m$, then $P(r) = Q(r)(r-r_0)^{m}$,
  and
  \begin{equation*}
    \displaystyle  \displaystyle \frac{d^{m}}{dr^ m}P(r_0) \displaystyle  \neq \quad 0,
  \end{equation*}
  however, all lower derivatives are zero at $r$
  \begin{align*}
    \displaystyle  \displaystyle \frac{d^{m-1}}{dr^{m-1}}P(r_0)
    \displaystyle  &=& \displaystyle  0 \\
    \displaystyle \quad \vdots & & \displaystyle  \vdots \\
    \displaystyle P(r_0) \displaystyle  &=& \displaystyle  0.
  \end{align*}
  Differentiate both sides of the equation $m$ times with respect to $r$ to obtain a relationship:
  \begin{equation*}
    \displaystyle  \displaystyle P(D)\left(t^ m e^{rt}\right)
    \displaystyle  =
    \displaystyle  \left(P^{(m)}(r) + m\  t\  P^{(m-1)}(r)
      + \frac{m(m-1)}2 t^2P^{(m-2)}(r) + \dotsb + t^ mP(r)\right)e^{rt}
  \end{equation*}
  Evaluating at $r_0$, only the $m$th derivative term survives
  \begin{equation*}
    \displaystyle  \displaystyle P(D)\left(t^ m e^{r_0t}\right)
    \displaystyle  = \displaystyle  P^{(m)}(r_0)e^{r_0t},
  \end{equation*}
  which gives a particular solution to the differential equation
  $P(D)y = e^{r_0t}$ of the form
  $\displaystyle y_ p= \frac{t^ me^{r_0t}}{P^{(m)}(r_0)}$. 
\end{proof}
\clearpage
\subsubsection{Worked example}
\paragraph{Linear, constant coefficient, inhomogeneous ODEs}
We'll be solving linear ODEs with constant coefficients. \\
\begin{enumerate}
\item $\dot{x} + kx = 1$
\item $\dot{x} + kx = e^{-5t}$
\item $\dot{x} + kx = 4 + 7 e^{-5t}$
\end{enumerate}
The first step in solving these problems is finding the homogeneous solution.
We do this by setting the right hand side equal to 0.
Actually, note that all these three problems have the same exact left hand side.
So the homogeneous solution will be common to all of them.
The characteristic polynomial of $\dot{x} + kx$ is
\begin{equation*}
  P(s) = s + k 
\end{equation*}
We find the solution to the homogeneous problem by setting
\begin{equation*}
  P(s) = s + k = k 
\end{equation*}
That will tell us that $s = -k$.
Therefore, our homogeneous solution,
\begin{equation*}
  x_h = C e^{-kt}
\end{equation*}

For part 1, now we need to find a particular solution
that satisfies the differential equation $\dot{x} + kx = 1$.  
Now, this is an equation that we can actually solve by inspection.
If we assume that $x$ is a constant, then $\dot{x} = 0$.
\begin{align*}
  0 + kx &= 1 \quad \text{ if } x = \text{constant} \\
  x &= \frac{1}{k} 
\end{align*}
We'll denote this particular solution by the subscript $a$.
So $x_a$ is the particular solution to part 1.
Now, the general solution to pat 1 would be given by
\begin{equation*}
  x = x_a + x_h = \frac{1}{k} + C e^{-kt}.
\end{equation*}
We used \textbf{the method of inspection}, and that's not the most systematic way of doing it.
But for now, I just wanted to show you that there are ways to solve ODEs that are very simple just
by looking at them.\\

Now, for part 2 we're looking $\dot{x} + kx = e^{-5t}$.
And now we use the systematic method.
We're going to use the exponential response formula, or ERF.
We can only use the ERF because the right hand side or the forcing of this equation is exponential.
In order to calculate the the particular solution $x_b$,
We take the right hand side $e^{-5t}$ and divide by the characteristic polynomial $P$,
\begin{equation*}
  x_b = \frac{e^{-5t}}{P(-5)} = \frac{e^{-5t}}{-5 + k}  
\end{equation*}
We can write the general solution to part 2 by adding the particular solution
and the homogeneous solution, that is 
\begin{equation*}
  x = C e^{-kt} + \frac{e^{-5t}}{-5 + k}. 
\end{equation*}
Note an important thing here. What happens if $k = 5$?
Then this denominator goes to $0$, and this solution is undefined.
We'll need to look at another method to solve this.
But just one last thing that I want to point out is that the ERF could also have been used here. 
if we think of $1$ as $e^{0t}$.
That's true. That's an exponential.
And if we use the exponential response formula,
we would have gotten the same answer.\\

All right so we just found a solution for part 2.
But we notice that there is a problem when $k = 5$.
So this solution actually
\begin{equation*}
  x = C e^{-kt} + \frac{e^{-5t}}{k + 5} \quad \text{if } k \ne 5.   
\end{equation*}

If $k = 5$, we're actually solving the differential equation
\begin{equation*}
  \dot{x} + 5x = e^{-5t}
\end{equation*}
Now, since we saw that the exponential response formula as we know it didn't work,
we can use the method of variation of parameters.
So for variation of parameters we need a homogeneous solution $C e^{-5t}$.
But for the method of variation of parameters we only need one homogeneous solution.
So we can take $C$ to be $1$.
So now
\begin{equation*}
  x = u(t) e^{-5t}
\end{equation*}
We plug this into our ordinary differential equation, and we get 
\begin{equation*}
  \frac{d}{dt}(u e^{-5t}) + 5ue^{-5t} = e^{-5t}
\end{equation*}
Now, we use the product rule to simplify this term, which
\begin{align*}
  \frac{du}{dt} e^{-5t}n -5ue^{-5t} + 5u e^{-5t} &= e^{-5t} \\
  \frac{du}{dt} &= e^{5t} e^{-5t} \\
  \frac{du}{dt} &= 1 \\
  \int \frac{du}{dt} &= \int 1 \\
  u = t + c
\end{align*}
Now we plug that back into our expression for $x$.
\begin{equation*}
  x = (t+c)e^{-5t} = te^{-5t} + ce^{-5t} \quad \text{where } k = 5 
\end{equation*}
Note that contains a particular solution $te^{-5t}$, as well as the homogeneous solution
$ce^{-5t}$. 
So by using the method of variation of parameters, we got both solutions at the same time.
That's a nice feat of the method a variation of parameters. \\

Now, in part 3, we're asked to solve $\dot{x} + kx = 4 + 7 e^{-5t}$.
We need to note that the right hand side is a superposition of the problems that we solved.
\begin{equation*}
  \dot{x} + kx = 4\cdot (1) + 7 \cdot e^{-5t}.
\end{equation*}
And these were the right hand sides on parts 1 and 2.
So we're solving this problem.
Because this equation is linear, we can superimpose the solutions that we found in part 1 and 2
and get the particular solution for part 3.
So the particular solution for part 3 is recovered by
\begin{equation*}
  x_c = 4 x_a + 7 x_b  
\end{equation*}
Note that, once again, this is the particular solution.
If I want the full solution, I would get the general solution by adding the particular solution plus the homogeneous solution that we found before.
\begin{equation*}
  x = 4 x_a + 7 x_b + c e^{-kt}
\end{equation*}
And this is the most general solution
\clearpage

\subsubsection{Review: Basis of homogeneous solutions with linear operators}
Recall the method for finding a basis of solutions to a constant-coefficient homogeneous linear ODE, now written as $P(D)y=0$. Using operators, we can explain why it works.
\begin{example}
  Consider the ODE
  \begin{equation*}
    y^{\prime \prime \prime } - 10 y^{\prime \prime } + 31y'-30y = 0.
  \end{equation*}
\end{example}
Rewriting this using operator notation we have $P(D)y = 0$, where $P(r) = r^3-10r^2+31r-30$
is the characteristic polynomial of the ODE.
The trick now is to factor the operator, and proceed from there.
The characteristic polynomial factors as $P(r) = (r-2)(r-3)(r-5)$.
The order is $3$, so the dimension of the vector space of solutions is $3$.\\

Now
\begin{align*}
  e^{2t} &\text{  is a solution since } P(D) e^{2t} = p(2) e^{2t} = 0 e^{2t} = 0 \\
  e^{3t} &\text{  is a solution since } P(D) e^{2t} = p(2) e^{3t} = 0 e^{3t} = 0 , \text{ and}\\
  e^{5t} &\text{  is a solution since } P(D) e^{5t} = p(5) e^{5t} = 0 e^{5t} = 0 
\end{align*}

Just because we wrote down $3$ solutions does not mean that they form a basis:
If we had written down $e^{2t}\, , e^{3t}\, , 4e^{2t} + 6e^{3t}$, 
then they would \textbf{not} have been a basis, because they are linearly dependent
(and their span is only $2$-dimensional).\\

To know that $e^{2t}, \, e^{3t} , \, e^{5t}$ really form a basis,
we need to know that they are {\color{orange}linearly independent.}
Could it instead be that
\begin{equation*}
  e^{5t} = c_1 e^{2t}+c_2 e^{3t} \qquad \text{(as functions)}
\end{equation*}
for some numbers $C_1, \, c_2$? \\

One way to see that is no possible is to apply $(D-2)(D-3)$ to both side,
which would give
\begin{align*}
  \displaystyle  \displaystyle (D-2)(D-3)e^{5t}
  \displaystyle &= \displaystyle  (D-2)(D-3)\left( c_1 e^{2t} + c_2 e^{3t} \right) \\
  \displaystyle &= \displaystyle  c_1(D-2)(D-3)e^{2t} + c_2 (D-2)(D-3) e^{3t}
                  \qquad \text {(by linearity)} \\
  \displaystyle &= \displaystyle  c_1\cdot 0 + c_2 \cdot 0 = 0  
\end{align*}

The left hand side gives us
\begin{equation*}
  (D-2)(D-3)e^{5t} = (5-2)(5-3) e^{5t} \neq 0.
\end{equation*}
This contradiction implies that $e^{5t}$ is not a linear combination of $e^{2t}$ and $e^{3t}$.
The same argument shows that no one of $e^{2t}, \, e^{3t} , \, e^{5t}$ is a linear combination
of the other tow. Since $e^{2t}, \, e^{3t} , \, e^{5t}$ are linearly independent,
they form a basis of a $3$-dimensional space (which must be the space of all solutions,
since that too is $3$-dimensional). 
\clearpage

\subsubsection{Review: repeated roots with linear operators}

What about the case of repeated roots?\\

This case is less important than the others,
because if someone hands you a polynomial it's unlikely that two or more of its roots will coincide.
But it does happen, and it is interesting to think about what happens.\\

\begin{example}
  Find a basis of solutions to $D^3 y=0$. (
  The characteristic polynomial is $r^3$, whose roots with multiplicity are
  $0,\, 0,\, 0$.) 
\end{example}

\Solution Integrate three times:
\begin{align*}
  D^2 y &= c_1\\
  Dy &= c_1 t + c_2 \\
  y &= c_1 \frac{t^2}{2} + c_2 t + c_3 \\
  \displaystyle &= \underset {{\color{orange}{\textrm{general solution}}} }{C_1 t^2 + c_2 t + c_3}.          
\end{align*}
for some numbers $C_1=c_2 / 2, c_2,$ and $c_3$.
Since $t^2,\, t,\, 1$ are linearly independent, they form a basis for the space of solution.
\begin{example}
  Find a basis of solutions to $(D - 5)^3 y = 0$. 
\end{example}

\Solution \\
The characteristic polynomial is $(r-5)^3$, which a repeated root $5,\, 5,\, 5$,
so based on the previous example, we might hope that $e^{5t},\, te^{5t},\, t^2 e^{5t},\,$
are basis.
Let's now explain \textbf{why} these are solutions. \\
We know that $D - 5$ sends $e^{5t}$ to $0$.
What does $D - 5$ do to $ue^{5t}$, if $u$ is a function of $t$?
Using product rule:
\begin{align*}
  \displaystyle  (D-5) \,  u e^{5t}
  \displaystyle &= \left( \dot{u}\cdot e^{5t} + u \cdot 5 e^{5t} \right) - 5 u \cdot e^{5t} \\
  \displaystyle &= \dot{u} \cdot e^{5t}.
\end{align*}
Replacing $(D-5)$ by $(D-5)^2$ and $(D-5)^3$ on the left we get the following:
\begin{align*}
  \displaystyle  (D-5)^2 \,  u e^{5t}
  \displaystyle = \ddot{u} e^{5t}\\
  \displaystyle (D-5)^3 \,  u e^{5t}
  \displaystyle = u^{(3)} e^{5t}.
\end{align*}
So our guess was right! In order for $ue^{5t}$ to be a solution to
$(D-5)^3y = 0$ the function$u^{(3)}$ must be 0:i.e,
$u = a + bt + ct^2$ for some numbers $a,\, b,\, c,$ so the solutions are
\begin{equation*}
  \displaystyle u e^{5t} = a \,  e^{5t} + b \,  t e^{5t} + c \,  t^2 e^{5t},
\end{equation*}
are expected. \\

Since $1,\, t,\, t^2$ are linearly independent functions, so are $e^{5t},\, te^{5t},\, t^2 e^{5t}$
(any relation between the last there function could by divided by $e^{5t}$ to get a relation
between the first three). Thus $e^{5t},\, te^{5t},\, t^2 e^{5t}$ form a basis. \\

If a characteristic polynomial $P(r)$ has a root $r$ is repeated $k$ times, then
$e^{rt},\, te^{rt},\, t^2 e^{rt},\, \cdots ,\, t^{k-1} e^{rt}$ are independent solutions
to the differential equation $P(D)=0$. 
\clearpage

\subsubsection{Worked Examples}
\begin{example}
  Find the general solution to the differential equation
  \begin{equation*}
    2\ddot{x}+\dot{x}+x=1+2e^{t}.
  \end{equation*}
\end{example}

This is an inhomogeneous linear equation, so the general solution has the form
$x_p + x_h$, where $x_p$ is any particular solution and $x_h$ is the
general homogeneous solution.
The characteristic polynomial is $P(s) = 2s^2 + s + 1$, with roots$(-1 \pm \sqrt{7})/4$,
So and the general homogeneous solution is given by
\begin{align*}
  \displaystyle x_h(t)
  \displaystyle &=e^{-t/4}\left(a_1\cos \left( \frac{\sqrt {7} t}{4}\right)
                  +c_2\sin \left( \frac{\sqrt {7}t}{4}\right)\right) \\
  \displaystyle &=Ae^{-t/4}\cos \left(\frac{\sqrt {7}t}{4}-\phi \right)
                  \quad \text{ where } A = \sqrt{a_1^2 + c_2^2}
                  \quad, \phi = \arctan(c_2 / a_1)                    
\end{align*}
The inhomogeneous equation is $P(D)x = 1 + 2e^{t}$.
The input signal is a linear combination of $1$ and $e^t$,
so we can find particular solution to $P(D)x =1$ and $P(D)x = e^t$ separately,
then use superposition to construct a particular solution is $P(D) = 1 + 2e^t$.
if $x_1$ is a particular solution to $P(D)x = 1$, and $x_2$ is a particular solution
to $P(D)x = e^t$, then superposition says that a particular solution to
$P(D)x = 1 + 2e^t$ is given by
\begin{equation*}
  x_p = x_1 + 2x_2
\end{equation*}
The construct function $1$ is an exponential: $1=e^{0t}$. Thus $P(D)x = 1$ has a particular solution
\begin{equation*}
  x_1 = \frac{1}{P(0)} = 1
\end{equation*}
(More elementarily, we've learned to look for constant solutions when the right hand side is constant, and solving $P(D)c = 1$ gives $c = 1$.)\\
Similarly, the ERF tells us that
\begin{equation*}
  x_2 = \frac{1}{P(1)} e^t = \frac{1}{4} e^t  
\end{equation*}
is a particular solution to $P(D)x = e^t$. Thus
\begin{align*}
  x_p &= 1 + 2 \left( \frac{1}{4} e^t\right) \\
      &= 1 + \frac{1}{2} e^t
\end{align*}
is a particular solution to $P(D)x = 1 + 2e^t$. So the general solution
$2\ddot{x}+\dot{x}+x=1+2e^{t}$ is
\begin{equation*}
  \displaystyle  x\left(t\right)=
  1+\frac{1}{2}e^ t+Ae^{-t/4}\cos \left(\frac{t\sqrt {7}}{4}-\phi \right).
\end{equation*}

\begin{example}
  Find a particular solution to the equation
  \begin{equation*}
    \displaystyle  \ddot{x} + 8\dot{x} + 15 x = e^{-5t}
  \end{equation*}
\end{example}

The characteristic polynomial is $P(r) = r^2 + 8r + 15$. Since $P(-5) = 0$
we need to use the generalized ERF.\\

Computing we see that $P \prime (r) = 2r + 8$, and $P \prime (-5) = -2$.
Therefore the generalized ERF gives a particular solution of
\begin{equation*}
  x_p = \frac{te^{-5t}}{P \prime (-5)} = - \frac{te^{-5t}}{2}.
\end{equation*}

(Of course, there are other particular solution.)

\begin{exercise}
  ERF Concept Check
\end{exercise}
For which values $a$ does not the differential equation
\begin{equation*}
  \displaystyle  \frac{d^5 y}{dt^5} - y = e^{at}
\end{equation*}
\textbf{not} have a solution of $ce^{at}$ for some number $c$? 


The characteristic polynomial of the differential equation
\begin{equation*}
  \displaystyle  \frac{d^5 y}{dt^5} - y = e^{at}
\end{equation*}
is $r^5 -1$, which has roots are $1,\, e^{2 \pi i / 5},\, e^{4 \pi i / 5},\, , e^{(n-1) 2 \pi i / 5}$.
When $a$ is one of these values, substituting $y = ce^{at}$ into left side of the differential
equation gives $0$, not $e^{at}$.
Thus the differential equation does \textbf{not} have a solution of the form $ce^{at}$ when
$a=1$ or $a = e^{(n-1)2 \pi/5}$. 

\clearpage
\subsection{Recitation}

\subsubsection{Operator Notation}

\begin{problem}
  Operations on Operators
\end{problem}
Express $(D^3 -D)(x+y)$ as an operators acting on $x$ plus and operator action on y.
(if $(D^3 -D)(x+y) = Ax + By$, what do $A$ and $B$ equal in terms of $D$? )\\

By product rule ,
\begin{equation*}
  (D^3 -D)(x+y) = (D^3 -D)x + (D^3 -D)y  
\end{equation*}

\begin{problem}
  Superposition
\end{problem}

Let $x_1$ and $x_2$ be two solution to $(D^3 - D)x = 0$. Which of the following are also
solution to the same equation? Choose all that apply.
\begin{itemize}
\item $x_1 - x_2$
\item $x_1 x_2$
\item $ax_1$ for any constant $a$
\item $x_2 + 1$
\item $ax_1 + bx_2$ for any two constant $a$ and $b$.
\item $x_1 + x_2 + x_1 x_2$
\end{itemize}
By superposition, the solution is form of
\begin{equation*}
  c_1 x_1 + c_2 x_2 \qquad \text{ where } c_1,\, c_2 \text{ is any number.}, 
\end{equation*}
and the right side is equation is $0$, so both $x_1$ and $x_2$ can be constant function.\\

The answer is
\begin{align*}
  &x_1 - x_2\\
  &ax_1 \quad \text{ for any constant } a \\
  &x_2 + 1 \\
  &ax_1 + bx_2 \quad \text{for any two constant } a\, \text{and} \,b.
\end{align*}

\begin{problem}
  Homogeneous solutions
\end{problem}
Which of the following are solutions to the homogeneous equation
$(D^3 - D)x = 0$?\\

The characteristic polynomial is
\begin{equation*}
  r^3 - r = 0
\end{equation*}
The root of $r$ is $0,\, -1,\, 1$, so the homogeneous solutions for $(D^3 - D)x$ is
\begin{equation*}
  c_1 e^{0t},\, c_2e^{t},\, \text{ and } c_3 e^{-t}.   
\end{equation*}

\begin{problem}
  Particular solutions
\end{problem}

\begin{enumerate}
\item Use ERF (or generalized ERF) to find a particular solution to $(D^3 - D)x = e^{2t}$. \\
  The particular solution using ERF is
  \begin{equation*}
    x_p = \frac{e^{rt}}{P(r)} = \frac{e^{2t}}{2^3 - 2} = \frac{1}{6} e^{2t} 
  \end{equation*}
\item Use ERF (or generalized ERF) to find a particular solution to $(D^3 - D)x = e^{-t}$. \\
  The $P(r) = 0$, so suing generalized ERF is
  \begin{equation*}
    x_p = \frac{te^{rt}}{P^{\prime}(r)} = \frac{te^{-t}}{3 - 1} = \frac{1}{2} te^{-t}. 
  \end{equation*}
\item Use ERF (or generalized ERF) to find a particular solution to $(D^3 - D)x = 1$. \\
  The $(D^3 - D)x = 1 = e^{0t}$, so $r=0$.
  Using generalized ERF, $P^{ \prime} (r) = 3 r^2 - 1$. So
  The particular solution is
  \begin{equation*}
    x_p = \frac{te^{rt}}{P^{\prime} (r)} = \frac{te^{0t}}{0 - 1} = - t. 
  \end{equation*}
\end{enumerate}
\clearpage
\subsubsection{ERF}

Use ERF to find a particular solution of $\ddot{x} + 9x = e^{3t} + 9$.\\

Let particular solution of $\ddot{x} + 9x = e^{3t}$ as $x_a$ and
$\ddot{x} + 9x = 1$ as $x_b$. \\
By superposition,  the particular solution of  a particular solution of
$\ddot{x} + 9x = e^{3t} + 9$ is
\begin{equation*}
  x_p = x_a + 9 \cdot x_b
\end{equation*}
The characteristic polynomial is $r^2 + 9$, so the particular of
$x_a$ is
\begin{equation*}
  \frac{e^{rt}}{r^2 + 9} = \frac{e^{3t}}{3^2 + 9} = \frac{e^{3t}}{18}.  
\end{equation*}
The particular solution $x_b$ is
\begin{equation*}
  \frac{e^{rt}}{r^2 + 9} = \frac{e^{0t}}{0^2 + 9} = \frac{1}{9}. 
\end{equation*}
So,
\begin{equation*}
  x_p = x_a + 9 \cdot x_b = \frac{e^{3t}}{18} + 9 \cdot \frac{1}{9} =  1 + \frac{e^{3t}}{18}. 
\end{equation*}

\begin{problem}
  Higher order inhomogeneous
\end{problem}
Use ERF to find a particular solution of $\displaystyle {\frac{d^{4}x}{dt^{4}}-x=e^{-2t}}$. \\

The characteristic polynomial is $r^4 - 1$.
The particular solution using ERF is
\begin{equation*}
  \frac{e^{rt}}{P(r)} = \frac{e^{-2t}}{(-2)^4 -1} = \frac{e^{-2t}}{15}.  
\end{equation*}
\clearpage

\subsubsection{Using Operators, a Review}
\begin{problem}
  Review damped oscillators
\end{problem}
Consider the damped harmonic oscillator corresponding to the differential operator
$10D^2 + 2D + 5I$. \\
\begin{itemize}
\item If this were an undamped oscillator (damping term put to zero), find the natural angular frequency
  $\omega _n$\\
  
  The characteristic polynomial for undamped oscillator is $10r^2 + 5$.
  The roots of polynomial is $\pm i \frac{\sqrt{2}}{2}$.
  The homogeneous solution is
  \begin{equation*}
    e^{i \frac{\sqrt{2}}{2} t},\, e^{i \frac{\sqrt{2}}{2} t}
  \end{equation*}
  By Eular's Formular,
  \begin{equation*}
    e^{i \frac{\sqrt{2}}{2} t} = \cos(\frac{\sqrt{2}}{2} t) + i \sin (\frac{\sqrt{2}}{2} t). 
  \end{equation*}
  So the natural angular frequency $\omega _n $ is $\frac{\sqrt{2}}{2}$.

\item For the damped oscillator, what is the rate $a$ of exponential decay? \\
  
  The characteristic polynomial for undamped oscillator is $10r^2 + 2r + 5$.
  The roots of polynomial is $-\frac{1}{10} \pm i \frac{7}{10}$
  The homogeneous solution is
  \begin{equation*}
    \displaystyle e^{(-\frac{1}{10} + i \frac{7}{10})t},\, e^{(-\frac{1}{10} - i \frac{7}{10})t}
  \end{equation*}
  \begin{equation*}
    \displaystyle e^{(-\frac{1}{10} + i \frac{7}{10})t}
    \displaystyle = e^{(-\frac{1}{10})t} e^{(i \frac{7}{10})t}
  \end{equation*}
  So, the exponential decay $a$ is $-\frac{1}{10}$.

\item For the damped oscillator, find the angular frequency $\omega _d$?\\
  
  The angular frequency $\omega _2$ is $\frac{7}{10}$. 
\end{itemize}
\clearpage

\clearpage
\subsection{Complex Replacement, Gain and Phase Lag, Stability}

\subsubsection{Complex Replacement}
\textbf{\color{blue}Objectives}
\begin{itemize}
\item Use {\color{blue}complex replacement} to solve any inhomogeneous LTI
  system with {\color{blue} sinusoidal input}.
\item Find the {\color{blue}complex gain} of an LIT system in terms of
  the complex system response, and the complexified system input.
\item Describe the {\color{blue}phase shift} and {\color{blue} amplitude gain} of
  any LTI system with sinusoidal input signal in terms of the
  {\color{blue}complex gain}.
\item Describe {\color{blue}conditions for stability} in physical systems,
  and distinguish between {\color{blue}long term (steady state)} and
  {\color{blue}transient} behavior in a {\color{blue}stable} system. 
\end{itemize}
\clearpage
\subsubsection{Boston Harbor Example}

Suppose we are studying the tides in Boston Harbor.
Let $x$ the water level in Boston Harbor.
Let $y$ be the water level of the ocean.
Then the input is the ocean level $y$, which is responsible for the changing tides $x$ in Boston Harbor,
the system response.\\

\textbf{\color{blue} The Physics} \\

We assume that the ocean and the harbor are connected by a narrow channel
so that the flow is slow and not turbulent.
This allows us to assume that the flow rate is pressure driven, and is linearly proportional to the pressure difference.
Furthermore the pressure difference is linearly proportional to the difference in water level between the ocean and the harbor.

\begin{exercise}
  Modeling sinusoidal tides
\end{exercise}

Represent this system as ODEs.\\

The change in water level in the harbor is proportional to the difference in water level between the ocean
and the harbor, therefore
\begin{equation*}
  \dot{x} = k(y - x)
\end{equation*}

which is the same as
\begin{equation*}
  \dot{x} + kx = ky 
\end{equation*}

We wrote $k(y−x)$ not $k(x−y)$ because we make a habit of choosing positive parameters where possible.
We can see that $k$ must be positive because
if the water level in the ocean is higher than the water level in the harbor,
then the water level in the harbor will increase.
Similarly, if the water level in the ocean lower than the water level in the harbor,
the water level in the harbor will decrease.
Therefore the proportionality constant $k$ must be positive.
Notice that our equation is the same equation as Newton Cooling.
\clearpage
\subsubsection{Complex replacement method}

\paragraph{Introduction to complex replacement}
So I'm going to take the equation in the form
\begin{equation*}
  \dot{y} + ky = kq _e (t) \quad \text{ where} q _e(t) = \cos \omega t. 
\end{equation*}
And the input that I'm interested in is when this
is a simple one that you used on the visual that you did,
got two points worth of work for handing in today, $\cos \omega t$,
which is the physical input. \\

And $\omega$ is the angular frequency.
\begin{align*}
  \omega &= \quad \text{Angular Frequency} \\
         &= \quad \text{Number of Complete Oscillations in } 2 \pi
\end{align*}
In other words, it's the number of complete oscillations.
This $\cos \omega t$ is going up and down.
So a complete oscillation is it goes down and then returns to where it started.
So the number of complete oscillations in how much time, well, in the distance $2 \pi$
on the $t$-axis in the interval of length $2 \pi$.
For example, if $\omega = 1$, $\cos t$ takes $2 \pi$ to repeat itself.
If omega were 2, it would make two complete oscillations in the interval $2 \pi$.
So it's what happens in the interval $2 \pi$, not what happens in the time interval $1$ second,
which is the natural meaning of the word frequency.
There's always this factor of $2 \pi$ that floats around to make all your formulas
and solutions incorrect.\\

Now, what I'm out to do is, the problem is for the physical input $q_e =  \cos \omega t$,
find the response. In other words, solve the differential equation.
So we got to solve the differential equation using complex number. 
I'm going to \textbf{complexify}.
To use complex numbers, what you do is complexification of the problem.
So I'm going to complexify the problem, turn it into the domain of complex numbers.
So take the differential equation, turn it into a differential equation involving
complex numbers, solve that, and then go back to the real domain to get the answer.\\

In other words, I didn't give you the real answer.
The real answer, why one would do it, is because it's easier.
And what makes it easier is the fact that exponentials are very easy to integrate.
There's hardly anything easier to integrate than an exponential, well they're polynomials,
but unfortunately, polynomials occurred distressingly little in real life.
Since it's easier to integrate exponential,and therefore, try to change the trigonometric functions
into complex exponentials simply because the work will be easier to do.\\

So let's do it.
To change this differential equation $\dot{y} + ky = k \cos \omega t$, got $\cos \omega t$, 
I am  going to use the fact that $e^{i \omega}$, Euler's formula, that the real part of it
is $\cos \omega t$.
So I'm going to view this as the real part of this complex function, but I'll
throw in the imaginary part too since at one point we'll need it.\\

Now, what is the equation then it's going to turn into?
The complexified equation is going
\begin{equation*}
  \dot{y} + ky = k e^{iwt}
\end{equation*}
Now, I have a problem because $y$, in this equation, $y$ means the real function which solves that problem.
I, therefore, cannot continue to call this $y$ because I want $y$ to be a real function.
I have to change its name Since this is a complex function on the right side of the equation,
I'll have to expect a complex solution to the differential equation.
I'm going to call that complex solution $\tilde y$.
\begin{equation*}
  \tilde y ^{\prime} + k \tilde y = k e^{i \omega t}
\end{equation*}
Now, then that's what I would also use as the designation for the variable.
So $\tilde y$  is the complex solution, and it's going to have the form
\begin{equation*}
  \tilde y = y_1 + i y_2 \qquad \text{Complex Solution}
\end{equation*}

Find this complex solution, so the program is to find $\tilde y$.
That's the complex solution. And then I say all you have to do is take the real part of that, and that will
answer the original problem.
Then $y1$, will solve the original problem, the original real ODE.\\

\textbf{\color{blue}Complex replacement} is a method for finding \textbf{a particular solution} to
an inhomogeneous ODE
\begin{equation*}
  P(D) x = {\color{orange}{\cos \omega t}} ,
\end{equation*}
where $P$ is a real polynomial, and $\omega$ is real number.
\begin{enumerate}
\item Write the right hand side of the equation ${\color{orange}{\cos \omega t}}$
  as $\mathrm{Re\, }\left(e^{i\omega t} \right)$
  \begin{equation*}
    P(D)x = {\color{orange}{\mathrm{Re\, }\left( e^{i\omega t}\right)}} .
  \end{equation*}
\item Replace the right hand side of the differential equation with the complex exponential
  $e^{i \omega t}$.
  We need a new variable for the solution, which will be a complex function. Give the name
  $z$ for the unknown complex function. This \textbf{\color{blue} complexified}
  differential equation is this:
  \begin{equation*}
    P(D) {\color{blue}{z}}  =
    \underset {{\color{blue}{\text {complex replacement}}} }{{\color{blue}{e^{i\omega t}}} .}
  \end{equation*}
\item Use ERF(or generalized ERF if $P(i \omega) = 0$) to find a particular solution ${\color{blue}{z_ p}}$
  to the complexified ODE.
\item Computer $x_ p = \mathrm{Re\, }({\color{blue}{z_ p}} )$. Then $x _p$ is a particular solution
  to the \textbf{original} ODE. 
\end{enumerate}

\begin{exercise}
  Complexification concept check
\end{exercise}
The ocean tides are periodic with angular frequency $\omega$.
The input can be modeled as the sinusoidal function
$y = A \cos (\omega t)$.\\
What is the complexified ODE for the tide problem above?\\

Because $A \cos \omega t = \mathrm{Re\, }\left(Ae^{i\omega t}\right)$,
the complexified differential equation is
\begin{equation*}
  \dot z + kz = kAe^{i\omega t}.
\end{equation*}
\clearpage
\subsubsection{Why does complex replacement work?}

\begin{question}
  Why does the complex replacement method work?
\end{question}

If $z = x _1 + ix _2$ is solution to complex replacement ODE, i.e.,
\begin{eqnarray*}
  P(D)z &=& e^{i \omega t} \\
  P(D)(x _1 + i x _2) &=& \cos \omega t + i \sin \omega t  
\end{eqnarray*}
Since $P$ has real coefficients, taking the real parts of both side gives
\begin{equation*}
  P(D)x _1 = cos \omega t
\end{equation*}
which says that $x_ 1$ is a solution to the original ODE.
Note that for free we have actually solve another equation.
Taking the imaginary parts of both gives us
\begin{equation*}
  P(D)x _2 = \sin \omega t
\end{equation*}

Complex replacement is helpful also with other real input signals,
with any real-valued function that can be written
as the real part of a reasonably simple complex input signal.
Here are some simple examples that would be helpful to have memorized:
\begin{align*}
  &\text{\color{blue} Real input signal}  \qquad  &\text{\color{blue}Complex replacement} \\
  &\cos \omega t \qquad &e^{i \omega t} \\
  &A \cos(\omega t - \phi) \qquad &A e^{i(\omega t - \phi)} \\
  &e^{at} \cos \omega t \qquad &e^{(a + i \omega t)}
\end{align*}

Using complex arithmetic, there is a more complicated formula which can be derived as well 
\begin{align*}
  &\text{\color{blue} Real input signal}  \qquad  &\text{\color{blue}Complex replacement} \\
  &a \cos \omega t + b \sin \omega t \qquad &(a -bi)e^{i \omega t}
\end{align*}

Each function in the first column is the real part of the corresponding function in the second column. The nice thing about these examples is that the complex replacement is a constant times a complex exponential, so ERF (or generalized ERF) applies.
\clearpage
\subsubsection{Worked examples using complex replacement}

\begin{example}
  Find a solution to $\ddot{x} + 4x = \cos (2t)$
\end{example}
\Solution
\begin{enumerate}
\item Replace $cos (2t)$ by $\mathrm{Re\, }e^{2it}$, and complexify the ODE:
  \begin{equation*}
    \ddot{z} + 4z = e^{2it}. 
  \end{equation*}
\item Find the characteristic polynomial and apply ERF: $P(r) = r^4 +4$.
  We find that $P(2i) = 0$, so we must use the $ERF^{\prime}:\, P^{\prime} (r) = 2r$.
  The particular solution is
  \begin{equation*}
    \displaystyle z_ p
    \displaystyle = \frac{te^{2it}}{P^{\prime} (2i)} = \frac{te^{2it}}{4i}. 
  \end{equation*}
\item Take the real part to find the solution to the original equation
  \begin{equation*}
    \displaystyle x_ p \displaystyle = \mathrm{Re\, } (z_ p)
    \displaystyle = \frac{t\left( \cos (2t) + i \sin (2t)\right)}{4i}
    \displaystyle = \frac{t}{4} \sin (2t)
  \end{equation*}
\end{enumerate}

\begin{example}
  Find a particular solution $x_ p$ to 
\end{example}
\begin{equation*}
  \ddot{x} + \dot{x} + 2x = {\color{orange}{\cos (2t)}} .
\end{equation*}

\Solution
\begin{enumerate}
\item Since ${\color{orange}{\cos (2t)}}$ is the real part of ${\color{orange}{e^{2it}}}$, 
  replace ${\color{orange}{\cos (2t)}}$ by ${\color{orange}{e^{2it}}}$.
  \begin{equation*}
    {\color{blue}{\ddot{z}}}  + {\color{blue}{\dot{z}}}  + 2{\color{blue}{z}}
    = {\color{orange}{e^{2it}}} .
  \end{equation*}
\item Find the characteristic polynomial $P(r) = r^2 + r + 2$ and apply ERF,
  which says that one particular solution to this new ODE is
  \begin{equation*}
    \displaystyle z_ p
    \displaystyle = \frac{1}{P(2i)} e^{2it}
    \displaystyle = \frac{1}{-2 + 2i} e^{2it}. 
  \end{equation*}
\item A particular solution to the original ODE is
  \begin{equation*}
    {\color{orange}{x_ p = \mathrm{Re\, }(z_ p)
        = \mathrm{Re\, }\left( \frac{1}{-2+2i} e^{2it} \right).}}
  \end{equation*}
  This is a sinusoid expressed in complex form. \\
  It might be more useful to have answer in amplitude-phase form or
  as a linear combination of cosine and sine. \\

  \textbf{\color{blue} Converting to amplitude-phase form.} We have
  \begin{equation*}
    \displaystyle z_ p
    \displaystyle = \frac{1}{-2 + 2i} e^{2it}. 
  \end{equation*}
  To get the amplitude-phase form, we convert numerator and denominator to
  polar form so the division is easier. The denominator $-2 + 2i$ has absolute value
  $2 \sqrt{2}$ and angle $3 \pi / 4$, so in polar form
  \begin{equation*}
    \displaystyle -2 + 2i
    \displaystyle = 2 \sqrt{2} e^{i (3 \pi / 4)}. z_ p
    \displaystyle = \frac{e^{2it}}{2 \sqrt{2} e^{i (3 \pi / 4)}}
    \displaystyle = \frac{1}{2 \sqrt{2}} e^{i(2t - 3 \pi / 4)}
  \end{equation*}
\end{enumerate}
\Conclusion In amplitude-phase form
\begin{equation*}
  \boxed {x_ p = \frac{1}{2\sqrt {2}} \cos (2t - 3\pi /4). }
\end{equation*}

\textbf{\color{blue} Converting to a linear combination of $\cos$ and $\sin$:}
We have
\begin{equation*}
  \displaystyle z_ p
  \displaystyle = \frac{1}{-2 + 2i} e^{2it} 
\end{equation*}
To get the linear combination, we express numerator and denominator in rectangular form and then rationalize the denominator. We have

\begin{align*}
  \displaystyle e^{2it} \displaystyle &= \displaystyle  \cos 2t + i \sin 2t \\
  \displaystyle \frac{\cos 2t + i \sin 2t}{-2+2i}
  \displaystyle &= \displaystyle  \frac{\cos 2t + i \sin 2t}{-2+2i} \left(\frac{-2-2i}{-2-2i}\right) \\
  \displaystyle &= \displaystyle (\cos 2t + i \sin 2t) \frac{-2-2i}{8}
                  =(\cos 2t + i \sin 2t) \frac{-1-i}{4} \\
  \displaystyle &= \displaystyle  \frac{(-\cos 2t +\sin 2t) - i(\cos 2t +\sin 2t)}{4}.                  
\end{align*}

\Conclusion
\begin{equation*}
  \boxed {x_ p = -\frac{1}{4} \cos (2t) + \frac{1}{4} \sin (2t)}
\end{equation*}
\clearpage

\subsubsection{Damped sinusoidal inputs}
We can use \textbf{complex replacement} to solve any ODE of the form
\begin{equation*}
  P(D)x = e^{at}\cos (\omega t -\phi ).
\end{equation*}
Let us see how by working through an example.

\begin{example}
  Find a solution to $\ddot{x} + 2x = e^{-t} \cos (3t - \phi)$, where $\phi$ is a real number. 
\end{example}

\Solution
\begin{enumerate}
\item Replace the right hand side with a complex exponential:
  \begin{equation*}
    \displaystyle e^{-t}e^{i3t -i\phi}
    \displaystyle = e^{-i\phi}e^{(-1 + 3i)t}.
  \end{equation*}
  The complexified equation is
  \begin{equation*}
    \ddot{z} + 2z = e^{-i\phi}e^{(-1 + 3i)t}.
  \end{equation*}
\item Apply ERF. The characteristic polynomial is
  \begin{align*}
    P(r) &= r^2 + 2 \\
    P(-1 + 3i) &= (-1 + 3i)^2 + 2 = -6 -6i.     
  \end{align*}
  and complex solution is
  \begin{equation*}
    \displaystyle z_ p
    \displaystyle = \frac{e^{-i\phi}e^{(-1 + 3i)t}}{-6 -6i}, 
  \end{equation*}
  A solution to the original problem is then $X_ p = \mathrm{Re\, } (z_ p)$.
  The solution involves a sinusoid. Polar form is the most convenient for
  this problem, so we put the denominator in polar form:
  \begin{equation*}
    -6 -6i = 6 \sqrt{2} e^{-i 3 \pi / 4}. 
  \end{equation*}
  Then
  \begin{equation*}
    \displaystyle z_ p
    \displaystyle = \frac{e^{-i\phi}e^{(-1 + 3i)t}}{-6 -6i}
    \displaystyle = \frac{1}{6 \sqrt{2}} e^{-i\phi + i 3 \pi / 4}e^{(-1 + 3i)t} 
  \end{equation*}
  and
  \begin{equation*}
    \displaystyle x_ p
    \displaystyle = \mathrm{Re\, } (z_ p)
    \displaystyle = \frac{1}{6 \sqrt{2}} e^{-t} \cos (3t - \phi + 3 \pi / 4)
  \end{equation*}
\end{enumerate}

\paragraph{Complex replacement of sine}

\begin{equation*}
  y^{\prime \prime} - y^{\prime} + 2y = 10e^{-x} \sin x 
\end{equation*}

So the input of this function is $10e^{-x} \sin x $. 
It's a decaying oscillation.
A decaying exponential, and I want to find a particular solution.
Well, let's find a particular. \\

So we want a particular solution, and our equation is
\begin{equation*}
  (D^2 -D +2)y = 10e^{-x} \sin x
\end{equation*}
Now, let's complexify it to make this right side of
\begin{equation*}
  10e^{-x} \sin x = 10 e^{(-1 + i)x}
\end{equation*}
What is $e^{(-1 + i)x}$? This is the \textbf{imaginary part} of this complex exponential.
So this is imaginary part of $e^{-x} \sin x$. \\

Since this is a complex equation, I shouldn't call this $y$ anymore.
By my notation, I like to call it $\tilde y$.

\begin{equation*}
  (D^2 -D +2) \tilde y = 10 e^{(-1 + i)x}.
\end{equation*}

The $\tilde y$ to indicate that the solution we get to this is not going to be the solution to the original problem, but you'll have to take the imaginary part of it to get it.\\
So we're looking now for the complex solution to this complexified equation.
Well, the complex particular solution I can write down immediately.
\begin{align*}
  \displaystyle \tilde y_ p
  \displaystyle &= \frac{10 e^{(-1 + i)x}}{(-1 + i)^2 - (-1 + i) + 2} \\
  \displaystyle &= \frac{10 e^{(-1 + i)x}}{3 -3i} \\
  \displaystyle &= \frac{10}{3} \frac{(1 + i)}{(1 - i) \cdot (1 + i)} e^{(-1 + i)x} \\
  \displaystyle &= \frac{10}{3} \frac{(1 + i)}{2} e^{(-1 + i)x} \\
  \displaystyle &= \frac{10}{3} \frac{(1 + i)}{2} e^{-x} \left( \cos (x) + i \sin (x) \right) \\
  \displaystyle &= \frac{10}{3} \frac{(1 + i)}{2} e^{-x} \left( \cos (x) + i \sin (x) \right) \\
  \displaystyle &= \frac{5}{3}  e^{-x} (1 + i) \left( \cos (x) + i \sin (x) \right) 
\end{align*}

So we're practically at our solution.
The solution then finally is going to be $y_p $ is the imaginary part of $\tilde y_ p$, and what's that?
\begin{equation*}
  y_ p = \mathrm{Im\, } \left( \tilde y_ p \right)
  = \frac{5}{3}  e^{-x} \left(\cos (x) + \sin (x) \right)   
\end{equation*}
In amplitude-phase form
\begin{equation*}
  y_ p = \frac{5}{3} e^{-x} \sqrt{2} \cos (x - \pi / 4)
\end{equation*}

\clearpage

\subsubsection{Complex gain}

Our goal is to explain how the amplitude and phase lag depend on system parameters and the input frequency.
To do so, we will use our method of complex replacement and introduce the complex gain.\\

Let's state the general picture here for reference.
We have an LTI system modeled by the differential equation
\begin{equation*}
  P(D)x = Q(D)y 
\end{equation*}

with input signal $y$ and system response $x$.
The ERF together with complex replacement shows that if $y = \cos (\omega t)$,
then a particular solution is given by
\begin{equation*}
  x_ p  = \mathrm{Re\, } \left( G(\omega) e^{i \omega t}\right)
\end{equation*}
where
\begin{equation*}
  \displaystyle G(\omega)
  \displaystyle = \displaystyle \frac{Q(i \omega)}{P(i \omega)} \quad
  \text{if } P(i \omega) \neq 0. 
\end{equation*}

\paragraph{Complexification and ERF worked out}
The complexified equation is
\begin{equation*}
  P(D)z = Q(D) e^{\i \omega t}. 
\end{equation*}

If $P(i \omega) \neq 0$, the ERF gives us a particular solution of the form
\begin{equation*}
  \displaystyle z_ p
  \displaystyle = \frac{Q(i \omega)}{P(i \omega)} e^{\i \omega t}. 
\end{equation*}

We can express this as
\begin{equation*}
  \displaystyle Z_ p = G(\omega) e^{\i \omega t}.
\end{equation*}

Writing the complex gain in polar form, $G(\omega) = |G(\omega)| e^{-i \phi}$. we find
\begin{equation*}
  z_ p =\left|G(\omega )\right| e^{i(\omega t-\phi )}.
\end{equation*}
So the solution to the original OED is
\begin{equation*}
  x_ p =\left|G(\omega )\right| \cos {(\omega t-\phi )}.
\end{equation*}

This form leads directly to the polar form of the sinusoidal function $x_ p$.
Let's see how this works with the equation we used to model the tide in Boston Harbor
\begin{equation*}
  \dot x + kx = k\cos (\omega t).
\end{equation*}

Recall that we model the ocean tide by $\cos (\omega t)$, and this is regraded as the input signal.
The system response $x$ is the high of the water in harbor.\\

We use the method of complex replacement to solve the ODE for Boston harbor problem.

\begin{itemize}
\item The complex replacement ODE is
  \begin{equation*}
    \dot z + kz = k e^{i \omega t}
  \end{equation*}
  with input signal $e^{i \omega t}$.
\item One particular response determined by ERF is
  \begin{equation*}
    \displaystyle z_ p
    \displaystyle = \frac{Q(i \omega)}{P(i \omega)} e^{i \omega t}
    \displaystyle = \frac{k}{i \omega + k} e^{i \omega t}
  \end{equation*}
\item The ERF shows that the system response to a complex exponential input signal is a
  \textbf{constant multiple} of that input signal.
  That constant is the \textbf{\color{blue}complex gain}:
  \begin{equation*}
    \displaystyle  G(\omega ) \displaystyle  =
    \displaystyle
    \frac{\text {complexified system response}}{\text {complexified system input}}.
  \end{equation*}
  in the present case,
  \begin{equation*}
    G(\omega) = \frac{k}{i \omega + k}
  \end{equation*}
  and
  \begin{equation*}
    \displaystyle  z_ p =G(\omega )e^{i\omega t} =  \frac{k}{i\omega + k}e^{i\omega t}.
  \end{equation*}
  This complex number $G$, expressed as a ratio of two function of time is \textbf{constant}.
  It depends upon the system parameters, of course, but we regard them as fixed.
  We are interested in how it varies with the input angular frequency $\omega$, and
  write $G(\omega)$ to stress that functional dependence.
\item To get a particular solution to the original real ODE, take the real part of $z_ p$:
  \begin{equation*}
    x_ p = \mathrm{Re\, }\left( G(\omega ) e^{i\omega t}\right)
    = \mathrm{Re\, }\left( \frac{k}{i\omega + k}e^{i\omega t} \right)
  \end{equation*}
\item Now comes the best part of this method.
  Writing the particular solution in terms of $G$ leads directly to the polar form for $x_ p$.
  To find it, write out the polar expression for the complex number $G(\omega)$.
  \begin{equation*}
    G(\omega ) = \left| G(\omega ) \right|e^{-i\phi }.
  \end{equation*}
  In our case
  \begin{equation*}
    \left| G(\omega ) \right| = \frac{k}{\sqrt {\omega ^2+k^2}}
  \end{equation*}
  and so
  \begin{equation*}
    \displaystyle  z_ p \displaystyle =
    \displaystyle \frac{k}{\sqrt {\omega ^2+k^2}}e^{i(\omega t-\phi )}
    \qquad \text {and}
    \qquad x_ p \displaystyle  = \displaystyle \frac{k}{\sqrt {\omega ^2+k^2}}\cos (\omega t-\phi ).    
  \end{equation*}
  We have discovered that the gain is given by
  \begin{equation*}
    \text {gain} = g(\omega ) = \frac{k}{\sqrt {\omega ^2+k^2}}.
  \end{equation*}

  \textbf{\color{orange}General rule:}\\
  In general, the
  \begin{equation*}
    \text {gain} = g(\omega ) = \left|G(\omega ) \right|
  \end{equation*}
  and the
  \begin{equation*}
      \text {phase lag} = \phi = -\arg G(\omega ).
  \end{equation*}

\end{itemize}
Observe that the amplitude of the response is different than the amplitude of the input.
That difference in amplitude is the gain $g$. which is the magnitude of the complex gain
$g = |G|$.
The \textbf{\color{blue}phase lag} is $\phi = -\arg {G}$.

For the system,
\begin{equation*}
  \text {gain} = |G| = \frac{k}{\left| k+i\omega \right|} = \frac{k}{\sqrt {k^2 + \omega ^2}}
\end{equation*}
and
\begin{align*}
  \displaystyle  \displaystyle \text {phase lag} = -\arg G
  \displaystyle &= \displaystyle  -\left(\arg \frac{k}{k+i\omega }\right) \\
  \displaystyle &= \displaystyle  - \arg k + \arg (k+i\omega ) = \arg (k+i\omega ).  
\end{align*}

Note the last equality follows because $\arg k =0$
since $k$ is real and positive. Observe that in general, we have
\begin{align*}
  \displaystyle  \text {complex gain} = G
  \displaystyle &= \displaystyle  \frac{Q(i\omega )}{P(i\omega )}\\
  \displaystyle \text {phase lag} = -\arg G
  \displaystyle &= \displaystyle  \arg P(i\omega ) - \arg Q(i\omega ).
\end{align*}

\begin{exercise}
  Complex gain and LTI
\end{exercise}

Why is it enough to consider input signals of the type $\cos (\omega t)$?\\
What about an input signal of the type $A\cos (\omega t - \theta )$? \\
If you know that
\begin{equation*}
  x_ p = g\cos (\omega t - \phi )
\end{equation*}
is steady state response to
\begin{equation*}
  P(D) x = \cos (\omega t),
\end{equation*}
what is a steady state response to
\begin{equation*}
  P(D) x = A\cos (\omega t-\theta )?
\end{equation*}

This is the key idea behind LTI. A steady state response is given by
\begin{equation*}
  x_ p = g\cdot A \cos (\omega t - \theta - \phi ).
\end{equation*}

\clearpage
\subsubsection{Driving through the spring}

Now we will apply the method of complex gain to understand solutions to a more
complicated system — a mass-spring-dashpot system driven through the spring.
The general differential equation is
\begin{equation*}
  \ddot x + b\dot x + kx = ky
\end{equation*}

where $y$ is the input signal, namely the position of the far end of the spring,
and $x$ is the displacement of the mass.

\begin{exercise}
  Complex gain check
\end{exercise}

Find the complex gain for the system
\begin{equation*}
  \ddot x + b\dot x + 2x = 2\cos (t).
\end{equation*}

\begin{itemize}
\item The input is $\cos (t)$.
\item The complex replacement ODE is $\ddot z + b\dot z + 2z = 2e^{it}$.
\item Use ERF to find a particular solution
  \begin{equation*}
    z_ p=\frac{2 e^{it}}{i^2 + bi + 2} = \frac{2 e^{it}}{1+ bi}.
  \end{equation*}
\item The complex gain is
  \begin{equation*}
      \frac{z_ p}{e^{it}} = \frac{2}{1+bi}.
  \end{equation*}
\end{itemize}

\begin{exercise}
  Phase lag
\end{exercise}
Consider the equation
\begin{equation*}
  \ddot x + b\dot x + 2x = 2\cos (t).
\end{equation*}
If the damping constant $b$ starts at $1$ and is increased,
what happens to the \textbf{phase lag} ?\\

The phase lag increases—the phase lag is the argument of $\frac{2}{P(i)} = \frac{2}{1+bi}$.
As $b$ increases the argument increases.

\begin{exercise}
  Amplitude gain
\end{exercise}
Consider the equation
\begin{equation*}
  \ddot x + b\dot x + 2x = 2\cos (t).
\end{equation*}
If the damping constant $b$ starts at $1$ and is increased,
what happens to the \textbf{phase lag} ? \\

The amplitude decreases. The amplitude of the solution is the gain
$g=\frac{2}{|P(i)|}$.
As $b$ increases, the value of $|P(i)| = |1+bi|$ increases, so gain $g=\frac{2}{|P(i)|}$. 

\clearpage
\subsubsection{The meaning of LTI}
Here we explain why it is really enough to understand
the system response to the input signal $\cos (\omega t)$\\

We have been studying LTI, or Linear-invariant system.
``Time-invariant'' means that the system parameters are not changing,
or are change very slowly relative to the time scale we are interested in.
The implication of our input/output analysis is this:
If $x(t)$ is a system response to the input signal $f(t)$,
then if we delay the input signal by $t_0$ seconds, the output signal is the same
as before but delayed by $t_0$ seconds as well: $x(t - t_0)$ is system response
to the input $f(t - t_0)$. The system parameters are the coefficients
in differential equation; so ``time-invariant'' is the same as ``constant coefficients.''\\

Because our systems are also \textbf{linear}, if $x(t)$ is system response to $f(t)$,
then $Ax(t)$ is system response to $Af(t)$. Thus when studying a a linear time-invariant
system with sinusoidal input $A \cos (\omega - \phi)$, it is enough to consider
the input $\cos (\omega t)$ since linearity and time-invariance give the response
to all other sinusoidal inputs for free.

\begin{example}
  If
  \begin{equation*}
    P(D) x = \cos \omega t
  \end{equation*}
\end{example}
has $x_ p = A \cos (\omega t - \phi )$ as a particular solution, then shifting time by
$a = \alpha / \omega$ show that
\begin{equation*}
  P(D) x = \cos (\omega t - \alpha )
\end{equation*}
has $x_ p = A \cos (\omega t - \alpha - \phi )$ as a particular solution. \\

The gain and phase lag represent a relationship between the input and the output (both sinusoidal).
The gain is the ratio of the output amplitude to the input amplitude,
and the phase lag is the number of radians the output signal falls behind the input signal.
Because our system is time-invariant, those relationships between input and response are
\textbf{unchanged} by replacing the input signal $cos(\omega t)$ with $cos(\omega t - \alpha)$:
the gain is $A$ and the phase lag is $\phi$.\\

\begin{exercise}
  Time Invariance concept check
\end{exercise}

If $x_ p = A\cos (\omega t - \phi )$ is a particular solution to $P(D)x = \cos (\omega t)$, 
what is a solution to $P(D) x = \sin (\omega t)$?\\

Since $\sin (\omega t) = \cos (\omega t - \pi /2)$, the system response is
$A\cos (\omega t - \pi /2 - \phi ) = A\sin (\omega t - \phi )$,
as the gain and phase lag are unchanged.
\clearpage

\subsubsection{Worked example}
\paragraph{Review of complex replacement and time-invariance}
We'll review how to solve first order differential equations
with sinusoidal inputs using complex replacement and the exponential response formula.\\

\begin{enumerate}[label=\textbf{Part.\arabic*}]
\item \label{Part.1} Use complex techniques to solve
  \begin{equation*}
    \dot{x} + kx = \cos (\omega t) \qquad \text{ where } k,\, \omega \text{ is constant}. 
  \end{equation*}
  The general solution is
  \begin{equation*}
    x = x_ h + x_ p 
  \end{equation*}
  Set right side of equation to zero,
  \begin{equation*}
      \dot{x} + kx = 0
  \end{equation*}
  then solve it by separate variable,
  \begin{align*}
    \dot{x} + kx &= 0 \\
    \frac{1}{x} dx  &= -k dt \\
    \int \frac{1}{x} dx  &= \int -k dt \\
    \ln |x|  &= -kt + c_1 \\
    e^{\ln |x|}  &= e^{-kt + c_1} \\
    x  &= e^{c_ 1} e^{-kt}. 
  \end{align*}

  So, the homogeneous solution for $\dot{x} + kx$ is
  \begin{equation*}
    C_1 e^{-kt}. 
  \end{equation*}

  The particular solution of $\dot{x} + kx = \cos (\omega t)$. \\
  
  The complex replacement of $\cos (\omega t)$ is $\cos (\omega t) = \mathrm{Re\, }(e^{i \omega t})$. \\
  Complexified this ODE, 
  \begin{equation*}
    \dot{z} + kz = e^{i \omega t}. 
  \end{equation*}
  The characteristic polynomial is $P(D) = D + k$, Using ERF, 
  \begin{equation*}
    z_ p = \frac{1}{P(i \omega)} e^{i \omega t} = \frac{1}{i \omega  + k} e^{i \omega t}
    = \frac{k - i \omega}{k^2 + \omega ^2} e^{i \omega t}
    = \frac{k - i \omega}{k^2 + \omega ^2} \left( \cos (\omega t) + i \sin (\omega t) \right). 
  \end{equation*}
  Take real part of $z_ p$ is
  \begin{equation*}
    x_p = \mathrm{Re\, }(z_ p) 
    = \frac{1}{k^2 + \omega ^2} \left( k \cos (\omega t) + \omega \sin (\omega t) \right). 
  \end{equation*}
  So the general solution is
  \begin{equation*}
    x = C_1 e^{-kt} +
    \frac{1}{k^2 + \omega ^2} \left( k \cos (\omega t) + \omega \sin (\omega t) \right). 
  \end{equation*}
  
\item \label{Part.2}Use our work from \ref{Part.1} to solve
  \begin{equation*}
    \dot{x} + kx = F \sin (\omega t) \qquad \text{ where } F \text{ is constant}.
  \end{equation*}
  Complex replacement
  \begin{equation*}
    \sin (\omega t) = F \mathrm{Im\, }(e^{i \omega t}).
  \end{equation*}
  The complexified ODE
  \begin{equation*}
    \dot{z} + kz = Fe^{i \omega t}. 
  \end{equation*}
  Using linearity
  \begin{equation*}
    z_ p = \frac{F \left(k - i \omega \right)}{k^2 + \omega ^2}
    \left( \cos (\omega t) + i \sin (\omega t) \right). 
  \end{equation*}
  Take imaginary part of $z_ p$ is
  \begin{equation*}
    x_ p = \mathrm{Im\, } (z_ p)
    = \frac{F}{k^2 + \omega ^2} \left( - \omega \cos (\omega t) + k \sin (\omega t) \right). 
  \end{equation*}
  
\item Use the superposition principle to solve
  \begin{equation*}
    \dot{x} + kx =  \cos (\omega t) + 3 \sin (\omega t)
  \end{equation*}
  Let the solution of \ref{Part.1} as $x_a$ and \ref{Part.2} as $x_b$ then,
  \begin{align*}
    \dot{x}_ a + kx_ a &= \cos (\omega t) \\
    \dot{x}_ b + kx_ b &= F \sin (\omega t) = 3 \sin (\omega t), \text{ where } F = 3
  \end{align*}
  By linearity
  \begin{align*}
    \frac{d}{dt} (\underbrace{x_ a + x_ b}_{x}) +
    k(\underbrace{x_ a + x_ b}_{x}) =  \cos (\omega t) + 3 \sin (\omega t). 
  \end{align*}
  We can write down the solution to this equation by simply taking
  the sum of the solutions from \ref{Part.1} and \ref{Part.2}. \\
  So the particular solutions is
  \begin{align*}
    x_ p &= \frac{1}{k^2 + \omega ^2} \left( k \cos (\omega t) + \omega \sin (\omega t) \right)
    + \frac{3}{k^2 + \omega ^2} \left( - \omega \cos (\omega t) + k \sin (\omega t) \right) \\
         &= \frac{1}{k^2 + \omega ^2} \left( \left(k -3 \omega \right) \cos(\omega t) +
           \left( \omega + 3k \right) \sin (\omega t) \right)    
  \end{align*}  
  
\item use our work from \ref{Part.1} again, to solve
  \begin{equation*}
    \dot{x} + kx = \cos (\omega t - \phi) \qquad \text{ where } \phi \text{ is constant}.
  \end{equation*}
  Reform the equation
  \begin{equation*}
      \dot{x} + kx = \cos (\omega (t - \phi / \omega).   
  \end{equation*}
  This equation is same form of \ref{Part.1}. So using LTI
  \begin{align*}
    x_ p & = \frac{1}{k^2 + \omega ^2} \left( k \cos \left( \omega ( t - \phi / \omega) \right)
    + \omega \sin \left( \omega (t - \phi / \omega) \right) \right) \\
    &= \frac{1}{k^2 + \omega ^2} \left( k \cos (\omega t - \phi)  + \omega \sin (\omega t - \phi ) \right) \\
  \end{align*}
\end{enumerate}

\clearpage
\subsubsection{Stability} 
To do with the first order linear DE
\begin{equation*}
  \dot{y} + ky = q(t).
\end{equation*}

In this lecture, we've been discussing the case of exponential inputs
$q(t) = e^{(a + bi)t}$.
This is the case where complex numbers and ERF help.
But earlier in the class we used {\color{blue} variation of parameters} to find
a general integral solution for any input $q(t)$.\\
Considering the most general input $q(t)$.

\paragraph{Steady state and transients} \label{stablilty}

Let's look at the equation
\begin{equation*}
  q(t) = e^{(a + bi)t}.
\end{equation*}

When you solve it, let me remind you how the solutions look,
because that explains the terminology. \\
The solution looks like
--after you've done the integrating factor and multiplied through and integrated both sides--
in short, what you're supposed to do.
\begin{equation*}
  y = e^{{\color{orange}-k}t} \int q(t) e^{{\color{orange}k}t} dt + c e^{-k}
\end{equation*}
It will help you to remember the opposite signs.
If you think that when $q$ is a constant, $1$ for example,
you want these two $e^{-kt}$ and $e^{kt}$ to cancel out and produce a constant solution.
That's a good way to remember that the signs have to be opposite.
But I don't encourage you to remember the formula at all.
It's just a convenient thing for me to be able to use right now.
And then there's the other term which
comes by putting out the arbitrary constants explicitly, $c e^{-kt}$. 
So you could either write it this way, where this is somewhat vague, or you could
make it definite by putting $\int_0 ^t$ and change the dummy variable inside
according to the way the notes tell you to do it. \\

When you do this and if $k > 0$--
that's absolutely essential-- only when that is so, then this term, as I told you a week or so ago,
this $c e^{-k}$ goes to 0 because $k > 0$ as $t$ goes to infinity.
So this goes to 0 as t goes-- and it doesn't matter what $c$ is--as $t$ goes to infinity.
The terms, $e^{{\color{orange}-k}t} \int q(t) e^{{\color{orange}k}t} dt$, stays some sort of function.
And so this term is called the \textbf{\color{blue}steady state} or long term solution.\\

And this term $c e^{-kt}$ disappears, gets smaller and smaller as time goes on,
is therefore called the \textbf{transient} because it disappears as the time increases to infinity.
So $c e^{-kt}$ uses the initial value, let's call it $y(0)$, assuming that you started
the initial value when t was equal to $0$, which is a common thing to do, although of course
not necessary.
The starting value appears in this term. This one is just some function.\\
The general picture of the way that looks
\begin{figure}[ht!]
  \centering
  \includegraphics[width=0.5\textwidth]{image-steady_state}
  \caption{Image of steady state}
\end{figure}

Well, the steady state solution has this starting point.
Other solutions can have any of these other starting points.
So in the beginning, they won't look like the steady state solution.
But we know that as time goes on, they must approach it,
because $c e^{-kt}$ represents the difference between the solution and the steady state solution.
So this term is going to $0$, and therefore whatever these solutions of graph do to start out with,
after a while they must follow the steady state solution more and more closely.
They must, in short, be asymptotic to it.
So the solutions to any equation of that form will look like this.
Even if up here maybe it started at 127, that's OK.
After a while it's going to start approaching that green curve.\\

Of course, they won't cross each other.
But they are sort of like that's the rock star and these are the groupies trying to get close to it.
But something follows from that picture, which is the steady state solution?
What, in short, is so special about this green curve?
All these other white solution curves have the same property that all the other white curves
and the green curve, too, are trying to get close to them.
In other words, there is nothing special about the green curve.
It's just that they all want to get close to each other.
And therefore, though you can write a formula like this, there isn't one steady state solution.
There are many. Now, this produces vagueness.
So you talk about the steady state solution, which are you talking about?
I have no answer to that. The usual answer is whichever one looks simplest.
Normally the one that will look simplest is the one where $c$ is zero.
But if this is a peculiar function, it might be that for some other value of $c$,
you get an even simpler expression.\\

So the steady state solution, about the best I can say, either you integrate that, don't use an arbitrary constant and use what you get, or pick the simplest.
Pick the value of $c$ which gives you the simplest answer.
Pick the simplest function.
That's what's usually called the steady state solution.\\

\textbf{\color{orange} Remarks on \ref{stablilty}: }
For higher order systems, we haven't presented the general integral formula
for solving differential equations with arbitrary input.
So here we only discuss transients, steady state, and stability
in the case of complex exponential input for higher order systems.\\

A certain spring/mass/dashpot system is modeled by the LTI ODE
\begin{equation*}
  \ddot x + 7\dot x+12 x = 12f(t)
\end{equation*}
where $f(t)=\cos (2t)$ is the input signal.\\

\textbf{\color{blue}Question:} Suppose that the solution to this equation
with initial conditions $(x(0), \dot x(0)) = (2,3)$ is $x(t)$.
What can you say about the solution $y(t)$ to the same ODE with initial condition
$(y(0), \dot y(0)) = (3,-8)$?
\begin{itemize}
\item It isn't related to $x(t)$ in any way.
\item It oscillates.
\item ts graph becomes asymptotic to that of $x(t)$ as $t \to \infty$.
\item Its graph diverges from that of $x(t)$ as $t \to \infty$.
\end{itemize}

\Solution \\
The characteristic polynomial is $r^2 + 7r + 12$, and the roots are $-3$ and $-4$.
Thus the general homogeneous solution has form
\begin{equation*}
  x_ h = c_1 e^{-3t} + c_2 e^{-4t},
\end{equation*}

where $c_ 1$ and $c_ 2$ are constants determined by initial conditions.
All of these homogeneous solutions decay (rapidly!) to zero as $t$ grows to $\infty$.
So any solution with different initial conditions will become
asymptotic to that. Why is this?\\

The general solution to the original inhomogeneous ODE is

\begin{align*}
  \displaystyle  x \displaystyle &= x_ p + x_ h \\
  \displaystyle &=
                  \underbrace{\mathrm{Re\, }
                  \left( \frac{1}{8+14i} e^{2it} \right)}_{{\color{blue}{\text {steady state solution}}} }
                  + \underbrace{c_1 e^{-3t} + c_2 e^{-4t}}_{{\color{orange}{\text {transient}}} }.
\end{align*}
In general, for a damped oscillator forced with a sinusoidal input produces a sinusoidal output signal.
That output is a particular solution called a \textbf{\color{blue} steady-state solution},
because this is what the solution looks like as $t \to \infty$.
Every other solution is the steady-state solution plus a \textbf{\color{orange} transient},
where the transient is a function that decays to $0$ as $ t \to + \infty$. \\

Changing the initial conditions changes only the constants $c_ 1$ and $c_ 2$ in the homogeneous
solution above, so the steady-state solution is the same.
A system like this,in which changes in the initial conditions have vanishing effect on
the long-term behavior of the solution, is called $stable$. 
\clearpage

\clearpage
%%% Local Variables:
%%% mode: latex
%%% TeX-master: "NoteForDifferentialEquation"
%%% End:

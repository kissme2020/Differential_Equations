\section{UNIT5}

\clearpage

\subsection{Nonlinear DEs: Graphical methods }

\subsubsection{Nonlinear differential equations: Graphical Methods}

\emph{Objectives}
\begin{itemize}
\item Use \emph{\color{blue} slope fields} and \emph{\color{blue}isoclines}
  to aid in the drawing of solution curves for \emph{\color{blue}first order ODEs}.

\item Draw \emph{\color{blue}integral curves} aided by existence and uniqueness theorems, and assess long term behavior of solutions.
\end{itemize}

\subsubsection{Introduction}

\begin{exercise}
  Review: Linear vs. nonlinear DEs
\end{exercise}
\begin{align*}
  y'=y(1-y) &\qquad y' = xy \\
  y'= x-2y &\qquad y' = -x/y \\
  y'= y^2- x^2 &\qquad y' = y/x \\
  y'= y^3 -3y -x &\qquad yy' = x^3 -y
\end{align*}

\emph{Linear equations:} The following DEs are linear because they can be written in the form
$\, y'+p(x)y=q(x)\,$.
In this form,
\begin{itemize}
\item $y = xy\,$ becomes $y' - xy = 0\, $.
\item $y = x - 2y\,$ becomes $y' + 2y = x\, $.
\item $y = y/x,$ becomes $y' - \frac{1}{x}y = 0\, $. 
\end{itemize}

\emph{Nonlinear equations:} he rest of the DEs are nonlinear because they cannot be written in the form
$\, y'+p(x)y=q(x)\,$ mentioned above. They failed for the reason below:
\begin{itemize}
\item $y' = y(1-y),\,$ rewritten as $y' = y - y^2,\,$ contains a $y^2$ term.
\item $y' = y^2 - x^2\,$ contains $y^2$ term.(The $x^2$ term does not contribute to the classification.)
\item $y' = y^3 -3y -x\,$ contains a $y^3$ term.
\item $y' = -x/y\,$ contains a term with $y$ od degree $-1$.
\item $yy' = x^3 -y\,$ contains the term $yy'$ in which the total degree of $y$ and its derivative is $2$. 
\end{itemize}

\paragraph{Introduction}
First order ODE, ODE, I'll only use three, two acronyms.
ODE is ordinary differential equations.
I think all of MIT knows that whether they've been taking a course or not.\\

So we're talking about first order ODEs, which
in standard form are written, you isolate the derivative of $y$
with respect to $x$, let's say on the left hand side.
And on the right hand side, you write everything else.
\begin{equation*}
  y' = f(x, y)
\end{equation*}
You can't always do this very well,
but for today, I'm going to assume that it has been done and it's doable.
So for example,
\begin{equation*}
  y' = x/y, \quad y' = x - y^2, \quad y' = y - x^2 
\end{equation*}

Now you look at this.
This $\, y' = x/y,\,$ of course, you can solve by \textit{separating variables}.
So this is solvable.
This one $\, y' = y - x^2\, $ is, in neither of these can you separate variables
and they look extremely similar, but they are extremely dissimilar.
The most dissimilar about them is
that this one $\, y' = y - x^2\, $ is easily solvable and you'll
learn if you don't know already next time, next Friday, how to solve this one.
This one $\, y' = x - y^2 ,\,$ which looks almost the same, is unsolvable in a certain sense.
Namely, there are no elementary functions,
which you can write down which will
give a solution of that differential equation.\\

So right away, one confronts the most significant fact
that even for the simplest possible differential
equations,
\begin{equation*}
  \boxed{y' = x - y^2}
\end{equation*}
those which only involve the first derivative,
it's possible to write down extremely simple guys.
I mean not really bad but recalcitrant.
It's not solvable in the ordinary sense in which you
think an equation is solvable.
And since those equations are the rule rather than
the exception, I'm going to devote this first day
to not solving a single differential equation,
but indicating to you what you do
when you meet a boxed equation like that,
what do you do with it.
So this first day is going to be devoted
to geometric ways of looking at differential equations
and numerical.
At the very end, I'll talk a little bit about numerical.\\

First order (nonlinear) equations can be written in the \emph{\color{blue} standard form}:

\begin{equation*}
  y' = f(x,y)
\end{equation*}

where $f(x, y)$ is a function of the variables $x$ and $y$. \\

Notice we now have two standard forms for first order differential equations.
Recall first order \emph{\color{blue}linear} equations
can also be written in the standard form for linear equations:

\begin{equation*}
  y'+p(x)y=q(x).
\end{equation*}


This unit is concerned with first order nonlinear differential equations, such as the one boxed

\begin{equation*}
  \boxed{y' = x - y^2}
\end{equation*}

The sad fact is that we can hardly ever find formulas for the solutions to nonlinear DEs.
Instead we try to understand the qualitative behavior of solutions using geometric methods or approximations.
\clearpage

\subsubsection{Geometric view of DEs}

\paragraph{The geometry}
What's our geometric view of differential equations?

Well, it's something that's contrasted
with the usual procedures by which you solve things and find
elementary functions which solve them.
I'll call that the \textit{analytic method}.
So on the one hand, we have the analytic ideas
in which you write down explicitly the equation
\begin{equation*}
  y' = f(x, y). 
\end{equation*}
And you look for certain functions, which are called solutions.
So there's the ODE.
And $y_1(x)$, notice I don't use a separate letter.
I don't use $g$ or $h$ or something like that for the solution
because the letters multiply so quickly,
multiply in the sense of rabbits that after a while, if you keep
using different letters for each new idea
you can't figure out what you're talking about.
So I'll use $y_1$ means it's a solution of this differential
equation.
Of course, the differential equation
has many solutions containing an arbitrary constant.
So we'll call this the solution.\\


Now \textit{the geometric view}.
The geometric guy that corresponds to this version of writing the equation
is something called a \textit{direction field}.
And the solution is from the geometric point of view,
something called an \textit{integral curve}.
\begin{align*}
  \text{Analytic}  &\qquad  \text{Geometric} \\
  y' = f(x,y) &\Longleftrightarrow \text{Direction Field} \\
  y_1 x &\Longleftrightarrow \text{Integral Curve}
\end{align*}


So let me explain if you don't know what the direction field is.
So what's a direction field?
Well, a direction field is you take the plane.

\begin{figure}[ht!]
  \centering
  \includegraphics[width=0.5\textwidth]{image-direction_field}
  \caption{Direction Field}
\end{figure}

And at each point of the plane-- of course, that's an impossibility.
But you pick some points of the plane.
You draw what's called a little line element.
So there is a point.
It's a little line.
And the only thing which distinguishes it outside of its position in the plane--
so here's the point $x$, $y$ at which we're drawing this line element is its slope.
What is its slope?
Its slope is to be $f(x,y)$.
And now you fill up the plane with these things until you're tired of putting them in.
So I'm going to get tired pretty quickly.
So I don't know.
Let's not make them all go the same way.
There's a few randomly chosen line elements that I put in.
And I put in the slopes at random since I didn't have any particular differential
equation in mind.\\

Now the integral curves.

\begin{figure}[ht!]
  \centering
  \includegraphics[width=0.5\textwidth]{image-integral_curves}
  \caption{Integral Curves}
\end{figure}

Those are the line elements.
And the integral curve is a curve which goes through the plane and at every point
is tangent to the line element there.
So this is the integral curve.
Hey, wait a minute.
That's not tangent to the line element there.
It didn't even touch it.
Well, I can't fill up the plane with line elements.
Here at this point, there was a line element which I didn't bother drawing in.
And it was tangent to that.
Same thing over here.
If I drew the line element here, I would find that the curve had exactly the right slope there. \\

So the point is what distinguishes
the integral curve is that everywhere it
has the direction-- that's the way I'll indicate that it's tangent--
has the direction of the field everywhere at all points, at all points on the curve, of course.
Where it doesn't go, it doesn't have any mission to fulfill.
Now I say that this integral curve
is the graph of the solution to the differential equation.
In other words, writing down analytically a differential equation is the same geometrically
as drawing this direction field.
And solving analytically for a solution to the differential
equation is the same thing as geometrically drawing an integral curve.

\clearpage

The figure below is the geometric picture of the differential equation
$y' = f(x, y).\,$
in previous paragraph this sketch is called a direction field.
In these notes, we will call it a \emph{slope field} instead.
We reserve the word direction field for diagrams whose line elements have arrows.

\begin{figure}[ht!]
  \centering
  \includegraphics[width=0.5\textwidth]{images_u2s1_slopefield}
  \caption{The line element at$(x,y)$ has slop $f(x,y)$}
\end{figure}

\begin{definition}
  For a differential equation $\, y'=f(x,y),\,$ a \emph{\color{blue}slop field} is
  a diagram is a diagram which includes at each point $(x, y)$ a short line
  element (or line segment) whose is the \emph{value} $f(x,y)$. 
\end{definition}

\begin{figure}[ht!]
  \centering
  \includegraphics[width=0.5\textwidth]{images_u2s1_slopefieldsolutioncurve}
  \caption{A solution curve (blue) is tangent to each line segment that it touches.}
\end{figure}

The graph of a solution $y_1 (x)$ to the DE in the $xy$-plane is called a
\emph{\color{blue} solution curve} or an \emph{\color{blue} integral curve}. \\
An integral curve must be tangent to the slope field at every point:
$\displaystyle \, y_1'(x)=f\left(x, y_1(x)\right).$
\clearpage

\subsubsection{Slope field}

\begin{example}
  Sketch the slope field for $y'= y^2 - x$
\end{example}

\Solution Let $f(x,y):=y^2 - x$. Then

\begin{align*}
  f(1,2)=3. &\qquad \text{so at $(1,2)$ draw a short segment of slope $3$}; \\
  f(0,0)=0. &\qquad \text{so at $(0,0)$ draw a short segment of slope $0$}; \\
  f(1,0)=-1. &\qquad \text{so at $(1,0)$ draw a short segment of slope $-1$}; \\
  f(1,2)=3. &\qquad \text{so at $(1,2)$ draw a short segment of slope $3$}; \\
  f(1,0)=1. &\qquad \text{so at $(1,0)$ draw a short segment of slope $1$};\\
            & \vdots
\end{align*}

The diagram of all these short segments is the slope field.
You can see how tedious this process is; a computer will sketch the slope field much more quickly.
You can see the slope field for this example in the Mathlet below by choosing the right function and parameter.\\

\href{http://mathlets.org/mathlets/isoclines/}
{ISOCLINES}

\emph{\color{blue}Slope Fields} The mathlet above shows many slope fields for various functions.
You can click on points to see the solution curve through that point. \\
Why draw a slope field?
The ODE is telling us that the slope of the solution curve at each point is the value of
$f(x,y)$, so the short segment is, to first approximation,
a little piece of the solution curve.
To get an entire solution curve, follow the segments! \\
We will get practice thinking about slope field and solution curves next.

\clearpage

\subsubsection{First exercises}

\begin{exercise}
  Identify the slope field
\end{exercise}

Which of the following is the slope field for $\displaystyle \frac{dy}{dx} = x + y$ ?\\
Take some point and calculate the function's slop

\begin{align*}
  f(-4,0)=-4. &\qquad \text{so at $(-4,0)$ draw a short segment of slope $-4$}; \\
  f(-3,0)=-3. &\qquad \text{so at $(-3,0)$ draw a short segment of slope $-3$}; \\
  f(-2,0)=-2. &\qquad \text{so at $(-2,0)$ draw a short segment of slope $-2$}; \\
  f(-1,0)=-1. &\qquad \text{so at $(-1,0)$ draw a short segment of slope $-1$}; \\
  f(0,0)=0. &\qquad \text{so at $(0,0)$ draw a short segment of slope $0$}; \\
  f(1,0)=1. &\qquad \text{so at $(1,0)$ draw a short segment of slope $1$}; \\
  f(2,0)=2. &\qquad \text{so at $(2,0)$ draw a short segment of slope $2$}; \\
  f(3,0)=3. &\qquad \text{so at $(3,0)$ draw a short segment of slope $3$}; \\
  f(4,0)=4. &\qquad \text{so at $(4,0)$ draw a short segment of slope $4$};\\
            & \vdots
\end{align*}

And if sign of $x$ and $y$ is opposite and absolut valus is same, the slop is zero.\\
So the slope field is

\begin{figure}[ht!]
  \centering
  \includegraphics[width=0.5\textwidth]{images_u5s3_slopeplot2}
  \caption{Slope Field $y' = x + y$}
\end{figure}

Identifying locations where the slopes are zero is a good practice
when trying to use slope fields since these places are where the solution curve has a critical point.

\clearpage

\begin{exercise}
  Identify the solution curves 1
\end{exercise}

\href{http://mathlets.org/mathlets/isoclines/}
{ISOCLINES}

Use the dropdown menu in the mathlet above to find the slope field for the differential equation
$\displaystyle y' = y(1-y)$\\

Given in the initial condition to see the solution curve through that point.
What happens to the given solution curve as $x$ tends towards infinity? \\
\begin{enumerate}
\item $y(0)=1/2$
  No matter what value of $a$ you choose,
  for initial conditions in $1 < y(a) < \infty$ the solution curve decreases
  to $1$ as x tends towards infinity. 
\item $y(0)=-1$
  No matter what value of $a$ you choose,
  for initial conditions in $0 < y(a) < 1$ the solution curve increases to $1$ as $x$ tends towards infinity.
\item $y(0)=2$
  No matter what value of $a$ you choose,
  for initial conditions in $y(a) < 0$ the solution curve decreases to $-\infty$ as $x$ tends towards infinity.
\end{enumerate}

\begin{figure}[ht!]
  \centering
  \includegraphics[width=0.5\textwidth]{image_exercise_identify_solution_curves}
  \caption{Identify The Solution Curves}
\end{figure}

\clearpage

\begin{exercise}
  Identify the solution curve 2
\end{exercise}

Use the dropdown menu in the mathlet above to find the slope field for the differential equation
$y' = y/x$. \\

Click on the point $(1,1)$ to see the solution curve through that point.
Which choice best describes the solution curve?\\

\begin{figure}[ht!]
  \centering
  \includegraphics[width=0.5\textwidth]{image-exercise_solution_curve2}
  \caption{Exercise Soluation Curve 2}
\end{figure}

The curve through the given initial condition is a ray emanating out of
the origin along the line $y = x$.
Note that infinitely many solutions emanate out of the origin,
and this is a point where the slope field is undefined!
This tells us in particular that there is no unique solution given the initial condition $y(0) = 0$.

\clearpage

\subsubsection{Isoclines}

\paragraph{How to draw slope fields}
Well, this leaves us the interesting question is, how
do you draw a direction field?
Well, this being 2003, mostly computers draw them for you.
Nonetheless, you do have to know a certain amount.
I've given you a couple of exercises,
where you have to draw the direction field yourself.
This is so you get a feeling for it
and also, because humans don't draw direction fields
the same way computers do.\\

let's first of all, how to \textit{computers} do it?
They are very stupid. There's no problem.
Since they go very fast and have unlimited amounts of energy
to waste, the computer method is the naive one.
\begin{enumerate}
\item Pick the point $(x, y)$, they are usually are equally spaced.
\item And at each point, it computes $f(x, y)$ at that point.\\
  Finds means it computes the value of $f(x, y)$ that function.
  And the next thing is on the screen it draws at $x,\, y$ the little line element having this,
  having slope $x, \,y.$.
\end{enumerate}

In other words, it does what the differential equation tells it to do.
And the only thing that it does is you can--
if you're telling the thing to draw the direction field,
about the only option you have is telling
what the spacing should be.
And sometimes people don't like to see a whole line.
They only like to see a little bit of a half line.
And you can sometimes tell, according to the program,
tell the computer how long you want that line to be,
if you want a teeny or a little bigger.
Once in a while you want an arrow on it, but not right now.
OK, that's what a computer does.\\


Now what does a human do?
This is what it means to be human.
You use your intelligence.
From the human point of view, this stuff has been done in the wrong order.
And the reason it has been done in the wrong order,
because for each new point it requires a recalculation of $f(x,y)$.
That is horrible.
The computer doesn't mind, but a human does.
\begin{enumerate}
\item Picking the slope that you would like to see.\\
  Pick the slope $C$,  So you pick a number; $C$ is $2$.
  I want to see where are all the points in the plane where
  the slope of that line element would be $2$.
  Well, they will satisfy an equation.
\item The equation is $f(x, y)$, in general, it will be $C$.
  So what you do is plot this, plot the equation. \\
  Notice, it's not the differential equation.
  You can't exactly plot a differential equation.
  It's a curve, an ordinary curve.  But which curve will depend?
  It's, in fact, from the 18.02 point of view,  it's a level curve of
  $f(x y)$ corresponding to the level of value $C$.
  But we're not going to call it that, because this is not 18.02.
  Instead, we're going to call it a \textit{isocline}.
  And then you plot.
\item Well, you've done it. So you've got this isocline.\\
  Except, I'm going to use solution solid lines only
  for integral curves, or when we do plot isoclines to indicate
  that they're not solutions, we'll use dashed lines for doing them.
  One of the computer things does, and the other one doesn't.
  But it won't. They use different colors also.
  There are different ways of telling you what's an isocline
  and what's a solution curve.
  So and what do you do?
  So these are all the points where the slope is going to be $C$. And now
  what you do is draw in as many as you want of line elements having slope $C$.
  Notice how efficient that is. If you want 50 million of them and have the time
  draw in 50 million.
  If two or three are enough, draw in two or three.
  You'll look be looking at the picture.
  You'll see what the curve looks like.
  And that will give you your judgment as to how to do that.

\end{enumerate}

So in general, a picture drawn that way,
you'll have a, so let's say, an isocline corresponded to $C$ equals $0$.
And the line elements along it--
I think for an isocline, for the purposes of this lecture,
it would be a good idea to put isoclines--
OK, so I'm going to put some solution curves in pink or whatever this color is.
And isoclines are going to be in orange, I guess.

\begin{figure}[ht!]
  \centering
  \includegraphics[width=0.5\textwidth]{image-isoline}
  \caption{Isocline}
\end{figure}

So I use isocline represented by a dash line.
And now you'll put in the line elements.
We'll need lots of chalk for that.
So I'll use white chalk. Why horizontal?
Well, because according to this, the slope
is supposed to be $0$ there.
And in the same way, how about an isocline
where the slope is $C = -1$?
Let's suppose here $C$ is equal to negative $1$, OK,
then it will look like this.
These are supposed to be lines of slope $-1$.
Don't shoot me if they're not.
So that's the principal.
So this is how you'll fill up the plane
to draw a direction field, by plotting the isoclines first.
And then once you have the isoclines there,
you'll have line elements and you can draw a direction field. \\


OK, so for the next few minutes, I'd like to work a couple of examples for you
to show how this works out in practice.\\

So the first equation is going to be
\begin{equation*}
  y' = -x/y
\end{equation*}

OK, first thing, what are the isoclines?
The isoclines are going to be,

\begin{align*}
  \frac{-x}{y} &= C \\
  y &= \frac{-1}{C} x
\end{align*}

So there's our isocline, since-- why don't I put that up
in orange, since it's going to be--
that's the way the color I'll draw it in also.
In other words, for different values of $C$,
now this thing is a line.
It's a line, in fact, through the origin.
This looks pretty simple, OK.
So here's our plane.
The isoclines are going to be lines through the origin.
And now let's put them in-- suppose, for example, $C = 1$,  
, then it's the line $y = - x$.
So it's this is the isocline. I'll put down here $C = 1$
So it should be little line segments of slope $1$ will be the line elements, things of slope $1$. \\
OK, now about $C$ equals negative $-1$?
If $C = -1$, then it's the line $y = x$.
And so that's the isocline.

\begin{figure}[ht!]
  \centering
  \includegraphics[width=0.5\textwidth]{image-example1-isocline}
  \caption{Example Isoclines}
\end{figure}

Notice it's still dash, because these are isoclines.\\
And so the slope elements look like this. Notice they're perpendicular.
Now notice that they are always going to be perpendicular,
because--
to the line-- because the slope of this line is $-1/C$,
but the slope of the line element is going to be $C$.
Those numbers $-1/C$ and $C$ are negative reciprocals.
And you know the two lines whose slopes are negative reciprocals are perpendicular.
So the line elements are always going to be perpendicular to these.
And therefore, I'd hardly even have to bother calculating doing any more calculation.
Here's going to be--
well, how about this one?
Here's a controversial isocline.
Is that an isocline?
Well, wait a minute.
That doesn't correspond to anything looking like this.
Aha, but it would if I put $C$, multiplied through by $C$.
And then it would correspond to $C$ being $0$.
In other words, don't write it like this.
Multiply through by $C$. It will read $Cy = - x$.
And then when $C$ is $0$, I'll have $x$ equals $0$, which is exactly the $y$-axis.
So that really is included.
How about the $x$-axis?
Well, the $x$-axis is not included.
However, most people include anyway.
This is very common to be a sort of sloppy and bending the edges of corners a little bit, you know,
and hoping nobody will notice.
We'll say that corresponds to $C$ equals $\infty$.
I hope nobody wants to fight about that.
If you do, go fight with somebody else.
So if $C$ is $\infty$, that means the line, little line segment,
should have infinite slope.
And by common consent, that means it should be vertical.
And so we can even count this as sort of an isocline.
And I'll make the dashes smaller to indicate it has
a lower status than the others.

\begin{figure}[ht!]
  \centering
  \includegraphics[width=0.5\textwidth]{image-example1-isocline1}
  \caption{Example Isoclines}
\end{figure}

And I'll put this in--
do this weaselly thing of putting it in quotation marks
to indicate that I'm not responsible for it.

\clearpage

\begin{definition}
  For a number $C$, the \emph{\color{blue}$C$-isocline} is the set of points
  in the $(x,y)$-plane such that the solution curve through that point has slope $C$.
  (Isocline means “same incline", or “same slope".)
\end{definition}

\begin{question}
  What is the equation for the $C$-isocline?
\end{question}

\Answer The ODE says that the slope of the solution curve through a point
$(x, y)$ is $f(x,y)$, so the \emph{\color{blue}equation of the $C$-isocline} is
\begin{equation*}
  \displaystyle  \displaystyle f(x,y) \qquad \displaystyle = \qquad \displaystyle C
\end{equation*}

\begin{exercise}
  Distinguishing Isoclines and Solution curves
\end{exercise}

\begin{figure}[ht!]
  \centering
  \includegraphics[width=0.5\textwidth]{images_u2s1_isoclinessolution}
  \caption{Exercise Distinguishing Isoclines and Solution curves}
\end{figure}

On the slope field below you can see two curves.
Identify which of them is an isocline and which one is a solution curve.\\

The blue curve is a solution curve since it is tangent to all line segments on the slope field. 
\emph{Note:} The domain of the blue curve is at the maximum possible
since the curve cannot be extended any further while remaining both tangent
to the slope field and the graph of a function.\\
The orange line is an isocline, since all line segments on it have the same slope. 
\emph{Note:} In general an isocline does not have to be a line.

\begin{exercise}
  Equation of an Isocline
\end{exercise}

The slope field shown in the problem above is for the DE $y' = - \frac{x}{y}$.
Find the equation of the $2$-isocline.\\

The 2-isocline is the set of all points where $\displaystyle \, f(x,y)=-\frac{x}{y}=2.\, \,$
Solving for $y,\,$ we get $\, y= -\frac{1}{2}x\,$
which is the equation of the orange line in the figure in the previous problem.

\begin{exercise}
  Equation of Solution Curve
\end{exercise}

As above, consider the differential equation $y' = - \frac{x}{y}$. \\
Use separation of variables to solve the DE,
and find the equation for the solution curve that goes through
$(x,y)=(0,2).\, \,$
Note that the solution curve shown in the figure above is exactly for this initial condition.) \\

\begin{align*}
  y' &= - \frac{x}{y} \\
  \frac{d}{dy} y  &=  -x dt\\
  \int y dy   &=  \int -x dt \\
  \frac{1}{2} y^2 + c_1 &= \frac{-1}{2} x^2 + c_2 \\
  y^2 &= -x^2 + C  \\
  y &= \pm sqrt{-x^2 + C}
\end{align*}

Apply initial condition $(x,y)=(0,2)$ on $y^2 = -x^2 + C\, $
\begin{align*}
  2^2 &= 0^2 + C \\
  C &= 4 
\end{align*}

So, the solution for DE is
\begin{equation*}
  y = \pm sqrt{-x^2 + 4}
\end{equation*}

This is indeed the equation of the semi-circle shown in the figure above.
Notice that the domain of this semi-circular solution curve is $(−2,2)$. \\
\emph{Food for thought:} Why are the solution curves not a full circle? 

\clearpage

\subsubsection{Zero Isoclines}

Isoclines organize the slope field.
The $0$-isocline, also called the \emph{\color{blue}nullcline}, is especially helpful.
The critical points of all solutions to the DE lie on the $0$-isocline.\\

\begin{example}
  For $y'=y^2-x$, what is the $0$-isocline?
\end{example}

\Solution Here $f(x, y):= y^2 - x$, so the $0$-isocline is the curve $y^2 - x = 0$,
which is a parabola concave to the right. At every point of this parabola,
the slope of the solution curve is $0$.

\begin{figure}[ht!]
  \centering
  \includegraphics[width=0.5\textwidth]{images_u2s1_0isocline}
  \caption{$0$-Isocline}
\end{figure}

\clearpage

\begin{example}
  For $y′=y2−x$, where are the points at which the slope of the solution curve is positive?
\end{example}

\Solution This will be the region in which $f(x,y)>0$.
The $0$-Isocline $f(x,y)=0$ divides the plane into regions, and $f(x,y)$ has constant sign on each region.
To test the sign, check one point in each region.
Recall $f(x,y) = y^2-x.\, \,$ Since $\, f(-1,0)=0^2-(-1)>0,\,$, by continuity it follows that
$f(x,y)>0\,$ in entire rerion to the left of the parabola.
Similarly, since $\, f(1,0)=0^2-(1)<0,\,$ it follows that $f(x,y)<0\,$
in the region to the right of the parabola.
Therefore, the answer is that the slope of the solution curve is negative in the region to the right of the parabola, and the slope of the solution is positive in the region to the left.

\begin{figure}[ht!]
  \centering
  \includegraphics[width=0.5\textwidth]{images_u2s1_0isoclineshaded}
  \caption{$0$-Isocline and Slope Region}
\end{figure}

\clearpage

\subsubsection{Another worked example}

\paragraph{Example}
A slightly more complicated example
is going to be,

\begin{equation*}
  y' = 1 + x -y
\end{equation*}

It's not a lot more complicated.
And as a computer exercise, you'll work with still more complicated ones.
But here, the isoclines would be what?
Well, I set that equal to $C$.
\begin{equation*}
  y = 1 + x - C = x + (1 - C)
\end{equation*}

So there is the equation of the isocline.
Let's quickly draw the direction field and see what--
notice, by the way, it's a simple equation but you \textit{cannot} separate variables.
So I will not, today at any rate, be able to check the answer.
I will not be able to get an analytic answer.
All we'll be able to do now is get a geometric answer.
But notice how relatively quickly one can get it.
So I'm feeling for how the solutions behave to this equation.\\

All right, let's see. What should we plot first?
Let's do $C = 0$ first.
That's the line. $y = x + 1$.
So all isoclines are in orange.
If so, when $C = 0,\,$ and $y = x +  1$.
So let's say it's this curve.\\
How about $C = -1$?
Then it's y equals x plus 2. It's this curve.
Let's label it down here $C = -1$.
$C = -2$ would be $y = x + 2$. \\
Well, how about the other side? If $C = 1$, well, then it's going to go through the origin.
Looks like a little more room down here.
That one must-- so if this is going to be $C = 1$,
then I sort of get the idea. $C = 2$ will look like this. \\
They're all going to be parallel lines
because all that's changing is the $y$-intercept as I do this thing.

\begin{figure}[ht!]
  \centering
  \includegraphics[width=0.5\textwidth]{image-example_isoclines}
  \caption{Isoclines}
\end{figure}

\clearpage

All right, let's put in the line elements.
All right, $C = -1$. These will be perpendicular.
$C = 0$, like this. $C = 1$-- oh, this is interesting.
I can't even draw in the line elements because they seem to coincide with the curve itself,
with the line itself.
They lie along the line, and that makes it hard to draw them in.
How about $C = 2$?
Well here, the line elements will be slanty.
They'll have slope $2$, so pretty slanty up.
And I can see for $C = 3$, in the same, way they're going to be even more slantier up.
And here, they're going to be even more slanty down.
This is not very scientific terminology or mathematical, but you get the idea.
OK, so there's our quick version of the direction field.

\begin{figure}[ht!]
  \centering
  \includegraphics[width=0.5\textwidth]{image-example_direction_field}
  \caption{Direction Field}
\end{figure}

All we have to do is put in some integral curves now.
Well, looks like it's doing this. It gets less slanty. Here, it levels out, has slope $0$.
And now in this part of the plane, the slope seems to be rising, so it must do something like that.
This guy must do something like this.
I'm a little doubtful of what I should be doing here $C = 1$.
How about going from the other side?
Well, it rises, gets a little-- should it cross this?
What should I do?
Well, there's one integral curve which is easy to see.
It's this one.\\
This line, $C = 1$is both an isocline and an integral curve.

\begin{figure}[ht!]
  \centering
  \includegraphics[width=0.5\textwidth]{image-example_integral_curve}
  \caption{Direction Field}
\end{figure}

It's everything except drawable.
So OK, you understand this is the same line.
It's both orange and pink at the same time,
but I don't know what combination color
that would make.
It doesn't look like a line, but you know.

\clearpage

Now the question is what's happening in this corridor?
In the corridor-- that's not a mathematical word either--
between the isoclines for, well, what are they?
They're the isocline for $C =  2$ and $C = 0$.\\

\begin{wrapfigure}{r}{5.5cm}
  \includegraphics[width=5.5cm]{image-integral_curve_corridor}
  \caption{Integral Curve on the Corridor}
\end{wrapfigure}

How does that corridor look? Well, it's something like this.
Over here, the lines are all look like that. And here, they all look like this.
The slope is $2$. And a hapless solution gets in there.
What's it to do?\\
Well, do you see that if a solution gets in that corridor,
an integral curve gets in that corridor, no escape is possible?
It's like a lobster trap. You know, the lobster can walk in, but it cannot walk out
because the things are always going in. How could it escape?
Well, it would have to double back somehow.
And remember, to escape on the left side, it must be going horizontally.
But how could it do that without doubling back first and having the wrong slope?
The slope of everything in this corridor is positive,
and to double back and escape it would, at some point, have to have negative slope.
It can't do that. Well, could it escape on the right hand side?
No, because at the moment when it wants to cross, it will have to have a slope less than this line,
but all these spiky guys are pointing up. It can't escape that way either.
So no escape is possible. It has to continue on there. But more than that is true.\\
\emph{So solution can't escape}.
Once it's in there, it can't escape. \\

\emph{\color{orange}Note on previous paragraph: }
In previous paragraph, all solutions are eventually trapped between the isoclines
$C = 0$ and $C = 2$.
These isoclines are sometimes called \emph{\color{blue}fences}.
Once solutions are trapped inside of the fence, they do not escape.

\clearpage

\subsubsection{Existence and uniqueness revisited}
\paragraph{Principles of existence and uniqueness}
And because there are two principles involved here
that you should know, that help a lot at drawing these pictures.
\emph{Principle number one} is that two integral curves cannot cross at an angle.
Two integral curves can't cross, I mean by crossing at an angle like that.
I'll indicate what I mean by a picture like that.
Now, why not?\\

This is an important principle.
They can't cross, because if two integral curves
are trying to cross, because it's an integral curve,
because it has this slope.
And the other integral curve has this slope.
And now they fight with each other.
What is the true slope at that point?
Well, the direction field only allows you to have one slope.
If there's a line element at that point,
it has a definite slope. And therefore, it cannot have both this slope and that one.
It's as simple as that.
So the reason is \emph{you can't have two slopes}.
The direction field doesn't allow it.
Well, that's a big, big help, because if I know here's
an integral curve and if I know that none of these other pink integral curves are allowed to cross it,
how else can I do-- well, they can't escape. They can't cross.

\begin{wrapfigure}{r}{5.5cm}
  \includegraphics[width=5.5cm]{image-existence_uniqueness}
  \caption{Integral Curves}
\end{wrapfigure}

It's sort of clear that they must get closer and closer to it.
I'd have to work a little to justify that, but I think nobody would have any doubt of it who
did a little experimentation.
In other words, all these curves join that little tube
and get closer and closer to this line, $y = x$.
And therefore, without solving the differential equation,
it's clear that all of these solutions, how do they behave?
As $x$ goes to $\infty$, they become asymptotic to--they become closer and closer to--
the solution $x$.
Is $x$ a solution?
Yeah, because $y = x$ is an integral curve.
Is $x$ a solution?
Yeah, because if I plug in $y = x$, I get what?
\begin{align*}
  y' &= 1 + x - y \\
  x' & = 1 + x - x \\
  1 & = 1  
\end{align*}
So this is a solution.
Let's indicate that it's a solution. 
So analytically, we've discovered an analytic solution
to the differential equation, namely $y = x$, just by this geometric process. 

\clearpage

Now, there's \emph{one more principal} like that,
which is less obvious, which is less obvious, but you do have to know it.
And so they're not allowed to cross. That's clear.
But it's much, much, much less obvious that \emph{two integral curves cannot touch}.
That is, they cannot even be tangent. Two integral curves cannot be tangent.
I'll indicate that by the word touch, which is what a lot of people say.
In other words, if this is illegal, so is this.

\begin{wrapfigure}{r}{5.5cm}
  \includegraphics[width=5.5cm]{image-existence_uniqueness2}
  \caption{Two integral curves cannot touch}
\end{wrapfigure}

Can't have it.
Without that, for example, I might feel that there would be nothing in this to prevent
those curves from joining.
Why couldn't these pink curves join the line $y  = x$?
It's a solution.
They just hitch a ride, as it were.
The answer is they cannot do that, because they have to just get asymptotic to it ever,
ever closer.
They can't join $y = x$, because at the point where they joined, you'd have that situation.
Now, why can't you have this?
That's much more sophisticated than this.\\

And the reason is because of something called the \emph{\color{blue}existence and uniqueness theorem}.
Existence and uniqueness theorem, which says through a point $(x_0,\, y_0),\,$
that $y' =  f(x, y)$ has only one and only one solution.
One has one solution-- in mathematics speak, that means at least one solution.
It doesn't mean it has just one solution, OK?
That's mathematical convention.
It has one solution-- at least one solution.
But the killer is only one solution.
That's what you have to say in mathematics if you
want just one-- one and only one solution through the point $(x_0, y_0)$.\\
So the fact that it has one, that's the existence part.
The fact that it has only one is the uniqueness part of the theorem.
Now, like all good mathematical theorems,
this one does have hypotheses.
So this is not going to be a course--
I warn you, those of you with theoretically inclined--
very rich in hypotheses, but you do have to-- the hypotheses
for this one are that $f(x, y)$ should be a \emph{continuous function}.
Now, you know polynomial, sines, stuff like--
should be continuous in the vicinity of that point.
That guarantees existence.
And what guarantees uniqueness is the hypothesis that you
would not guess by yourself.
Neither would I. What guarantees the uniqueness is that also its partial derivative with respect
to $f_y(x, y)$ should be continuous near $(x_0, y_0)$.

\clearpage

We have already discussed the existence and uniqueness theorem for linear ODEs.
Here is a version of the theorem that also works for first order nonlinear ODEs. \\

\begin{theorem}
  Existence and uniqueness theorem for a first order (linear or nonlinear) ODE Consider a first order ODE
  \begin{equation*}
    y' = f(x,y)
  \end{equation*}
\end{theorem}

For any point $\, (x_0, y_0),\,$ if $\, f(x,y)\,$ and $\, \frac{\partial f}{\partial y}\,$
are continuous $\, (x_0,y_0),\,$ then there is a unique solution to the first order DE through the point
$\, (x_0, y_0).\,$\\

As a consequence of uniqueness, we have the following two geometric features:
\begin{enumerate}
\item Solutions curves cannot cross.
\item Solutions curves cannot become tangent to one another; that is, they cannot touch.
\end{enumerate}

Use the Mathlet below to investigate the behavior of solution curves.\\

This is a game to find a solution curve through a specific point called the target.
If you click on the target itself, the mathlet will draw the solution curve through it.
When you click on another point, the mathlet will draw the solution curve through that point.
You can try to aim for the solution through the target by clicking on other points than the target.\\

You may notice in some examples, the consequences of uniqueness seem to fail. Why is that the case?

\href{http://mathlets.org/mathlets/solution-targets/}
{SOLUTION TARGETS}

\clearpage

\subsubsection{When Existence and Uniqueness fails}

\begin{problem}
  Find the general solution to the ODE
  \begin{equation*}
    xy' = y -1
  \end{equation*}
\end{problem}

\begin{align*}
  x \frac{dy}{dx} &= y-1 \\
  \frac{dy}{y - 1} &= \frac{dx}{x} \\
  \ln |y - 1| &= \ln |x| + c \\
  |y - 1| &= C|x| \qquad (C > 0) \\
  y - 1= \pm Cx \qquad (C > 0) \\
  y = 1 + Cx \qquad (c \neq 0). 
\end{align*}

To bring back the solution $y=1,\,$ we allow $C = 0\,$ as well.\\
Below paragraph will plot the solution to the DE $xy' = y -1$. 

\paragraph{When existence and uniqueness fail}
Let's write this in a more human form.

\begin{equation*}
  y = 1 + Cx
\end{equation*}

All right, let's just plot those.
So these are the solutions.
Pretty easy equation, pretty easy solution method.
Just separation of variables.
What do they look like?
Well, these are all lines whose intercept is at $1$.
And they have any slope whatsoever.
So these are the lines that look like that.\\

\begin{wrapfigure}{r}{5.5cm}
  \includegraphics[width=5.5cm]{image-function_graph}
  \caption{Graph of $y = 1 + Cx$}
\end{wrapfigure}

OK, now let me ask-- existence and uniqueness.
\emph{Existence}-- through which points of the plane
does the solution go?
Answer-- through every point of the plane, through any point
$y < 1$, I can find one and only one of those lines,
except for these stupid guys here on the stalk of the flower.
Here, for each of these points, there is no \emph{existence}.
There is no solution to this differential equation, which
goes through any of these wiggly points on the y-axis,
with one exception.
This point ,$y = 1,\,$is oversupplied.
At this point, it's not existence that fails.
Its \emph{uniqueness} that fails. No uniqueness.

There are lots of things which go through here.
Now, is that a violation of \emph{the existence and uniqueness theorem}?
It cannot be a violation, because a \emph{theorem has no exceptions}.
Otherwise, it wouldn't be a theorem.
So let's take a look.
What's wrong? We thought we solved it, modulo putting the absolute value
signs on the log.
What's wrong with the answer?
What's wrong is to use the theorem,
you must write the differential equation in standard form,

\begin{equation*}
  y' = f(x, y)
\end{equation*}

Let's write the differential equation the way
we were supposed to.

\begin{equation*}
  \frac{dy}{dx} = \frac{1 - y}{x}
\end{equation*}

And now I see the right hand side is not \emph{continuous}.
In fact, not even defined when $x = 0$, along the $y$-axis.
And therefore, the existence and uniqueness is not guaranteed along the line $x$ equals $0$,
the $y$-axis.
And in fact, we see that it failed.
Now, as a practical matter, the way existence and uniqueness
fails in all ordinary life work with differential equations
is not through sophisticated examples
that mathematicians can construct,
but normally because $f(x,y)$ fail to be defined somewhere.
And those will be the bad points.\\

\textbf{\color{orange}Note on previous Paragraph} 
The DE on the paragraph was $xy′=1−y,\,$
but the intended one is $xy′=y−1,\,$ which you just solved in the problem above.
Although the two equations are different,
both of them have difficulty of existence and uniqueness at $x=0\,$
because the denominator of the right hand side of each ODE in standard form is $0$ at $x=0$.\\

At the points where the hypotheses of the existence and uniqueness theorem fail,
the conclusion of the theorem may also fail. Here is another example demonstrating this. \\

\begin{example}
  Draw the solution curves for $y' = \frac{2y}{x}$. 
\end{example}

\Solution Here $f(x, y) = \frac{2y}{x},\,$ which is undefined when $x = 0,\,$
so things might go wrong along the $y$-axis, and in fact they \emph{do} go wrong. \\

Solve the ODE by separation of variables:

\begin{align*}
  \frac{dy}{dx} &= \frac{2y}{x} \qquad (x \neq 0) \\
  \frac{dy}{y} &= 2 \frac{dx}{x} \qquad (\text{assuming also that}\, y \neq 0) \\
  \int \frac{dy}{y} &= 2 \int \frac{dx}{x} \\
  \ln |y| &= 2 \ln |x| + C \quad (\text{for some constant}\, C) \\
  y &= \pm e^{2 \ln |x| + C} \\
  y &= \pm |x|^2 e^C \\
  y &= c x^2 \qquad (x \neq 0),                       
\end{align*}

\clearpage
where $\, c\,  \colon =\pm e^ C,\,$ which can be any nonzero real number.
To bring back the solution $y=0,\,$ we allow $c=0\,$ as well.

\begin{figure}[ht!]
  \centering
  \includegraphics[width=0.5\textwidth]{images_u2s1_halfparabolas}
  \caption{Several half parabolas as solution curves to $y' = \frac{2y}{x}$}
\end{figure}

Weird behavior happens along $x = 0\,$ where $y' = \frac{2y}{x}\,$ is not even defined:

\begin{itemize}
\item Through any point $\, (0,b)\,$ on the $y$-axis, there is \emph{no} solution curve.
  the existence theorem does not apply.
\item Geometrically, the parabolas become tangent at the origin.
  This would be ruled out by uniqueness if the uniqueness theorem applied.
  The full parabolas are not solutions;
  the solution curves are half parabolas defined for either all $\, x<0\,$ or all
  $\, x>0$. 
\end{itemize}

Both existence and uniqueness apply to every point outside the $y$-axis.
The rest of the plane (outside the $y$-axis) is covered with good solution curves,
one through each point, none touching or crossing the others.

\clearpage

\subsubsection{Interval of validity}

The characteristics of solution curves
that we have been discussing are true for both linear and nonlinear equations.
But there is often a big difference
between the domain of definition of solutions to linear and nonlinear DEs. \\

For first order \emph{linear DEs} of the form

\begin{equation*}
  \displaystyle  \displaystyle y'=f(x,y)
  \qquad \text {where}\, f(x,y)=q(x)-p(x)y,
\end{equation*}

recall that any solution defined at $\, x = a \,$ is defined on the \emph{entire interval}
which contains $a$ and on which $p(x)$ and $q(x)$ are continuous.
(Note that $p$ and $q$ being continuous is equivalent
to the hypothesis of the general existence/uniqueness theorem presented
in this lecture, namely, that $f$ and $\, \frac{\partial f}{\partial y}\,$ are continuous.)\\

For \emph{nonlinear DEs,} solutions do not have to be defined on the entire interval
on which $f$ and $\, \frac{\partial f}{\partial y}\,$ are continuous.
Consider the following examples. \\

\begin{example}
  Draw the solution curve to $y'=y^2\,$ that satisfies the condition $\, y(0)=1\,$.
  What is the domain of definition of this solution curve?
\end{example}

\Solution This DE is nonlinear, so our methods of solving linear DEs do no apply.
We solve it by separation of variables:

\begin{align*}
  \frac{dy}{dx} &= y^2 \\
  y^{-2} dy &= dx \qquad (y \neq 0) \\
  \int y^{-2} dy &= \int dx \\
  \frac{-1}{y^{-1}} &= x + C \qquad (\text{for some constant\,} X)
\end{align*}

The condition $\, y(0) = 1\,$ leads to

\begin{equation*}
  1 = \frac{-1}{0 + C} \quad \Longrightarrow \quad C = -1, 
\end{equation*}

which gives the solution

\begin{equation*}
  y = \frac{-1}{x -1} = \frac{1}{1 - x} \qquad \text{where} \, -\infty <x<1.
\end{equation*}

\clearpage

Therefore, here is the solution curve:

\begin{figure}[ht!]
  \centering
  \includegraphics[width=0.5\textwidth]{images_u2s1_halfhyperbola}
  \caption{Solution Curve $y' = y^2$}
\end{figure}

The solution approaches $\infty$ as $x\rightarrow 1^-$.
We say that the solution \emph{\color{blue} blows up} at $x=1.\,$
The domain of definition of the solution is $-\infty <x<1$.
It is called the \emph{\color{blue} interval of validity} of the solution.
The \emph{\color{blue} interval of validity} of a solution is the largest interval
on which it can be defined. \\

The graph if $\, y = \frac{1}{1-x}\,$ consists of both branches of the hyperbola.
But \emph{\color{orange} be careful}, the full hyperbola is not a solution;
only the branch through the point $\, (x,y)=(0,1)\,$ is. \\

For nonlinear DEs, the interval of validity of a solution cannot be read off from the equation.
It can be much smaller than the domain on which the equation is defined.\\

\begin{exercise}
  Interval of validity
\end{exercise}

Let $y(x)$ be the solution to $\dot y = x y^4$ with initial condition $y(0) = 1$.
Find the interval of validity $\, a<x<b\,$ of $y(x)$.

The DE is nonlinear, solve it by separation of variable:

\begin{align*}
  \frac{dy}{dt} &= x y^4 \\
  y^{-4} dy &= x dt \qquad (y \neq 0) \\
  \int y^{-4} dy &= \int x dt \\
  \frac{-1}{3} y^{-3}  &= \frac{1}{2} x^2 + C \quad (\text{for some constant}\, C) \\
  y^{-3} &= \frac{-3}{2} ( x^2 + C ) \\
  y^{3} & = \frac{-2}{3} \frac{1}{( x^2 + C )}
\end{align*}

\clearpage

\begin{align*}
  y &=  \left( \frac{-2}{3} \frac{1}{\left( x^2 + C \right)} \right)^{1/3}
\end{align*}

The condition $\, y(0) = 1\,$ leads to

\begin{equation*}
  1 = \left( \frac{-2}{3} \frac{1}{\left( 0^2 + C \right)} \right)^{1/3}
  \Longrightarrow C = \frac{-2}{3}, 
\end{equation*}

which gives the solution

\begin{equation*}
  y = \left( \frac{-2}{3} \frac{1}{( x^2 - 2/3)} \right)^{1/3}
  \quad \text{where\,} -\sqrt {\frac{2}{3}}<x<\sqrt {\frac{2}{3}}. 
\end{equation*}

In fact, $y\rightarrow \pm \infty$ as $x\rightarrow \pm \sqrt {2/3};\, \,$
that is the solution blows up at $x=\pm \sqrt {2/3}$

\clearpage

\subsubsection{Review}

\paragraph{Worked example: Slope fields, isoclines, and solution curves}
So today, I'd like to tackle a problem on direction fields.\\
So given a differential equation,

\begin{equation*}
  y' = \frac{-y}{x^2 + y^2}
\end{equation*}
We're asked to

\begin{enumerate}
\item Sketch the direction field. \\
  So to start off this problem, we're
  asked to sketch the direction field.
  Now, when asked to sketch a direction field,
  the first thing to do is to look at a couple isoclines.\\
  So what is an isocline?
  Well, it's a curve where the derivative y
  prime $,\,y' = m, \,$which $m$ is some constant.\\
  So what are the isoclines for this differential equation?
  
  \begin{equation*}
    \frac{-y}{x^2 + y^2} = m 
  \end{equation*}

  Now, of particular interest, there's
  a very special isocline, which is usually the easiest to plot,
  and that's a \emph{nullcline}.
  And this is just the special case where $\, m = 0$.\\
  So what's the nullcline claim for this ODE?
  Well, when $\, m = 0, \,$ the only way that $\,y' = 0 \,$
  is when $\,y  = 0$.\\
  So the nullcline for this ODE is $\,y = 0$. \\
  Now, for more generalized isoclines,
  we're left with this relation between $x$ and $y$.
  And we can massage this expression to see exactly what the isoclines are.\\
  So specifically, I'm going to multiply through
  by $\, x^2 + y^2\,$.
  And I'm going to divide by $m$.
  
  \begin{equation*}
    -\frac{1}{m}y = x^2 + y^2
  \end{equation*}
  
  And we see that we have a quadratic in $x$ and $y$, and we have one linear term.
  So whenever we have a relation like this
  and we want to understand what it looks like,
  typically, the approach is just to complete the square.
  So I'm going to bring the $1/m y$ to the other side of the equation.

  \begin{equation*}
    x^2 + \underbrace{y^2 + \frac{1}{m} y}_{\text{complete square}} = 0
  \end{equation*}
  
  And I'm going to combine it with $\,y \,$ squared to complete the square.
  And when we do this, we obtain the following equation.

  \begin{equation*}
    x^2 + (y^2 + \frac{1}{2m})^2 = \frac{1}{4m^2}
  \end{equation*}
  
  And we recognize this equation as the equation for a circle.\\
  Specifically, it's a circle that's
  centered at $\, x = 0, \, $ and $\, y = \frac{-1}{2m}\, $.
  And in addition, the circle has a radius
  $\, r^2  = \frac{1}{4m^2},\, $ which  means its radius is
  $\, \frac{1}{2 |m|}\, $.\\
  
  So why don't we take a look and plot a couple of isoclines?
  So as I mentioned before, typically, the first isocline   that we should plot is the nullcline,
  which is $\, m = 0\, $.
  And we know that this is $\, y = 0 \, $.
  So all along the line $y$ is equal to $0$.
  I'm just going to draw dashes that correspond to a slope of $y$ is equal to $0$.
  For the other isoclines, we just have to pick some values of $m$ and start plotting.\\
  So I'll take the value $m$ is equal to $-1$.
  And when $\, m  = -1,\, $ we have a circle which is centered at $0$ and $1/2,\,$
  with radius $1/2$.
  And at every point along this circle,
  we just draw a little slope of $-1$. So at every point along this curve,
  the solution has slope $-1$.\\
  Now, in addition, we can also get another circle, which
  is centered at $- 1/2$.
  And this corresponds to the isocline of $\, m = 1\,$. 
  And every point on this circle has slope $1$.
  Those should all be the same.\\
  And now we can pick some other values of $m$.
  So for example, $\, m = 2\, $.
  If $m$ is equal to $2$, then we have a circle
  which is centered at $- 1/4$ and radius $1/4$.
  So it might look something like this.
  And $m$ is equal to $-2$ might look something like this.
  So this is $m$ is equal to $- 2$.
  And this is $m$ is equal to $2$.
  And the slope on this curve is going to be steeper.
  It's going to be $-2$.
  And the same goes for this circle.
  And then lastly, I can draw a sketch of $\, m = -3/4 ,\,$ which would go up something like this.
  And that would have a slightly weaker slope, like this.

  \begin{figure}[ht!]
    \centering
    \includegraphics[width=0.5\textwidth]{image-review_direction_field}
    \caption{Direction Field of $y' = \frac{-y}{x^2 + y^2}$}
  \end{figure}

  So notice how the collection of isoclines are a family of circles that all are tangent to the origin.
  And in fact, if we think about it, the nullcline,
  the $m$ equals $0$ line, is, actually, in some sense, a limit circle where we take the radius
  and the center going to infinity.
  So as the circles become larger and larger, they tend to approach this line, $\, y = 0\,$.

  \clearpage

\item For solution through $\, y(0) =1$\\

  So I'll just sketch what this solution curve might look like.
  And it's going to start off at a slope of $-1$,
  and then it's going to hit these circles, which have a steeper slope.
  And then, eventually, it's going to go   through the circles with the steeper slope
  and come back to the circle that has slope $-1\, ,$  and eventually decay outwards.\\
  So this is what the solution curve might look like.

  \begin{figure}[ht!]
    \centering
    \includegraphics[width=0.5\textwidth]{image-review_solution_curve}
    \caption{Solution Curve $y' = \frac{-y}{x^2 + y^2}$}
  \end{figure}

  \begin{itemize}
  \item Why is $\,y(x) > 0 \,$ for $x > 0$? \\
    Well, we see that this solution curve stays in the upper half plane.
    And we note that $\, y = 0 \,$, which is the nullcline, is actually a very special curve.
    Not only is it the nullcline, but it's actually a solution  to the differential equation.
    So if we look back at the differential equation,
    we see that $\, y = 0 \,$ has $0$ derivative and the right-hand side of the differential equation
    is also $0$.\\
    So this is not only a nullcline, but it's a solution to the differential equation.
    Note that this is extremely special.
    In no way does every nullcline have  to be a solution to a differential equation.
    But in this case, we get lucky.
    Now, we know from the theory of ODEs that solution curves can't cross.
    So A solution curve that starts in the upper half plane
    can't cross another solution, which in this case is the $\, y = 0 \,$ curve.
    Hence, it must be bounded in the upper half plane for all $\, x < 0\, $.

  \item Why is $y(x)\,$ decreasing for $x > 0$?  \\
    Well, what we have to look at  the sign of y prime.
    
    \begin{equation*}
      y' = \frac{-y}{x^2 + y^2}
    \end{equation*}
    
    So as I've just argued, a solution  that starts in the upper half plane
    and goes through the point $\, y(0) = 1 \,$ stays in the upper half
    plane for all $\, x > 0 \, $.\\
    What this means is that the solution curve has $y$ bigger than $0$ for all $x$ bigger than $0$.\\
    So that means the numerator is always positive.
    The denominator is also always positive.
    So we have negative a positive number divided by a positive number.
    And so this quantity is always going to be less than $0$.\\
    So for $\, x > 0 \,$, then $\, y' <  0\, $.
    Hence, the solution that starts at $\, y(0) = 1\, $ is going to monotonically decay it
    to $\, y = 0 \, $.
  \end{itemize}
  So this concludes the problem. \\    
  Now, just to recap.
  We were given an ODE.
  The standard approach when sketching direction fields is to pick a few nullclines or
  to pick a few isoclines,  and specifically pick the nullcline, sketch the isoclines. \\
  And then if you're asked to plot any integral curves
  or solutions to the differential equation,  you just simply connect the dots.
\end{enumerate}

\clearpage

\subsubsection{Optional}

\begin{exercise}
  Change of variables
\end{exercise}

Consider the nonlinear ODE:

\begin{equation*}
  \dot{y}+y^2\, t-y =0.
\end{equation*}

This is known as the \emph{Bernoulli equation}.
Make the change of variable  $\, y = \frac{1}{u} \,$
and write the DE in terms of $u$. \\
Enter the equation in the form $\, \dot{u}= f(u,t)$.

if $\, y = \frac{1}{u},\,$ then $\, \dot y = - \frac{\dot u}{u^2}\, $
by chain rule.
Therefore, the DE in terms of $u$ is :
\begin{equation*}
  -\frac{\dot u}{u^2} + \frac{1}{u^2} t - \frac{1}{u} = 0
\end{equation*}

Multiplying by $\, u^2, \,$ we have

\begin{equation*}
  - \dot u + t - u = 0
\end{equation*}

Putting it in the requested form:

\begin{equation*}
  \dot u = t - u 
\end{equation*}

Notice that this equation linear.

\clearpage

\subsection{Recitation }

\subsubsection{Mathlet Activity}

\emph{\color{blue} Practice with isoclines, slope fields, and solution Curves} \\

Consider the ODE

\begin{equation*}
  \displaystyle \frac{dy}{dx} = x-2y
\end{equation*}

On a piece of paper, draw by hand a big axis system and plot some isoclines, especially the nullcline.
Use them to illustrate the slope field.
Using the slope field, plot a few solutions.
Then, to verify your drawing, choose the ODE from the dropdown menu in the mathlet below.\\

\href{http://mathlets.org/mathlets/isoclines/}
{ISOCLINES}

\begin{question}
  For a general differential equation $\displaystyle \frac{dy}{dx} = f(x, y),\,$
  if a straight line is an integral curve, how is it related to the isoclines of the equation?
\end{question}

If there is a straight line solution $\, y=mx+b\,$ , then $\, y'(x)=m,\,$
and the straight line solution curve is (part of) the $m$-isocline.

\begin{problem}
  Straight line integral curves
\end{problem}

For the ODE

\begin{equation*}
  \displaystyle \frac{dy}{dx} = x-2y
\end{equation*}

one of the integral curves seems to be a straight line.
What straight line is it? In other words, for what $m$ and $b$ is $\, y = mx + b\,$
a solution to the given DE?\\

The straight line solution is form of $\, y=mx+b\,$ and the slop of solution is $m$.
Plug in a solution $\, y=mx+b\,$ to DE

\begin{align*}
  (mx + b)' &= x -2(mx + b) \\
  m &= x - 2mx - 2b \\
  m &= (1- 2m)x - 2b. 
\end{align*}

The coefficient of $x$ is zero, so
\begin{equation*}
  1 - 2m = 0 \qquad \Longrightarrow m = \frac{1}{2}. 
\end{equation*}

The $\, m = -2b \,$, substitute $\frac{1}{2}$ on $m$

\begin{equation*}
  \frac{1}{2} = -2b \qquad \Longrightarrow n = - \frac{1}{4}. 
\end{equation*}

So, the straight line solution is

\begin{equation*}
  y = \frac{1}{2} x - \frac{1}{4}
\end{equation*}

\begin{question}
  It seems that all the solutions become asymptotic to each other as $x \rightarrow \infty$.
  Explain why solutions get trapped between parallel lines of some fixed slope.  
\end{question}

All isoclines have the form $\, c= x-2y$, rearranging to $y = x/2 - c/2,\,$
his tells us that every isocline has slope $\frac{1}{2}$.
Thus the isoclines are all parallel lines. \\

The $\, c>1/2\,$ isoclines are the parallel lines below our straight line integral curve
$\, y=x/2 - 1/4.\, \,$
The slope is positive and greater than the slope of the isoclines themselves,
which means that every integral curve that starts below the $\, c=1/2\,$
integral curves moves upwards across the isoclines $\, c>1/2$. \\

The $\, c<1/2\,$ isoclines are the parallel lines above our straight line integral curve
$y=x/2 - 1/4.\, \,$
Any integral curve through this region must pass through the isocline
since the slope of a solution curve through the isocline is always less than the slope of the isocline.

\begin{problem}
  Critical Points
\end{problem}
\begin{enumerate}
\item Where are the critical points of the solutions of $\frac{dy}{dx}=x-2y$?
  In other words, for what values of $y_0$ does the solution $y$
  with $y(0)=y_0$ have a critical point? \\

  To find general solution using integral factor $e^{2x}$
  
  \begin{align*}
    e^{2x} y' + 2 e^{2x} y &=  e^{2x} x \\
    \int (e^{2x} y) ' dx &= \int e^{2x} x dx \\
    e^{2x} y &= \frac{1}{2} e^{2x} x  - \frac{1}{4} e^{2x} + c  \footnotemark \\
    y &= \frac{1}{2} x  - \frac{1}{4} + c e^{2x} \\
  \end{align*} \footnotetext{
    Integration by Parts \\
    From $(fg)' = f'g + fg'$, integrate both side
    $\int (fg)' = \int f'g dx + \int fg' dx$.
    Rewrite it as
    \begin{equation*}
      \int fg' dx = fg - \int f'g dx  
    \end{equation*}
    Let's do a couple of substitute
    \begin{align*}
      u = f(x) &\qquad v = g(x) \\
      du = f'(x) dx &\qquad dv = g'(x)dx
    \end{align*}
    This substitutions gives
    \begin{equation*}
      \int u dv = uv - \int v du 
    \end{equation*}
  }

  Apply condition $y(0)=y_0$,

  \begin{equation*}
    y _0 = - \frac{1}{4} + c. 
  \end{equation*}

  So, $ y _0 > - \frac{1}{4}.$
\item The critical point is a
  $y _0 = - \frac{1}{4} + c\,$, therefore $\, -\frac{1}{4} \,$ is minimum.  
\end{enumerate}

\begin{problem}
  How many critical points? 
\end{problem}

What is the maximum number of critical points a single solution can have? \\

The derivative of the general solution is

\begin{equation*}
  y' = 2 c e^{2x} - \frac{1}{2},  
\end{equation*}

so there is one critail solution.

\clearpage

\subsection{Autonomous equations }

\subsubsection{Autonomous Equations}

\emph{Objectives}
\begin{itemize}
\item Identify  \textbf{\color{blue} autonomous} ODEs.
\item Use the \textbf{\color{blue}phase line} to determine \textbf{\color{blue}stability} of solutions to autonomous ODEs.
\item Model population growth with the \textbf{\color{blue}logistic equation}
  and find qualitative behavior of solutions.
\item Determine how varying parameters in a system can change behavior of solutions using a
  \textbf{\color{blue} bifurcation diagram }.
\end{itemize}

\clearpage

\subsubsection{Definition}

\paragraph{Definition}
The topic is a special kind of differential equation which occurs a lot.
It's one in which the right-hand side doesn't have
any independent variable in it.
Now since I'm going to use as the independent variable $t$
for time, maybe it would be better to write the left-hand
side to let you know that since you won't be able to figure out
any other way what it is,

\begin{equation*}
  \frac{dy}{dt} = f(y)
\end{equation*}

and the point is that there is no t on the right-hand side.
So there's no $t$.\\
Well, there's a name for such an equation.
Now some people call it time-independent.
The only problem with that is that sometimes the independent variable isn't time.
It's something else.
We need a generic word for there being no independent variable
on the right-hand side.
So the word that's used for that is \emph{autonomous}.\\
So that means no independent variable on the right-hand side.
It's a function of $y$ alone, the dependent variable.\\
Now your first reaction should be, oh, well, big deal.
If there's no $t$ on the right-hand side,
then we can solve this by separating variables.
So why is he even talking about it in the first place?
So I admit that.
We can separate variables.
And what I'm going to talk about today is how to get critical--
how to get useful information out
of the equation above how its solutions look
without solving the equation. \\
The reason for wanting to do that is, A, it's fast.
It gives you a lot of insight.
And the actual solution --I'll illustrate one or two-- may, first place,
take you quite a while.
You may not be able, actually, to do the integrations to get an--
required in separation of variables to get an explicit solution.
Or it might simply not be worth the effort
of doing if you only want certain kinds of information
about the solution.\\
So the thing is, the problem is, therefore,
to get qualitative information about the solutions without actually having to solve the equation.\\

An \emph{\color{blue}autonomous} ODE is a differential equation
that does not explicitly depend on the independent variable.
If time is the independent variable, this means that the ODE is time-invariant.
We will use time as the independent variable in this lecture.\\

The standard form for a first order autonomous equation is

\begin{equation*}
  \displaystyle  \displaystyle \frac{dy}{dt} = f(y)
  \qquad (\text {instead of }\, f(t,y))
\end{equation*}

where the right hand side does not depend on $t$.
As usual, we write $\displaystyle \dot{y}=\frac{dy}{dt}$. \\

For example, the (nonlinear) equation

\begin{equation*}
  \displaystyle  \displaystyle \dot{y}=y-3y^2
\end{equation*}

is autonomous. We will find out later that this DE models population growth in an environment with limited resources. It is called a \textit{logistic equation}.\\

\textbf{Why is this called autonomous?} In ordinary English,
a machine or robot is called autonomous if it operates without human input.
A differential equation is called autonomous if its coefficients are not changed over time,
such as might happen if a human adjusted a dial on a machine and let it run.

\begin{exercise}
  Quiz
\end{exercise}

Which of the following are autonomous differential equations?

\begin{itemize}
\item $\dot{y}=t-y^2$
\item $\dot{x}=2x-5x^2$
\item $\displaystyle \dot{y}=\frac{3}{y}$
\item $\displaystyle \dot{x}=\frac{2x}{t}$
\item $\dot{y}=\cos y$
\end{itemize}

The right hand sides of $\, \dot{x}=2x-5x^2,\, \, \dot{y}=\frac{3}{y},\,\, \dot{y}=\cos y\,$
do not depend on the independent variable $t$.
The remaining DEs are not autonomous because their right hand sides explicitly include $t$.

\clearpage

\subsection{First properties}

Here are two consequences of the time invariance of an autonomous equation:

\begin{itemize}
\item Each isocline (in the $(t,y)$-plane) consists of one or more horizontal lines.
\item Solution curves (in the $(t,y)$-plane) are horizontal translations of one another.
  That is , it $y(t)$ is a solution, then so is $\, y(t - a) \,$ for any $a$. 
\end{itemize}


For example, here is the slope field for $\, \dot{y}=y(1-y)\,$ with the $(-1)$-isocline
and a few solution curves:

\begin{figure}[ht!]
  \centering
  \includegraphics[width=0.5\textwidth]{images_autonomousproperties2_newcolors}
  \caption{The $(−1)$-isocline (orange) consists of 2 horizontal lines;
Solution curves (blue) are horizontal translations of one another.}
\end{figure}

\textbf{Why are all translates of solutions also solutions? a non-geometric argument} \\

Let $y(t)$ be a solution to $\, \dot y = f(y) \,$.
Let $\, u = t - a, \,$ then
\begin{align*}
  \displaystyle  \displaystyle \frac{d}{dt} y(t-a)\, =\, \frac{d}{dt}y(u(t))
  \displaystyle = 
  \displaystyle \frac{dy}{du}\, \frac{du}{dt} \qquad (\text {chain rule}) \\
  \displaystyle \, f\left(y(u)\right) (1)\, =\, f\left(y(t-a)\right).
\end{align*}

Therefor $\, y(t -a)\,$ is also a solution. 


\begin{exercise}
  Horizontal isoclines
\end{exercise}

Find the $(-1)$-isocline for the DE $\, \dot y = y(y-1)\,$.
Answer by giving the equation of the lines that the isocline consists of.\\

The equation for $(-1)$-isocline is

\begin{align*}
  & \qquad \displaystyle y(1-y)\, =\, -y^2+y \\
  & \Rightarrow \qquad y^2 - y + c = 0 \\
  & \Rightarrow \qquad y = \frac{1 \pm \sqrt{1 - 4c}}{2}
\end{align*}

For $\, c<1/4,\,$ the $c$-isocline consists of the two horizontal lines
$\, y= (1\pm \sqrt {1-4c})/2.\, \,$
As expected, the isocline are horizontal for this autonomous equation. \\
\textbf{Note:} $(1/4)$-isocline consists of the single horizontal line
$\, y=1/2\,$ and there are no isoclines for
$c>1/4$. \\

\textbf{\color{orange}Food for thought} \\
The linear DE $\, \dot{y}=ay\,$ is autonomous, and its solutions are
$\, y(t) = ce^{at}\,$ where $c$ is any constant.\\
Are the solutions horizontal translations (or time shifts) of one another?

\clearpage

\subsubsection{Critical points}

\begin{definition}
  The values of $\, y\,$ at which $\, f(y) = 0 \,$ are called
  the \emph{\color{blue} critical points} or \emph{\color{blue} equilibria} of
  the autonomous equation $\, \dot y = f(y)\,$. 
\end{definition}

If $y_0$ is a critical point of an autonomous equation, then $\, y = y_0\,$
is a constant (or horizontal, or equilibrium) solution,
because the derivative of a constant function is $0$.
The $0$-isocline of an autonomous equation consists of all the constant solutions.

\begin{figure}[ht!]
  \centering
  \includegraphics[width=0.5\textwidth]{images_autonomous0isocline_newcolors}
  \caption{The constant solutions $\,y=y_0\,$ and $\,y = y_1\,$ 
    (where $y_0$ and $y_1$ are critical points) divide 
    the $(t,y)$-plane plane into "up" and "down" regions.}
\end{figure}

Recall that for any first order DE, the $0$-isocline divides the $(t,y)$-plane to
``up'' regions, where $\, f > 0\,$ and the solutions are increasing,
and ``down'' regions where $\, f < 0\,$ and the solutions are decreasing.
For autonomous equation, the qualitative behaviour of all solutions is
encoded by the critical points and the sign of $f(y)$ in the intervals
between the critical points. \\

To find the qualitative behaviour of solution to $\, dot y = f(y),\,$ we will follow two steps:

\begin{itemize}
\item Find the critical points. That is, solve $\, f(y) = 0.\,$
\item Determine the intervals of $y$ in which $\, f(y) > 0\, $ and in which $\,f(y) < 0\,$.
  There are interval in which solutions are increasing and decreasing respectively. 
\end{itemize}

We will use these steps in the next two examples.

\clearpage

\subsubsection{First example}

\paragraph{Critical points and qualitative behavior of solutions}
All right, so let's do our bank account.
So $y$ is money in the bank, in the account, in the bank account.
$r$ is the interest rate.
Let's assume it's a continuous interest rate.
All banks nowadays pay interest continuously.
The continuous interest rate.

\begin{align*}
  &y = \text{Money in the bank} \\
  &r = \text{Continuous interest rate}
\end{align*}

So if that's all there is and money is growing at that-- the differential equation says
that the rate at which it grows is equal to $r$,
the interest rate, times the principal, the amount that's in the bank at that time.

\begin{equation*}
  \frac{dy}{dt} = ry 
\end{equation*}

So that's the differential equation that governs that.
Now, that's, of course, just the next solution is simply an exponential curve.
There's nothing more to say about it. Now, let's make it more interesting. \\

Let's suppose there's a shifty teller in the bank.
And your money is being embezzled from your accountat a constant rate.\\
So let's let w equal $w$ is the rate of embezzlement, thought of as continuous.
So every day, a little bit of money is sneaked out of your account, because you're not
paying any attention to it. You're off skiing somewhere and not
noticing what's happening to your bank account.

\begin{equation*}
  \frac{dy}{dt} = ry - w
\end{equation*}
So since $w$ is the rate--
it's the time rate of embezzlement-- I simply subtract it from this.
It's not $w$ times $y$, because the embezzler
isn't stealing a certain fraction of your account.
It's simply stealing a certain number of dollars every day,
the same number of dollars being withdrawn from the accountant.\\


OK, now of course, you could solve this.
This separates variables immediately.
You get the answer and there's no problem with that.
But let's analyze the behavior of the solutions
without solving the equation by using these two points.
So I want to analyze this equation using the method of \emph{critical points}.\\

So the first thing I should do is-- so here's

\begin{equation*}
  \frac{dy}{dt} = ry - w, 
\end{equation*}

our equation-- is find the critical points.
Notice it's an autonomous equation, all right,
because there's no $t$ on the right hand side.
OK, so the critical points--
well, that's where

\begin{equation*}
  ry - w = 0.
\end{equation*}

In other words, there's only one critical point.
And it occurs when $\, y =  w/r\,$.
So that's the only critical point.\\
Now, I want to know what's happening to the solution.
So in other words, if I plot, I can right away,
of course, negative values aren't particularly interesting here.
There is definitely a horizontal solution.
And it has the value--it's at the height $\, w/r \,$.\\
That's a solution.
Question is, what do the other solutions look like?
Now, watch how I make the analysis, because I'm going to use two now.
So if to find critical point was step one, then step two--
what do I do?
Well, I'm going to graph $\, f(y)$.
Well, $\, f(y) =  ry - w\, $.
What does that look like?

\begin{wrapfigure}{r}{5.5cm}
  \includegraphics[width=5.5cm]{image-autonomous_example1_graph}
  \caption{Graph of $\frac{dy}{dt} = ry - w$}
\end{wrapfigure}

OK, so here is the $y$-axis.
Notice the $y$-axis is going horizontally,
because what I'm interested in is the graph of this function.
So that function-- what do I call the other axis?
I'm going to use the same terminology that's
used on the little visual that describes this.
And that's $\, dy \,$.
You could call it the $\, dy/dt$-axis, because it's, so
to speak, the other variable.
That's not great either.
But worst of all would be introducing yet another letter for which that we
would have no use whatever.
So let's think of it--
we're plotting now the graph of $f(y)$.
The $f(y)$ is this function $\, ry - w \,$.
Well that's a line.
Its intercept is down here at $w$.
And so the graph looks something like this.
It's a line. This is the line $\, ry - w\,$. It has slope $r$.
Well, what am I going to get out of that line?
Just exactly this.
What am I interested in about that line?
Nothing other than where is it above the axis
and where is it below.
This function is positive over here.
And therefore, I'm going to indicate that symbolically,
by putting an arrow here.
The meaning of this arrow is that $\,y (t) \,$ is increasing.
See where it's the right-hand side of that last board?
$y(t)$ is increasing when $f(y)$ of y is positive. 
And therefore, to the right of this point, it's increasing.
Here, to the left of it, $f(y)$ negative.
And therefore, over here, it's going to be decreasing.
What point is this, in fact?
Well, that's where it crosses the axis.
That's exactly the critical point, $w/r$.
Therefore, what this says is that a solution,
once it's bigger than $y/r$, it increases.
And it increases faster and faster because this function is higher and higher.
And that represents the rate of change.
So in other words, once a solution--
let's say a solution starts over here at time $0$.
So this is the $t$-axis and here is the $y$-axis.
So now I'm plotting solutions.
If it starts at t equals $0$, above this line, that
is starts with the value $w/r$, which is bigger than $0$,
the value bigger than $w/r$, then it increases and increases
faster and faster.\\
If it starts below that, it decreases and decreases
faster and faster.
Now, in fact, I only have to draw two of those.
Because what do all others look like ?
They're \emph{translations}.
All the other curves look exactly like those.
They're just translations of them.
This guy, if I start closer, it's still going to decrease. \\

\textbf{Phase line} \\
We repeat the example in the previous paragraph to introduce phase lines explicitly.\\
Recall the first order autonomous DE in the video:

\begin{equation*}
  \dot y = f(t) = ry - w \qquad (r, w > 0). 
\end{equation*}

The variable $y$ models the amount of money (in dollars) in a saving accout;
$r$ is the interest rate (in units of $\text{year}^{-1}$); and $w$ (in dollars per yaer)
is the rate of money being stolen out of accout. \\
To find qualitative behavior of the solution, we follow the two steps described before.

\begin{enumerate}
\item Find the critical points.

  \begin{equation*}
    f(y) = ry - w = 0 \quad \Rightarrow \quad y = \frac{w}{y}
  \end{equation*}

  So, $\, y = w/r\, $ is the constant solution.

\item Determine the intervals of $y$ in which $\, f(y) > 0 \,$ and in which $\, f(y) < 0 \,$.
  One way to do this by graphing $f(y)$.

  \begin{figure}[ht!]
    \centering
    \includegraphics[width=0.5\textwidth]{images_autonomouslinear}
    \caption{Graph of $f(y)$}
  \end{figure}

  For $\, y > w/r,\, f(y) > 0\,$ and any solution $y(t)$ is increasing;
  for $\, y < w/r\, , f(y) < 0 $ and any solution $y(t)$ is decreasing. \\
  
\end{enumerate}

We can summarize the above information by adding arrows on the (horizontal)
$y$-axis as in the previous paragraph. The critical poimts divides the $y$-axis
into intervals.
We place a right arrow (int the $+y$ direction) to each interval of $y$ where
$\, f(y) > 0 \,$ and and any solution $y(t)$ is increasing.
Similarly, we place a left arrow (in the $−y$ direction) to each interval
where $\, f(y)<0 \,$ and solutions are decreasing.
The $y$-axis with this extra information is called a phase line.

\begin{figure}[ht!]
  \centering
  \includegraphics[width=0.5\textwidth]{images_u2s2_horizontalphaseline1}
  \caption{Phase line of $\dot y = ry - w\, (r,w > 0)$}
\end{figure}

\clearpage

If the phase line is drawn vertically, the qualitative behaviour of solution curves can be read directly from it, as you see in the diagram below.

\begin{figure}[ht!]
  \centering
  \includegraphics[width=0.5\textwidth]{images_autonomousphaselinesolutioncurves2}
  \caption{Vertical phase line (left); Solution curves on the $ty$-plane (right)}
\end{figure}

\begin{example}
  Use the phase line above to determine what happens to solutions
  with different initial conditions as time passes.
\end{example}

\Solution \\

\begin{itemize}
\item If $\, y(0) = w/r,\,$ which is the critical point,
  then the solution will be $\, y(t)=w/r\,$ for all $t$.
  This corresponds to the constant solution shown on $(t,y)$-plane in the figure above.
\item If $\, y(0)>w/r,\, $ then the up arrow (in the positive $y$ direction) on the phase line
  tells us that as t increases, $y(t)$ increases and will tend to $+ \infty$.
  This corresponds to a solution curve above the horizontal solution,
  such as the one shown on the $(t,y)$-plane above.
\item If $\, y(0)<w/r, \,$ then the down arrow (in the negative $y$ direction)
  on the phase line says that as $t$ increases, $y(t)$ decreases and
  will tend to $- \infty$.
  This corresponds to a solution curve below the horizontal solution,
  such as the one shown on the $(t,y)$-plane above.
\end{itemize}

\begin{exercise}
  Maximum rate of embezzlement
\end{exercise}

Consider the same example as above:

\begin{equation*}
  \dot y = f(t) = ry - w \qquad (r, w > 0). 
\end{equation*}

where $y$ models the amount of money (in dollars) in a savings account;
$r$ is the interest rate (in $\text{year}^{-1}$); and $w$ (in dollars per year) is
the rate of money being stolen out of the account.\\

Suppose that the current amount in an account is \$ 10000,
and the annual interest rate is 2\% (so r=0.02).
What is maximum amount of money that can be stolen from the account per year so that the amount of money in the account remains increasing and the account owner will likely not notice? \\

For amount of money $y(t)$ to be increasing,
the solution has to satisfy $\, y(0) = 10000 > w/r, \,$
where the interest rate is given as $r = 0.02$.
This gives $\, w = 10000r =200\,$.
So the maximum amount of money stolen per year has to be less than \$ 200.

\begin{exercise}
  Phase line concept check
\end{exercise}

As in the above example, consider $\, \dot y = ry - w\,$. \\
If $\, y(0) < \frac{w}{r},\,$ what is $\, \lim_{t \to - \infty} y(t)$? \\

Since $\, y(0) < w/r, \,$ the solution curve is below the horizontal solution.
We see from the figure above that as $\, t \to −\infty,\, y(t) \to w/r$.\\
We can also use the phase line to find the answer.
To run time backwards, reverse the arrows in the phase line.
As $\, t \to − \infty,\,$ we have $\, y(t) \to w/r$.

\clearpage

\subsection{Logistic equation}

\paragraph{Modeling population growth}
Next example-- the \emph{logistic equation}.
This is a population equation.
This is the one that section 7.1 and section 1.7
is most heavily concerned with, this particular equation.
The derivation of it is a little vague,
so it's an equation which describes how population increases.
And the one minute -- the population of behavior of some
population-- let's call it $y(t)$ is the only thing I know to call
anything today.
Now, simple, the basic population equation
runs $dy,\, dt$. There's a certain growth rate--
let's call it $ky$.

\begin{equation*}
  \frac{dy}{dt} = ky 
\end{equation*}

So $k$ is what's called the growth rate.
Sometimes it's talked about in terms of birth rate.
But it's the net birth rate.
It's the rate at which people or bacteria or whatever are dying,
are being born minus the rate at which they're dying.
So it's a net birth rate.
But let's just call it the growth rate.\\
Now, if this is the equation, we can think of this--
if $k$ is constant, that's what's called simple population
growth. And you're all familiar with that.\\

\emph{Logistical growth} allows for a slightly more complex
situation.
Logistic growth says that calling $k$ a constant is unrealistic,
because the earth is not filled entirely with people.
What stops it from having unlimited growth?
Well, the resources, the food that the organism has to live on gets depleted.
And in other words, that the growth rate $k$ declines as $y$ increases.
As the population increases, one expects the growth rate
to decline, because resources are being used up.
And they're not indefinitely available.\\
Well, in other words, we should replace $k$ by a function with this behavior.
What's the simplest function which declines as $y$ increases?
The simplest choice-- and if you are ignorant about what else
to do, stick with the simplest.
At least you won't work any harder
than you have to-- would be to take $k$ equal to the simplest
declining function of $y$ there is, which is simply
a linear function, $\, a -  by\,$.\\
So if I use that as the choice of the declining growth rate,
the new equation is

\begin{equation*}
  \frac{dy}{dt} = (a - by)y = ay -by^2
\end{equation*}

The $y$ stays the same.\\
This equation is what's called the \emph{logistic equation}.
It has many applications, not just to population growth.
It's applied to the spread of disease, the spread of a rumor,
the spread of many things.\\

The simplest model for population $y(t)$ is the ODE $\, \dot y = ky$ for a positive growth constant $k$,
which is the birth rate minus the death rate of the population.
This DE says that the rate of population growth is proportional to the current population,
and we know the solutions to be $\, y(t) = Ce^{kt}$. 

\clearpage

\subsubsection{Qualitative behavior}

\paragraph{Qualitative behavior of the solutions to the logistic equation}

\begin{equation*}
  \frac{dy}{dt} = (a - by)y = ay -by^2
\end{equation*}

Now, those of you who've solved it, actually solved it,
know that the solution, the explicit solution,
involves-- well, you separate variables.
But you'll have to--
you'll have to use partial fractions.
I hope you love partial fractions.
You're going to need them later in the term,
but I can avoid them now by not solving
the equation explicitly.
And anyway, you get a solution, which
I was going to write on the board for you,
but you can look it up in your book.
It's unpleasant enough looking to make
you feel that there must be an easier way, at least to get
the basic information out. \\
Let's see if we can get the basic information out.
What are the \emph{critical points}?\\
Well, this is pretty easy.
a, I want to set the right hand side equal to $0$,

\begin{equation*}
  y(a -by) = 0
\end{equation*}

And therefore the critical points are where $\, y = 0 ,\,$, that's one.
And the other factor is when $\, a - by\, $ factor is $0$.
And that happens when $\, y = b/a\,$. \\
So there are my two critical points.
What does-- let's start drawing pictures of solutions.
Let's put in those right away.
The critical point, $0$, gives me a solution
that looks like this.
And the critical point $a/b$, those are positive numbers, so that's somewhere up here.
So those are two solutions, constant solutions.
In other words, if the population by dumb luck started at $0$, it would stay at $0$ for all time.
That's not terribly surprising.
But it's a little less obvious that if it starts
at that magic number $a/b$, it will also stay at that magic number for all time, without moving up
or down or away from it.\\
Now the question is therefore, what happens in-between.
So for the in-between, I'm going to make that same analysis
that I made before.
And it's really not very hard.
Look.

\begin{wrapfigure}{r}{5.5cm}
  \includegraphics[width=5.5cm]{image-logistic_equation_graph1}
  \caption{Graph of $\frac{dy}{dt} = ry - w$}
\end{wrapfigure}

So here is my $(dy,\, dt)$-axis, I'll call that $y'$, OK?
And here's the $y$-axis. 
The function that I want to graph is this $,\, ay - by^2, \,$ or in factored form
$,\, y(a - by)\,$. 
Now this function we know has a $0$.
It has a $0$ here.
And it has a $0$ at the point $a/b$.
At these two critical points, it has a $0$.
What's it's doing-- what is it doing in between?
Well, in between it's a parabola.
It's a quadratic function.
It's a parabola.
Does it go up or does it go down?
Well, when $y$ is very large, it's very negative.
That means it must be a downward opening parabola.
And therefore this curve looks like this.\\
So I'm interested in knowing where is it positive
and where is it negative.
Well, it's positive here, for this range of values of $y$.
Since it's positive there, it will be increasing there.
Here it's negative, and therefore will be decreasing.
Here it's negative, and therefore $\frac{dy}{dt}$ will be negative also.
And therefore the function $y$ will be decreasing here.\\

So how do these other solutions look?
Well, we can put them in.
I'll put them in in white, OK?
Cause I think this has got to last till the end of the term.
So how are they doing?

\begin{wrapfigure}{r}{5.5cm}
  \includegraphics[width=5.5cm]{image-logistic_equation_graph2}
  \caption{Graph of $\frac{dy}{dt} = ry - w$}
\end{wrapfigure}

They're increasing in between the two curves.
They're not allowed to cross either of these yellow curves.
But they're always increasing.
Well, if they're always increasing, they must-- well,
it starts here and increases, it must--
and I'm not allowed to cross.
It must do something like that.
This must be a translation of it.
In other words, the curves must look like that.
Those are supposed to be translations of each other.
I know they aren't, but use your imaginations.\\
What's happening above?
So in other words, if I start with a population anywhere
bigger than $0$ but less than $a/b, \,$ it increases to asymptotically to the level $a/b$. \\
What happens if I start above that?
Well, then it decreases to it because this way,
for values of $y$ bigger than $a/b,\,$ it decreases as time increases.
So these guys up here are doing this.
And how about the ones below the axis?
Well, they have no physical significance.
But let's put them in any way.
What are they doing?
They are decreasing away from $0$.
So these guys don't mean anything physically,
but mathematically they exist as solutions.
They're going down like that.

\begin{example}
  Describe qualitatively the solutions to $\, \dot y = 3y - y^2\,$.
  (This is a special case of the logistic equation
  $\, \dot y = ay - by^2\,$)
\end{example}

\Solution \\
To find qualitative behavior of the solutions, we follow the two steps described before:

\begin{enumerate}
\item Find the critical points by solving $\, f(y) = 0 \,$.
  \begin{equation*}
    \displaystyle  \displaystyle f(y)=3y-y^2=0
    \quad \Rightarrow
    \quad \displaystyle  y=0\, \, \, \text {or} \, \, \, y=3.
  \end{equation*}
  So the constant solutions are $\, y(t) = 0\,$ an $\, y(t) = 3$.
\item Determine the intervals of $y$
  in which $\, f(y) > 0 \,$ and in which $\, f(y) < 0\,$.
  To do this, we can graph the function $f(y)$ as in the previous paragraph,
  or simply test one point in each region.
  \begin{itemize}
  \item Since $\, f(−1)<0,\,$ the region $y < 0$ is a down region.
  \item Since $\, f(1)>0,\,$ the region $0< y < 3$ is a down region.
  \item Since $\, f(4)<0,\,$ the region $y > 3$ is a down region.
  \end{itemize}
\end{enumerate}

\clearpage

We can summarize the above information by a phase line:

\begin{figure}[ht!]
  \centering
  \includegraphics[width=0.5\textwidth]{images_u2s2_horizontalphaseline2}
  \caption{Phase line of $\dot y = 3y - y^2$}
\end{figure}

\begin{example}
  Sketch some solutions to the equation $\, \dot y = 3y - y^2\,$.
  Make sure you include at least one of each type.
\end{example}

\begin{figure}[ht!]
  \centering
  \includegraphics[width=0.5\textwidth]{images_logisticsolutions}
  \caption{Phase line drawn vertically (left); Sample solution curves on $(t,y)$-plane (right)}
\end{figure}

There are five fundamental solutions, depending on the initial conditions:

\begin{itemize}
\item If $\,y(0)=0,\,$ the solution will be $y(t)=0$ for all $t$.
\item If $\,y(0)=3,\,$ the solution will be $y(t)=3$ for all $t$.
\item If $\ 0< y(0) <3 \,$ will increase as $t$ increases.
  It tends to $3$ as $\, t \to + \infty \, $ and tends to $0$ as $\, t \to −\infty$.
\item If $\, y(0)>3,\,$ then $y(t)$ decreases as $t$ increases, tending to $3$ without reaching $3$.
\item If $\, y(0)<0,\,$ then $y(t)$ decreases, and $\, y(t) \to −\infty\,$ as $t$ grows.
\end{itemize}

\textbf{\color{orange} Food for thought: }
For a solution $y(t)$ with $\, 0<y(t)<3,\,$
is it possible that $y(t)$ tends to a limit less than $3$? \\

No. If $y(t)$ tends to any number a between $0$ and $3$,
then as the solution curve levels off while approaching $\,a, \, \dot y (t)\,$ must tend to $0$.
But we know that $\, f(y)>0 \,$ for all $\, 0<a<3,\,$ so this is impossible.

\begin{exercise}
  Interval of validity of a solution
\end{exercise}

By separable of variables, the general solution to $\, \dot y = 3y -y^2\,$ is

\begin{equation*}
  \displaystyle \displaystyle y(t) = 0 \displaystyle \text {or}
  \displaystyle \frac{3}{1+c e^{-3t}}\qquad (c \, \, \text {a constant}).
\end{equation*}

Find the interval of validity of the solution satisfying $\, y(0)=6$.\\
(Use the interval notation in your answer.
That is, find $a$ and $b$ such that the solution is valid on the interval
$\, a<t<b$.) \\

To satisfy the initial condition $\,y(0) = 6,\, c = -1/2\,.$
Now, Now, we find $t$ when the denominator is undefined:

\begin{equation*}
  \displaystyle  \displaystyle 1- (1/2) e^{-3t}\, =\,  0
  \displaystyle \Longrightarrow
  \displaystyle  t=-\frac{\ln (2)}{3}.
\end{equation*}

The solution is defined at $\, t = 0,\,$
where the condition $\, y(0)=6 \,$ is defined,
so the interval of validity of this solution the largest interval
on which the solution is defined: $\, (− \ln⁡(2)/3\, ,+\infty)\,$.
Below is the graph of the solution curve:

\begin{figure}[ht!]
  \centering
  \includegraphics[width=0.5\textwidth]{images_u2s2_logisticintervalofvalidity}
  \caption{Graph of $\dot y = 3y - y^2$}
\end{figure}

(Check also that since $\, e^{-3t} \to 0\,$ as
$\,t \to \infty, \, y(t)= \frac{3}{1−(1/2)e^{−3t}} \to 3\,$ as expected.)

\textbf{Remark:}\\
Let us verify the given solution by separation of variables:

\begin{align*}
  \dot y &= 3y - y^2 \\
  \Longrightarrow \int \frac{dy}{y(3-y)} &= \int dt \qquad (y \neq 0,\,3)
\end{align*}

We evaluate the integral on the left using partial fraction (omitting the constant of integration):

\begin{align*}
  \int \frac{dy}{y(3-y)} &= \int \left( \frac{1/3}{y} + \frac{1/3}{3 -y} \right) \\
                         &= \frac{1}{3}(\ln |y| - \ln |3 - y|) \\
                         &= \frac{1}{3}\ln \left| \frac{y}{3-y} \right|.  \\
\end{align*}

Therefore, we have

\begin{align*}
  \displaystyle \frac{1}{3}\ln \left|\frac{y}{3-y}\right|
  \displaystyle &= \displaystyle t + C \\
  \displaystyle \Longrightarrow
  \displaystyle  \left|\frac{y}{3-y}\right|
  \displaystyle &= \displaystyle  c_1 e^{3t}\qquad (c_1>0)\\
  \displaystyle \Longrightarrow
  \displaystyle  \frac{y}{3-y}
  \displaystyle &= \displaystyle  c_1 e^{3t}\qquad (c_1\neq 0) \\
  \displaystyle \Longrightarrow y 
  \displaystyle \frac{3}{1+ce^{-3t}}\qquad (c \neq 0)
\end{align*}

To bring back the solution $\,y=3,\,$ we allow $\,c=0.$

\begin{exercise}
  Determine the model
\end{exercise}

In an environment without constraints, the population of a species of frogs is described by the DE

\begin{equation*}
  \dot y = 3y, 
\end{equation*}

where the constant $3$ is the natural birth rate minus the natural death rate
(in $\text{month}^{-1}$). \\

The same species of frogs living in a pond grows according to the logistic equation.
The population eventually reaches an equilibrium of 6000 frogs.
Find $f(y)$ such that differential equation $\, \dot y = f(y) \,$ models the population of frogs in a pond, measured in kilofrogs.\\

The logistic equation has growth rate $\, k(y)=a−by\,$.
The constant a is the constant growth rate of the frogs in an environment without constraints,
given to be $3$. So the logistic equation that models the frog population is

\begin{equation*}
  \dot y = (3−by)y
\end{equation*}

for some constant $\, b>0\,$.
Since the population settles at $\,y=6\,$ (six thousand frogs),
$\, \dot y = (3−by)y=0 \,$ at $\,y=6\,$; thus $\,b=3/6=1/2\,$.\\
\textbf{Remark: rewriting the logistic equation. } \\
Note that $\, b=a/y_0,\,$ where $y_0$ is the non-zero equilibrium population.
Renaming $a$ to $k_0,\,$ the natural growth constant if the environment is unconstrained,
the growth rate in the logistic equation can be rewritten as:

\begin{equation*}
  \displaystyle  \displaystyle k(y)=k_0\left(1-\frac{y}{y_0}\right).
\end{equation*}

The unit $k_0$ is $\text{month}^{-1},\,$ and the unit of $y_0$ is kilofrog. 

\clearpage

\subsubsection{Stability of critical points}

\paragraph{Stability of critical points using logistic equation}

Now you notice from this picture that there are--
that even though both of these are constant solutions,
they have dramatically different behavior.

\begin{wrapfigure}{r}{5.5cm}
  \includegraphics[width=5.5cm]{image-logistic_equation_graph2}
  \caption{Graph of $\frac{dy}{dt} = ry - w$}
\end{wrapfigure}

This one $,\, a/b\,$ this solution is the one
that all other solutions try to get-- approach as time goes to infinity.
This one, the solution $0$, is repulsive
as it were, Any solution that starts near $0$--
if it starts at 0, of course, it stays there for all time--
but if it starts just a little bit above $0$, it increases to $\,a/b$. \\
This is $,\, a/b ,\,$ called a \emph{stable solution} because everybody tries
to get closer and closer to it. \\
This is called-- it's an $0$ is also a constant solution, but this
is an \emph{unstable solution}.
And now usually, solution is too general a word.
I think it's better to call it a \emph{stable critical point},
and an \emph{unstable critical point}.
But of course, it also corresponds to a solution.\\

So critical points are not all the same.
Some are stable and some are unstable.
And you can see which is which just
by looking at this picture.

\begin{figure}[ht!]
  \centering
    \includegraphics[width=5.5cm]{image-logistic_equation_graph3}
  \caption{Graph of $\frac{dy}{dt} = ry - w$}
\end{figure}

If the arrows point toward them, you've got a stable critical point.
If the arrows point away from them, you've got an unstable critical point.\\

\clearpage

Now there is a third possibility.
What if-- I think we better address it,
because otherwise you're going to sit there wondering, hey,
what did he mean by that?
It could look like-- suppose it looked like this.

\begin{wrapfigure}{r}{5.5cm}
  \includegraphics[width=5.5cm]{image-logistic_equation_graph4}
  \caption{Graph of Third Possibility}
\end{wrapfigure}

Suppose it were just tangent.
So, well, this is the picture of that curve, the pink curve.
What would the arrows look like then?
Well, what would the arrows look like then?
Well, since they're positive--
it's always positive, the arrow goes like this.
And then on this side it also goes in the same direction.
So is this critical point stable or unstable?
It's stable if you approach it from the left.
So how do the-- in fact do the curves--
how would the corresponding curves look?
Well, there's our long term solution that corresponds--
this corresponds to that point.
Let's call this a and then this will be the value a.
If I start below it. I rise to it.
If I start above it, I leave it. I increase.
So if I start above it, I do this.
Well now that's stable on one side, and unstable on the other.
And that's indicated by saying it's semistable.
Wonder how long it took to think that one up.
Semistable critical point.
Stable on one side, unstable on the other depending
whether you start below it or-- and of course, it could be reversed.
If I'd drawn the picture the other way,
could have approached it from the top and left it from below.
You get the idea of the behavior.\\

A critical point $\,x=a \,$ is called

\begin{itemize}
\item \emph{\color{blue} stable} if solutions starting near it move towards it,

  \includegraphics[width=0.5\textwidth]{images_u2s2_stablecp}
  
\item \emph{\color{orange} unstable} if solutions starting near it move away from it,

  \includegraphics[width=0.5\textwidth]{images_u2s2_unstablecp}

\item \emph{semistable} if the behavior depends on \emph{which side} of
  the critical point the solution starts.
  
  \includegraphics[width=0.3\textwidth]{images_u2s2_semistabler}
  \quad or \quad  \includegraphics[width=0.3\textwidth]{images_u2s2_semistabler1}

\end{itemize}

\begin{example}
  $\dot y = 3y - y^2 $
\end{example}

\begin{figure}[ht!]
  \centering
  \includegraphics[width=0.5\textwidth]{images_logisticsolutions_colored}
  \caption{Phase line (left) and solutions curves (right) of $\, \dot y = 3y - y^2$}
\end{figure}

In the previous example of $\, \dot y = 3y − y^2,\,$, the critical points are $0$ and $3$.
The phase line shows that $0$ is unstable, and $3$ is stable.
Solutions move away from the unstable critical point $0$
and approach the stable critical point $3$ as $t$ increases, as shown in the figure above.\\

\begin{remark}
  A solution corresponding to an unstable critical point is an example of a separatrix
  because it separates solutions having very different fates.
  In the example above, $\, y = 0\,$ is a separatrix.
  A solution starting just below $0$ tends to $− \infty$,
  while a solution starting just above $0$ tends to $3$: very different fates!
\end{remark}

\textbf{\color{blue} Summary: steps for understanding solutions
  to $\dot y = f(y)$ qualitatively.}

\begin{enumerate}
\item Find the critical points by solving $\,f(y) = 0\,$.
  These divide the $y$-axis into open intervals.
\item Determine the intervals of $y$
  in which $\, f(y)>0 \, $ and $\, f(y)<0,\,$ by either graphing $f(y)$
  or evaluating $f(y)$ at one point in each interval.
\item Draw the phase line,
  which consists of a line marked with $−\infty,\,$ the critical points,
  $+ \infty,\,$ and arrows between these.
\item Solutions starting at a critical point are constant.
\item Other solutions tend to the limit that the arrow points to as $t$ increases.
  As $t$ decreases, solutions tend to the limit that the arrow originates from.
\end{enumerate}

\clearpage

\subsubsection{Worked Example}

\paragraph{Worked example: autonomous equations and phase lines}

OK, so in this problem, I'd like to take
a look at autonomous equations and phase lines.
And specifically we're going to take a look
at this simple equation

\begin{equation*}
  \dot x = ax + 1, 
\end{equation*}

 models births and death rates, and a fixed replenishment rate for a population.
So in this case, the variable $a$ represents births minus deaths in a population,
and $1$ represents the constant input of new creatures.

\begin{enumerate}
\item we're asked to find intervals of the variable $a$,
  which determine the longtime stability of the population. \\

  \begin{equation*}
    \dot x = ax + 1
  \end{equation*}

  And as mentioned in the title of the problem,
  this is an autonomous equation, which means that the right hand
  side is not a function of time.
  The right hand side only depends on $x$.
  And for these types of problems, the critical points or the place
  that the points at which the right hand side vanish tend to determine
  the longtime characteristics of the solution.
  So we can get a handle on what intervals of a
  determine the longtime stability by sketching what
  $\dot x$  versus $x$ might look like. \\
  
  See that if $a$ is positive,

  \begin{figure}[ht!]
    \centering
      \includegraphics[width=0.5\textwidth]{image_autonomous_working_example1}
    \caption{Graph of $\dot x = ax -1 \quad (a > 0)$}
  \end{figure}

  this represents a line with a positive slope $a$.
  So on the right hand side of the intercept with the $x$-axis,
  $\dot x$ is positive, which means that when the line goes
  through the point x dot equals zero, the term $\, ax + 1\,$ is going to be positive.
  I'll just point out that this point right here
  is when $\, \dot x = 0 \,$ is equal to zero.
  And this happens when $\, x = 1/a \,$.
  So this point right here is $-1/a$. 
  And when $\, x < -1 / a$,  $\dot x$ is negative, which means that solutions
  will want to move away from the point $1/a$. \\

  What other qualitative behavior could we have?
  Well, when $\, a < 0,\,$  the slope of this curve is going to be negative.
  So I'll plot again, $\dot x$ versus $x$.

  \begin{figure}[ht!]
    \centering
      \includegraphics[width=0.5\textwidth]{image_autonomous_working_example2}
    \caption{Graph of $\dot x = ax -1 \quad (a < 0)$}
  \end{figure}
  
  And again, the intercept is going to be at $-1/a$. 
  However, when $x$ is above the critical point, $\dot x$  is negative.
  So solutions will want to tend to move back towards $-1/a$.
  And when $x$ is below the critical point, we see that $\dot x$ is in the upper plane, which
  means it's positive. So solutions will want to grow.
  So we have two cases right now. 
  One is when $\, a > 0\,$ is , one is when $\, a < 0 \,$.\\

  And then, of course, there's going to be an intermediate point,
  which is when $\, a = 0 \,$.

  \begin{figure}[ht!]
    \centering
    \includegraphics[width=0.5\textwidth]{image_autonomous_working_example3}
    \caption{Graph of $\dot x = ax -1 \quad (a = 0)$}
  \end{figure}
  
  And in this case, the curve $\dot x$ versus $x$ is just going to
  be a straight line that goes through $1$. \\

  So this gives us three regions that will determine the long time behavior.
  So the three regions are $\, a = 0,\,  a < 0, \,$ and $\, a > 0\,$. 

\item For each typical case of $a$ in the part 1 of the problem,
  we're asked to sketch a phase line and then also to sketch the solution $x(t)$ versus $t$.\\
  
  So for part 2, let's take a look at the case when $\,  a > 0\, $ first.
  And I'm just going to pick one value of $a$.
  I'll just pick $\,a = 1 \,$ to work things out concretely.
  And we're asked to sketch a phase line, so I'll plot the phase line $x$.

  And we know from the ODE that the critical point occurs at $-1/a$.
  So in this case, there's going to be one critical point,
  and it's going to be at $-1$.
  So when $\, x  > -1 \,$, in this region,  $\dot x$ positive.
  So above negative $1$, $\dot x$ is positive,
  which means that if a solution starts at a point above
  $-1$, it's going to constantly increase forever.\\ 
  And likewise, if it's below $-1$, it's going to decrease forever.
  So this is $\dot x$ is negative in this region.
  And of course, $\dot x$ is zero at the critical point, which
  means if we start at the critical point, we stay there for all time. \\
  So let's sketch some solutions.

  \begin{figure}[ht!]
    \centering
    \includegraphics[width=0.5\textwidth]{image_autonomous_working_example4}
    \caption{Graph of Phase Line and Solution $\dot x = ax -1 \quad (a > 0)$}
  \end{figure}

  So whenever I have to sketch solutions for a phase line,
  the first solution I always plot is going to be the critical point.
  So note that $\, x = -1\,$ solves the differential equation.
  So $x$ equals a critical point always solves the differential  equation.
  So that means when $\, x = -1,\,$ we solve the differential equation.
  When $\, x > -1, \,$ we obtain exponential growth.
  And when $\, x > -1, \,$ we have exponential growth in the opposite direction.
  So this is sometimes said to be an unstable critical point,
  because any small perturbation away from $-1$ will make the solution diverge.
  Also note that each one of these curves--
  or once I have one of these curves, I can generate all the others by just picking it up and shifting
  it over.
  And the reason this works is because the original equation is autonomous,
  meaning that the right hand side didn't depend on time.
  So whenever you have an autonomous equation, when you sketch one solution,
  you can always get a family of others
  by picking up that solution and just shifting it to the right and to the left.\\

  Now, when $\, a < 0,\,$, I'll just pick the value of $\,a = -1$.
  We have a critical point at $1$.
  And in this case, the arrows are going to point towards 1 like this.
  And again, I can sketch some curves.
  So I'll first draw the solution when $\,x = 1$.

  \begin{figure}[ht!]
    \centering
    \includegraphics[width=0.5\textwidth]{image_autonomous_working_example5}
    \caption{Graph of Phase Line and Solution $\dot x = ax -1 \quad (a < 0)$}
  \end{figure}
  
  And in this case, solutions are going to converge towards the critical point.
  We say that the critical point $\, x = 1\,$  is stable.
  And that's because any small perturbation makes the solution come back--
  any small perturbation in the solution will eventually just come back 
  to the critical point at $\, x = 1$. \\

  
  And now lastly, we have the third case, which is when $\, a = 0$.
  And in this case, there are no critical points.
  In fact, $\, \dot x  = 1$.
  So $\dot x$ is positive for all values of $x$.
  So our phase line is just a line with arrows.
  And in this case, if I were to sketch some solutions $x(t)$
  there are no critical points.  The $x$ is just increasing.
  So the solutions are going to look something like this.

  \begin{figure}[ht!]
    \centering
    \includegraphics[width=0.5\textwidth]{image_autonomous_working_example6}
    \caption{Graph of Phase Line and Solution $\dot x = ax -1 \quad (a = 0)$}
  \end{figure}

  They're just going to be straight lines,
  a family of straight lines. \\

  So we've just sketched some phase lines,
  and we sketched several solutions for each phase line.
  We've seen how the critical points and the arrows
  around the critical points affect the longtime behavior of the solutions.
  And this is typically the process that you go through when looking at autonomous equations.
  You always first, a-- find the critical points, then b--
  find the regions between the critical points,
  and figure out if $\dot x$ is either positive or negative.
  And then that automatically gives you the phase line.
  Once you have the phase line, you can always just sketch a family of solutions.
\end{enumerate}

Using the analysis from the previous paragraph,
we can predict the effect of immigration on the population of a country.

\begin{example}
  The population of a wealthy and sparsely populated country is currently at
  $7$ million and is declining from low birth rate.
  The natural growth rate is $\, a=−0.01\,$ (in $\text{year}^{−1}$).
  What will be the effect on the population If the country admits $10,000$ immigrants per year? 
\end{example}

\textbf{Solution }\\
Let $\, x(t) \,$ (in ten thousands) be the population at time $t$ (in years).
Since the country is sparsely populated and wealthy, we can assume resources are unlimited.
In other words, the quadratic term in the logistic equation is negligible.
Therefore, without immigration, the population can be modeled by

\begin{equation*}
  \dot x = ax = \frac{-1}{100} x. 
\end{equation*}

With immigration, the rate of change of population $\dot x$ must include an additional term:

where $1$ (in ten thousand immigrants per year) is the immigration rate.
This DE is the same as in the previous paragraph with $\, a=−1/100\,$.
Therefore, as in the paragraph, the phase line is

\includegraphics[width=0.5\textwidth]{images_u2s2_populationphaseline}

This means that no matter what the initial population is,
the population will approach the equilibrium of $\displaystyle -\frac{1}{a} = 100$ (in ten thousands),
that is, $1$ million people.
The current population of the country is $7$ million;
so the population will be decreasing while tending to $1$ million. 

\begin{exercise}
  Maintaining population
\end{exercise}

In the example problem above, what rate of immigration is needed for
the country to maintain its population?\\

To maintain a population of $7$ million (or 700 ten thousands), we solve the differential equation

\begin{equation*}
  \dot x = - \frac{1}{100}x + a
\end{equation*}

for $a$.  At steady state, we want

\begin{equation*}
  0 = -\frac{1}{100} 700 + a
\end{equation*}

Thus to maintain the population, the immigration rate must be
$\, a = 7\,$ ten thousands of immigrants per year.

\clearpage

\subsubsection{Phase line concept problems}

\begin{exercise}
  Phase line from DE
\end{exercise}

Find the critical points and draw a phase line for the autonomous ODE:
$\, \dot y = y^2 + 2y \,$\\


The critical values are the zeros of $\, y^2 + 2y = y (y+2),\,$
which are $−2$ and $0$. We see that

\begin{align*}
  \displaystyle  \displaystyle & \dot{y}>0 \quad
                                 \displaystyle  \text {for}\, \,  y<-2 \, \text {and}\, \,  y>0; \\
  \displaystyle & \dot{y}<0 \quad
  \displaystyle \text {for}\, \,  -2<y<0.
\end{align*}

Therefore, the phase line is:

\begin{figure}[ht!]
  \centering
  \includegraphics[width=0.5\textwidth]{images_u2s2_phsln1}
\end{figure}
  
\clearpage

\begin{exercise}
  Phase line from slope field
\end{exercise}

Draw the phase line for the differential equation associated with the following slope field.

\begin{figure}[ht!]
  \centering
  \includegraphics[width=0.5\textwidth]{images_u2s2_slopefieldtophaseline}
  \caption{A slope field of an autonomouse equation}
\end{figure}

The answer is

\begin{figure}[ht!]
  \centering
  \includegraphics[width=0.5\textwidth]{images_u2s2_phsln2}
  \caption{A slope field of an autonomouse equation}
\end{figure}

\clearpage

\begin{exercise}
  Graph from slope field
\end{exercise}

Consider the same slope field as above.

\begin{figure}[ht!]
  \centering
  \includegraphics[width=0.5\textwidth]{images_u2s2_slopefieldtophaseline}
  \caption{A slope field of an autonomouse equation}
\end{figure}

Graph $\, f(y) \,$ from the slope field.

\begin{figure}[ht!]
  \centering
  \includegraphics[width=0.5\textwidth]{images_u2s2_function}
  \caption{A slope field of an autonomouse equation}
\end{figure}

\clearpage

\subsubsection{Concept challenge}

\begin{exercise}
  Critical points
\end{exercise}

In the autonomous equation $\, \dot y=f(y), \,$, where $f(y)$ has the graph shown,
describe stability of the rightmost critical point. \\

\begin{figure}[ht!]
  \centering
  \includegraphics[width=0.5\textwidth]{images_autonomousfunctiona}
  \caption{A slope field of an autonomouse equation}
\end{figure}

Labelling the $3$ critical points as $\,a,\, b,\, c,\,$ (as in the figure above), the phase line is:

\begin{equation*}
  -\infty \quad \longleftarrow \quad a \quad \longrightarrow
  \quad b \quad \longleftarrow \quad c \quad \longrightarrow \quad \infty .
\end{equation*}

The phase line shows that the right most critical point at $\, y=c \,$ is unstable.

\begin{exercise}
  Local maxima
\end{exercise}

Consider the autonomous equation $\, \dot y = f(y) \,$ where $f(y)$ and $f'(y)$ are continuous everywhere.
Can non-constant solutions to this equation have a local maximum? \\

For a solution to have local maxima we know that it has to contain points
where $\, \dot y = 0$.
But all points where $\, \dot y=0 \,$ lie on the $0$-isocline,
which we know are the constant solutions to the autonomous DE.
Then by the existence and uniqueness theorem,
other solution curves cannot contain points where $\, \dot y=0$.
Therefore, no non-constant solutions can have a local maximum (or local minimum).
Geometrically, non-constant solution curves cannot intersect the $0$-isoclines.

\begin{exercise}
  Inflection points
\end{exercise}

When do non-constant solutions of the autonomous ODE $\, \dot y=f(y)\,$
have inflection points?\\

By the chain rule

\begin{equation*}
  \frac{dy}{dt} = f(y) \quad
  \implies \quad \frac{d^2y}{dt^2} = f'(y)\frac{dy}{dt}.
\end{equation*}

An inflection point is one where $\displaystyle {\frac{d^2y}{dt^2} = 0}$.
By the above formula this occurs when either $\, f'(y)=0 \,$ or $\dot y = 0$.
Since $\, \dot y=0\,$ only on constant solutions,
which have no inflection points, all that's left is $\, f'(y)=0$. 
Note: For this reason, an inflection point never has $\, y'=0$. 

\clearpage

\subsubsection{Harvesting models}

\paragraph{Logistic equation with harvesting}
I'm going to soup up this \emph{logistic equation} just a little bit more.
So let's talk about the logistic equation,
but I'm going to add to it with \emph{harvesting}.
So this is a very late 20th century concept.

So we imagine, for example, a bunch of formerly free range
Atlantic salmon penned in a large--
one of these huge factory farms off the coast of Maine or someplace.
They've made salmon very much cheaper than it used to be,
but at a certain cost to the salmon and possibly to our environment.
So what happens?
Well, the salmon grow and grow, and do what salmon do, and they
are harvested. \\
So what's the equation?
I'm going to assume that the \emph{harvest is at a constant time rate}.
That corresponds to, in other words,
it's not a certain fraction of all the salmon that are being
caught each day, and canned.
It's a certain-- the factory has a certain capacity.
So 400 pounds of salmon each day are pulled out and canned.
So it's a time-- constant time rate.
That means that the equation is

\begin{equation*}
  \frac{dy}{dt} = ay - by^2 - \underbrace{h}_{\text{Constant harvesting}}
\end{equation*}
Not h times y.
Then I would be harvesting a certain fraction
of all the salmon there, which is not what I'm doing. \\
Now, I want to analyze what the critical points of this look
like.
Now this a little more subtle, because there's now
a new parameter there.
And what I want to see is how that
varies with the new parameter.
The best thing to do is--
I mean, the thing not to do is make this equal to $0$,
fiddle around with the quadratic formula,
get some massive expression, and then spend the next half
hour scratching your head trying to figure out what it means,
or what information you're supposed to be getting out of it.\\
Draw pictures instead.
If $\, h = 0 \,$ that's the smallest harvesting rate I could have.
The picture looks like our old one.
So if $\,h = 0\,$ , the picture looks like--
what color did I? Purple.

\begin{wrapfigure}{r}{5.5cm}
  \includegraphics[width=5.5cm]{image-harvesting}
  \caption{Graph of $\frac{dy}{dt} = ay - by^2 - h$}
\end{wrapfigure}

So this is the one-- original one corresponding to $\,h = 0 \,$.
Or in other words, it's the equation $\, ay^2 - by \,$.
Now if I want to find--
I now want to increase the value of $h$, as I increase the value of $h$,
in other words, harvest more and more, what's happening?
Well, I simply lower this function by $h$.
Now, so for example, if I lower h somewhat, it will come to here.
So this is some value ay minus by squared minus $h_1$. 
If I lower it a lot, it'll look like this, so $\, ay - by^2 - h_{2}\,$.
Now obviously, there's one interesting value to lower it by.
It's the value which would lower it exactly by this amount.
Let me put that in special.
If I lower it by just that amount, the curve always looks the same.
It's just been lowered. I'm going to say this one is--
so this one is the same thing, except that I've subtracted $h_m$.
Where is $h_m$ on the picture?
Well, I lowered it by this amount. So this height is $h_m$.
In other words, if I find the maximum height here--
which is easy to do because it's a parabola-- and lower it by exactly that amount,
I will have lowered it to this point. 
This will be the \emph{critical point}. \\
Now the question is, what's happened to the critical points
as I did this?
I started with the critical points here and here.
As I lower $h$, the critical point changed to this and that.
And now it changed to this one, when I got to the purple line.
And as I went still further down, there were no critical points.
So this curve has no critical points attached to it.\\

What are the corresponding pictures?
The corresponding pictures, well, we've already drawn--
the picture for $\, h = 0 \,$.

\begin{wrapfigure}{r}{5.5cm}
  \includegraphics[width=5.5cm]{image-harvesting1}
  \caption{Graph of solutions curve of $\frac{dy}{dt} = ay - by^2 - h$}
\end{wrapfigure}

How would the solution look like for this one, for $h_1$?
For $h_1$ the solutions look like, here is $a/b$.
Here is $a / b,\,$ but the critical points aren't at $0$ and $a/b$ anymore.
They've moved in a little bit. So they are here and here.
So otherwise the solutions look just like they did before,
and the analysis is the same.
And similarly, if $h_2$ goes very far, if $h_2$ is very large,
there are no critical points.
Are the solutions decreasing all the time, or increasing?
Well, they're always \emph{decreasing}, because the function is always negative.
Solutions always go down, always.
The interesting one is this last one, where I've decreased it just to $h_m$.
And what happens there is, there is this certain magic critical point,
whose value we could calculate.
There's one constant solution.
So this is one that has the value $y$ and $t$ here.
So here it's the value $,\, h_m,\,$ is the value by which it has been lowered.
So this is the picture for $h_m$.
And how do the solutions look?

\begin{wrapfigure}{r}{5.5cm}
  \includegraphics[width=5.5cm]{image-harvesting2}
  \caption{Graph of Maximum harvesting rate of $\frac{dy}{dt} = ay - by^2 - h$}
\end{wrapfigure}

Well, to the right of that, they are decreasing.
And to the left, they're also decreasing,
because this function is always negative.
So the solutions look like this If you start above.
And if you start below, they decrease.
And of course, they can't get lower than $0$,
because these are salmon.
What is the significance of $h_m$?
The $h_m$ is the \emph{maximum} rate of harvesting.
It's an extremely important number for this industry.
It's the maximum rate-- time rate
at which you can pull the salmon daily out of the water
and can them without what happening?
Without the salmon going to $0$.
As long as you start above, and don't harvest it more
than this rate, you'll be following these curves.
You'll be following these curves.
And you'll still have salmon.
If you harvest just a little bit more
you'll be on this curve, that has no critical points.
And the salmon in the tank will decrease to 0. \\

Recall that a population $y(t)$ in an environment
with limited resources is described by the logistic equation,

\begin{equation*}
  \dot y = ay - by^2 \qquad (a,\, b > 0) \qquad (\text{no harvesting}). 
\end{equation*}

If this population is also harvested at a constant rate,
we add a term to the logistic equation to include the effect of harvesting:

\begin{equation*}
  \dot y = ay - by^2  - h \qquad (a,\, b > 0,\, h \geq 0) \quad (\text{with harvesting}). 
\end{equation*}

where $h$ is the harvesting rate (in number of animals per time).

This is an \emph{infinite family} of autonomous equations,
one for each value of $h$, and each has its own phase line.
We will explore how the phase line changes with $h$ in the example and problems below.

\begin{example}
  Frogs grow in a pond according to the logistic equation
  with natural growth constant $3$ (in $\text{month}^{−1}$),
  and eventually the population reaches an equilibrium of $3000$ frogs.
  Then the frogs are harvested at a constant rate. Model the population of frogs.
\end{example}

\textbf{\color{blue} Variables and functions:}

\begin{align*}
  t &\qquad \text{time}\, \text{(month)} \\
  y(t) &\qquad \text{size of population (kilofrogs) at time $t$} \\
  h &\qquad \text{harvest rate (kilofrogs per month)}
\end{align*}

\textbf{\color{blue} Differential equation:} \\
\textbf{Without harvesting:} \\
We have modeled the frog population in a pond without harvesting in a previous problem.
We learned then that the logistic equation can also be written as

\begin{equation*}
  \dot y = k_0 \left( 1 - \frac{y}{y_0} \right) y. 
\end{equation*}

where $k_0$ is the natural growth constant
if the environment is unconstrained, and $y_0$ is the non-zero equilibrium population.
In our example, $\,k_0 = 3,\,$ and $\, y_0 = 3\,$ (kilofrogs); therefore the DE is

\begin{equation*}
  \dot y = 3 \left( 1 - \frac{y}{3} \right) y = 3y - y^2. 
\end{equation*}

\emph{With harvesting: } We add the term $\,−h\,$ to the DE above to account for the harvesting:

\begin{equation*}
  \dot y = 3y - y^2 -h
\end{equation*}

for a harvesting rate $h$.
Since the population has reached the equilibrium of $3$ (kilofrogs) when harvesting starts,
we use the initial condition $\, y(0)=3 \,$.

\clearpage

\begin{exercise}
  Draw the phase lines for two different harvesting rates
\end{exercise}

As in the example above, consider the DE

\begin{equation*}
  \dot y = 3y - y^2 - h \qquad (h > 0). 
\end{equation*}

\begin{figure}[ht!]
  \centering
  \includegraphics[width=0.5\textwidth]{images_u2s2_harvestingproblem}
  \caption{Graph of $\, \dot y=3y - y^2 - h\,$
    for $\, h = h_1\,$ (green) and $\,h = h_m\,$ (black)}
\end{figure}

Use the graphs above to draw the phase line of $\, \dot y=3y − by^2 −h \,$ for $\, h=h_1 \,$.
The exact location of each critical point is not required.

For $\, h = h_1 \,$

\includegraphics[width=0.7\textwidth]{images_u2s2_harvestproblemphaseline1}

\begin{exercise}
  Draw the phase lines for two different harvesting rates
\end{exercise}

As in the problem above,

\begin{equation*}
  \dot y = 3y - y^2 - h \qquad (h > 0). 
\end{equation*}

Use the graphs in the previous problem to draw
the phase line of $\, y˙= 3y − y^2 − h \,$ for $\, h = h_m$.
The exact location of each critical point is not required. \\

For $\, h = h_m \,$

\includegraphics[width=0.7\textwidth]{images_u2s2_harvestproblemphaseline2}

\clearpage

\begin{exercise}
  population evolution from phase line
\end{exercise}

As above, let $h_m$ be the value of $h$ such that
the graph of $\, f(y) = 3y − y^2 − h_m\,$ is as in the previous problem:

\begin{figure}[ht!]
  \centering
  \includegraphics[width=0.5\textwidth]{images_u2s2_harvestingproblemhm}
  \caption{Graph of $\, f(y) = 3y - y^2 - h_m \,$}
\end{figure}

Let $y(t)$ be a solution to the DE $\, \dot y = 3y − y^2 −h_m$.
According to the phase line at $\, h = h_m\,$ that you drew in the previous problem,
can $y(t)$ go all the way from $ + \infty$ to $ + \infty$ as $t$ increases? \\

There are three types of solutions: one to the left of the critical point,
one at the critical point, and one to the right of the critical point.
Let $y_0$ be the critical point.
The three types of solutions correspond to the three possible evolutions,
which depend on the initial conditions.

\begin{itemize}
\item If $\, y(0)> y_0\,$, then $\, y(t) \to y_0 \,$ as $\, t \to + \infty \,$.
\item If $\, y(0) =  y_0\,$, then $\, y(t) =  y_0 \,$ for all $t$.
\item If $\, y(0) < y_0\,$, then $y(t)$ tends to $- \infty$.
  (We interpret this as a population crash : the frog population reaches $0$ in finite time;
  the part of the trajectory with $\, y<0 \,$ is not part of the population model).
\end{itemize}

\clearpage

\begin{exercise}
  Phase lines for different harvesting rates
\end{exercise}

As above, consider the DE

\begin{equation*}
  \dot y = 3y - y^2 - h \qquad (h > 0). 
\end{equation*}

The 4 phase lines given below are for 4 different values of
$\, h: h=a,\, h=b, \, h=c,\,$ and $h=d$.
Order $\,a,\, b,\, c,\, d\, $ in increasing order.


\begin{figure}[ht!]
  \centering
  \includegraphics[width=0.5\textwidth]{images_u2s2_4phaselines_a}
  \caption{Graph of phase lines}
\end{figure}

Solution is $\, b < a < d < c \,$.\\
This be because as $h$ increases, the graph of $\, f(y) = 3y − y^2 − h\,$
(shown in the previous problem) is translated vertically downwards,
and therefore the distance between two critical points shrinks,
until eventually there is only one critical point, and then none.

\begin{exercise}
  Number of critical points
\end{exercise}

As above, consider the DE

\begin{equation*}
  \dot y = 3y - y^2 - h \qquad (h \geq 0). 
\end{equation*}

Let $h_m$ be the value of h for which the DE has exactly $1$ critical point,
and let $y_0$ be the value of that critical point. Find $h_m$ and $y_0$. \\
The quantities $h_m$ and $y_0$ are of physical importance;
they correspond to the maximal harvesting rate
and the equilibrium population at the maximum harvesting rate respectively.
We will come back to this shortly.\\

The critical points are the real roots of $\, 3y − y^2 − h\,$.
Using the quadratic formula, we get

\begin{equation*}
  \displaystyle \displaystyle y_0(h) = \frac{3 \pm \sqrt{9 - 4h}}{2}. 
\end{equation*}

Therefore, the DE has exactly $1$ critical point $\, y_0 = 3/2\,$ when $\, 9 − 4h = 0\,$.
(The DE has $2$ critical points when $\, 9 − 4h > 0,\,$
and no critical point when $\, 9 − 4h < 0\,$.)

\clearpage

\subsubsection{Bifurcation diagrams}

Let us continue with our previous example of frogs being harvested from a pond.
Their population is modeled by

\begin{equation*}
  \dot y = 3y - y^2 - h \qquad (h \geq 0). 
\end{equation*}

You have just seen that the phase lines corresponding to different values of $h$,
can be very different, and they tell us how the frog population evolves for different harvesting rate.
Our goal now is to summarize all this information in a single diagram.\\

If we draw phase lines vertically at several $h$-values in the $(h,\,y)$-plane,
we get a diagram on the left below.
If we draw phase lines at all $h$-values, we get a diagram on the right below.

\begin{figure}[ht!]
  \centering
  \begin{subfigure}[b]{.49\textwidth}
    \centering
    \includegraphics[width=\textwidth]{images_bifurcation4a}
    \caption{Vertical phase lines of $\, \dot y = 3y - y^2 - h\,$ at discrete values of $h$}
  \end{subfigure}
  \begin{subfigure}[b]{.49\textwidth}
    \centering
    \includegraphics[width=\textwidth]{images_bifurcation1}
    \caption{Bifurcation diagram for $\, \dot y = 3y - y^2 - h\,$}
  \end{subfigure}
\end{figure}

The diagram on the right is called a \emph{\color{blue}bifurcation diagram}.\\

\begin{definition}
  A \emph{\color{blue} bifurcation diagram} of a family of autonomous equations
  depending on a parameter $h$ is a plot of the values of the critical points as functions of $h$,
  along with one arrow in each region in the $\, (h, \, y)$-plane defined by the curve,
  indicating whether solutions are increasing or decreasing in that region.
\end{definition}

As usual, the arrows tell us the stability of the critical points.
In the diagram, we have also explicitly indicated the stability of the critical points by colors:
\emph{\color{blue}stable} , \emph{\color{orange}unstable}, and \emph{semistable}.

\begin{question}
  What is the shape of the curve on the bifurcation diagram for $\dot y = 3y - y^2 - h\,$?
\end{question}

The curve on this bifurcation diagram is a parabola.
It is defined by $\, 3y - y^2 - h = 0,\,$ or equivalently

\begin{equation*}
  h = 3y - y^2
\end{equation*}

a sideways parabola. \\

\textbf{Remark:} The shape of the curve on the bifurcation diagram depends on the differential equation,
and so there are many possibilities.
We will see bifurcation diagrams with other curves in the mathlet and paragraph that follows.

\begin{question}
  How do we find the direction of the arrows (and stability of critical points)
  on the bifurcation diagram above without drawing many phase lines?
\end{question}

To get the arrow on the bifurcation diagram directly,
check the sign of $\, \dot y = f(y)\,$ of one point in each region defined by the graph on the diagram. \\
In our example of $\, \dot y = 3y - y^2 -h,\,$ check one point inside the parabola,
such as $\, (h,\, y) = (0,\,1) ,\,$ where $\, \dot y = 3y - y^2 -h\,$.
This gives an up arrow for the entire region inside the parabola.
Similarly check one point outside the parabola, such as $\, (h,\, y) = (1,\, 0),\,$
where $\, \dot y = 3y - y^2 -h < 0\,$.
This gives an up arrow for the entire region outside the parabola.\\
Therefore, the upper branch of the parabola consists of \emph{\color{blue}stable} critical points,
and the lower branch consists of \emph{\color{orange}unstable} critical points,
when $\,9 - 4h<0,\,$ When $\, 9 - 4h = 0,\,$ the critical point is \emph{semistable}.\\

Let us now use the bifurcation diagram to answer an important question about harvesting frogs.

\begin{example}
  What is the maximum sustainable harvest rate?
\end{example}

(\emph{Sustainable} means that the harvesting does not cause the population to crash to $0$,
but that instead $\, \displaystyle \lim_{t \to + \infty} y(t) \,$ is positive,
so that the harvesting can continue indefinitely.) \\

\textbf{\color{blue}Solution:}
The maximum sustainable harvesting rate is $\, h = 9/4,\,$ i.e., $2250$ frogs/month. \\
\textbf{Why?}
\begin{itemize}
\item For $\, h < 9/4,\,$ the phase line is
  
  \includegraphics[width=0.5\textwidth]{images_u2s2_harvestproblemphaseline2withnumbers}

  Recall the starting population of frogs is $3000$. So we have the initial condition is
  $\, y(0) = 3,\,$ which is above the top (blue) branch of the parabola.
  Let the top branch of the parabola be called $y_{top} (h)$.
  Then for each value of $\, h < 9/4,\, y(t) \to y_{top}(h)\,$ as $t$ increases.
\item For $\ h = 9/4 \,$ the phase line is

  \includegraphics[width=0.5\textwidth]{images_u2s2_harvestproblemphaselinesswithnumbers}

  Since $\, y(0) = 3 > 3/2,\,  y(t) \to 3/2,\,$ as $t$ increases.
\item For $\, h > 9/4,\,$ the phase line is

  \includegraphics[width=0.5\textwidth]{images_u2s2_harvestproblemphaselinenocrit}

  In this case, no matter what the starting population is,
  $y(t)$ will reach $0$ in finite time and a population crash is inevitable (overharvesting).    
\end{itemize}

\begin{remark}
  Harvesting at exactly the maximum rate is a little dangerous,
  however, because if after a while $y$ becomes very close to $\,3/2,\,$
  and a little kid comes along and takes a few more frogs out of the pond
  so that the population is just below $\, 3/2,\,$ the whole frog population will crash!
\end{remark}

\begin{exercise}
  Avoiding population crash
\end{exercise}

Recall the bifurcation diagram for the harvesting problem above.

\begin{figure}[ht!]
  \centering
  \includegraphics[width=0.5\textwidth]{images_bifurcation1}
  \caption{Bifurcation diagram for $\, \dot y = 3y - y^2 - h\,$}
\end{figure}

A farmer would like to harvest $2000$ frogs per month.
What is the minimum starting population $y_{min}$ needed in order to avoid a population crash? \\
(Enter your answer in kilofrogs.) \\

From the phase line at $\, h=2, \,$ as shown in the diagram above,
any starting population above and at the critical point is safe from population crash,
while any population below the critical point will experience a crash.\\
The critical point at $\, h = 2\,$ is
the smaller root of $\, 3y - y^2 - 2=0\,$. Factoring the quadratic equation, we find:

\begin{equation*}
  \displaystyle  -y^2+3y-2=(-y+2)(y-1)=0
  \displaystyle \Longrightarrow y=1, 2.
\end{equation*}

Therefore, the minimum starting population needed
for harvesting at a rate of $2000$ frogs per month is $1000$ frogs. \\
\textbf{Remark:}\\
Alternatively, we can use the quadratic formula to solve $\, -y^2 + 3y -2 = 0 \,$:

\begin{equation*}
  \displaystyle  y = \frac{3 \pm \sqrt {9 - 4(2)}}{2}\, =\, 1\, \text {or}\,  2.
\end{equation*}

\begin{exercise}
  Minimum population for different harvesting rates
\end{exercise}

Consider the same frog harvesting problem as above, and recall its bifurcation diagram:

\begin{figure}[ht!]
  \centering
  \includegraphics[width=0.5\textwidth]{images_bifurcation1}
  \caption{Bifurcation diagram for $\, \dot y = 3y - y^2 - h\,$}
\end{figure}

Give a formula for the minimum starting population $y_{min} (h)$
of frogs required for a farmer to harvest at a rate of $h$ (kilofrogs per month)
without crashing the population. Assume $\, h < 9/4 \,$.
(Your answer should be in terms of $h$.)\\

For any harvest rate $\, h < 9/4,\,$
the minimum starting population of frogs is the value of the lower branch of the parabola at $h$,
which is the smaller root of $\, 3y - y^2 - h = 0$.
Solving for $y$ in terms of $h$ by the quadratic formula, we have

\begin{equation*}
  \displaystyle  \frac{3 \pm \sqrt {9 - 4h}}{2}.
\end{equation*}

Therefore,

\begin{equation*}
  \displaystyle \displaystyle y_{min}
  \displaystyle =
  \displaystyle  \frac{3 - \sqrt {9 - 4h}}{2}, \qquad \left(h>\frac{9}{4}\right).  
\end{equation*}

\clearpage

\subsubsection{Phase line and bifurcation diagram mathlet}

The mathlet below shows the slope fields, solution curves,
phase lines and bifurcation diagrams for different families of autonomous DEs.

To consider a harvesting model similar to the previous example,
set the autonomous function to $\, f(y) = y(1 - y) + a,\,$
and change the harvesting rate by changing the parameter $a$.
You can hide the slope field, the phase line, the bifurcation diagram
and the plot of $f(y)$ by clicking the respective boxes.\\

\textbf{Remark:}
Observe that the direction fields are independent of time.
This is the most important property of autonomous equations,
and it is exactly for this reason that the phase line is able to capture
qualitatively the behavior of all solution curves.
A phase line makes sense only if the DE is autonomous!

\href{http://mathlets.org/mathlets/phase-lines/}
{Phase Lines}

\clearpage

\subsubsection{Application: Bead on wire}

\paragraph{Bead on wire: Finding the equilibria}
We'll consider a physical system
and draw the bifurcation diagram based on how it behaves.

\begin{wrapfigure}{r}{5.5cm}
  \includegraphics[width=5.5cm]{image-bead_wire}
  \caption{Figure of bead and wire}
\end{wrapfigure}

Then we'll write down a mathematical model
for the system that would give us quantitative insight if we
later decided to put it into Matlab or Mathematica.
So let's get started.
Our system is made up of a bead that
has a horizontal wire running through it.
This wire is attached to the walls on either end,
and there's also a spring attached to the bead.
The spring is fixed on the ground here,
and we're going to assume that
the height of the bead and the wire is given by $h$, 
and that $x$ is the displacement of the bead from the center,
or the displacement of the bead from where the spring is fixed.

\begin{align*}
  h &= \text{Height of wire} \\
  x &= \text{Displacement of bead from center}
\end{align*}

So our question is, what are the equilibrium positions
of the bead? \\
And we'd also like to know, what are the stability
of those equilibrium positions? \\
So let's suppose that we can change
the position of the wire, either up or down, so that we can change $h$.
And the equilibrium positions will
be dependent upon the relationship between $h$ and the relaxed length of the spring.
The relaxed length of the spring is just
the length at which the spring is being neither stretched nor compressed.
So since we want to think about equilibrium positions
as a function of some parameter, we're going to be drawing a \emph{bifurcation diagram}.
And this will be given in the $(h,\,x)$-plane, where
$h$ is on the horizontal axis and $x$ is on the vertical axis.
So it may seem a little bit strange to have $x$ on the vertical axis, but the reason for that
is that $h$ is actually the parameter here that we're varying.
So it's more of an independent variable, and we want to see how that affects $x$.
So like I said, this is going to depend
on the relationship between $h$ and the relaxed length of the spring.
So let's give that length a name and call it $L$.
\begin{equation*}
  L = \text{Relaxed of spring}
\end{equation*}
In this case, we're only considering positive values of $h$.
So I'm only in the right half plane. And let's say $L$ is somewhere here.

\clearpage

Now, if $\, h > L,\,$ that means we're stretching the spring
because we're moving it beyond its relaxed length.

\begin{wrapfigure}{r}{5.5cm}
  \includegraphics[width=5.5cm]{image-bead_wire1}
  \caption{Figure of bead and wire $h > L$}
\end{wrapfigure}

And that would look something like this.
We have a tall wire, and the bead is at the center,
and the spring is being stretched.
So this is $h$ greater than $L$.
If we tried to nudge this bead in either direction, the tension in the spring would cause the bead
to remain at the position $\, x  = 0 \,$.
So that means that $x$ equals $0$ is a stable equilibrium position.
So let's draw a phase line for what this might look like.
We're choosing an $h$ that's greater than $L$.
And we said that $x$ equals $0$ is a stable equilibrium position,
so our phase line would look like this. 
If I drew all the phase lines for values of $h$ that are greater than $L$, that is,
all the phase lines in which we're stretching the spring,
we'd have a solid line of stable equilibria position here at $x$ equals $0$.\\

But what happens when $\, h < L\,$?

\begin{wrapfigure}{r}{5.5cm}
  \includegraphics[width=5.5cm]{image-bead_wire2}
  \caption{Figure of bead and wire $h < L$}
\end{wrapfigure}

Well, when that happens we are actually
compressing the spring like this.
So this is $h$ less than $L$. If we do this carefully,
we can keep the bead at $\,x  = 0\,$ here.
But if we nudge the bead in either direction, it's going to move along the wire in the direction
that we nudge it.
So if we nudge it to the right, it would end up here, or to the left, it would end up here.
And it will move until the spring has reached
its relaxed length, so there's no more outward spring
force being applied.
So if we draw phase line for this scenario,
we're choosing an $\, h <  L\,$. And $\, x = 0 \, $ has now become an unstable equilibrium position
because any perturbation causes the bead to move away from that position.
So here, the arrows would be pointing away.
But we've also gained two other possible stable equilibrium
positions, namely, $1$ for positive $x$ and 1 for $- x$, so this would be the phase line
for this particular $h$ value.

Now, I want to think about drawing all of the phase
lines for h being less than $L$. And notice that if I decrease $h$ even more,
the bead will have to move farther along the wire in order for the spring
to reach its relaxed length.
So as I decrease $h,\, x$ is either becoming more positive--
so you have a curve like this-- or more negative.
So drawing all the phase lines for $h$ being less than $L$,
I recover this branch of stable equilibria, this branch of stable equilibria,
and a branch of unstable equilibria at $\, x = 0 \,$. \\

Now, we can actually write down what
this curve is mathematically by considering the system when the bead is at equilibrium.

\begin{wrapfigure}{r}{4cm}
  \includegraphics[width=4cm]{image-bead_wire3}
  \caption{Figure of bead and wire Pythagorean theorem}
\end{wrapfigure}

So that would form a triangle with a height of $h$.
The bead is here, so this side has a length of $x$.
And the spring at equilibrium is at its relaxed length, $L$.

So using the Pythagorean theorem, we can write down that $x$ is given

\begin{equation*}
  x = \pm \sqrt{L^2 - h^2 }
\end{equation*}

And that determines this curve here, where $x$ is now on the vertical axis.
And we have this sideways parabola, or you can think of it as a semi-circle,
given by this equation, where the positive is the top branch
and the negative is the bottom branch of equilibria.

\clearpage

Now, when you draw bifurcation diagram,
you won't include all of the phase lines for each value of $h$.
Instead, you can notice that the bifurcation diagram
is divided into regions based on the equilibrium branches.

\begin{figure}[ht!]
  \centering
  \includegraphics[width=0.5\textwidth]{image-bead_wire4}
  \caption{Bifurcation Diagram of bead and wire system}
\end{figure}

So we have a region here, here, here, and here.
In this region, all of the phase lines will be pointing downwards,
and so we replace this portion of the phase lines by a downwards arrow.
Similarly, in this region, all of the phase lines will be pointing upwards,
so we have an upwards arrow.
In this region, they'll be pointing downwards.
We have a downwards arrow.
And in this region, they'll be pointing upwards.

So a bifurcation diagram gives us all the possible equilibrium positions and their stability.
So we have curves of equilibria and arrows depicting their stability.
Now, this system was somewhat simple as a physical system, as well as geometrically.
So we were able to draw this just by reasoning through it.
Typically, you'll need to write down the equations of motion
and find the equilibria by setting the derivative equal to $0$.
So we'll continue going through that in the next paragraph.

\clearpage 

\paragraph{Bead on wire: Setting up the DE}
Now we're going to write down a differential equation that
models the horizontal motion of the bead
by considering the forces that are acting on the bead.
Then we'll rederive the bifurcation diagram.

\begin{wrapfigure}{r}{4cm}
  \includegraphics[width=4cm]{image-bead_wire5}
  \caption{Diagram of bead and wire}
\end{wrapfigure}

There are two forces that are acting on the bead.
The first is a \emph{friction force}.
And this is due to the bead sliding along the wire
and feeling friction from the wire.
I'm going to denote this force by lowercase $f$.
And it's proportional to the velocity of the bead $\dot x$.

\begin{equation*}
  f = -b \dot x \qquad (b > 0)
\end{equation*}

There's a negative sign because friction acts against this bead's motion.
And $b$ is called the friction coefficient and is greater than $0$.
The second force acting on the bead is the \emph{force due to the spring}.
I'll call this force $F_ s$.
Now we're only interested in the horizontal motion of the bead,
and so we're only interested in the horizontal component
of this force.
For the horizontal component, I'll denote this $F_x$.
\begin{exercise}
  Find the spring force on the bead
\end{exercise}

What is the horizontal component $F_x$ of the spring force acting on the bead?
\begin{figure}[ht!]
  \centering
  \includegraphics[width=0.5\textwidth]{images_u2s2_beadsystem}
  \caption{Diagram of bead and wire}
\end{figure}

A bead is threaded to a wire and the wire is attached horizontally to a box.
The bead can only slide horizontally along the wire.
The bead is then attached to a spring, whose other end is pinned to the center of the bottom of the box.\\

We have the following parameters and variables:

\begin{align*}
  &k \quad \text{the spring constant} \\
  &L \quad \text{the relaxed length of the spring} \\
  &h \quad \text{the height of the wire from the bottom of the box} \\
  &x \quad \text{the horizontal displacement of the bead from the center of the wire}
\end{align*}

Find the horizontal component $F_x$ of the spring force on the bead in terms
of $\,k,\, L,\, h\,$ and $x$.

\clearpage

First, consider the case in which the spring is stretched.
In this case, the spring force is a vector pointing
from the bead towards the center of bottom of the box as shown below:

\begin{figure}[ht!]
  \centering
  \includegraphics[width=0.5\textwidth]{images_u2s2_beadsystemforce}
  \caption{Diagram of bead and wire2}
\end{figure}

The vector $\, \overrightarrow {F}_ s\,$ has magnitude

\begin{equation*}
  \displaystyle  \displaystyle |\overrightarrow {F_ s}|
  \displaystyle =
  \displaystyle  k|d|
\end{equation*}

where

\begin{equation*}
  \displaystyle d
  \displaystyle =
  \displaystyle  \sqrt {x^2+h^2}-L
\end{equation*}

is the difference between the stretched length and relax length of the spring.
To find the horizontal component,
we multiply by the sine of the angle between the spring and the dotted vertical line in the image.
This quantity is given by $\displaystyle \frac{x}{\sqrt {x^2+h^2}}$.
This implies the magnitude of the horizontal component $F_x$ of the spring force is

\begin{align*}
  \displaystyle  \displaystyle |F_ x|
  \displaystyle &=  \displaystyle  \left(\frac{|x|}{\sqrt {x^2+h^2}}\right)
                  \left|\overrightarrow {F}_ s\right| \\
  \displaystyle &= \displaystyle  k \frac{|x|\, |d|}{\sqrt {x^2+h^2}}.                  
\end{align*}

Let us now figure out the direction of $F_x$.
\begin{itemize}
\item If the bead is to the right of the center $\,(x>0)\,$
  and the spring is stretched $\,(d>0),\,$ then $F_x$ pulls the bead towards the center of the wire,
  which is in the $- x$ direction.
\item If the bead is to the right of the center $\,(x>0)\,$
  but the spring is compressed $\, (d<0),\,$
  then $F_x$ pushes the bead away the center of the wire, which is in the $+ x$ direction.
\end{itemize}

Therefore,

\begin{equation*}
  \displaystyle \displaystyle F_x
  \displaystyle =
  \displaystyle -k\frac{xd}{\sqrt {x^2+h^2}}.
\end{equation*}

This means in the stretched case, $\,x>0,\, d>0,\, F_x < 0\,$;
and in the compressed case, $\,x>0,\, d<0,\, F_x > 0$.
Check that this formula also has the correct sign when the bead is to the left of the center $\,(x<0)\,$.
Finally, plugging in $d$, we have

\begin{align*}
  \displaystyle F_x
  \displaystyle &= \displaystyle - k \frac{x\left(\sqrt {x^2+h^2}-L\right)}{\sqrt {x^2+h^2}} \\
  \displaystyle &= \displaystyle -kx \left(1-\frac{L}{\sqrt {x^2+h^2}}\right).
\end{align*}

\paragraph{Bead on wire: Bifurcation diagram from the DE}
Now that we have our two forces, the \emph{friction force}, which
the horizontal component of the \emph{spring force}, which you just calculated,

\begin{align*}
  f &= -b \dot x, \qquad (b > 0) \\
  F_x &= -kx \left(1-\frac{L}{\sqrt {x^2+h^2}}\right).
\end{align*}

We can use Newton's second law to write down a differential equation for the system. \\
Newton's second law says that the mass times the acceleration
is equal to the sum of the forces acting on our bead.
In mathematical terms,

\begin{equation*}
  m \ddot x = -b \dot x -kx \left(1-\frac{L}{\sqrt {x^2+h^2}}\right).
\end{equation*}

Now, we're going to make the simplifying assumption that we're considering a very, very light bead,
and that the mass can be approximated by $0$.
So we're going to assume, $\, m =  0\,$.
That means that this term $\, \ddot mx = 0,\,$ and we can then solve for $\dot x$.
Doing this gives that

\begin{equation*}
  \dot x = \frac{k}{b} x \left(\frac{L}{\sqrt {x^2+h^2}} - 1 \right).
\end{equation*}

Where are these terms have been switched due to distributing the negative sign.
Now this is the differential equation that describes the horizontal motion of the bead. \\
So we could plug this equation into an ODE solver,
like Matlab or Mathematica, to see the quantitative behavior
of the solutions.
But right now, we're just interested in the equilibrium,
and the stability of those equilibria.
So to describe this we need to draw a bifurcation diagram.\\

To do that, we first need to find the equilibria.
Equilibria occur when $\, \dot x  = 0\,$. 
We know that $k$ and $b$ are both positive quantities.
So $k/b$ is positive.
Now, we can say that the terms that would cause $\, \dot x = 0\,$
are $x$ and this factor $\displaystyle \left(\frac{L}{\sqrt {x^2+h^2}} - 1 \right)$.
So if $\, x = 0, \, \dot x = 0,\,$
meaning that I'm going to call this equilibrium $\, x_0 ^{*} = 0 \,$  one equilibrium position.
And now if we set this term $\, \displaystyle \left(\frac{L}{\sqrt {x^2+h^2}} - 1 \right) = 0 \, $
and solve for $x$, we get two equilibrium positions which I will
call $\, x_{\pm} ^{*}\,$.
And those are equal to $\, \pm \sqrt{L^2 - h^2} \,$. \\

So now we're going to plot these equilibrium positions
in the $(h,\,x)$ plane, where we have $x$ on the vertical axis
and $h$ on the horizontal axis.
Now, the equilibrium position $\,x = 0 \,$ just gives us the axis.
$\, x_{\pm} ^{*}\,$ gives us a semicircle with a radius $L$ and centered at the origin.
So now we're going to determine the sign of $\dot x$ in each of these four regions
that are bounded by the equilibrium branches.

\begin{wrapfigure}{r}{4cm}
  \includegraphics[width=4cm]{image-bead_wire6}
  \caption{Bifurcation diagram of bead and wire}
\end{wrapfigure}

And this will tell us more about the long-term behavior of the solutions.
So in this region we know that $x$ is positive, because we're in the upper half plane.
And we have that the $\, \sqrt {x^2+h^2} \,$ is larger than the radius, $L$.
So quantity $\, \displaystyle (\frac{L}{\sqrt {x^2+h^2}} - 1) < 1 , \,$ and subtracting $1$
makes this factor negative.

\begin{equation*}
  \dot x = \frac{k}{b} \underbrace{x}_{+}
  \underbrace{\left(\frac{L}{\sqrt {x^2+h^2}} - 1 \right)}_{-}.
\end{equation*}

So a positive times a negative gives us a negative,
and so we have a downwards arrow in this region.
Inside the semicircle, we have that the $\, \sqrt {x^2+h^2} \,$ less than the radius,
because we're inside the semicircle.
So it's less than L. This quantity $\, \displaystyle (\frac{L}{\sqrt {x^2+h^2}} - 1) > 1 , \,$
and subtracting $1$ gives us a positive quantity here.
The $x$ is still positive, because we're in the upper half plane.

\begin{equation*}
  \dot x = \frac{k}{b} \underbrace{x}_{+}
  \underbrace{\left(\frac{L}{\sqrt {x^2+h^2}} - 1 \right)}_{+}.
\end{equation*}

So a positive times a positive gives us that $\dot x$ is positive, and we denote just
by an upwards arrow.
You can continue these arguments for the other two regions,
and you'll see that we'll get an upwards arrow here and a downwards arrow here.
These arrows will help us determine the stability of the branches.
Branches that have arrows pointing towards them are stable.
So that would be the semicircle and this part of the $h$ axis.
Branches that have arrows pointing away from them
are unstable equilibria. So that you denotes this area here. \\

So now we have the bifurcation diagram that describes the equilibrium positions,
and we found this by using Newton's second law, writing down our differential equation,
and setting $\dot x$ equal to 0.

\begin{remark}
  The arrows in a bifurcation diagram often switch directions across adjacent regions.
  However, always double check using the differential equation.
\end{remark}

\clearpage

\subsection{Recitation}

\subsubsection{Phase lines}

Horseshoe crab blood (which is bright blue since it is based on copper rather than iron)
contains cells that respond to bacteria at about one part in a trillion.
Starting in 1970 it has been used to detect bacterial infections in humans.
A start-up in Maryland is considering creating a horseshoe crab farm,
and seeks advice about how large a herd they should create
in order to sustain the level of harvesting that would make a profit, namely, 250 crabs per month.\\

Here's a model. The unknown $a$ is proportional to the size of the farm.
This proportionality constant is the maximum population in the absence of harvesting
(measured in kilocrabs).
The larger the farm, the larger the maximum population with no harvesting.
Crabs multiply fast in an unconstrained environment, at the rate of $1 \text{month}^{−1}$.
So the logistic growth rate is $\, (1−y/a),\,$ and the logistic equation is

\begin{equation*}
  \dot y = \left( 1 - (y/a) \right)y = y - y^2/a.  
\end{equation*}

If we now include the desired harvest rate we get

\begin{equation*}
  \dot y = -\frac{1}{4} + y - \frac{y^2}{a}. 
\end{equation*}

Our job is to determine the behavior of the crab population for various values of $a$.

\begin{problem}
  Phase line for particular value of $a$
\end{problem}

Just as an experiment, sketch the phase line for $\,a=2.$
Be sure to include all critical points; say whether they are stable or unstable
(or neither), and include the direction of travel of solutions. \\

The DE is

\begin{equation*}
  \dot y = -\frac{1}{4} + y - \frac{y^2}{2}.
\end{equation*}

To find critical point set $\, dot y = 0\,$

\begin{align*}
  -\frac{1}{4} + y - \frac{y^2}{2} &= 0 \\
  y^2 -2y + \frac{1}{4} &= 0 \\
  y = 1 + \pm \sqrt{2}
\end{align*}

\begin{itemize}
\item $\, y < 1 - \sqrt{2}\,$ then, $\dot y$ is decrease.
\item $\, 1 - \sqrt{2} < y < 1 + \sqrt{2}\, $ then $\dot y$ is increases.
\item $\, y > 1 + \sqrt{2}\,$ then, $\dot y$ is decrease.
\end{itemize}

\clearpage

The phase line is

\begin{figure}[ht!]
  \centering
  \includegraphics[width=0.5\textwidth]{image-recitation_phase_line}
  \caption{Phase line of $\, \dot y = -\frac{1}{4} + y - \frac{y^2}{2} $}
\end{figure}

\begin{problem}
  Sketch solutions
\end{problem}

Again for $\,a = 2,\,$ sketch representative graphs of the population evolution,
one of each type of solution.

The solution curves is

\begin{figure}[ht!]
  \centering
  \includegraphics[width=0.5\textwidth]{image-recitation_soluation_curve}
  \caption{Solution Curve of $\, \dot y = -\frac{1}{4} + y - \frac{y^2}{2} $}
\end{figure}

\clearpage

\subsubsection{Horseshoe crab farm continued}

Our model for a profitable horseshoe crab farm with
harvest rate $250$ crabs per month is given by

\begin{equation*}
  \dot y = -\frac{1}{4} + y - \frac{y^2}{2}.
\end{equation*}

where $y$ is has units of kilocrabs.

\begin{problem}
  Invoking the mathlet
\end{problem}

It looks as if the farm could be made smaller and still produce the desired harvest.
To see how small a can be made, it will be useful for you invoke the Phase Lines Mathlet.
Select the appropriate autonomous equation from it, and set $\, a=2$.
You can create solutions by clicking on the graphing window,
and visualize the phase line by selecting the [Phase Line] button.\\

\href{http://mathlets.org/mathlets/phase-lines/}
{Phase Lines}

\begin{figure}[ht!]
  \centering
  \includegraphics[width=0.5\textwidth]{image-mathlet_Bifurcation_Diagram}
  \caption{Mathlet of $\, \dot y = -\frac{1}{4} + y - \frac{y^2}{2} \,$}
\end{figure}

The \emph{\color{blue} blue} is used to denote stable equilibria and
the \emph{\color{orange} orange} is used to denote unstable equilibria.

\begin{problem}
  Vary the parameter
\end{problem}

Now vary $a$ on the tool.
What seems to be the minimum value of a resulting in a stable positive population of crabs? \\

When vary $a$ the minimum of $\, a  = 1\,$ and $\, y_{min} = 0.5 \,$.

\begin{problem}
  Bifurcation diagram
\end{problem}

You can visualize all these phase lines simultaneously using the [Bifurcation diagram] option.
Select it, and then vary $a$.
You can see the various phase lines being formed and changing as $a$ varies.
The pair of critical points for large $a$ merge at a certain value of $a$;
and for smaller values of $a$ the population crashes. \\

What is the relation between $a$ and $y$ describing
the curve you see on the bifurcation diagram?
Write it in the form (an implicit function of $a$ and $y$) $\,=0 \,$. \\

The function is

\begin{equation*}
  0 = -\frac{1}{4} + y - \frac{y^2}{a} 
\end{equation*}

Finally, determine by computation the position of the semi-stable critical point
and compare your answers with what you measured earlier and with what you see on the Mathlet. \\

Put the  $\, y_{min} = 0.5 \,$ in the equation

\begin{equation*}
  -\frac{1}{4} + \frac{1}{2} - \frac{(1/2)^2}{a}  = 0
  \qquad \Rightarrow a = 1. 
\end{equation*}

\clearpage

\subsection{Numerical Methods}
\subsubsection{Numerical Methods}

\textbf{Objectives}

\begin{itemize}
\item Apply \emph{\color{blue}Euler's Method} to both linear
  and nonlinear differential equations
  with a given step size to \emph{approximate} values of
  the solution function near some given initial values.
\item Describe how \emph{\color{blue}concavity or convexity}
  contribute to the error using Euler's Method.
\item Gain a sense of how \emph{\color{blue}higher order methods} reduce the error.
\end{itemize}

\clearpage

\subsubsection{Euler's method}

Consider an ODE $\, \dot y = f(t,y)\,$.
It specifies a slope field in the $\,(t,\,y)\,$-plane, and solution curves follow the slope field. \\

Suppose that we are given a starting point $\,(t0,\,y0),\,$
and that we are trying to approximate the solution curve through it.

\begin{question}
  Where, approximately,
  will be the point on the solution curve at a time $h$ seconds later?
\end{question}

We have $\,y(t_0) = y_0,\,$ and $\,y'(t_0) = f(t_0,\,y_0)$.
Using linear approximation, we get

\begin{equation*}
  y(t_0+h)=y_0+hf(t_0, y_0).
\end{equation*}

The geometrical picture as follows:

\begin{figure}[ht!]
  \centering
  \includegraphics[width=0.5\textwidth]{images_u1c5s3_Euler1step}
\end{figure}

\begin{equation*}
  {\color{orange}{ y_1-y_0\, = \, h y'(t_0)\, =\, hf(t_0,y_0)}}
\end{equation*}

\paragraph{Euler's method}
I think it's a good idea-- since in real life,
most of the differential equations are solved by numerical methods--
to introduce you to those right away.
Even when you see where you saw the computer screen,
the solutions being drawn, of course what really was happening is that the computer
was calculating the solutions numerically and plotting the points.
So this is the main way. \\
Numerically is the main way differential equations
are actually solved if they have any complexity at all.
Now, so the problem is that initial value problem--
let's write a first-order problem the way

\begin{equation*}
  y' = f(x, y). 
\end{equation*}

And now I'll specifically add to that the starting point
that you used when you did the computer experiments.
And I'll write the starting point this way.

\begin{equation*}
  y(x_0) = y_0. 
\end{equation*}

So this is the initial condition,
and this is the first-order differential equation.
And as you know, the two of them together are called an \emph{IVP},
an initial value problem, which means two things--
the differential equation and the initial value that you want to start the solution at. \\

Now the method we're going to talk about,the basic method, of which many others
are merely refinements in one way or another, is called \emph{Euler's method}. \\
Euler, who did of course everything in analysis,
as far as I know, didn't actually use it to compute solutions of differential equations.
His interest was theoretical.
He used it as a method of proving the existence theorem,
proving that solutions existed.
But nowadays, it's used to calculate the solutions numerically. \\

And the method is very simple to describe.
It's so naive you probably think that if you had been living
300 years ago, you would have discovered it and covered yourself with glory for all eternity.\\

So here is our starting point, $x_0,\, y_0$.



Now, what information do we have?
At that point, all we have is the little line element whose
slope is given by $f(x, \, y)$.
So if I start the solution, the only way the solution could possibly go would
be to start off in that direction, since I have no other information.
At least it has the correct direction at $x_0, \, y_0,\,$
but of course, it's not likely to have the correct direction
anywhere else.
Now, what you do then is choose a step size.
I'll draw just a few, two steps of the method, and that's, I think, good enough.
You choose a uniform step size, which is usually called $h$,
and you continue that solution until you get to the next point,
which will be $\, x0 +  h,\,$ as I've drawn it on the picture.
So we get to here. We stop at that point.
And now you recalculate what the line element is here.
Suppose here, the line element now through this point
goes like that.
Well, then that's the new direction that you should start out with, going from here.
And so the next step of the process will carry you to here.
That's two steps of Euler's method.
Notice it produces a broken line approximation to the solution,
but in fact you only see that broken line
if you're at a computer, if you're looking at the computer
visual, for example, whose purpose is to illustrate
for you Euler's method.
In actual practice, what you see is just the computer is simply calculating
this point, that point, that point, and the succession of points.
And many programs will just automatically connect
those points by a smooth-looking curve, if that's what you prefer to see.
Well, that's all there is to the method. \\

What we have to do now is derive the equations for the method.
Now, how are we going to do that?

\begin{wrapfigure}{r}{5.5cm}
  \includegraphics[width=5.5cm]{image-method_Eulers_formular}
  \caption{Method of Euler's Formula}
\end{wrapfigure}

Well, the essence of it is how to get from the $n$ th step to the $n$ plus first step.
So I'm going to draw a picture just to illustrate that.
So now we're not at $x_0$, but let's say we've already gotten to $x_n, \, y_n$.
How do I take the next step?
Well, I take the line element and it goes up like that,
let's say, because the slope is this.
I'm going to call that slope $A_n$.
Of course, $A_n$ is the value of the right hand side
at the point $\, x_n,\, y_n, \,$ and we'll need that in the equation.
But I think it will be a little clearer if I just give it a capital letter at this point.
Now, this is the new point, and all I want to know is, what are its coordinates?
Well, the $\, x_n + 1 \, $ is there. The $\, y_n + 1 \,$ is here.
Clearly I should draw this triangle, complete the triangle.
This side of the triangle, the hypotenuse, has slope $A_n$.
This side of the triangle has length $h$. The $h$ is the step size.
Perhaps I'd better indicate that, actually put that up,
so that you know the word. Step size.
It means how-- it's the step size on the $x$-axis, how far you have to go to get
from each $x$ to the next one.
What's this? Well, if that slope has the slope $A_n, \,$ this is $h, \,$
then this must be $\, h A_n, \,$ the length of that side,
right, in order that the ratio of the height to this width should be $A_n$.
And that gives us the method. \\
How do I get from--
clearly, to get from $x_n$ to $x _{n+1}, \,$ I simply add $h$.
So that's the trivial part of it.
The interesting thing is, how do I get the new $\, y_{n+1}\,$?
And so the best way to write it as that

\begin{equation*}
  y _{n+1} - y_n = h A_n 
\end{equation*}

So that's the way to write it.
Or, since the computer is interested in calculating
$y_{n+1}$ itself, put this on the other side.
You take the old $y_n, \,$ the previous one, and to it you add $h A_n $.

\begin{equation*}
  y_{n+1} = y_n + h A_n
\end{equation*}

And what, pray tell, is $A_n$?
Well, the computer has to be told that An is the value of $f(x, \, y)$.
So now, with that, let's actually write the Euler program.
Not the program, but the Euler method equations.
Let's just call it the Euler equations.
What will they be?

\begin{align*}
  x_{n + 1} &= x_n + h \\
  y_{n+1} &= y_n + h A_n \\
  A_n &= f(x_n, \, y_n)
\end{align*}

So it's these three equations which define Euler's method.
And these are the equations.
These would be the recursive equations that you would put in to do that. \\

\textbf{\color{orange}Note on the previous paragraph:}
In the above paragraph $x$ uses instead of $t$, as the independent variable. \\

Consider again the ODE $\, \dot y = f(t,y) \,$ and the starting point $(t_0 , y_0)$.
We try to approximate the solution curve through it.

\begin{question}
  Where, approximately, will be the point on the solution curve at
  time $\, t_0 + 3h \,$?
\end{question}

The crude answer would be to take $3$ steps each using the initial slope
$\, f(t_0, y_0) \,$ (or equivalently, one big step of width $3h$).

\clearpage

Geometrically:

\begin{figure}[ht!]
  \centering
  \includegraphics[width=0.5\textwidth]{images_u1c5s3_crude3step}
\end{figure}

The more refined answer is called \emph{\color{blue}Euler's method} :
take $3$ steps, \emph{but reassess the slope after each step, using the slope field at each successive position} :

\begin{figure}[ht!]
  \centering
  \includegraphics[width=0.5\textwidth]{images_u1c5s3_Euler3stepsnew}
\end{figure}

\begin{align*}
  t_1 = t_0 + h & \qquad y_1 = y_0 + f(t_0, y_0)h \\
  t_2 = t_1 + h & \qquad y_1 = y_1 + f(t_1, y_1)h \\
  t_3 = t_2 + h & \qquad y_3 = y_2 + f(t_2, y_2)h 
\end{align*}

The sequence of line segments from
$(t_0 , y_0)$ to $(t_1 , y_1)$ to $(t_2 , y_2)$ to $(t_3 , y_3)$ is
a piecewise linear approximation to the solution curve.
The more refined answer to the question is $(t_3 , y_3)$. \\

\textbf{Euler's Method:} \\
Given an initial value problem

\begin{equation*}
  y' = f(t, x) \qquad y(t_0) = y_0, 
\end{equation*}

and a choice of step size $h$ (in seconds if time is the independent variable),
the Euler method gives an approximation to the solution curve between
$\, t = t_0 \,$ and $\, t = t_0 + (n + 1)h , \,$ by a sequence of
line segments connecting the points
$ \,  (t_0,y_0),\, (t_1,y_1),\, \ldots \, (t_ n,y_ n), \, (t_{n+1},y_{n+1}),\,$
where for each $0\leq k\leq n,\,$

\begin{align*}
  t_{k+1} &= t_k + h \\
  y_{k+1} &= y+k + hf(t_k , y_k).
\end{align*}

\begin{figure}[ht!]
  \centering
  \includegraphics[width=0.5\textwidth]{images_u1c5s3_Eulernsteps}
\end{figure}

\begin{equation*}
  {\color{orange}{y_{k+1}-y_ k\, = \, h y'(t_ k)\, =\, f(t_ k,y_ k)}}
\end{equation*}

These calculations are usually done by computer, and there are round-off errors in calculations.
But even if there are no round-off errors, Euler's method hardly ever gives the exact answer.
The problem is that the actual solution is rarely a straight line.  \\

\paragraph{Worked example}
Let's try an example then.

\begin{align*}
  x_{n + 1} &= x_n + h \\
  y_{n+1} &= y_n + h A_n \\
  A_n = f(x_n , \, y_n)
\end{align*}

So let's calculate.
I'll use a simple example, but it's not entirely trivial.
My example is going to be the equation

\begin{align*}
  y' &= x^2 - y^2 \\
  y(0) &= 1 \\
  h &= 0.1 \\ 
\end{align*}

And so this is my initial value problem.
That pair of equations, and I have to specify a step size.
So let's take the step size to be $0.1$.
You choose the step size, or the computer does,
we'll have to talk about that in a few minutes.
Now, what do you do?\\
Well, I say this is a non-trivial equation,
because this equation, as far as I know,
cannot be solved in terms of elementary functions.
So this equation will be, in fact, a very good candidate for a numerical method like Euler's.
And you had to use it-- or maybe it was the other way around.
You drew a picture of the direction, field,
and answered some questions about the isoclines, how the solutions behave.
All right now, the main thing I want you to get,
this is not just for Euler's, talking about Euler's equations, but in general,
for the calculations you have to do in this course,
it's extremely important to be \emph{systematic}.
Because if you are not systematic,
if you just scribble, scribble, scribble, scribble,
you can do the work, but it becomes impossible to find mistakes.
You must do the work in a form in which it
can be checked, which you can look over it and find and try
to see where the mistakes are, if in fact there are any.
So I strongly suggest that you make a little table to do Euler's method by hand.
I'd only ask you for a step or two,
since I'm just trying to make sure you have some idea of these equations
and where they come from.
So first the value of $n$, then the value of $x_n,\,$
then the value of the $y_n$.
And then a couple of more columns which tells you how to do the calculation.
You're going to need the value of the slope $,\, A_n \,$
and it's probably a good idea, also--
because otherwise, you'll forget it--
to put in $h A_n,\,$ because that occurs in the formula.
All right, let's start doing it. \\

Well, the first value of $n$ is $0$.
That's the starting point.
At the starting $x_0,\,  y_0,\,$  $x$ has the value $0,\,$
and $y$ has the value $1$.
In other words, I'm carrying out exactly what
I drew pictorially, only now I'm doing it arithmetically using
a table and substituting into the formulas.
OK, the next thing we have to calculate is $A_n$.
Well, since An is the value of the right hand
side$,\, x^2 - y^2, \,$ at the point $\, 0, \, 1,\,$ you have to plug that in.
The value of the slope there is $-1$.
Now, I have to multiply that by $h$.
$h$ is $-0.1$. 

\begin{table}[ht!]
  \centering
  \begin{tabular}{lllll}
    $n$ & $x_n$ & $y_n$ & $A_n$ & $h A_n$ \\
    0  & 0     & 1     &  -1   &  -0.1    \\
    1  & 0.1   & 0.9   & -0.80 &  -0.08   \\
    2  & 0.2   & 0.82  & 0.6324&  0.06324 \\ 
  \end{tabular}
\end{table}

What's the value of $x _n$?
Well, to the old $1$, I add one tenth.
What's the value of $y$?
Well, at this point, you have to do the calculation.
It's the old value of $y$--
to get this new value, it's the old value plus $-0.1$.
Well, that's $\,y_{1} = y_{0} + h A_n  = 1 - 0.1 = 0.9 \,$.
An, now I have to calculate the new slope at this point.
OK, that is $\, 0.1^2 - 0.9^2 = -0.80 \,$. 
I now multiply that by $h$, and which
means it's going to be $- 0.08$, perhaps, with a $0$ after.
I didn't tell you how many decimal places.
Let's carry it out to two decimal places.
I think that will be good enough.

And finally the last step, $2$ here,
add $1$, add another one tenth, so the value of $x$ is now two tenths.
And finally, what's the value of $y$?
Well, I didn't tell you where to stop.
Let's stop at $y$ of $0.2,\,$ because there's no more room on the blackboard.
Seems like an excellent--
about approximately how big is that, in other words.
Then, this is going to be this, the old $y$,
plus this number, which seems to be $0.82$ to me.
So the answer is, the new value is $0.82$.

\clearpage

\begin{exercise}
  Practice
\end{exercise}

Continue with the example from the above paragraph
and use Euler method with time step $\, h=0.1, \,$ to approximate the solution
to the initial value problem

\begin{equation*}
  y'\, =\, f(x,y)\, =\, x^2-y^2, \qquad y(0)=1
\end{equation*}

at $\, x = 0.3 $.

\begin{table}[ht!]
  \centering
  \begin{tabular}{lllll}
    $n$ & $x_n$ & $y_n$ & $A_n$ & $h A_n$ \\
    0  & 0     & 1     &  -1   &  -0.1    \\
    1  & 0.1   & 0.9   & -0.80 &  -0.08   \\
    2  & 0.2   & 0.82  & -0.6324&  -0.06324 \\
    3  & 0.3   & 0.75676  &  &   
  \end{tabular}
\end{table}

So,

\begin{equation*}
  y(0.3) \approx \quad 0.76
\end{equation*}

\clearpage

\subsubsection{Overestimate, underestimate and concavity}

\paragraph{Overestimate, underestimate and concavity}
Well, now let's ask a few questions.
One of the first, most basic thing
is, you know,

\begin{table}[ht!]
  \centering
  \begin{tabular}{lllll}
    $n$ & $x_n$ & $y_n$ & $A_n$ & $h A_n$ \\
    0  & 0     & 1     &  -1   &  -0.1    \\
    1  & 0.1   & 0.9   & -0.80 &  -0.08   \\
    2  & 0.2   & 0.82  & -0.6324&  -0.06324 
  \end{tabular}
\end{table}

\begin{equation*}
  y(0.2) \approx 0.82
\end{equation*}

how right is this?\\

How can I answer such a question if I have no explicit formula
for that solution?
That's the basic problem with numerical calculation.
In other words, I have to wander around
in the dark to some extent, and yet
have some idea when I've arrived at the place that I want to go. \\
Well, the first question I'd like to answer,
question is, is this too high or too low?
Is Euler-- sorry, he will forgive me in heaven.
I will use him.
By this, I mean, is the result--
well, let me say something first, and then I'll criticize it.
Is Euler too high or too low? \\
In other words, is the result of using Euler's method, i.e.
is this number too high or too low?
Is it higher than the right answer, what it should be?
Or is it lower than the right answer?
Or, god forbid, is it exactly right?
It's almost never exactly right.
That's not an option.

Now, how will we answer that question?
Well, let's answer it geometrically.
Basically, if the solution were a line, were a straight line,
then the Euler method would be exactly right all the time.\\
But it's not a line.
It's a curve.
Well, the critical question is, is it curved-- is the solution-- so here's the solution.
Let's call it $y_1 (x)$.
And let's say here was the starting point.

\begin{wrapfigure}{r}{5.5cm}
  \includegraphics[width=5.5cm]{image-Euler_convex_concave}
  \caption{Convex and Concave on Euler's formula}
\end{wrapfigure}

Here, the solution is \emph{convex}.
And here, the solution is \emph{concave}.
Right, a concave up, a concave down. 
If you learned those words, but I think those have, by now, pretty--
I hope pretty well disappeared from the curriculum.
Call it-- if you haven't up till now--
what mathematicians call it, convex is that, and the other one is concave.
Well, how do the Euler solutions look?
Well, I'll just sketch--
I think from this you can see already, when you start out
on the Euler solution, it's going to go like that.
Now you're too low.
Well, let's suppose after that the line element here is approximately the same as what it is there,
you know, roughly parallel.
After all, they're not too far apart.
And the direction field is continuous.
That is, the directions don't change drastically from one point to another.
Well, then it'll sort of-- but now you see it's still too low.
It's even lower.
As it pathetically tries to follow, it's losing territory.
And that's basically because the curve is convex.
Exactly the opposite would happen \emph{if the curve were concave}, if the solution curve
were concave.
Now it's \emph{too high}, and it's not going to be able to correct that as long as the solution
curve stays concave.
Well, that's probably too optimistic.
It's probably more like this.
So in other words, in this case, \emph{if the curve is convex}, Euler is going to be \emph{too low}.
Let's put $E$ for Euler.
How about that?
Euler is too low.
If it's concave, then Euler is too high.
There's just one little problem left, namely,
if we don't have a formula for the solution,
and we don't have a computer that's
busy drawing the picture for us, in which case
we wouldn't need any of this anyway, how are we supposed to tell if it's convex or concave? \\

Ah, back to calculus.
Calculus to the rescue.
When is a curve convex? \\
A curve is convex if its second derivative $\, y'' > 0,\, $
because the first-- to be convex means the first derivative is increasing
all the time, and therefore the second derivative,
which is the derivative of the first derivative, should be positive.
Just the opposite here-- the curve,
the slope is, the first derivative is decreasing all the time, and therefore the second derivative is $\, y'' < 0 \,$. 
So all we have to do is decide what the first and second--
what the derivative, the second derivative of this solution is.
We should probably call it a solution.
$y(x)$ is a little too vague.
$y_1$ means the solution that started at this point.
So in fact, probably it would have
been better from the beginning to call that $\, y_1, \,$ except there's no room.
y1, let's say.
That means the solution which started out at the point $(0, 1)$.
So I'm still talking about a solution like that.
All right.
So I want to know, if this is positive,
the second derivative is positive at the starting point
$0$, or it's negative.
Now, again, how do you calculate the second derivative if you don't know what the solution is explicitly?
And the answer is, you can do it from the differential equation itself.

How do I do that?
Well, easy.

\begin{equation*}
  y' = x^2 - y^2
\end{equation*}

OK, that tells me how to calculate $y'$
if I know the value of $x$ and $y,\,$ in other words, the point $(0,\, 1)$.
What would be the value of $y''$?
Well, differentiate the equation.

\begin{equation*}
  y'' = 2x - 2yy' 
\end{equation*}

Don't forget to use the chain rule.
So if I want to calculate at $(0, 1),\,$ in other words, at my starting point,
is that curve convex or concave?\\
Well, let's calculate.

\begin{align*}
  y(0) &= 1 \\
  y'(0) & = -1 \\
  y''(0) & = 2 \cdot 0 - 2 \cdot 1 \cdot -1 = 2. 
\end{align*}

I've calculated-- without having the foggiest idea of what the solution is or how it looks, I've
calculated that its second derivative
at the starting point is $2$.
Therefore, my solution is convex at the starting point.
And therefore this Euler approximation,
if I don't carry it out too far, will be too low.
So it's convex, Euler, too low.

\clearpage

One of the first questions to ask about the Euler method is
whether the approximation it gives is too high or too low.
This is in general difficult to decide,
but when the actual solution is known to be either concave or convex, we do know the answer. \\

Recall in Euler's method, we start with the initial value problem

\begin{equation*}
  y' = f(t,y), \qquad y(t_0) = y_0. 
\end{equation*}

Let $y(t)$ be the actual solution and let $\, \bar{y}(t)\,$
be the approximate solution given by Euler's method with stepsize $h$.
As before, $\, t_n = t_0 + nh \,$ is the time after $n$ steps.
\begin{itemize}
\item If $y(t)$ is \emph{convex} (or \emph{concave up}), that is,
  if $\, \ddot y > 0,\,$ in the interval $[t_0 , t_n],$ then
  $\, \bar{y}(t_ n)< y(t_ n)$.

  \includegraphics[width=0.5\textwidth]{images_u1c5s3_convex}

\item If $y(t)$ is \emph{concave} (or \emph{concave down}), that is,
  if $\, \ddot y < 0,\,$ in the interval $[t_0 , t_n],$ then
  $\, \bar{y}(t_ n) > y(t_ n)$.

  \includegraphics[width=0.5\textwidth]{images_u1c5s3_concave}
  
\end{itemize}

Often we do not know whether the actual solution $y(t)$ is convex or concave
in an entire interval.
What we are able to compute without knowledge of the solution is the value of $\ddot y$
at the starting point $t_0$.
This is still useful because this means that as long as we carry out Euler's method
for a short enough time interval,
we can still predict whether the Euler approximation overshoots or undershoots. \\

To find $\ddot y$ at the starting point $t_0,\,$
we differentiate the DE $\, \dot y = f(t,y)$ to find $\ddot y$.
The differentiation is often called implicit because the variable $y$ depends on $t$.
This gives us a formula for $\ddot y$ in terms of $\, t,\, y, \,$ and $\dot y$.
To evaluate $\ddot y(t_0),\,$ plug in the initial condition
$\, t_0,\, y(t_0),\,$ as well as $\, \dot y(t_0) = f(t_0,y_0)$.

\clearpage

\subsubsection{Error and step size}

\paragraph{Error and step size}
The question, then, is naturally this is not the world's best method.
It's not as bad as it seems.
It's not the world's best method because that convexity
and concavity means that you're automatically introducing a \emph{systematic error}.
If you can predict which way the error is
going to be by just knowing whether the curve is convex or concave, then it's not what you want.
You want to at least have a chance of getting
the right answer, whereas this is telling you you're definitely going to get the wrong answer.
All it tells you is--
and it's telling you your answer's going to be too high or too low.
We'd like a better chance of getting the right answer.
So the question is, how do you get a better method? \\

The search is for a \emph{better method}.
Now, the first method--
which will occur I'm sure to anyone
who looks at that picture--
is look, if you want this yellow line to follow the white one--
the white solution-- more accurately, for heaven's sake,
don't take such big steps.
Take small steps, and then it will follow better.
All right, let's draw a picture. \\

So, use a \emph{smaller step size}.
And the picture roughly-- which is
going to justify that-- it will look like this.

\begin{wrapfigure}{r}{5.5cm}
  \includegraphics[width=5.5cm]{image-Error_Euler_method}
  \caption{Error of Euler's Method}
\end{wrapfigure}

If the solution curve looks like this, then
with a big step size, I'm liable to have
something that looks like that.
But if I take a smaller step-- so I suppose I halve the step
size--
how is it going to look then?
Well, I better switch to a different color.
If I halve the step size I'll get a littler--
goes like that-- and now it's following closer.
Of course, I'm stacking the deck,
but see how close it follows?
No, I'm definitely not to be trusted on this, but--
OK, let's do the opposite and make really big steps.
Suppose instead of the yellow one
I use the green one of double step size.
What would have happened then?
Well I've started out, but now I've
gone all the way to there, and now on my way up-- of course,
it has a little further to go-- but if for some reason
I stopped there, you can see I would be still lower.
In other words, the bigger the step size, the more the error.
And where are the errors that we're talking about?
Well, the way to think of the errors--
this is the error--
that number of the error.
You can make it a positive, negative,
or just put it automatically an absolute value sign around it.
That's not so important.
So in other words, the conclusion
is that the error $e$--
the difference between the true value that I should have gotten
and the Euler value that the calculation produced--
\emph{the error $e$ depends on the step size}. \\
Now, how does it depend on the step size?
Well, it's impossible to give an exact formula,
but there's an approximate answer,
which is by and large true.
And the answer is--
so $e$ is going to be a function of $h$.
What function?
Well, asymptotically-- which is another way of putting
quotation marks around what I say--
it's going to be a constant-- some constant--
times $h$.

\begin{equation*}
  e \sim c_1 h 
\end{equation*}

It looks like this.
And for this reason, it's called a first order--
the Euler is a \emph{first order method}.
And now, first order does not refer to the first order of the differential equation.
It's not that use of the word first.
It's not the first order because it's $\, \dot y = f(x, y)$. 
The first order means the fact that $h$ occurs to the first power.
The way people usually say this is, since the normal way
of decreasing the step size--
as you'll see as soon as you try to use a computer visual that
deals with Euler method-- which I highly recommend by the way--
so highly recommended that you have to do it--
is that, the way to say it--
since you-- each new step halves the step size--
that's the usual way to do it.
If you halve the step size--
since this is a constant, if I halve the step size,
I halve the error, approximately.
Halve the step size, halve the error.
That tells you how the error varies with step size for Euler's method. \\

To improve the approximation by the Euler method, we can use a smaller step size $h$,
so that the slopes of the line segments are reassessed more frequently. \\

The cost of this, however, is that to increase $t$ by a fixed amount, more steps will be needed. \\

Under reasonable hypotheses on $f$ – the right hand side of the DE $\, \dot y= f(t,y) \,$
– one can prove that in the limit as $\, h \to 0,\,$ this process converges and produces
an exact solution curve.
This is one way to prove that the solution to the initial value problem exists.
In fact, this is the way Euler proved it.\\

\textbf{Error of approximation:}\\
Let $\, \bar{y}(t)\,$ be the approximate solution given by Euler's method with step size $h$.\\
Let the \emph{\color{blue} error of approximation} $e$
over the interval $\, [a,b] \,$ be

\begin{equation*}
  e\, = \max _{a\leq t\leq b} \left|y(t)-\bar{y}(t)\right|.
\end{equation*}

That is, $e$ is the maximum absolute difference between the actual solution
$y(t)$ and $\bar{y} (t)$. The error $e$ is bounded above by a \emph{linear} function of $h$

\begin{equation*}
  e \leq C h
\end{equation*}

where $C$ is a constant depending on $f$.
Because $h$ occurs as a first power above, not $\, h^{1/2}\,$ or $\, h^2,\,$
Euler's method is called a \emph{\color{blue}first order method}.

\clearpage

\subsubsection{Euler's Method mathlet}

\href{http://mathlets.org/mathlets/eulers-method/}
{EULER'S METHOD}

\begin{exercise}
  Convex Function Example
\end{exercise}

\begin{figure}[ht!]
  \centering
  \includegraphics[width=0.5\textwidth]{image-euler_method_mathlet}
  \caption{Euler's Method $\, y' = 0.5y + 1$}
\end{figure}

In the Euler's method mathlet above, choose the ODE $\, y' = 0.5y+1$. \\
Use the initial point $\,(−2, \, −1)\,$. Construct and
compare Euler solutions using step size $\, h= 0.5,\, 0.25,\, 0.125\,$.
Then compare to the actual solution. \\

Which is the best approximation? \\
The smaller the step size, the smaller the error.
Thus the best choice is the smallest step size $\, h =.125 \,$.

\begin{exercise}
  Error accumulation at each step
\end{exercise}

With the same ODE $\, y' = 0.5y+1 \,$ and initial point $\, (−2,\, −1)\,$.
Compare Euler solutions using step size $\, h = 1.0,\, 0.5,\, 0.25,\, 0.125\, $ to the actual solution.\\

At each step, is the approximation overshooting or undershooting the actual function? \\
The smaller the step size, the more accurate the approximation.
We see that each approximation is undershooting the actual value of the function.
This makes sense because the solution function $y$ is convex,
so the solution function curves upwards at every point, away from the tangent line approximation.

\clearpage

\begin{exercise}
  Concave Function and Error
\end{exercise}

Consider the ODE $\, y' = y^2 - 2y +1\,$. Make a prediction before using the mathlet.
From the initial point $\, (−1,  −1),\,$ will the Euler approximation be overshooting or undershooting?\\

The solution function is concave, and thus is curving downwards,
below the tangent line approximation everywhere.
The Euler approximation is everywhere overshooting.

\begin{figure}[ht!]
  \centering
  \includegraphics[width=0.5\textwidth]{image-euler_method_mathlet1}
  \caption{Euler's Method $\, y' = y^2 - 2y +1\,$}
\end{figure}

\clearpage

\subsubsection{Worked example}

\paragraph{Worked example: Euler's method}
So today, I'd like to tackle a problem
in numerical integration of ODE's, specifically on Euler's method.
And the problem we're interested in considering today
is,

\begin{equation*}
  y' = y^2 - xy. 
\end{equation*}

And we're interested in integrating the solution
that starts at $\, y(0) = -1 ,\,$ using a step size of $0.5$,
and we want to integrate it to $y(1)$.
And then, for the second part, we're interested in if our first step of integration
either overestimates or underestimates the exact solution. \\

So as I mentioned before, this is a problem in numerics
and specifically, whenever you're given an ODE,
you can almost always numerically
integrate it on a computer.
And this is quite possibly the simplest algorithm for numerical integration.
So specifically, what we do is, we take the left-hand side,
the derivative $, \, y' ,\,$ and we approximate it using a very simple,
finite difference formula. \\
So if I take $y'$and approximate it

\begin{align*}
  y_{n + 1} &= y_n + hf(x_n , y_n) \\
  h \qquad &\text{Step size} \\
  f(x, y) &= y^2 - xy \\
  x_{n + 1} &= x_n + h. 
\end{align*}

And we see that $\, y_{n + 1} = y_n + hf(x_n , y_n) \,$
is an approximation to $y'$.
And I'm using subscripts $n$ here just to denote the step of the algorithm.
So for part A, we're asked to integrate the solution
that starts at $\, y(0) = -1\,$ to $y(1)$.
So what this means, for part A, is we want $x_0$  to be $0$
and we want $\, y_0 = -1 \,$. 
Now to further integrate this equation, the quickest way
to do it, especially if you're in a test scenario, is to build a table.

\begin{table}[ht!]
  \centering
  \begin{tabular}{lllll}
    $n$ & $x_n$ & $y_n$ & $f_n = f(x_n, y_n)$ & $ h f_n$ \\
    0  & 0     & -1     &  1   &  0.5    \\
    1  & 0.5   & -0.5   & 0.5 &  0.25   \\
    2  & 1   & -0.25  & &  
  \end{tabular}
\end{table}

And we know that this is the answer we're looking for.
So just to conclude,

\begin{equation*}
  y_2 \approx y(1) = 0.25. 
\end{equation*}

So for part B, we're asked, does our approximation,
$, \, -0.25 , \,$  overestimate or underestimate the actual exact solution of the ODE?
Now in general, what you want to do is, you want to take the second derivative.
However, for this problem, we're only going to consider the first step.
So our first step, does it overestimate or underestimate the exact solution?
And to do this, what we want to do
is, we want to take a look at the concavity.
So we want to look at $y^{\prime \prime}$.

\begin{equation*}
  y^{\prime \prime} = \frac{d}{dx} \left( y^2 - xy \right) = 2yy' -y - xy' |_{x = 0\, y =-1}
\end{equation*}

So this is the first step.
So at $\, x = 0 ,\, y = -1 \,$. 
This is going to give us $\, y^{\prime \prime} = -1\,$. 
And we note that this is less than $0$.
We've just shown that the concavity
at our starting point $\, , x = 0,\,  y = -1,\,$ is less than $0$.
So what this means is that our initial approximation is going
to overestimate the solution.
We can see that it's going to overestimate it just
by a quick sketch.

\begin{wrapfigure}{r}{4cm}
  \includegraphics[width=4cm]{image_eulers_method_working_example}
\end{wrapfigure}

For example, if I were to plot $y$ and $x$,
we're starting off at this point $x$ is equal to $0$,
$y$ is equal to $-1$.
We know the solution, the exact solution, is increasing
and it's concave down because the second derivative is $-1$. 
And, by Euler's formula, what we're doing is,
we're approximating the solution using a tangent line at this point.
So we can see that our approximate solution, when
we take one step to go from here to here--
$x_0$ to $x_1$--
our solution, which is now going to be $y_1$, is going to overestimate the curve.
And the reason it overestimates it, I'll just reiterate again,
is because our solution is concave down.

\clearpage

\subsubsection{Reliability}

\begin{question}
  How can we decide whether answers obtained numerically can be trusted?
\end{question}

Here are some heuristic tests.
(“Heuristic" means that loosely speaking, these tests work in practice,
but they are not proved to work always.)

\begin{itemize}
\item \textbf{\color{blue}Self-consistency:} Solution curves should not cross!
  If numerically computed solution curves cross, a smaller step size is needed.
  (E.g., try the mathlet “Euler's Method" with $\, \dot y = y^2 -  x ,\,$ step size $1$,
  and starting points $(0,0)$ and $(0,1/2)$.)
\item \textbf{\color{blue}Convergence as $h \to 0$:} The estimate for $y(t)$ at a fixed later time $t$
  should converge to the true value as $\, h \to 0$.
  If shrinking $h$ causes the estimate to change a lot, then $h$ is probably not small enough yet.
  (E.g., try the mathlet “Euler's Method" with $\, \dot y = y^2 - x\,$ with starting point
  $(0,0)$ and various step sizes.)
\item \textbf{\color{blue}Structural stability:}Small changes in the DE's parameters should not change the outcome completely.
  If small changes in the parameters drastically change the outcome, the answer should not be trusted.
\item \textbf{\color{blue}Stability:}Small changes in the DE's initial conditions do not change the outcome much.
  One reason for instability could be a separatrix,
  a curve such that nearby starting points on different sides lead to qualitatively different outcomes;
  this is not a fault of the numerical method, but rather an instability in the answer.
\end{itemize}

\href{http://mathlets.org/mathlets/eulers-method/}
{EULER'S METHOD}

\begin{exercise}
  Concept check for self consistency
\end{exercise}

Choose the ODE $y' = y^2 - x$ on the mathlet.
From the mathlet, for which of the following sizes does Euler's method mathlet
show solutions that cross?

\begin{itemize}
\item Failure of self-consistency with step size $1.0$ can be seen by plotting the Euler solutions curves
  with initial points $(0,0)$ and $(0,.5)$.
\item Failure of self-consistency with step size $.5$ can be seen by plotting to Euler solution curves
  with initial points $(0,0)$ and $(0,−3)$.
\item Failure of self-consistency with step size $.25$ can be seen by plotting to Euler solution curves
  with initial points $(0,−2.4)$ and $(0,−3)$.
\item From the mathlet only, step size $0.125$ doesn't seem to cause any consistency issues.
\end{itemize}

\begin{exercise}
  Concept check for reliability
\end{exercise}

Choose the ODE $\, y' = y^2 - x\,$.
Use the initial point $(−0.98,0)$.
Compare the step size $h=0.5$ to the actual solution. Which of the reliability tests fail?\\

As the step size decreases,
the solution does converge to the actual solution.
However, nearby points, such as $(−1, 0)$ and $(−1, −0.1)$ lead to different qualitative behavior,
so this is not stable.

\clearpage

\subsubsection{Change of variable}

Euler's method generally can't be trusted to give reasonable values when $(t,y)$ strays very far from the starting point.
In particular, the solutions it produces usually deviate from the truth as $\, t \to  \pm \infty ,\,$ or
in situations in which $\, y \to \pm \infty$ in finite time.
Anything that goes off the screen can't be trusted.

\begin{example}
  The solution to $\, \dot y = y^2 - t\,$ starting at $(−2,0)$ seems to go to $+ \infty$ in finite time.
  But Euler's method never produces a value of $+ \infty$.
\end{example}

To see what is really happening in this example, try the \emph{\color{blue} change of variable}
$\, u = 1/y \,$.
To rewrite the DE in terms of $u, \,$ substitute $\, y = 1/u \,$ and $\dot y = - \dot u / u^2\,$:

\begin{align*}
  - \frac{\dot u}{u^2} &= \frac{1}{u^2} - t \\
  \dot u &= -1 + tu^2 
\end{align*}

This is equivalent to the original DE, but now,
when $y$ is large, $u$ is small,
and Euler's method can be used to find the time when $u$ crosses $0$,
which is when $y$ blows up.

\clearpage

\subsubsection{Runge–Kutta methods}

When computing $\, \int_a ^b f(t) dt \,$ numerically,
the most primitive method is to use the left Riemann sum:
divide the range of integration into subintervals of width $h$,
and estimate the value of $f(t)$ for each t on the subinterval by the value at the left endpoint.
More sophisticated methods are the trapezoid rule and Simpson's rule , which have smaller errors.\\

There are analogous improvements to Euler's method.

\begin{table}[ht!]
  \centering
  \begin{tabular}{ccc}
    \emph{\color{blue} Integration} & \emph{\color{blue} Differential equation} & \emph{\color{blue} Error} \\
    left Riemann sum   & Euler's method     & $\mathcal{O}(h)$ \\
    trapezoid rule & second-order Runge–Kutta method (RK2) & $\mathcal{O}(h^2)$ \\
    Simpson's rule & fourth-order Runge–Kutta method (RK4) & $\mathcal{O}(h^4)$
  \end{tabular}
\end{table}

The \emph{\color{blue} big $\mathcal{O}$ notation} $\, \mathcal{O}(h^4) \,$ 
means that there is a constant $C$(depending on the differential equation but not on $h$)
such that the error is at most $\, Ch^4,\,$ assuming that $h$ is small.
The error estimates in the table are valid for reasonable differential equations.

\textbf{\color{blue}Better methods:}\\

The Runge–Kutta methods evaluate at more points on the interval $[t_0,t_0 + h]$
to get a better estimate of what happens to the slope over the course of that interval.\\
Below is an example of a \emph{\color{blue}second-order Runge–Kutta method (RK2)}.
It is also called the \emph{\color{blue}midpoint method} or the \emph{\color{blue}modified Euler method}. 
Here is how one step of this method goes:

\begin{figure}[ht!]
  \centering
  \includegraphics[width=0.5\textwidth]{images_u1c5s3_RK2midpoint}
\end{figure}

\begin{enumerate}
\item Starting from $(t_0,y_0),\,$ look ahead to see where one step
  of Euler's method would land, but do not go there!
  Call this temporary point $\, (t_1,\tilde{y}_1).$
\item Find the \emph{midpoint} between the starting point and the temporary point:
  $\, \left( \frac{t_0+t_1}{2}, \frac{y_0+\tilde{y}_1}{2} \right).$
\item Use the slope at this midpoint to find $y_1$

  \begin{equation*}
    y_1\, =\,  y_0 +h f\left( \frac{t_0+t_1}{2}, \frac{y_0+\tilde{y}_1}{2} \right).
  \end{equation*}

  Repeat the steps above using $\, (t_1,y_1)\,$ as the starting point.\\
  Here is a summary of the equations: 

  \begin{align*}
    t_1 &= t_0 + h \\
    \tilde{y}_1 &= y_0+h f(t_0,y_0) \\
    y_1 &= y_0+ h f\left(\frac{t_0+t_1}{2},\frac{y_0+\tilde{y}_1}{2}\right) \\
    \left(t_0, y_0 \right) & \mapsto \left(t_1, y_1\right).
  \end{align*}

\end{enumerate}

\textbf{\color{blue} Even better methods:}\\

The \emph{\color{blue}fourth-order Runge–Kutta method (RK4)} is similar,
but more elaborate, averaging several slopes.
It is the most commonly used method for solving DEs numerically.
Some people simply call it the Runge–Kutta method.
The mathlets we have been playing use RK4 with a small step size to compute the ``actual'' solution to a DE.

\textbf{\color{orange} Note on below paragraph:}
In the below paragraph Prof Mattuck starts by explaining another example of
a second order Runge–Kutta method also called \emph{\color{blue}Heun's method} and
then comments on the error of this method.
Even though the RK2 methods in the text above and in the paragraph are different,
both are second order methods to which Prof Mattuck's comment on error apply.

\paragraph{Heun's method, errors, and Runge–Kutta 4}
However, there are more efficient methods which gets the result faster.
And so if that's our good method, let's call this a still better method.
The better methods aim at being better.
They keep the same idea as Euler's method,
but they say, look, let's try to improve that slope $A_n$.
In other words, since the slope $A_N$ that we started with is guaranteed to be wrong
if the curve is \emph{convex} or \emph{concave},
can we somehow correct it so that for example,
instead of immediately aiming there,
can't we somehow aim it so that by luck, the next step just lands us back on the curve again?
In other words, we're sort of looking for a shortcut
path which, by good luck, will end up back on the curve again.
And all the simple improvements on Euler's method,
and they are the most stable in ways
to solve differential equations numerically, aim at finding a better slope.
So they find a better value for a \emph{better slope}.
They find a better value than $A_n$.
Try to improve that slope that you found.\\
Now, once you have the idea that you should look for a better
slope, it's not very difficult to see what in fact you should try.
Again, I think most of you would say, hey, I would have thought of that.
And you'd be closer in time, since these methods were only
found about maybe 100--
around the turn of the last century is when I place them.
Mostly by some German mathematicians interested in solving equations numerically. \\
All right, so what is the better method?
Our better slope, what should we look for in our better slope?
Well, the simplest procedure is once again,
we're starting from there.

\begin{wrapfigure}{r}{4cm}
  \includegraphics[width=4cm]{image_RK2}
\end{wrapfigure}

And the Euler slope would be the same as the line element.
So the line element looks like this, and our yellow slope $A_n$--
I'll still continue to call it $A_n$-- goes like that, gets to here.
OK, now if the curve were convex, this would be too low,
and therefore the next step would be--
I'm going to draw this next step in pink.
Well, let's continue in here.
Would be going up like that.
I'll call this $B_n$, just because it's the next step of Euler's method.
It could be called $A_n$ prime or something like that.
But this will do.
And now what you do is-- let me put an arrow on it to indicate parallelness.
Go back to the beginning, draw this parallel to $B_n$. So here is $B_n$.
Again, just a line of that same slope.
And now what you should use, the simplest improvement
on Euler's method is take the average of these two,
because that's more likely to hit
the curve than An will, which is sure to be too low if the curve is convex.
In other words, use this instead.
Use that.
So this is our better slope.
OK, what will we call that slope?
We don't have to call it anything.
What would the equations for the method be?\\
Well,

\begin{equation*}
  x_{n+1} = x_n + h
\end{equation*}

is gotten by adding the step size.
So here's my step size, just as it was before.
The new thing is how to get the new value of $y$.
So, $y_{n+1}$ should be the old $y_n$ plus  $h$ times not this crummy slope $A_n$,
but the better, the pink slope.
What's the formula for the pink slope?
Well, let's do it in two steps.

\begin{equation*}
  y_{n+1} = y_n + h \left( \frac{A_n + B_n}{2} \right) 
\end{equation*}

It's the average of $A_n$ and $B_n$. \\
Hey, but you didn't tell me or I didn't tell you would $B_n$ was.
So you now must tell the computer, oh, yes, by the way,
you remember that $A_n$ was what it always was.
The interesting thing is what is $B_n$?
Well, to get $B_n$, $B_n$ is the slope of the line element at this new point.
Now, what am I going to call that new point?
I don't want to call this $y$ value $y_{n+1}$,
because it's this up here that's going to be the $y_{n+1}$.
All this is is a temporary value used to make another calculation, which will then
be combined with the previous calculations to get the right value.
Therefore, give it a temporary name.
That point, we'll call it-- it's not going to be the real $y_{n+1}$
we'll call it $\tilde{y}_{n+1}$, $y_{n+1}$ temporary.
And what's the formula for it?
Well, it's just going to be what the original Euler formula is.
It's going to be yn plus what you would have gotten
if you had calculated--
in other words, it's the point that the Euler method produced,
but it's not finally the point that we want.

\begin{equation*}
  \tilde{y}_{n+1} = y_n + h A_n \\
\end{equation*}

Now, do I have to say anything else?
Yeah, I didn't tell the computer what $B_n$ was.
OK, $B_n$ is the slope of the direction
field at the point $n + 1$.
And the computer knows what that is.
And this point $\tilde{y}_{n+1}$.
So you make a temporary choice of this, calculate that number,
and then go back and as it were correct that value
to this value by using this better slope.

\begin{equation*}
  B_n = f \left(x_{n+1}, \tilde{y}_{n+1} \right)
\end{equation*}

Now, that's all there is to the method,
except I didn't give you its name.
Well, it has three names.
Four names, in fact.
Which one shall I give you?
I don't care.
OK, the shortest name is Heun's method.
But nobody pronounces that correctly.
So it's Heun's method.
It's called also the improved Euler method.
It's called Modified Euler, very expressive word.
Modified Euler's method.
And it's also called RK2.
I'm sure you'll like that name best.
It has a Star Wars sort of sound to it.
RK stands for Runge--Kutta, and the reason for the 2 is not
that it uses--
well, it is that it uses two slopes.
But the real reason for the 2 is that it is a \emph{second-order method}.
So that's the most important thing to put down about it.
It's a second-order method, whereas Euler's was only a \emph{first-order method}.
So Heun's method, or RK2-- that's the shortest thing to write--
is a second-order method, meaning that the error varies
with a step size, like some constant--
it won't be the same as the constant for Euler's method--
times $h$ squared.

\begin{equation*}
  e \sim c_2 (h^2)
\end{equation*}

That's a big saving, because it now means that if you halve the step size,
you're going to decrease the error by a factor of $1/4$.
You will quarter the error.
Now, you say hey, great. Why should anyone use anything else?
Well, think a little second.
The real thing which determines how slowly one of these methods
run is you look at the hardest step of the method and ask how long does the computer take?
How many of those hardest steps are there?
Now the answer is, the hardest step is always
the evaluation of the slope, the evaluation of the function,
because the functions that are in common use are not $x^2 - y^2$.
They take half a page and have as coefficients 10 decimal place numbers, whatever
the engineers doing it, whatever their accuracy was.
So the thing that controls how long a method runs is how many times the slope of the function must be evaluated.
\emph{For Euler, I only have to evaluate it once.}
Here, I have to evaluate it twice. Now, roughly speaking, the number
of function evaluations will each, will give you the exponent.
The method that's called Runge--Kutta fourth order will
require four evaluations of slope,
but the accuracy will be the like $h^4$. Very accurate.
You halve the step size, and it goes down by a factor of $16$. \\
Great. But you had to evaluate the slope four times.
Suppose instead you had halved four times this thing.
What would you have done?
You would have decreased it to $1/16$ of what it was.
You still would increase the number of functions evaluations you needed to four,
and you would have decreased the error by $1/16$.
So in some sense, it really doesn't matter whether you use a very fancy method which requires
more function evaluations.
It's true, the error goes down faster, but you're having to do more work to get it.
So anyway, nothing is free.
Now, there is an RK4.
I think I'll skip that, since I wouldn't dare to ask you any questions about it.
But let me just mention it at least, because it's the standard.
It uses four evaluation. It's the standard method, if you don't
want to do anything fancier.
It's rather inefficient, but it's very accurate. Standard method, accurate.
And you'll see when you use the programs, it's in fact the program which is drawing
those curves, the numerical method which draws all
those curves that you believe in on the computer screen is the RK4 method.
I should give them their names.
Runge--Kutta fourth-order method.
Two mathematicians-- I believe both German mathematicians--
around the turn of the last century.
Runge--Kutta fourth-order method requires you to calculate four slopes.
I won't bother telling you what you do.
But it's just a procedure like that.
It's just a little more elaborate.
And you take two of these, you make up a weighted average for the super slope.
You use weighted average.

\begin{equation*}
  \frac{A_n + 2B_n + 2C_n + D_n}{6}
\end{equation*}

What should I divide that by to get the right--?
$6$. Why $6$?
Well, because if all these numbers were the same,
I'd want it to come out to be whatever that common value was.
Therefore, in a weighted average, you must always divide by the sum of the coefficients.
So this is the \emph{super slope}.
And if you plug that super slope in to here, you will be using the Runge--Kutta method and get
the best possible results.

\clearpage

\subsection{Recitation}

\subsubsection{Euler's method}

\textbf{Euler's Method:}
Numerical method to compute $y(x)$ of $\, y' = f(x,y) \,$ with initial condition
$\, y(x_0) =y_0\,$. \\

This is the only numerical method you are responsible for. (Memorize it.)
The idea is to follow the direction field at a fixed step size
$\, h > 0$.
Start at $P_0=(x_0,y_0)$ and the iterative formulas are given by

\begin{equation*}
  x_{n+1} = x_ n + h; \quad y_{n+1} = y_ n + h\,  f(x_ n,y_ n)
\end{equation*}

\textbf{Error Estimates:} $E = y_{\textrm{approx}} - y_{\textrm{exact}}$

\begin{equation*}
  |E|\le Ch
\end{equation*}

where $C$ is on the order of $|f| + |\partial f/\partial x| + |\partial f/\partial y|$.

\textbf{Remark.} If $y^{\prime \prime }<0$  in a region (concave down), then $E>0$.
(The approximation is bigger than the actual answer.)
Similarly, If $y^{\prime \prime } > 0$, then $E>0$.
(The approximation is smaller than the actual answer.)

\begin{problem}
  Task 1
\end{problem}

Use Euler's method to estimate the value at $\, x=1.5 \,$ of the solution of
$\, y' = y^2 − x^2 = F(x,y)\,$  at $y(0)=−1$. Use $h=0.5$.\\
Recall the notation

\begin{align*}
  x_0 &= 0 \\
  y_0 &= -1, \\
  x_{n+1} &= x_n + h,  \\
  y_{n+1} &= y_n + F(x_n, y_n)h, 
\end{align*}


Enter the values in a table with columns
$n, x_ n, y_ n, F(x_ n,y_ n) , F(x_ n,y_ n)h$.

\begin{table}[ht!]
  \centering
  \begin{tabular}{ccccc}
    $n$ & $x_n$ & $y_n$ & $F(x_n, y_n)$ & $h F(x_n, y_n)$ \\
    0  & 0     & -1     &  1   &  0.5    \\
    1  & 0.5   & -0.5   & 0 &  0   \\
    2  & 1   &  -0.5  & 0 &  0 \\
    3  & 1.5   & -0.875 & &   
  \end{tabular}
\end{table}

\begin{problem}
  Task 2
\end{problem}

Is the estimate from task 1. likely to be too large or too small? \\

Check $y^{\prime \prime}$ at start point for checking concavity

\begin{align*}
  y^{\prime \prime} = 2 y y' - 2 x = 2(-1)(1)  -2(0) = - 2 
\end{align*}

$\, y^{\prime \prime} = -2 < 0 \,$ (concave down) then over estimated.

\clearpage

\subsection{Part A Homework}

\begin{problem}
  Each nonempty isocline for the differential equation
  $ \, y' = \frac{1}{x + y} \,$ consists of 
\end{problem}

Let $m$ is a slope of DE then,

\begin{align*}
  m &= \frac{1}{x + y} \\
  \frac{1}{m} &= x + y \\
  y &= -x + \frac{1}{m}. 
\end{align*}

So, the $m$-isoclines is a line of negative slopes.

\begin{problem}
  Which of the following are true? 
\end{problem}

\begin{itemize}
\item There is a function $y(x)$ defined for all real numbers $x$ that
  satisfies the differential equation $\, y' = e^xy \,$
  and the initial condition $\, y(0)=2$. \\

  By separation of variables

  \begin{align*}
    \frac{dy}{dx} &= e^x y \\
    \frac{1}{y} dy &= e^x dx 
    \int \frac{1}{y} dy &= \int e^x dx \\
    \ln |y| &=  e^x + c_1  \\
    e^{\ln |y|} &=  e^{e^x + c_1}  \\
    y &= Ce^{e^x}. 
  \end{align*}

  Apply initial condition $\, y(0)=2$.

  \begin{equation*}
    2 = Ce^{e^0} \Rightarrow C = \frac{2}{e}
  \end{equation*}

  The $\, y = \frac{2}{e} e^x \,$ is continuous all real numbers, so
  this DE is true by the existence and uniqueness theorem for a linear ODE.

\item There is a differentiable function $y(x)$ defined for all real numbers $x$
  that satisfies the differential equation $\, y' = y^3 \,$
  and the initial condition $\, y(0) = 2$. \\

  By separation of variables

  \begin{align*}
    \frac{dy}{dx} &= y^3 \\
    y^{-3} dy &= dx \\
    \int y^{-3} dy &= \int dx \\
    - \frac{1}{2} y^{-2} &= x + C \\
  \end{align*}

  Apply initial condition $\, y(0) = 2 \,$

  \begin{equation*}
    - \frac{1}{2} 2^{-2} = 0 + C \Rightarrow C = - \frac{1}{8}
  \end{equation*}

  So,

  \begin{align*}
    - \frac{1}{2} y^{-2} &= x  - \frac{1}{8} \\
    y^{-2} &= -2x + \frac{1}{4} \\
    y & = \left( \frac{1}{4} - 2x \right). 
  \end{align*}

  Observe that this solution tends to $ + \infty $ as $\, x \to 1/8 \, $ from the left.
  Therefore it is not defined for all real numbers.
\end{itemize}

\begin{problem}
  Each nonempty isocline for the differential equation $\, y' = y^2 - 4 \, $
  consists of 
\end{problem}

To find $m$-isoclines,

\begin{align*}
  m &= y^2 - 4 \\
  y^2 &= m + 4 \\
  y &= \pm \sqrt{m + 4}
\end{align*}

So $m$-isoclines consist two line.

\begin{problem}
  Given that $x(t)$ is a nonconstant solution to $\, x' = x^2 - 3x -4,\,$
  which of the following are possibilities for $, \lim_{t \to + \infty} x(t)\,$?
\end{problem}

The DE has no independent variable, so it is autonomous equation.
Find critical point by set letf hand side of equation to zero,

\begin{align*}
  & x^2 - 3x - 4 = 0 \\
  &(x - 4)(x + 1 )= 0 \\
  & x = 4 \, \text{or} \, x = -1.  
\end{align*}

The phase line of DE is

\begin{equation*}
  -\infty \quad \longrightarrow \quad -1 \quad
  \longleftarrow \quad 4 \quad \longrightarrow \quad +\infty .
\end{equation*}

So, the possibilities is $\, -1 \,$ and $\, + \infty $.

\begin{problem}
  How many stable critical points does the differential equation
  $\, \dot x =(x-1) (x-3)^2 (x-5)^3 (x-7) (x-9)^2 \,$ have?
\end{problem}

The phase line is

\begin{equation*}
  -\infty \quad \longleftarrow \quad 1 \quad \longrightarrow
  \quad 3 \quad \longrightarrow \quad 5 \quad \longleftarrow
  \quad 7 \quad \longrightarrow \quad 9 \quad \longrightarrow \quad +\infty .
\end{equation*}

The arrows are pointing inward only at $5$, so $5$ is the only stable critical point.

\begin{problem}
  The number $\pi$ is given by $\, \pi = y(1) \,$ where $y$ is a solution to
  the differential equation $\, y' = \frac{4}{1+x^2}, \,  y(0) = 0$.\\
  Approximate $\pi$ using Euler's method with $\, h=.01$.
  Round to $3$ decimal places. (Use a computer.)
\end{problem}

\begin{lstlisting}[language=Python]

def DiffEquation(x):
    """ Return result of calculation of DE. """
    return 4 / (1 + x**2)

#test code
#Step size
h = 0.01
#initial point
x0 = 0; y0 = 0
#final step
x_f = 1.
#set valriable
x = x0; y = y0
print("")
while x <= x_f:
    slope = DiffEquation(x)
    delta_y = h * slope
    y = y + delta_y
    x = x + h 
    print("{:.5f}, {:.5f}, {:.5f}, {:.5f}".format(x, y, slope, delta_y))

\end{lstlisting}

We get an approximation $\pi \approx 3.152$.

\begin{problem}
  Suppose there are initially $3$ bunnies $\, (t=0) \,$ near East Campus dorms at MIT
  and the population satisfies the logistic equation:

  \begin{equation*}
    \displaystyle \dot P = .05P-.002P^2
  \end{equation*}

  (with time measured in months). \\
  
  Use Euler's method to approximate the number of bunnies after $10$ years using step size $1$.
  Round to the nearest integer.
\end{problem}

We implement Euler's method here using a python script.
We are interested in what happens in $10$ years,
which is $120$ months. Using step size $\, h=1, \,$ we can find:

\begin{lstlisting}[language=Python]
t=0 P=3 h=1 while t<120: P=(h+.05)*P-.002*P*P t+=h print P 
\end{lstlisting}

We approximate there to be $25$ rabbits in $10$ years.

\clearpage

\subsection{Part B Homework}

\subsubsection{Parachute jumper}

\begin{problem}
  The critical points
\end{problem}

The equation

\begin{equation*}
  m \dot v = Kv^2 - mg 
\end{equation*}

describes the velocity (\emph{negative velocity points downwards})
of a parachute jumper of mass $m$ subject to gravity $g$, and wind resistance
from the open parachute of $\, Kv^2, \,$ with $K$ a constant.\\

Find the critical points when $\, m = 100$kg, $\, g = 10 meter/sec^2,\,$ and 
$\, K=10kg/meter, \,$ and draw a phase line describing the stability of each point.

Substitute value, then
\begin{align*}
  100kg \dot v &= 10kg/meter v^2 - 100kg 10 meter/sec^2 \\
  \dot v &= \frac{1}{10} v^2 - 10. \\ 
\end{align*}

To find critical point set $\, \dot v = 0, \,$ then

\begin{align*}
  & \frac{1}{10} v^2  - 10 = 0 \\
  & \frac{1}{10} v^2   = 10 \\
  & v^2   = 100 \\
  & v = \pm 10. 
\end{align*}

The critical point $\, v = −10$m/s is a stable equilibrium,
which happens when the parachute reaches terminal velocity
(remember velocity is negative going downwards).
The critical value $\,v = 10$ m/s is an unstable equilibrium.

\includegraphics[width=0.5\textwidth]{image_unit5_homework_partB1}

\clearpage

\begin{problem}
  Sketch some solutions
\end{problem}

Sketch solutions for each of the given different initial condition types.

\includegraphics[width=0.5\textwidth]{image_unit5_homework_partB2}


\textbf{What are g-forces?} \\
Pilots, astronauts, and parachute jumpers are all concerned with the g-forces
they experience. The g-force is a measurement of the acceleration that causes weight.
The unit of measurement is $1$ g, the ordinary force of gravity on Earth.
When the parachute jumper is in free fall with closed parachute,
experiencing only the effect of gravity with no normal force opposing
the force due to gravity, the jumper feels weightless.
Thus the jumper experiences $0$ g's.
The g-force experienced by a parachute jumper is the number of g's of acceleration relative to free fall.

\begin{problem}
  g-force
\end{problem}

Suppose that at $\, t = 0,\, v(0)= - 20$m/s.
Find the time at which the acceleration $\dot v$ is largest in absolute value.
At that moment, what is the g-force experienced by the jumper?

\clearpage

\subsubsection{Nonlinear equations}

\begin{problem}
  Sketching isoclines
\end{problem}

Find the isoclines of slope $m$ for various specified values of $m$ of the differential equation

\begin{equation*}
  \frac{dy}{dx} = y^2 - x^2
\end{equation*}

\begin{itemize}
\item Isocline $\, m = 0 \,$ \\

  \begin{align*}
    &y^2 - x^2 = 0 \\
    &y^2 = x^2 \\
    &y = \pm x. 
  \end{align*}

  So, $0$-isoclines is two line.

\item Isocline $\, m = 2 \,$ \\

  \begin{align*}
    &y^2 - x^2 = 2 \\
    &y^2 = x^2 + 2 \\
    &y = \pm \sqrt{x^2 + 2}. 
  \end{align*}

  So, $0$-isoclines are both branches of a hyperbola.

\item Isocline $\, m = -2 \,$ \\

  \begin{align*}
    &y^2 - x^2 = 2 \\
    &y^2 = x^2 - 2 \\
    &y = \pm \sqrt{x^2 - 2}. 
  \end{align*}

  So, $0$-isoclines are both branches of a hyperbola.
\end{itemize}

\begin{problem}
  Long term behavior of solutions
\end{problem}

Set the mathlet to the DE

\begin{equation*}
  \frac{dy}{dx} = y^2 - x^2
\end{equation*}

There seem to be two different looking behaviors of solutions as $x$ increases.
Some curves go up and some go down.
Look for a value $A$ for which when $\, y(0)>A ,\,$ $y$ exits the frame upwards as $x$ increases,
and for $\, y(0)<A,\,$  $y$ exits much lower. 


%%% Local Variables:
%%% mode: latex
%%% TeX-master: "NoteForDifferentialEquation"
%%% End:


\section{UNIT5}

\clearpage

\subsection{Nonlinear DEs: Graphical methods }

\subsubsection{Nonlinear differential equations: Graphical Methods}

\emph{Objectives}
\begin{itemize}
\item Use \emph{\color{blue} slope fields} and \emph{\color{blue}isoclines}
  to aid in the drawing of solution curves for \emph{\color{blue}first order ODEs}.

\item Draw \emph{\color{blue}integral curves} aided by existence and uniqueness theorems, and assess long term behavior of solutions.
\end{itemize}

\subsubsection{Introduction}

\begin{exercise}
  Review: Linear vs. nonlinear DEs
\end{exercise}
\begin{align*}
  y'=y(1-y) &\qquad y' = xy \\
  y'= x-2y &\qquad y' = -x/y \\
  y'= y^2- x^2 &\qquad y' = y/x \\
  y'= y^3 -3y -x &\qquad yy' = x^3 -y
\end{align*}

\emph{Linear equations:} The following DEs are linear because they can be written in the form
$\, y'+p(x)y=q(x)\,$.
In this form,
\begin{itemize}
\item $y = xy\,$ becomes $y' - xy = 0\, $.
\item $y = x - 2y\,$ becomes $y' + 2y = x\, $.
\item $y = y/x,$ becomes $y' - \frac{1}{x}y = 0\, $. 
\end{itemize}

\emph{Nonlinear equations:} he rest of the DEs are nonlinear because they cannot be written in the form
$\, y'+p(x)y=q(x)\,$ mentioned above. They failed for the reason below:
\begin{itemize}
\item $y' = y(1-y),\,$ rewritten as $y' = y - y^2,\,$ contains a $y^2$ term.
\item $y' = y^2 - x^2\,$ contains $y^2$ term.(The $x^2$ term does not contribute to the classification.)
\item $y' = y^3 -3y -x\,$ contains a $y^3$ term.
\item $y' = -x/y\,$ contains a term with $y$ od degree $-1$.
\item $yy' = x^3 -y\,$ contains the term $yy'$ in which the total degree of $y$ and its derivative is $2$. 
\end{itemize}

\paragraph{Introduction}
First order ODE, ODE, I'll only use three, two acronyms.
ODE is ordinary differential equations.
I think all of MIT knows that whether they've been taking a course or not.\\

So we're talking about first order ODEs, which
in standard form are written, you isolate the derivative of $y$
with respect to $x$, let's say on the left hand side.
And on the right hand side, you write everything else.
\begin{equation*}
  y' = f(x, y)
\end{equation*}
You can't always do this very well,
but for today, I'm going to assume that it has been done and it's doable.
So for example,
\begin{equation*}
  y' = x/y, \quad y' = x - y^2, \quad y' = y - x^2 
\end{equation*}

Now you look at this.
This $\, y' = x/y,\,$ of course, you can solve by \textit{separating variables}.
So this is solvable.
This one $\, y' = y - x^2\, $ is, in neither of these can you separate variables
and they look extremely similar, but they are extremely dissimilar.
The most dissimilar about them is
that this one $\, y' = y - x^2\, $ is easily solvable and you'll
learn if you don't know already next time, next Friday, how to solve this one.
This one $\, y' = x - y^2 ,\,$ which looks almost the same, is unsolvable in a certain sense.
Namely, there are no elementary functions,
which you can write down which will
give a solution of that differential equation.\\

So right away, one confronts the most significant fact
that even for the simplest possible differential
equations,
\begin{equation*}
  \boxed{y' = x - y^2}
\end{equation*}
those which only involve the first derivative,
it's possible to write down extremely simple guys.
I mean not really bad but recalcitrant.
It's not solvable in the ordinary sense in which you
think an equation is solvable.
And since those equations are the rule rather than
the exception, I'm going to devote this first day
to not solving a single differential equation,
but indicating to you what you do
when you meet a boxed equation like that,
what do you do with it.
So this first day is going to be devoted
to geometric ways of looking at differential equations
and numerical.
At the very end, I'll talk a little bit about numerical.\\

First order (nonlinear) equations can be written in the \emph{\color{blue} standard form}:

\begin{equation*}
  y' = f(x,y)
\end{equation*}

where $f(x, y)$ is a function of the variables $x$ and $y$. \\

Notice we now have two standard forms for first order differential equations.
Recall first order \emph{\color{blue}linear} equations
can also be written in the standard form for linear equations:

\begin{equation*}
  y'+p(x)y=q(x).
\end{equation*}


This unit is concerned with first order nonlinear differential equations, such as the one boxed

\begin{equation*}
  \boxed{y' = x - y^2}
\end{equation*}

The sad fact is that we can hardly ever find formulas for the solutions to nonlinear DEs.
Instead we try to understand the qualitative behavior of solutions using geometric methods or approximations.
\clearpage

\subsubsection{Geometric view of DEs}

\paragraph{The geometry}
What's our geometric view of differential equations?

Well, it's something that's contrasted
with the usual procedures by which you solve things and find
elementary functions which solve them.
I'll call that the \textit{analytic method}.
So on the one hand, we have the analytic ideas
in which you write down explicitly the equation
\begin{equation*}
  y' = f(x, y). 
\end{equation*}
And you look for certain functions, which are called solutions.
So there's the ODE.
And $y_1(x)$, notice I don't use a separate letter.
I don't use $g$ or $h$ or something like that for the solution
because the letters multiply so quickly,
multiply in the sense of rabbits that after a while, if you keep
using different letters for each new idea
you can't figure out what you're talking about.
So I'll use $y_1$ means it's a solution of this differential
equation.
Of course, the differential equation
has many solutions containing an arbitrary constant.
So we'll call this the solution.\\


Now \textit{the geometric view}.
The geometric guy that corresponds to this version of writing the equation
is something called a \textit{direction field}.
And the solution is from the geometric point of view,
something called an \textit{integral curve}.
\begin{align*}
  \text{Analytic}  &\qquad  \text{Geometric} \\
  y' = f(x,y) &\Longleftrightarrow \text{Direction Field} \\
  y_1 x &\Longleftrightarrow \text{Integral Curve}
\end{align*}


So let me explain if you don't know what the direction field is.
So what's a direction field?
Well, a direction field is you take the plane.

\begin{figure}[ht!]
  \centering
  \includegraphics[width=0.5\textwidth]{image-direction_field}
  \caption{Direction Field}
\end{figure}

And at each point of the plane-- of course, that's an impossibility.
But you pick some points of the plane.
You draw what's called a little line element.
So there is a point.
It's a little line.
And the only thing which distinguishes it outside of its position in the plane--
so here's the point $x$, $y$ at which we're drawing this line element is its slope.
What is its slope?
Its slope is to be $f(x,y)$.
And now you fill up the plane with these things until you're tired of putting them in.
So I'm going to get tired pretty quickly.
So I don't know.
Let's not make them all go the same way.
There's a few randomly chosen line elements that I put in.
And I put in the slopes at random since I didn't have any particular differential
equation in mind.\\

Now the integral curves.

\begin{figure}[ht!]
  \centering
  \includegraphics[width=0.5\textwidth]{image-integral_curves}
  \caption{Integral Curves}
\end{figure}

Those are the line elements.
And the integral curve is a curve which goes through the plane and at every point
is tangent to the line element there.
So this is the integral curve.
Hey, wait a minute.
That's not tangent to the line element there.
It didn't even touch it.
Well, I can't fill up the plane with line elements.
Here at this point, there was a line element which I didn't bother drawing in.
And it was tangent to that.
Same thing over here.
If I drew the line element here, I would find that the curve had exactly the right slope there. \\

So the point is what distinguishes
the integral curve is that everywhere it
has the direction-- that's the way I'll indicate that it's tangent--
has the direction of the field everywhere at all points, at all points on the curve, of course.
Where it doesn't go, it doesn't have any mission to fulfill.
Now I say that this integral curve
is the graph of the solution to the differential equation.
In other words, writing down analytically a differential equation is the same geometrically
as drawing this direction field.
And solving analytically for a solution to the differential
equation is the same thing as geometrically drawing an integral curve.

\clearpage

The figure below is the geometric picture of the differential equation
$y' = f(x, y).\,$
in previous paragraph this sketch is called a direction field.
In these notes, we will call it a \emph{slope field} instead.
We reserve the word direction field for diagrams whose line elements have arrows.

\begin{figure}[ht!]
  \centering
  \includegraphics[width=0.5\textwidth]{images_u2s1_slopefield}
  \caption{The line element at$(x,y)$ has slop $f(x,y)$}
\end{figure}

\begin{definition}
  For a differential equation $\, y'=f(x,y),\,$ a \emph{\color{blue}slop field} is
  a diagram is a diagram which includes at each point $(x, y)$ a short line
  element (or line segment) whose is the \emph{value} $f(x,y)$. 
\end{definition}

\begin{figure}[ht!]
  \centering
  \includegraphics[width=0.5\textwidth]{images_u2s1_slopefieldsolutioncurve}
  \caption{A solution curve (blue) is tangent to each line segment that it touches.}
\end{figure}

The graph of a solution $y_1 (x)$ to the DE in the $xy$-plane is called a
\emph{\color{blue} solution curve} or an \emph{\color{blue} integral curve}. \\
An integral curve must be tangent to the slope field at every point:
$\displaystyle \, y_1'(x)=f\left(x, y_1(x)\right).$
\clearpage

\subsubsection{Slope field}

\begin{example}
  Sketch the slope field for $y'=y^2-x$
\end{example}


%%% Local Variables:
%%% mode: latex
%%% TeX-master: "NoteForDifferentialEquation"
%%% End:

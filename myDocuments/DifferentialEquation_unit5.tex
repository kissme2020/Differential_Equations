\section{UNIT5}

\clearpage

\subsection{Nonlinear DEs: Graphical methods }

\subsubsection{Nonlinear differential equations: Graphical Methods}

\emph{Objectives}
\begin{itemize}
\item Use \emph{\color{blue} slope fields} and \emph{\color{blue}isoclines}
  to aid in the drawing of solution curves for \emph{\color{blue}first order ODEs}.

\item Draw \emph{\color{blue}integral curves} aided by existence and uniqueness theorems, and assess long term behavior of solutions.
\end{itemize}

\subsubsection{Introduction}

\begin{exercise}
  Review: Linear vs. nonlinear DEs
\end{exercise}
\begin{align*}
  y'=y(1-y) &\qquad y' = xy \\
  y'= x-2y &\qquad y' = -x/y \\
  y'= y^2- x^2 &\qquad y' = y/x \\
  y'= y^3 -3y -x &\qquad yy' = x^3 -y
\end{align*}

\emph{Linear equations:} The following DEs are linear because they can be written in the form
$\, y'+p(x)y=q(x)\,$.
In this form,
\begin{itemize}
\item $y = xy\,$ becomes $y' - xy = 0\, $.
\item $y = x - 2y\,$ becomes $y' + 2y = x\, $.
\item $y = y/x,$ becomes $y' - \frac{1}{x}y = 0\, $. 
\end{itemize}

\emph{Nonlinear equations:} he rest of the DEs are nonlinear because they cannot be written in the form
$\, y'+p(x)y=q(x)\,$ mentioned above. They failed for the reason below:
\begin{itemize}
\item $y' = y(1-y),\,$ rewritten as $y' = y - y^2,\,$ contains a $y^2$ term.
\item $y' = y^2 - x^2\,$ contains $y^2$ term.(The $x^2$ term does not contribute to the classification.)
\item $y' = y^3 -3y -x\,$ contains a $y^3$ term.
\item $y' = -x/y\,$ contains a term with $y$ od degree $-1$.
\item $yy' = x^3 -y\,$ contains the term $yy'$ in which the total degree of $y$ and its derivative is $2$. 
\end{itemize}

\paragraph{Introduction}
First order ODE, ODE, I'll only use three, two acronyms.
ODE is ordinary differential equations.
I think all of MIT knows that whether they've been taking a course or not.\\

So we're talking about first order ODEs, which
in standard form are written, you isolate the derivative of $y$
with respect to $x$, let's say on the left hand side.
And on the right hand side, you write everything else.
\begin{equation*}
  y' = f(x, y)
\end{equation*}
You can't always do this very well,
but for today, I'm going to assume that it has been done and it's doable.
So for example,
\begin{equation*}
  y' = x/y, \quad y' = x - y^2, \quad y' = y - x^2 
\end{equation*}

Now you look at this.
This $\, y' = x/y,\,$ of course, you can solve by \textit{separating variables}.
So this is solvable.
This one $\, y' = y - x^2\, $ is, in neither of these can you separate variables
and they look extremely similar, but they are extremely dissimilar.
The most dissimilar about them is
that this one $\, y' = y - x^2\, $ is easily solvable and you'll
learn if you don't know already next time, next Friday, how to solve this one.
This one $\, y' = x - y^2 ,\,$ which looks almost the same, is unsolvable in a certain sense.
Namely, there are no elementary functions,
which you can write down which will
give a solution of that differential equation.\\

So right away, one confronts the most significant fact
that even for the simplest possible differential
equations,
\begin{equation*}
  \boxed{y' = x - y^2}
\end{equation*}
those which only involve the first derivative,
it's possible to write down extremely simple guys.
I mean not really bad but recalcitrant.
It's not solvable in the ordinary sense in which you
think an equation is solvable.
And since those equations are the rule rather than
the exception, I'm going to devote this first day
to not solving a single differential equation,
but indicating to you what you do
when you meet a boxed equation like that,
what do you do with it.
So this first day is going to be devoted
to geometric ways of looking at differential equations
and numerical.
At the very end, I'll talk a little bit about numerical.\\

First order (nonlinear) equations can be written in the \emph{\color{blue} standard form}:

\begin{equation*}
  y' = f(x,y)
\end{equation*}

where $f(x, y)$ is a function of the variables $x$ and $y$. \\

Notice we now have two standard forms for first order differential equations.
Recall first order \emph{\color{blue}linear} equations
can also be written in the standard form for linear equations:

\begin{equation*}
  y'+p(x)y=q(x).
\end{equation*}


This unit is concerned with first order nonlinear differential equations, such as the one boxed

\begin{equation*}
  \boxed{y' = x - y^2}
\end{equation*}

The sad fact is that we can hardly ever find formulas for the solutions to nonlinear DEs.
Instead we try to understand the qualitative behavior of solutions using geometric methods or approximations.

%%% Local Variables:
%%% mode: latex
%%% TeX-master: "NoteForDifferentialEquation"
%%% End:

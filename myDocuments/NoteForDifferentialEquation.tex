%%% Local Variables:
%%% mode: latex
%%% TeX-master: t
%%% End:

\documentclass[8pt, a4paper]{article}

\usepackage{amsfonts}
\usepackage{amsmath}
\usepackage{amssymb}
\usepackage{mathtools}
\usepackage{cancel}
\usepackage{listings} %for python code.
\usepackage{graphicx} %package for graph.
\usepackage{float} %Thin border around figure.
\floatstyle{boxed} 
\restylefloat{figure}
\usepackage{wrapfig}
\usepackage{sidecap}
\usepackage{caption} 
\usepackage{tcolorbox}
% color
\usepackage{xcolor}
\usepackage{boxedminipage} %Box text
\usepackage{framed} 
\definecolor{light_gray}{gray}{0.95}
\definecolor{dark_gray}{gray}{0.45}
% header note
\usepackage{fancyhdr}
% % Path for image
% \graphicspath{./figures/}

\title{Lecture Note for Differential Equation}
\author{KD KIM}
\date{\today}
\begin{document}
\maketitle

\section{Modeling : Natural population growth}
\subsection{Recitation 1}
A certain African government is trying to come up with a good policy regarding
the hunting of oryx in a specific game preserve.
They are using the following model:
the oryx population has a positive natural growth rate of $k$ $years^{-1}$, 
and there is assumed a constant harvesting rate of a $oryxes/year$.
\newline
\begin{enumerate}
\item Write down a differential equation modeling the oryx population. [First
  step: choose symbols and units.]
\item Discuss this model using the language of signals and systems.
 
\item Suppose $a = 0$: no hunters. What is the doubling time (in terms of $k$)?
What is the relation between the population now and the population after $k-1$
years?

\item Find the general solution of the equation you found in (1).
  Check that the proposed solution satisfies the ODE.
\item  There is a constant solution. Find it.
  Does the way the solutions depend upon $k$ and a make sense?
  That is, do the units come out right?
  Does the solution behave right when a is large or small? When $k$ is large or small?
  Sketch the graphs of some solutions.
\item Notice that for initial values less than the equlibrium,
  the solutions stop having meaning in terms of the real-world situation they are modeling
  when they become negative.
  In these cases, predict the time te at which oryxes will be extirpated from this area.
  For example, suppose that $x(0) = x_0$ is less than the equilibrium population.
  For this initial condition what is te? Check units.
\item Would you recommend that the government base a policy on this idea, for any value of a?

\end{enumerate}

\end{document}

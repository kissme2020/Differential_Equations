%%% Local Variables:
%%% mode: latex
%%% TeX-master: t
%%% End:

\documentclass[8pt, a4paper]{article}

\usepackage{amsfonts}
\usepackage{amsmath}
\usepackage{amssymb}
\usepackage{mathtools}
\usepackage{cancel}
\usepackage{listings} %for python code.
\usepackage{graphicx} %package for graph.
\usepackage{float} %Thin border around figure.
\floatstyle{boxed} 
\restylefloat{figure}
\usepackage{wrapfig}
\usepackage{sidecap}
\usepackage{caption}
\usepackage{color}
\usepackage{tcolorbox}
% color
\usepackage{xcolor}
\usepackage{boxedminipage} %Box text
\usepackage{framed} 
\definecolor{light_gray}{gray}{0.95}
\definecolor{dark_gray}{gray}{0.45}
% header note
\usepackage{fancyhdr}
% % Path for image
% \graphicspath{./figures/}

\title{Lecture Note for Differential Equation}
\author{KD KIM}
\date{\today}
\begin{document}
\maketitle
\clearpage

\section{Unit 1}

\subsection{1 Introduction to differential equations and modeling}

\subsubsection{Differential equations: notation and terminology}

\paragraph{A secret Function} $y(t)$, \\
\subparagraph{It satisfies the differential equation}
\begin{equation*}
  \dot{y} = 3y,
\end{equation*}
which could model population growth in some biological system, like the growth of yeast cells.
Note that \emph{\color{blue} $\dot{y}$} is common notation for the derivative with respect to time;
$\dot{y}$, $y'$, and $\frac{dy}{dt}$ all mean the same thing here.\\
$y = e^{3t}$ is a solution to the differential equation above because substituting it into the DE gives
$3e^{3t}=3e^{3t}$. But it's not the function I was thinking of! Some other solutions are
$y = 7e^{3t}$, $y = -5e^{3t}$, $y = 0$, etc. From calculus we know that the family of functions
\begin{equation*}
  y = ce^{3t}
\end{equation*}
is the \emph{\color{blue} general solution}.
\begin{itemize}
\item for each $c$, the function $y=ce^{3t}$ is a solution, and
\item there are no other solutions besides these.
\end{itemize}
So there is a 1-parameter family of solutions to this DE.
The constant $c$ is a \emph{\color{blue} parameter}.
\subparagraph{The function satisfies the initial condition $y(0)=5$.}
\begin{align*}
  y(0) &= ce^{0} \\
  5 &= c  
\end{align*}
\emph{The solution} satisfying the initial condition $y(0) = 5$ is
\begin{equation*}
  y(t) = 5e^{3t}
\end{equation*}

\subparagraph{Definition 1.1} An \emph{\color{blue} initial value problem}
is a differential equation together with initial conditions.\\

\emph{\color{orange} Important} Checking a solution to a DE is usually easier
than finding the solution in the first place, so it is often worth doing.
Just plug in the function to both sides, and also check that it satisfies the initial condition.\\

In the solution to the DE, only one initial condition was needed,
since only one parameter $c$ needed to be recovered.
\clearpage

\subsubsection{Modeling population growth}
\paragraph{Cell division}
Here we will see how the differential equation for our secret function appears when modeling a natural phenomenon – the population growth of a colony of cells.\\

In this example we'll model the number of yeast cells in a batch of dough.
As we work through this example, pay careful attention to the assumptions we make, and how the \emph{\color{blue} initial condition} plays a role in the resulting differential equation.
\subparagraph{\color{blue} The system}
For our system, we assume we have a colony of yeast cells in a batch of bread dough.
The first step is to identify the variables, the units, and give them names.
\begin{align*}
  y\qquad &\text{number of cells}\\
  t\qquad &\text{time measured in seconds}
\end{align*}

We also need to set some initial condition, $y_0$, the number of cells that we begin with at $t=0$.
In this system, this might be the number of yeast cells in a yeast packet.
\subparagraph{\color{blue} A differential model}
If $y$ denotes the number of yeast cells, what can we say about the derivative $\dot{y}$?
The derivative represents the rate at which the number of cells is growing.
How should it depend on the number of cells?
In nature, cells given plenty of space and food tend to divide through mitosis regularly.
If we assume that each cell is dividing independently of all other cells, then doubling the number of cells should double the rate at which new cells are born.
In fact, multiplying the number of cells by any scalar factor should do the same to the derivative.
So this directly implies that the growth rate of cells is proportional to the number of cells:

\begin{equation*}
  \dot{y} \propto y.  
\end{equation*}

We can make this into a true equation by simply inserting a proportionality constant $a$, such that

\begin{equation*}
  \dot{y} = ay. 
\end{equation*}

We say that $1/a$ is a ``characteristic'' timescale for our problem, setting the rate at which the cells divide. A solution to the above differential equation is

\begin{equation*}
  y = y_0e^{at},
\end{equation*}
where $y_0$ is the number of yeast cells we started with at $t=0$.
In our case, we assume that $y_0$ is the number of yeast cells in a packet, which is about $180$ billion yeast cells.
\clearpage

\paragraph{Classification of differential equations:}
There are two kinds:
\begin{itemize}
\item An \emph{\color{blue} ordinary differential equation (ODE)}  involves derivatives of a function of \emph{only one} variable.
\item A \emph{\color{blue} partial differential equation (PDE)} involves \emph{partial derivatives} of a \emph{multivariable} function.
\end{itemize}

Here we will study ordinary differential equations. In \textit{Fourier Series and PDEs}, we will study partial differential equations. When we consider ODEs, we will often regard the independent variable to be time.\\

Notation for higher derivatives of a function $y(t)$:
\begin{align*}
  \text{first derivative: } \qquad  &\dot{y}\qquad y'\qquad \frac{dy}{dt}\\
  \text{second derivative: } \qquad  &\ddot{y}\qquad y''\qquad \frac{d^2y}{dt^2}\\
  \text{third derivative: } \qquad  &y^{(3)}\qquad \qquad \frac{d^3y}{dt^3}\\
                                    &\centerdot \\
                                    &\centerdot \\
                                    &\centerdot \\
  n^{th} \text{derivative: } \qquad &y^{(n)}\qquad \qquad \frac{d^ny}{dt^n}                              
\end{align*}

\emph{\color{orange} Warning: } The dot notation $\dot{y}$ should only be used to refer to a time derivative. If for example $y$ is a function of a spacial variable $y=y(x)$, we will only use the notation $y'$ to denote the derivative with respect to $x$.\\

\subparagraph{Definition 1.2} The \emph{\color{blue} order} of a DE is the highest $n$ such that the $n^{th}$ derivative of the function appears.


\clearpage

\subsubsection{Modeling a bank acount with withdrawls}
\subsubsection{Newtons second law and differential equations}

\subsection{Recitation 1: An Oryx population model}

\subsection{2 Solving first order linear DEs}
\subsubsection{Standard linear form}
\subsubsection{Modeling temperature conduction}
\subsubsection{Solving homogeneous first order linear DEs}
\subsubsection{Solving inhomogeneous first order linear DEs}
\subsubsection{Linear combinations and the superposition principle}
\subsubsection{Existence and uniqueness of solutions}

\subsection{Recitation 2: first order linear differential equations}

\subsection{Homework1: Part A}

\subsection{Homework1: Part B}

\section{Unit 2}
\clearpage

\subsection{3 Complex numbers}
\subsubsection{Operations with complex numbers}
\subsubsection{Geometry of complex numbers}
\subsubsection{Polar form}
\subsubsection{Euler's formula}
\subsubsection{Worked example}

\subsection{Recitation 3}

\subsection{4 Complex exponential}
\subsubsection{Definition, notation, and proof}
\subsubsection{Using complex exponential in integrals}
\subsubsection{Finding roots of polynomials}
\subsubsection{Roots unity}
\subsubsection{Examples}

\subsection{Recitation 4}

\subsection{5 Homogeneous 2nd Order Linear DEs}
\subsubsection{Harmonic Oscillator}
\subsubsection{Damped Harmonic Oscillator}
\subsubsection{Solving the homogeneous case: Characteristic polynomial}
\subsubsection{Distinct real roots}
\subsubsection{Distinct complex roots}

\subsection{Recitation 5}

\subsection{Homework2: Part A}

\subsection{Homework2: Part B}








\end{document}
